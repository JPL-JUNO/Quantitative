\chapter{其他反转形态\label{ch06}}
相比较而言,我们在 \autoref{ch04} 和 \autoref{ch05} 中所介绍的反转形态都是较强的反转信号。一旦它们出现,就表明多头已经从空头手中夺过了大权(比如说看涨吞没形态、启明星形态,或者刺透形态等),或者空头已经从多头手中抢回了主动权(比如说看跌吞没形态、黄昏星形态,或者乌云盖顶形态等)。本章要讨论更多的反转形态。其中一部分形态通常——但并不总是——构成反转信号,因此,它们是较弱的反转信号。这些形态包括\textbf{孕线形态、平头顶部形态、平头底部形态、捉腰带蜡烛线、向上跳空两只乌鸦、反击蜡烛线}等。此外,本章还要探讨一些强烈的反转信号,包括三只乌鸦、三山形态、三川形态、圆形顶、圆形底、塔形顶和塔形底。
\section{孕线形态}
纺锤线(即小实体的蜡烛线)是特定蜡烛线形态的组成部分。孕线形态就是这些形态中的一个例子(星线也是一个例子,这在第五章已有探讨)。如 \autoref{fig6-1} 所示即\textbf{孕线形态},其中后面一根蜡烛线的实体较小,并且被前一根的实体包进去了,日本人描述前一根蜡烛线为“非常长的黑色实体或白色实体”。本形态的名字来自一个古老的日本名词,意思就是“怀孕”。在本形态中,长的蜡烛线是“母”蜡烛线,而小的蜡烛线则是“子”或“胎”蜡烛线。孕线形态的第二根线既可以是白色的,也可以是黑色的。

\figures{fig6-1}{孕线形态}

孕线形态与吞没形态相比,两根蜡烛线的顺序恰好颠倒过来。在孕线形态中,前一个是非常长的实体,它将后一个小实体包起来。而在吞没形态中,后面是一根长长的实体,它将前一个小实体覆盖进去了。孕线形态与吞没形态的另一个区别是,在吞没形态中,两根蜡烛线实体的颜色应当互不相同。而在孕线形态中,这一点倒不是必要条件。

孕线形态与西方技术分析理论中的\textbf{收缩日}概念有相似之处。按照西方的理论,如果某日的最高价和最低价均居于前一日价格区间的内部,则这一天就是一个收缩日。但是,孕线形态并没有如此严格的要求。对孕线形态来说,所要求的全部条件是,第二个实体居于第一个实体内部,即使第二根蜡烛线的影线超越了第一根线的高点或低点也无所谓。
\subsection{十字孕线形态}
作为一条普遍的经验,在孕线形态中,第二根实体越小,则整个形态越有力量。这条经验通常都是成立的,因为第二个实体越小,市场的矛盾心态就越甚,所以越有可能形成趋势的反转。在极端情况下,随着第二根蜡烛线的开市价与收市价之间的距离的收窄,其实体便越来越小,最后就形成了一根十字线。在下降行情中,前面是一根长黑色实体(或者在上涨行情中前面是一根高高的白色实体),后面紧接着一根十字线,构成了一类特殊的孕线形态,\textbf{十字孕线形态}。

\notes{十字孕线形态也可能引发底部过程,不过,当这类形态出现在市场顶部时更有效力}。

在 \autoref{fig6-5} 中,从上吊线开始出现了一轮跌落行情,终于在 11 月 4 日—5 日通过一个孕线形态探得底部。孕线形态的第二根实体很短,因此我把它视为一根十字线。于是,这是一个十字孕线形态。这个形态尤其有意义,原因在于它的出现有助于清晰地验证位于 61 美元的先前定义的支撑水平(图中用水平直线标注)。如果这是一张线图,那么基于贯穿 9 月份始终的行情,我们也将得到同一个支撑水平。虽然我们采用的是蜡烛线图,依然可以,也应该用传统的线图支撑水平或阻挡水平。由此,这里有一个东方技术信号(孕线形态)验证了传统的西方技术信号(支撑水平线)。

在本图中,更早些的 9 月 29 日和 30 日组成了另一个十字孕线形态。随着这个孕线形态的出现,短线趋势从上升转为横向延伸。这个形态强调了一个要点:\important{当趋势发生变化时,并不意味着行情必将从上升转为下降,或者从下降转为上升}。在本图所示的两例孕线形态中,在前者出现后,原来的上升趋势并没有改变。具体说来,在 11 月的孕线形态出现后,趋势从下降转为上升,而在 9 月的孕线形态出现后,趋势从上升转为中性。在上述意义上,两个孕线形态都正确地预告了趋势的转变。

\figures{fig6-5}{亚马逊(Amazon)——日蜡烛线图(十字孕线形态)}
\section{平头顶部形态和平头底部形态}
\textbf{平头形态}是由几乎具有相同水平的最高点的两根蜡烛线组成的,或者是由几乎具有相同的最低点的两根蜡烛线组成的。在上升的市场中,当几根蜡烛线的最高点的位置不相上下时,就形成了一个\textbf{平头顶部形态}。在下跌的市场中,当几根蜡烛线的最低点的位置基本一致时,就形成了一个\textbf{平头底部形态}。平头形态既可以由实体构成,也可以由影线或者十字线构成。在理想情况下,平头形态应当由前一根长实体蜡烛线与后一根小实体蜡烛线组合而成。这样就表明,无论在第一个时段市场展现了什么样的力量(如果是长白色实体,展现的便是看涨的力量;如果是长黑色实体,展现的便是看跌的力量),到了第二个时段都被瓦解了,因为第二个时段是一个小实体,且其高点与第一个时段的高点相同(在平头顶部形态中),或其低点与第一个时段的低点相同(在平头底部形态中)。如果一个看跌的(对于顶部反转)蜡烛图信号,或看涨的(对于底部反转)蜡烛图信号,同时构成了一个平头形态,该形态就多了一些额外的技术分量。

应当对不同时间框架下的平头形态区别对待,日蜡烛线、日内蜡烛线的短线图形不同于周蜡烛线图和月蜡烛线图等的长线图形。这是因为如果两个交易日或者两个日内时段具备相同的高点或低点,并没有什么大不了的。仅当这类形态同时具备其他蜡烛图的特征时(例如第一根蜡烛线为长实体,第二根为短实体;或者既符合其他蜡烛图形态的要领,又具备相同的高点或低点),才值得注意。由此可见,对于日线图或日内图表上的平头形态,必须牢记的一个主要方面是,\important{在同时具备其他特定的蜡烛线组合特征的前提下,才可以根据平头形态采取行动}。

对希望获得关于市场的长期看法的朋友来说,不妨选用周蜡烛线图和月蜡烛线图进行研究,其中由相邻的蜡烛线形成的平头顶部形态和平头底部形态可能构成重要的反转信号。这类形态甚至在没有其他蜡烛图信号相验证的条件下,也是成立的。我们不妨用下面的例子来说明其原因。在周蜡烛线图或月蜡烛线图中,如果前一个时间单位的低点在后一个时间单位内成功地经受了市场向下的试探,这个低点就可能构成重要的市场底部,引发上冲行情。在日蜡烛线图或日内蜡烛线图上,就不能这么说了。

\autoref{fig6-16} 展示了一个平头顶部形态。2 月 2 日为小实体,它不居于前一根实体范围之内,并不构成孕线形态。两根蜡烛线具有相同的高点,即 55 美元,因此这属于平头形态。不仅如此,2 月 2 日的小实体是一根上吊线(其上影线足够短,可以视之为上吊线)。当然,这根上吊线(与任何上吊线一样)需要得到看跌信号的验证,即市场收市于上吊线的实体之下。下一个时段正是如此。

从上述平头顶部形态开始,戴尔形成了下降行情,一直持续到 2 月底—3 月初的一系列锤子线,它们标志着行情进入盘整区域。2 月 26 日和 3 月 1 日是前面的两根锤子线,它们并不能构成常规的平头底部形态。为什么?虽然两者的低点确实差不多相同,但是两根锤子线不满足平头底部形态的一项常规要求——平头底部形态的第一根蜡烛线应为长实体。(\warning{我觉得应该没有这一条件才对})尽管 2 月 26 日和 3 月 1 日的两根锤子线不能归类为平头底部形态,但是由于两者看涨的长下影线突出显示了市场正在排斥位于 39 美元的低价位,我依然要强调它们的重要性。如此一来,我就把 2 月 26 日和 27 日的蜡烛线组合看成平头底部形态的变体。

\figures{fig6-16}{戴尔公司(Dell)——日蜡烛线图(平头顶部形态)}

\section{捉腰带线}
\textbf{捉腰带形态}是由单独一根蜡烛线构成的。\textbf{看涨捉腰带形态}是一根坚挺的白色蜡烛线,其开市价位于本时段的最低点(或者,这根蜡烛线只有极短的下影线),然后市场一路上扬,收市价位于或接近本时段的最高点。看涨捉腰带线又称为\textbf{开盘光脚阳线}。如果市场本来处于低价区域,这时出现了一根长长的看涨捉腰带线,则预示着上冲行情的到来。

\textbf{看跌捉腰带形态}是一根长长的黑色蜡烛线,它的开市价位于本时段的最高点(或者距离最高价只有几个最小报价单位),然后市场一路下跌。在市场处于高价区的条件下,看跌捉腰带形态的出现,就构成了顶部反转信号。看跌捉腰带线有时也称为\textbf{开盘光头阴线}。

如果市场收市于黑色的看跌捉腰带线之上,则意味着上升趋势已经恢复。如果市场收市于白色的看涨捉腰带线之下,则意味着市场的抛售压力重新积聚起来了。如果捉腰带线得到了阻挡区域的验证,或者接连出现了几根捉腰带线,或者有一阵子没有出现捉腰带线,突然来了一根,这样的捉腰带线就更加重要。

在 \autoref{fig6-22} 中,6 月初出现了一个向上跳空,很快转化为支撑区域,随后在 6 月上半个月里多次成功地经受了试探,支撑区域得到了验证。6 月 13 日的蜡烛线是看涨的捉腰带线。到 7 月底、8 月初,市场再次向下试探该窗口的支撑作用,又形成了一系列看涨的捉腰带线。后面这两根看涨捉腰带线也分别与其前一根蜡烛线组成了两个背靠背的刺透形态。从 8 月初的低点开始形成上冲行情,持续到 8 月 9 日的流星线。

\figures{fig6-22}{力博通信公司(Redback Networks)——日蜡烛线图(看涨捉腰带线)}
\section{向上跳空两只乌鸦}
如 \autoref{fig6-23} 所示,为\textbf{向上跳空两只乌鸦形态},它很罕见。“向上跳空”指的是图示的小黑色实体与它们之前的实体(第一个小黑色实体之前的实体,通常是一根长长的白色实体)之间的价格跳空。

\figures{fig6-23}{向上跳空两只乌鸦}

在理想的向上跳空两只乌鸦形态中,第二个黑色实体的开市价高于第一个黑色实体的开市价,并且它的收市价低于第一个黑色实体的收市价。

这个形态在技术上看跌的理论依据大致如下:市场本来处于上升趋势中,并且这一天的开市价同前一天的收市价相比,是向上跳空的,可是市场不能维持这个新高水平,结果当天反而形成了一根黑色蜡烛线。到此时为止,多头至少还能捞着几根救命稻草,因为这根黑色蜡烛线还能够维持在前一天的收市价之上。第三天,又为市场抹上了更深的疲软色彩:当天市场曾经再度创出新高,但是同样未能将这个新高水平维持到收市的时候。然而更糟糕的是,第三日的收市价低于第二日的收市价。如果市场果真如此坚挺,那么为什么它不能维持新高水平呢?为什么市场的收市价下降了呢?这时候,多头心中恐怕正在惴惴不安地盘算着上述两个问题。思来想去,结论往往是,市场不如他们当初指望的那样坚挺。如果次日(也就是指第四天)市场还是不能拿下前面的制高点,那么,我们可以想见,市场将会出现更低的价格。

\autoref{fig6-25} 说明了把蜡烛图形态与它贴身的周边环境相结合的重要意义。7 月中旬虽然出现了一个向上跳空两只乌鸦形态,但是它本身并不构成卖出信号。这是因为,7 月 17 日该股票向上跳空,通常向上跳空是行情坚挺的征兆——无论是在蜡烛图上还是在线图上都是如此。因而,尽管该向上跳空两只乌鸦形态发出了警告信号,我认为它没有那样疲软。

\figures{fig6-25}{康宁公司——日蜡烛线图(向上跳空两只乌鸦)}

\section{三只乌鸦}
在向上跳空两只乌鸦形态中,包含了两根黑色蜡烛线。如果连续出现了三根依次下降的黑色蜡烛线,则构成了所谓的\textbf{三只乌鸦形态}。如果三只乌鸦形态出现在高价格水平上,或者出现在经历了充分延伸的上涨行情中,就预示着价格即将下跌。有的时候,三只乌鸦形态又称作\textbf{三翅乌鸦形态}。从外形上说,这三根黑色蜡烛线的收市价都应当处于其最低点,或者接近其最低点。在理想的情形下,每根黑色蜡烛线的开市价也都应该处于前一个实体的范围之内。

如 \autoref{fig6-27} 所示,从 4 月 15 日开始形成了一个三只乌鸦形态。从三只乌鸦开始的下降行情一路几乎不受阻碍地延续到了 P 处的刺透形态。三只乌鸦形态中的第二根和第三根蜡烛线(4 月 16 日和 17 日)都开市于之前的实体之下。虽然在常规的三只乌鸦形态中后续蜡烛线的开市价居于之前的黑色实体内部,但是这两根蜡烛线的开市价低于之前的实体,可以视作更加疲软的信号。其原因在于,第二根和第三根黑色实体开市价低于前一日的收市价,之后在整个交易日里都无力夺得实质性的立足地。

三只乌鸦形态可能对长线交易者(做空)更有用处。这是因为本形态在第三根蜡烛线才能完成。显然,到了这个时候,市场已经回落了相当大的幅度。举例来说,上述三只乌鸦是从 70.75 美元处开始的。既然我们需要第三根黑色实体来完成形态,那么在得到信号的时候,股价已经跌到了 67.87 美元。

无论如何,在本例中,当三只乌鸦形态的第一根黑色蜡烛线出现的时候,我们就可以看出行情遇到麻烦的一点端倪了。个中缘由是,股票当日开市于之前 3 月份的历史高点 70 美元之上,然而牛方未能坚守上述新高,当日收市时,反而跌回到 70 美元以下。如果市场先创新高,之后却不能守住,可能带有看跌的意味。这里的情况便是这样。

\figures{fig6-27}{鹏斯公司(Pennzoil)——日蜡烛线图(三只乌鸦形态)}

我们再来看看之前 1 和 2 处的高点。在 2 月初的 1 处,鹏斯公司为当前行情创了新高,但这里的一组蜡烛线却向我们发出了强烈的“火光”警告信号,行情并不像表面上那么顺利。具体说来,在 2 月 2 日所在的一周的后几天,尽管股价不断创下更高的高点、更高的低点、更高的收市价,但是它们都是小实体,都有长长的上影线。这肯定显示出当前的行情变化其实并不是一面倒地有利于牛方。之后价格回落,直到B处的看涨吞没形态才结束。从此处开始的一轮上冲行情持续推升,持续到 3 月 2 日所在的一周,图中用 2 做了标记。2 处的上冲行情与 1 处的上涨行情有异曲同工之妙,2 处的行情也有更高的高点、更高的低点、更高的收市价,这样如果在线图上看起来,行情显得很健康。然而从蜡烛图的角度来看,3 月 4 日、5 日、6 日的价格攀升带有长长的上影线。这一点证明多方相对强势的力量正在涣散。3 月 6 日的蜡烛线是一根流星线。

\section{白色三兵挺进形态}
与三只乌鸦形态相对的形态称为\textbf{白色三兵挺进形态},或者更通俗的说法是\textbf{白三兵形态}(如 \autoref{fig6-28})。本形态是由接连出现的三根白色蜡烛线组成的,它们的收市价依次上升。当市场在某个低价位稳定了一段时间后,如果出现了这样的形态,就标志着市场即将转强。

白三兵形态表现为一个逐渐而稳定的上升过程,其中每根白色蜡烛线的开市价都处于前一天的白色实体之内,或者处在其附近的位置上。每一根白色蜡烛线的收市价都应当位于当日的最高点或接近当日的最高点。这是市场的一种很稳健的攀升方式(不过,如果这些白色蜡烛线伸展得过长,那么我们也应当对市场的超买状态有所戒备)。

\figures{fig6-28}{白色三兵挺进形态}

如果其中的第二根和第三根蜡烛线,或者仅仅是第三根蜡烛线,表现出上涨势头减弱的迹象,那就构成了一个\textbf{前方受阻形态}(如 \autoref{fig6-29} 所示)。这就意味着这轮上涨行情碰到了麻烦,持有多头头寸者应当采取一些保护性措施。特别是在上升趋势已经处于晚期阶段时,如果出现了前方受阻形态,则更得多加小心。\tips{在前方受阻形态中,作为上涨势头减弱的具体表现,既可能是其中的白色实体一个比一个小,也可能是后两根白色蜡烛线具有相对较长的上影线。}如果在后两根蜡烛线中,前一根为长长的白色实体,并且向上创出了新高,后一根只是一个小的白色蜡烛线,那么就构成了一个\textbf{停顿形态}(如 \autoref{fig6-30} 所示)。有时候,这种形态也称为\textbf{深思形态}。当这一形态出现时,说明牛方的力量至少暂时已经消耗尽了。在本形态中,最后一根小的白色蜡烛线既可能从前一根长长的白色蜡烛线向上跳空(这种情况下,该蜡烛线就变成了一根星线),或者正如日本分析师所描述的那样“骑在那根长长的白色实体的肩上”(这就是说,位于前一根长长的白色实体的上端)。这根小小的实体暴露了牛方能量的衰退。当停顿形态发生时,便构成了多头头寸平仓获利的紧要时机。

\figures{fig6-29}{前方受阻形态}

虽然前方受阻形态与停顿形态在一般情况下都不属于顶部反转形态,但是有时候,它们也能引导出不容忽视的下跌行情。\important{我们应当利用前方受阻形态和停顿形态来平仓了结已有的多头头寸,或者为多头头寸采取保护措施,但是不可据之开立空头头寸}。一般来说,如果这两类形态出现在较高的价格水平上,则更有预测意义。

前方受阻形态与停顿形态之间并没有太大差异。关于白三兵,需要考虑的主要因素是,如果三根蜡烛线的每一根的收市价都位于或接近本时段的最高价,则最具有建设性。如果后两根蜡烛线表现出犹豫的迹象,是小实体,或者是上影线,那么这些线索表明,上冲行情正变得衰弱。

\figures{fig6-30}{停顿形态}

\autoref{fig6-32} 是白三兵形态良好的实例。三根白色蜡烛线的收市价都非常接近本时段最高价,每一根的开市价都位于先前一根实体的内部或之上。关于白三兵形态需要考虑的一个方面是,等到白三兵形态完成时,市场可能已经明显脱离其低位了。在本例中,微软离开其低位差不多 4 美元,这可是较大比例的行情变化了。因此,\tips{除非交易者长线看好,在白三兵形态完成时买进或许并不具备有吸引力的风险报偿比。}

我发现,\tips{在白三兵形态出现后,一旦行情调整,则其中的第一根或第二根白色蜡烛线,即白三兵的起点处,经常构成支撑水平}。在本例中,在白三兵形态出现后股票进入整理阶段,缓缓回落,直到形成一根锤子线为止。这验证了白三兵形态中第二根蜡烛线内部形成的支撑水平。

\figures{fig6-32}{微软——日蜡烛线图(白三兵形态)}
\section{三山形态和三川形态}
与西方的三重顶形态相似,日本也有所谓的\textbf{三山顶部形态}(如 \autoref{fig6-35} 所示)。一般认为,本形态构成了一种主要顶部反转过程。如果市场先后三次均从某一个高价位上回落下来,或者市场对某一个高价位向上进行了三次尝试,但都失败了,那么一个三山顶部形态就形成了。在三山顶部形态的最后一座山的最高点,还应当出现一种看跌的蜡烛图指标(比如说,一根十字线,或者一个乌云盖顶形态等),对三山顶部形态做出确认。

在三山顶部形态中,如果中间的山峰高于两侧的山峰,则构成了一种特殊的三山形态,称为\textbf{三尊顶部形态}。

\figures{fig6-35}{三山形态}

\textbf{三川底部形态}恰巧是三山顶部形态的反面。在市场先后三度向下试探某个底部水平后,就形成了这类形态。市场必须向上突破这个底部形态的最高水平,才能证实底部过程已经完成。与西方的头肩形底部形态(也称为倒头肩形)对等的蜡烛图形态是变体三川底部形态,或者称为\textbf{倒三尊形态}。

\autoref{fig6-42} 是三尊顶部形态的又一个例子。既然本形态与头肩顶同质而异名,我们就可以转而采用西方技术分析,正如上一个图例所讨论的,以头肩形颈线的概念为基础来分析。

具体说来,一旦头肩顶颈线被向下突破,它从支撑作用转为阻挡作用。4 月 10 日 13:30,市场力图通过一个小小的看涨吞没形态来站稳脚跟,但是市场没能突破颈线的阻挡作用,将指数推升到颈线之上,说明空方保持着控制权。这反映出,关键的一点是要弄清楚蜡烛图形态到底是在什么样的位置形成的。在本例中,看涨的吞没形态是潜在的底部反转信号,不过,如果等市场以收市价突破到颈线阻挡水平之上之后,再从容买进,岂不是更有道理——即使拿到了看涨吞没形态的这张好牌?耐心等待是值得的,因为如果行情收市到颈线之上无疑有助于增强信心,表明多头更占上风。

\figures{fig6-42}{纳斯达克 100 指数——15 分钟蜡烛线图(三尊顶部形态)}

\section{反击线形态(约会线形态)}
当两根颜色相反的蜡烛线具有相同的收市价时,就形成了一个\textbf{反击线形态}(也称为\textbf{约会线形态})。

我们不妨把看涨反击线形态同看涨刺透形态做一番比较。如刺透形态与看涨反击线形态一样,也是由两根蜡烛线组成的。它们之间的主要区别是,看涨反击线通常并不把收市价向上推进到前一天的白色实体的内部,而是仅仅回升到前一天的收市价的位置。而在刺透形态中,第二根蜡烛线深深地向上穿入了前一个黑色实体之内。因此,刺透形态与看涨反击线形态相比较,是一种更为重要的底部反转信号。尽管如此,正如我们下面列举的一些实例所显示的,对看涨反击线形态还是不可小觑的,因为它的出现表明,行情流动的方向正在改变。

如果说看涨反击线形态与刺透形态有渊源的话,那么,看跌反击线形态与乌云盖顶形态也有类似的关系。在理想的看跌反击线形态中,第二天的开市价高于前一天的最高点,这一点与乌云盖顶形态是一致的。不过,与乌云盖顶形态不同的是,这一天的收市价并没有向下穿入前一天的白色蜡烛线之内。由此看来,乌云盖顶形态所发出的顶部反转信号,比看跌反击线形态更强。

在反击线形态中,一项重要的考虑因素是,\tips{第二天的开市价是否强劲地上升到较高的水平(在看跌反击线形态中),或者是否剧烈地下降到较低的水平(在看涨反击线形态中)}。这里的核心思想是,在该形态第二天开市时,市场本已经顺着既有趋势向前迈了一大步,但是后来,却发生了意想不到的变故!到当日收市时,市场竟然完全返回到前一天收市价的水平!如此一来,朝夕之间竟扭转了行情基调。

在 \autoref{fig6-48} 中的 10 月 15 日,展现了一例看跌的反击线形态。我们可以看出,该反击线的收市价并没有恰好处于前一时段白色蜡烛线的收市价,而是稍稍低于后者。在判断反击线形态时,对形态的定义应该留有适当变通的余地,这对于绝大多数蜡烛图信号都适用。举例来说,12 月 6 日有一个看涨的反击线。当日开市时,股价剧烈下跌向下跳空,当日收市时,收市价接近前一日的收市价,而不是恰好位于。即使第二天的收市价与第一天的收市价不是恰好相同,也肯定足够接近了,因此我认为这属于看涨反击线形态。判别这个看涨反击线形态的主要标准是,虽然白色蜡烛线的开市价非常疲软,却能够当日完全反弹,令人刮目相看。

\figures{fig6-48}{吉列公司(Gillette)——日蜡烛线图(看跌的和看涨的反击线形态)}
\section{圆形顶部形态和平底锅底部形态(圆形底部形态)}
在\textbf{圆形顶部形态}(如 \autoref{fig6-51} 所示)中,市场逐步形成向上凸起的圆弧状图案,在这个过程中,通常出现的是一些较小的实体。最后,当市场向下跳空时,就证明圆形顶部形态已经完成。这一形态与西方的圆形顶部形态是相同的。不同的是,在日本的圆形顶部形态中,应当包含一个向下跳空,作为市场顶部的附加验证信号。

\figures{fig6-51}{圆形顶部形态}

在平底锅底部形态中,市场从更低的低点转为相同的低点,再转为更高的低点。这个过程形象地证明了空头正在逐步丧失立足之地。在这之后,再添加一个向上跳空,带来进一步的证据,表明空头失去了对市场的控制权。

对圆形顶部形态来说,道理相同,而方向相反。也就是说,市场从更高的高点转为相同的的高点,再转为更低的高点。于是,上升行情的节奏松弛下来。在这之后,再来一个向下跳空,便完成了圆形顶部形态。跳空对多头行情做了进一步的盖棺论定。

\autoref{fig6-58} 清晰地显示了西方圆形底部形态和东方平底锅底部形态的区别。从 9 月 1 日到 9 月 14 日所在的一周里,该股票构造了一个圆形底部形态(因为它从更低的低点转向相同的低点,再转向更高的低点)。然而,既然在该圆形底部形态中没有向上跳空,那么它便不是平底锅底部形态。请记住,平底锅底部形态与西方的圆形底部形态同出一辙,只不过它附加了向上跳空作为最后的一推。与上述圆形底部形态同时出现的是数字 1 和 2 标注的两处信号,两条长长的上影线(且 2 处是一根流星线)。

现在让我们把注意力转向 10 月从头到尾的价格变化。在这期间,该股票逐步构筑了一个圆形底部形态(也就是从更低的低点转向更高的低点)。10 月 26 日出现了一个小幅的向上跳空,于是,这就完成了一个平底锅底部形态,带着其应有的全部看涨潜力。因为平底锅底部形态包含这个向上跳空(常规的圆形底部形态没有向上跳空),所以我认为平底锅底部形态比常规的圆形底部形态更重要。

\figures{fig6-58}{地壳公司(Earthshell)——日蜡烛线图(平底锅底部形态)}
\section{塔形顶部形态和塔形底部形态}
\textbf{塔形顶部形态}出现在高价格水平上。市场本来处在上升趋势中,在某一时刻,出现了一根坚挺的白色蜡烛线或者一系列白色蜡烛线,然后市场放缓了上涨的步调,接着出现了一根或者数根大的黑色蜡烛线,于是塔形顶部形态就完成了(如 \autoref{fig6-59} 所示)。在本形态中,中间有若干小实体,两侧长长的白色和黑色蜡烛线形如“高塔”。也就是一边由长蜡烛线组成下跌的一侧,另一边由长蜡烛线组成上涨的一侧。

\figures{fig6-59}{塔形顶部形态}

\textbf{塔形底部形态}发生在下降行情中(如 \autoref{fig6-60} 所示)。市场形成了一根或数根长长的黑色蜡烛线,表示空方动力丝毫不减。后来出现了几根小实体,缓和了行情看跌的气氛。最后出现了一根长长的白色蜡烛线,完成了一个塔形底部形态。

\figures{fig6-60}{塔形底部形态}

\autoref{fig6-61} 揭示了塔形顶部形态与圆形顶部形态的区别。该股票在 10 月的第一周上涨,形成了一系列白色实体,但之后开始踩水,留下了一系列小实体。10 月 15 日的向下跳空完成了一个圆形顶部形态。把注意力转向 12 月,我们看到了一系列延长的白色蜡烛线。蜡烛线1出现表明该股票继续上涨的机会不足了。长黑色蜡烛线 2 带来了第二支“塔”,完成了塔形顶部形态。

\figures{fig6-61}{CNB Bancshares——日蜡烛线图(塔形顶部形态与圆形顶部形态)}