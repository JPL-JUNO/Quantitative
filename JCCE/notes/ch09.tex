\chapter{蜡烛图技术汇总\label{ch09}}
日本老话说:“听说不等于经验。”只有自己在市场上,通过亲身经验,才能发掘出这些蜡烛图工具的全部潜能。

\autoref{fig9-1} 说明了以下的各种形态和蜡烛线。

1.一根长上影线的蜡烛线。这是一条线索,表明多方有些犹豫,不过这条线索微不足道,因为一个时段的长上影线不足以改变市场方向的基调。此处也没有足够长的历史行情帮助我们判断市场是否处于超买状态。

2.流星线的出现证实了之前蜡烛线 1 的长上影线构成的潜在阻挡水平。

3. 又一根长上影线的蜡烛线。三根看跌的长上影线的高点差不多在同一个水平,确实足以构成理由警醒我们,要加强注意。该蜡烛线与流星线的轮廓相同(即具有长上影线和位于整个交易区间下端的小实体),然而,流星线应当出现在上涨行情之后。在这里,市场正在横向延伸。如此一来,我们不把它视为流星线,这根蜡烛线的上影线是令人担心的原因,因为它证实了流星线 2 揭示的麻烦。

4. 向下的窗口进一步加强了 1、2、3 三处的长上影线的看跌意味。

5. 这是一个小型的刺透形态,或许会引发一点乐观的情绪。然而,从市场总体的技术图像来观察,该刺透形态形成于窗口 4 定义的阻挡区域之下。于是,如果我们缘于刺透形态来买进,就买在了阻挡水平之下。刺透形态之后,出现了一根拉长的黑色实体,使得空方重新执掌大权。

6. 这根锤子线暗示空方的动力正在松懈。之后的几天里,市场稳定下来,以锤子线的低点作为支撑水平——直到下一个蜡烛图信号出现。

7. 长黑色实体的收市价位于6处锤子线的低点之下。这导致趋势方向恢复向下。这一天同时还完成了一个看跌的下降三法形态。

8. 向下的窗口为空方的势头雪上加霜。6 月 10 日是一根十字线,一方面它打开了这个窗口,另一方面它也发出了一个微弱的暗示,显示空方可能后继乏力。但是,十字线在下降趋势中一般作用不大,不如在上升趋势中的作用。另外,现在窗口已经成为阻挡区域,要改变趋势首先必须克服这个障碍。

9. 另一个长黑色实体,其上影线证实了前一日向下的窗口的阻挡作用。6 月 14 日,市场大幅向上跳空(与前一个时段的收市价相比),令人印象深刻,而且在整个时段内费尽力气守住了开市时的价位,最后收市于较高水平。6 月 11 日和 6 月 14 日两个时段组成了孕线形态。这样在一定程度上中和了 6 月 11 日蜡烛线的看跌力量。然而,8处向下的窗口依然发挥作用,6 月 16 日的长上影线揭示了这一点,它的高点处于该窗口阻挡区域的范围内,接近 114 3/4。

10. 6 月 17 日长长的白色蜡烛线最终把市场推进到了该窗口的阻挡区域之上,把趋势转向了更积极的方向。还请注意,自从 6 月 11 日的长黑色蜡烛线以来,每个时段的蜡烛线都带有更高的高点和更高的低点。

11. 一根小实体是许多时段以来第一次出现更低的高点。此外,这根微小的黑色实体局限在之前拉长的白色蜡烛线范围之内,组成了一个孕线形态。这意味着多方已经喘不过气来了。从此处开始,市场稳步下降。

12. 蜡烛线的下影线维持了6月11日的低点构成的113.25附近的支撑水平,因此带来了一点点好消息,市场正在力图站稳脚跟。

13. 不幸的是(或者幸运的是,取决于您是多头还是空头),这根蜡烛线为当前行情创了新低,日内创了新低,收市价也创了新低。看起来,空头似乎又杀回来了。

14. 十字线带来了双倍的利好。首先,这根十字线位于前一根黑色实体内部,两者形成了一个十字孕线形态。更重要的是,其收市价重新回到了 113.25,显示前一天出现的低点不能维持。这一点可能促使空头立场动摇,而鼓舞那些寻机买进的人。

15. 这根白色蜡烛线把市场向看好的方向再推了一把,因为它包裹了前一根小实体,形成了一个看涨吞没形态。

16. 从 15 处开始的上涨行情一帆风顺,直到 7 月 1 日的小实体构成了一个孕线形态。有趣的是,本孕线形态具备一根超长的白色蜡烛线和一根小黑色实体,它出现的位置和它的轮廓与几个星期之前由蜡烛线 10 和 11 构成的孕线形态如出一辙。

17. 行情从 16 处的孕线形态开始下降,不过图形显示空方并不能完全控制局面,因为在这轮小规模的下跌行情中包括了一系列看涨的长下影线。这些蜡烛线也都具有小实体。

18. 另一根长白色蜡烛线(也是一根看涨的捉腰带线),位于与 6 月 30 日长长的白色蜡烛线相同的位置,为行情上涨奠定了基础。在这根长长的白色蜡烛线之后,下一个时段又是一根小实体。小黑色实体紧随高高的白色实体,让我们想起了蜡烛线 10 和 11 以及 16 处的孕线形态。其间的区别在于,这里的小黑色实体(7 月 9 日)并没有局限在之前长长的白色实体的内部。如此一来,它并不构成孕线形态,与蜡烛线 10 和 11 的组合以及 16 处的形态不同。另一方面,既然 7 月 9 日的黑色实体没有深深地扎入之前的白色实体内部,它也不构成乌云盖顶形态。

19. 在这个时间范围内,市场在当前上涨行情的最高位置附近波动。但是,这些小实体蜡烛线以及它们看跌的长上影线传递了一种感觉,市场正在与之前的上升趋势分道扬镳。这里的行情表现出的犹豫状态并不令人吃惊,因为 117 是一个阻挡水平,由5月底向下的窗口所构成,这在 4 处曾经讨论过。

20. 白色蜡烛线包裹了黑色实体,这是组成看涨吞没形态的正确的蜡烛线组合。然而,这并不是一个看涨吞没形态,因为这种形态属于底部反转信号,必须出现在价格下跌之后。

21. 7 月 26 日的白色蜡烛线开市价走低,后来的收市价却回升到与前一个时段收市价同样的水平上。这是一根看涨的反击线。这将趋势转向了不那么疲软的方面。

22. 两根小的白色实体从 7 月 26 日的蜡烛线出发,稍稍向上跳空,形成了向上跳空并列白色线形态。这是另一个正面指标。

23. 8 月 2 日的蜡烛线向下打破了在 7 月 26 日和 27 日之间打开的小型向上的窗口构成的支撑水平。尽管这里跌破了支撑水平,但是8月2日的蜡烛线构成了锤子线。这就在114上下(锤子线的低点)提供了潜在的支撑作用,第二天就得到了接踵而至的蜻蜓十字线的验证。

24. 虽然 8 月 2 日的蜡烛线在日内变化中一度向下突破锤子线的支撑水平,但市场挣扎着回升,到本时段结束时,收市价达到该支撑区域之上,并形成了一个看涨吞没形态。

25. 一根拉长的黑色实体夺去了市场的勇气,但市场好歹维持了来自

24. 处看涨吞没形态的低点的支撑水平。接下来一个时段,8 月 9 日,支撑水平终于被跌破。无论如何,现在市场已经接近主要支撑区域了,这是在 6 月下旬于 112.75-113 处形成的。于是市场接近支撑区域,不过尚且没有看到反转信号。

26. 倒锤子线发出了很有试探意味的线索,接近 113 的支撑水平或许能守得住。虽然如此,因为倒锤子线的形状是疲软的,我们必须等待下一个时段看涨的验证信号,其收市价向上越过了倒锤子线的实体,变得稍稍积极一点。验证信号来自27处。

27. 锤子线构成验证信号。

28. 8 月 13 日的白色实体完成了一个看涨的吞没形态。相应地,由于 26 处的倒锤子线、27 处的锤子线,以及本看涨吞没形态,我们得到了有力的图形信号,表明来自 6 月的接近 113 的支撑水平稳如泰山。

29. 出现在长长的白色蜡烛线之后的十字线可能构成值得戒惧的信号。面对长白色蜡烛线之后的十字线(或面对任何蜡烛图信号),首要的考虑是市场是否处在超买状态或超卖状态。由于这根十字线出现之处刚刚脱离了最近的低点,显然,我认为这里谈不上超买状态。因此,这根十字线并不带有太多的反转意义。

30. 在 8 月 16 日所在一周的下半周,冒出了一群纺锤线,使得趋势方向从向上转为中性。到 8 月 24 日白色蜡烛线收市时,完成了一个上升三法形态。本形态由 8 月 17 日到 8 月 24 日的蜡烛线组成。

31. 本十字线(它的实体如此之小,以至于在我看来它与经典的十字线具有同等效力)充分说明了观察市场总体技术环境的重要性。与 29 处的十字线相比,本十字线处在更为超买的市场环境下。因此,我们可以认为 31 处的十字线比 29 处的十字线更有预测意义。

32. 因为 31 处的十字线尚且位于最高点附近上下波动,我更倾向于等待进一步的看跌验证信号,以支持该十字线潜在的反转信息。验证信号来自此处的黑色实体,其收市价居于十字线的收市价之下。这根黑色实体完成了一个黄昏星形态。

33. 这个小型向下的窗口维持了看跌动力的继续。不过,此处也有一项小小的正面因素。9 月 2 日的蜡烛线依然保住了关键支撑水平的有效性,该支撑水平来自 6 月下旬,位于接近 112.75-113 处。

34. 一根长长的白色蜡烛线的开市价与前一根蜡烛线的开市价几乎处在同一个水平。如此一来,就可以把它归结为分手线。既然 113 附近的支撑水平继续维持坚固,就为乐观态度提供了一点由头。然而,下一日是一根黑色蜡烛线,未能延续上述势头,断送了任何看好的乐观苗头。

35. 一根相对长的黑色实体保持了疲软的市场基调,但是多头依然存有希望,因为 112.75-113 的支撑区域还是完好无缺的。

36. 一根小的白色实体,其收市价向上超越了前一根黑色实体,有助于巩固接近 113 处的支撑水平。不过,这不是一个刺透形态,因为刺透形态要求白色实体的收市价向上推进到之前黑色实体的中点之上。

37. 9 月 16 日的十字线向上打开了一个很小的窗口。因为市场并没有上涨得多么长远,我倾向于不把这根十字线归结为警告信号,特别是考虑到在十字线处向上打开的窗口具有潜在的支撑作用。然而,下一个时段,向上的窗口所形成的支撑区域被跌破了。

38. 通过从 9 月 14 日到 23 日的蜡烛线的低点,我们可以绘制一根上升支撑线。将趋势线的力量与蜡烛图结合起来,实际上也就是要把其他许多西方技术工具和蜡烛图结合起来,这是一个重要方面。

\figures{fig9-1}{债券期货——日蜡烛线图(汇总)}