\documentclass{book}
\usepackage{amsmath, amssymb}
\usepackage{ulem}
\usepackage{mathptmx}
\usepackage{ctex}
\usepackage{tcolorbox}
\usepackage{epigraph}
\usepackage{caption}
\usepackage{subcaption}
\usepackage{pdfpages}
\usepackage{graphicx}
\setkeys{Gin}{width=0.4\textwidth}
\usepackage{listings}
\newtheorem{theorem}{Theorem}
\newtheorem{definition}{Definition}
\usepackage{tikz}
\usepackage{pifont}
\usepackage{tabularx}
% \usepackage{algorithm}
\usepackage[lined,boxed,ruled]{algorithm2e}
\usepackage{bm}


% 在part页添加内容
\makeatletter
\let\old@endpart\@endpart
\renewcommand\@endpart[1][]{%
    \begin{quote}#1\end{quote}%
    \old@endpart}
\makeatother


\newcommand\tips[1]{\textcolor{green!50!black}{#1}}
\newcommand\notes[1]{\textcolor{blue!50!black}{#1}}
\newcommand\important[1]{\textcolor{red!90!black}{#1}}

\newcommand\figures[2]{
    \begin{figure}
        \centering
        \includegraphics{../img/#1.png}
        \caption{#2}
        \label{#1}
    \end{figure}
}

\usepackage{enumitem}
% 去掉enumerate、itemize、description中的间隙
\setlist{noitemsep, topsep=0pt}

\tcbuselibrary{breakable}
\tcbset{breakable}
% \newtcblisting[auto counter, number within =chapter]{py}[1]{listing engine=minted,
% 	minted style=colorful,
% 	minted language=python,
% 	minted options={breaklines,autogobble,linenos,numbersep=3mm},
% 	colback=red!5!white,colframe=orange!50!blue,listing only, left=5mm,enhanced,
% 	title=Examples~\thetcbcounter~#1,
% 	breakable,
% 	enhanced
% 	%						 overlay={
% 	%						 	\begin{tcbclipinterior}
% 	%						 		\fill[red!30!white] (frame.south west)
% 	%						 		rectangle ([xshift=5mm]frame.north west);
% 	%							\end{tcbclipinterior}}
% }
\usepackage[
    top=1.23in,
    bottom=1.23in,
    right=1in,
    left=1in]
{geometry}

\usepackage{hyperref}
% 将引用的chapter改写为Chapter
\usepackage[english]{babel}
\addto\extrasenglish{
    \def\chapterautorefname{Chapter}
}
\addto\extrasenglish{
    \def\sectionautorefname{Section}
}
\usepackage{orcidlink}
\definecolor{SWJTU}{HTML}{025483}
% 全局取消段前缩进
% \setlength{\parindent}{0pt}
% \setlength{\parskip}{5pt}
\hypersetup{
    colorlinks=true,
    linkcolor=SWJTU,
    filecolor=SWJTU,
    urlcolor=SWJTU,
    citecolor=SWJTU,
}



\setkeys{Gin}{width=0.55\textwidth}
\begin{document}
\frontmatter
\begin{multicols}{2}
    \tableofcontents
\end{multicols}
\chapter*{前言}
\section{一些私人理解}
很多 pattern 分析的心理都是空和多,但是下跌中,如果只是有本持有的多方卖出,他们并不一定需要回补,那么看涨情绪在不能做空的市场中应该要稍弱。
\mainmatter
% \chapter{个体心理}
\section{现实与幻觉}
\subsection*{交易时睁大双眼}
一厢情愿的力量甚于金钱。最近的研究证明,人们都有惊人的欺骗自己和回避现实的能力。
\begin{tcolorbox}
    你必须像分析你的交易那样分析你的心理情绪,以确保能做出正确的决定。你的交易必须建立在定义明晰的规则之上,你得学会规范资金管理,以确保在出现连续亏损后也不会被踢出局。
\end{tcolorbox}
\section{自我毁灭}
\subsection*{赌博}
相比于在赌马游戏中填数字,在金融市场中赌博有着更耀眼的光环,给人以高深莫测的感觉。
\subsection*{自毁}
当交易者陷入困境时,他们往往去责怪别人、坏运气或者其他随便什么。毕竟寻找导致自身失败的原因总是痛苦的。
\begin{tcolorbox}
    坚持写交易日志——记录每次建仓和清仓的理由,寻找你成功和失败的重复模式。
\end{tcolorbox}
\section{胜利者与失败者}
\subsection*{大海般的市场}
市场不知道你的存在,你的存在也无法影响市场。它不会故意关照你的财富,也不会特意伤害你的资金。你要做的,只是控制自己的行为。

经过一连串盈利之后,新手可能觉得他可以在交易市场上如鱼得水来去自如,他开始肆无忌惮地冒险,直到亏得一无所有。还有一种可能,业余交易者在经历连续损失之后,倍受打击,以至于当自己的交易系统给出明确的买卖信号时,他也无法下定决心做出指令。当交易让你得意忘形或垂头丧气时,你的神智已被遮蔽。当喜悦让你飘飘然时,你会非理性交易,然后亏损;当恐惧支配着你的心灵时,你会错过获利的机会。

作为交易者,当被市场整得死去活来的时候,该做的便是减少交易的规模。记住,尚在摸索学习中或感到紧张之时,千万要减小交易规模。

\subsection*{为你的生涯负责}
你可以根据自己的情况调整以下这个清单:
\begin{enumerate}
    \item 坚定自己在市场中长期作战的意念——从现在开始至少交易 20 年。
    \item 像海绵一般地学习,关注专家的观点,但对任何事情都要保持有益的怀疑态度。遇到有疑问的地方要刨根究底,而不是简单地接受专家的观点,或只理解他们字面上的意思。
    \item 不贪婪,不急于交易——要把你的时间用于学习,市场一直在这里,未来无尽的岁月中会有更多更好的机会。
    \item 培养分析市场的方法,换句话说,就是“如果 A 发生,那么 B 很可能会发生”。市场有很多维度,要使用多种解析方法来确认自己的交易决策。要学会用历史数据测试交易决策,随后在市场上真枪实弹地进行交易。市场瞬息万变,你需要的是根据牛市、熊市、震荡市等不同的特征采用不同的工具进行交易,同时还要有所区分。
    \item 建立一套资金管理计划。你的第一目标是必须长期生存下去,第二目标是资本的稳定增长,第三目标才是赚取高额利润。大多数交易者对第一目标和第三目标的重要性产生了混淆,将第三目标放在了第一位,更有甚者,都不知道第一目标和第二目标的存在。
    \item 要认识到交易者在任何交易系统中都是最为薄弱的一环。假如有条件的话,去匿名戒酒会学习一下如何避免损失,或者建立一套属于自己的方法来克制情绪化交易。
    \item 胜利者在思考、感受与行动上的方式与失败者是完全不同的。你必须深探自己的内心,驱赶那些幻觉,改变你原来的思考、感受与行动的方式。这样的改变通常都不容易,但如果你想成为一名专业交易者,你必须专注于自我改善和培养自己的个性。
\end{enumerate}
% \part{基本知识}
% % \chapter{其他买入看涨期权的策略}
我们将讨论另外两种买入看涨期权的策略。这两种策略都涉及卖空标的股票和买入看涨期权。当标的股票有场内看跌期权交易的时候,这些策略就没有使用看跌期权那么好。不过,这里的概念相当重要,并且当市场中看涨期权非常活跃而看跌期权不活跃时,这些策略就会更有活力。这些策略一般被称为“合成”(synthetic)策略。
\section{保护性卖空(合成看跌期权)}
在卖空标的股票的同时买入看涨期权,是将卖空的风险限制在一定数额里的一种手段。从理论上来说,卖空的风险是无限的,因此许多投资者在卖空时都会有所顾虑。对这些卖空股票的投资者来说,股票价格的上涨会让他们心绪不安。投资者有可能会由于情绪的缘故而被迫作出也许是不正确的决定——回补卖空头寸,以减低心理压力。如果在卖空股票的同时持有看涨期权,投资者就可以把亏损限制在一个固定的、通常是相当小的数额内。

当投资者买入看涨期权来保护卖空头寸时,有一个简单的公式可以计算最大风险金额:
\begin{equation}
    \text{风险}=\text{买入的看涨期权的行权价}+\text{看涨期权价格}–\text{股票价格}
\end{equation}

无论是上涨还是下跌,卖空者的风险都会因标的股票发放股息而略有增长,因为他必须为卖空的股票支付股息。

一般而言,最好是买入平值或略微虚值的看涨期权来对卖空头寸进行保护。买入深度虚值看涨期权在保护方面起不到什么作用,除非股价急剧上涨,否则它对风险没有什么改善。正常情况下,投资者会在卖空头寸对其产生严重不利后果之前就回补。因此,花钱买入这样一个深度虚值看涨期权是一种浪费。不过,如果投资者想要他的卖空头寸有足够的“活动”空间,并且非常肯定他对这个股票极度看空的看法是正确的,那么他可以买入相当深度的虚值看涨期权来作为灾难保护,以防股票价格突然向上爆发(例如,标的股票突然收到收购要约)。

\begin{figure}
    \centering
    \begin{tikzpicture}[scale=.4,shift={(35,0)}]
        \draw[->] (35, 0) -- (47, 0) node[anchor=north] {\small{到期时标的资产价格}};
        \draw[->] (35, -6) -- (35, 6) node[anchor=east] {\small{到期时盈亏}};
        \draw[thick] (35,2)--(37,0)node[anchor=north east]{0}--(40,-3)--(45, -3);
        \draw[dashed] (37,3)--(40,0)node[anchor=north east]{40}--(45, -5);
        \draw[dashed] (40, 0)--(40, -3)node[anchor=north east]{-3};
    \end{tikzpicture}
    \begin{tikzpicture}[scale=.4,shift={(35,0)}]
        \draw[->] (35, 0) -- (47, 0) node[anchor=south] {\small{到期时标的资产价格}};
        \draw[->] (35, -6) -- (35, 6) node[anchor=east] {\small{到期时盈亏}};
        \draw[thick] (36,3.5)--(39.5,0)node[anchor=north east]{0}--(45, -5.5)--(46,-5.5);
        \draw[dashed] (37,3)--(40,0)node[anchor=south west]{40}--(46, -6);
        \draw[dashed] (45, 0)--(45, -5.5)node[anchor=north east]{-5.5};
        \fill (45, 0) circle (3pt);
    \end{tikzpicture}
    \caption{卖空者愿意承担的风险可能不同,他也许想买入1手虚值看涨期权作为保护,而不是上面示例中的平值看涨期权。如果买入的是虚值看涨期权,保护成本就会低一些,卖空者所放弃的潜在盈利也少一些。但是他的风险就会大一些,因为只有在股票上涨到行权价之上的时候,这个看涨期权才具有保护功能。}
\end{figure}
\section{保证金要求}
根据最新的保证金规则,如果股票空头头寸有看涨期权多头保护,投资者在保证金要求方面就会有相当的优待。实际所需的保证金为下列两项的较小值:第一,看涨期权行权价的 10\% 加上虚值部分的金额;第二,卖空股票现有市场价值的 30\%。这个头寸会被逐日盯市,如果股票价格低于行权价,大部分经纪商会要求该卖空头寸按“正常”比率缴纳保证金。
\section{组合保证金}
一般而言,组合保证金要求是基于风险的,很难手工计算出来。
\section{后续行动}
保护性卖空者在这个策略中需要采用的后续行动基本上就是平仓。如果标的股票先迅速下跌,然后看上去会反弹,那投资者应该回补股票,而不是卖出看涨期权。这样做的话,如果股票反弹到最初的行权价之上,投资者还能从看涨期权中获利。如果标的股票价格上涨,那就不应该采取相似的、只卖出盈利头寸(看涨期权)的方法。也就是说,如果 XYZ 从 40 涨到了 50,而 7 月 40 看涨期权价格也从 3 涨到了 10,那就不应只卖出看涨期权获得 7 点盈利,并继续持有股票以希望其会下跌。理由是,当看涨期权为实值时,如果解除保护,这个投资者就会面临高度的风险。如果股票价格下跌,那么提走盈利就不是问题,这甚至是他所期望的。因为如果股票继续下跌,就没有或者只有很小的额外风险。但当股票上涨时,情况就不同了。在这种情况下,如果卖空者卖掉他的看涨期权拿走盈利,而股票随后继续上涨的话,就会产生大笔的亏损。

如果看涨期权是持平(at parity)或接近持平的,或者是实值的,那么通过行权而把头寸平仓,就常常是可取的做法。在大多数策略里,由于股票手续费比期权手续费高出许多,将看涨期权行权对期权持有者来说没有什么好处。但在保护性卖空的策略里,卖空者最终总是要回补他卖空的股票,因而总会有股票手续费。因此,行权并按行权价(也就是较低的价格)买入股票,或许还能因此少支付些手续费,这也许会给他带来好处。
\section{合成跨式价差(反向对冲)}
在这种策略里,投资者买入的看涨期权所对应的股数要多于其卖空的股数。如果在期权的存续期内标的股票上涨或下跌的幅度足够大,这个策略家就可以盈利。这个策略一般被称作反向对冲(reverse hedge)或合成跨式价差(synthetic straddle)。如果该股票有场内看跌期权交易,那这个策略就过时了,直接买入跨式价差(1 手看涨期权和 1 手看跌期权)所产生的结果会更好。因此,这个反向对冲策略又被称为“合成跨式价差”。

如果股票在任何方向上有足够幅度的运动,显然都可以得到盈利。事实上,投资者可以准确地判断出,如果要盈利,在到期时股票必须达到怎么样的价格。这些盈亏平衡点很容易计算出来。首先计算出最大风险,然后再确定盈亏平衡点。
\begin{equation}
    \begin{aligned}
        \text{最大风险}    & =\text{行权价}+2\times \text{涨期权价格}-\text{股票价格} \\
        \text{上行盈亏平衡点} & =\text{行权价}+\text{最大风险}                      \\
        \text{下行盈亏平衡点} & =\text{行权价}-\text{最大风险}                      \\
    \end{aligned}
\end{equation}

在到期之前,即使股价离行权价很近也有可能盈利,因为买入的看涨期权还剩有时间价值。

一般而言,进行合成跨式价差交易的股票的波动率应该较大。尽管这种股票的期权权利金会比较高,但当价格呈直线运动时,股票价格的变化幅度仍会大于这些权利金。使用波动率较大的股票的另一个好处是,一般它们很少或者没有股息。这是合成跨式价差所期望的,卖空者也就不必付或者只需付很少的股息。

在建立头寸时,标的股票的技术形态也会有帮助。交易者一般希望在策略亏损的区域内不存在技术性的支撑位和压力(resistance)位。在这样的形态里,股票可以上下快速运动。交易者有时也可以交易宽幅震荡的股票,它们的股价不断地在震荡区域的这一端摆动到另一端。如果反向对冲的亏损区域刚好位于该震荡区间之内,那这个头寸也会有吸引力。
\section{后续行动}
因为反向对冲内在的有限亏损特征,所以没有必要采取任何后续行动来限制亏损。投资者可以相当容易地建立头寸,在到期日前也不必采取任何后续行动。在这个策略里,这经常就是最好的后续行动。

另外,还有一种后续行动可以运用,尽管这种行动有一定的不利之处。它有时被称为针对跨式价差的交易(trading against the straddle)。当股价在任何一个方向运动得足够远的时候,就从这一侧提取盈利。然后等股价摆回到另一个方向时,就再从另一侧提取盈利。
\section{改变看涨期权多头和股票空头之间的比率}
资者不一定非要刚好买入 2 手看涨期权来对应 100 股股票空头。他可以就 100 股股票空头买入 3 或 4 手看涨期权,以建立一个更为看多的头寸。

不管比率是多少,都可以用一个公式来计算最大风险和盈亏平衡点。
\begin{equation}
    \begin{aligned}
        \text{最大风险}    & =(\text{行权价}-\text{股票价格})\times \text{卖空的股票手数}+\text{买入的看涨期权手数}\times \text{看涨期权价格} \\
        \text{上行盈亏平衡点} & =\text{行权价}+\frac{\text{最大风险}}{\text{买入的看涨期权手数}-\text{卖空的股票手数}}                     \\
        \text{下行盈亏平衡点} & =\text{行权价}-\frac{\text{最大风险}}{\text{卖空的股票手数}}                                      \\
    \end{aligned}
\end{equation}

这个策略可以使用的最后一种调整方法,就是在卖空 100 股股票的同时,买入 2 手行权价不同的看涨期权。于这个策略涉及两个行权价,因此被称为“合成宽跨式”(synthetic strangle),即由两个行权价不同的看涨期权或看跌期权构成的普通跨式。
\section{总结}
如果标的股票有场内看跌期权,这一章所描述的策略一般就没有用处。但是,如果没有看跌期权存在,或者看跌期权很不活跃,而策略家认为在看涨期权的存续期内股票会在某一方向上有相对较大的运动,他就应当考虑使用某种形式的合成跨式策略,也就是卖空一定数量的股票,同时买入对应更多股票数量的看涨期权。如果他所希望的运动确实出现了,就会产生可观的盈利。无论是哪种情况,亏损都被限制在某个固定的金额之内,一般是初始头寸的 20\%-30\%。虽然可以采取一些后续行动来锁住小额盈利和从股票的反向运动中获利,但更明智的做法是继续持有头寸,以便获得更大的盈利。这个策略一般使用 2:1 的比率(买入 2 手看涨期权,卖空 100 股股票),但如果投资者想要更为看多或者看空,可以调整这个比率。如果初始股票价格在两个行权价之间,可以通过在卖空股票的同时分别买入次高行权价和次低行权价的看涨期权,来建立一个中性的盈利区域。
% \chapter{酒田战法和其他蜡烛图组合}
\section{酒田战法}
\subsection{三山形态}
三山形态构成了市场的一个大型顶部,与西方技术分析中的“三重顶”形态比较类似,在三重顶形态中,价格上升和下跌各三次,形成了市场的顶部。三尊顶形态(san-son)与西方的头肩形顶部形态也类似。三尊顶形态有点儿像佛教中佛像的陈列,大殿里供奉的佛像通常有三尊,其中中间摆放的是一尊最大的,两边分别是小一点的佛像(见 \autoref{fig5-1a})。三山形态包含了西方技术分析中的“三重顶”形态,在这一形态中,价格三次向上测试,但随后都出现了一定幅度的调整,在三重顶形态中,三个顶部的高度是相同的,或者大致接近(见 \autoref{fig5-1b})。
\figures{fig5-1a}{}
\figures{fig5-1b}{}

\subsection{三川形态}
三川形态和三山形态正好相反。它同传统的三重底形态和头肩底形态类似。三川形态由三根位于市场底部的看涨蜡烛线组合而成,用于预测市场的转折点,这些 K 线组合包括启明星、白三兵等,在一些介绍酒田战法的日文著作中,也将启明星形态称作三川启明星形态 \autoref{fig5-2a}。

\figures{fig5-2a}{}
\subsection{三空形态}
在三空形态中(\autoref{fig5-3}),价格的跳空缺口意味着投资者进入和退出市场的时机。以向上跳空形态为例,市场出现底部后,当它再次上升时,投资者应该在出现第三个跳空缺口后做空。第一个向上跳空缺口意味着新入场的买方力量强大,第二个缺口代表继续有买方入场以及部分有经验的空头平仓,第三个跳空缺口是由犹豫的空头平仓和如梦初醒的多头买进造成的。酒田战法建议在第三个向上跳空的缺口后做空,因为买盘卖盘出现分歧以及随后市场出现超买的可能性越来越大。相反,在下降趋势中出现第三个向下跳空缺口后,投资者应该做多。在日语中,跳空缺口的弥补又被称为“anaume”,跳空缺口又被称为窗口(mado)。

\figures{fig5-3}{}
\subsection{三兵形态}
三兵形态指的是“向同一个方向站立的三名士兵”。白三兵是一个典型的看涨信号,表明市场正处于稳定的上升态势中,这种稳健的价格走势说明市场将进一步大幅走高。另外,酒田战法还给出了三兵形态的衰退形态,这些形态表明上升趋势的力度逐渐减弱,也就是通常所说的“上涨乏力”。三兵形态包含几种衰退形态,第一种衰退形态是“前进受阻形态”,该形态虽然跟白三兵形态比较类似,不同之处在于,该形态在第二天和第三天录得的 K 线都带有很长的上影线。第二个衰退形态则是“停滞形态”,在该形态中,同样也是第二天的 K 线带有很长的上影线,而第三天则是出现纺锤线,或者是十字星,这表明市场拐点的临近。
% \chapter{其他反转形态\label{ch06}}
相比较而言,我们在 \autoref{ch04} 和 \autoref{ch05} 中所介绍的反转形态都是较强的反转信号。一旦它们出现,就表明多头已经从空头手中夺过了大权(比如说看涨吞没形态、启明星形态,或者刺透形态等),或者空头已经从多头手中抢回了主动权(比如说看跌吞没形态、黄昏星形态,或者乌云盖顶形态等)。本章要讨论更多的反转形态。其中一部分形态通常——但并不总是——构成反转信号,因此,它们是较弱的反转信号。这些形态包括\textbf{孕线形态、平头顶部形态、平头底部形态、捉腰带蜡烛线、向上跳空两只乌鸦、反击蜡烛线}等。此外,本章还要探讨一些强烈的反转信号,包括三只乌鸦、三山形态、三川形态、圆形顶、圆形底、塔形顶和塔形底。
\section{孕线形态}
纺锤线(即小实体的蜡烛线)是特定蜡烛线形态的组成部分。孕线形态就是这些形态中的一个例子(星线也是一个例子,这在第五章已有探讨)。如 \autoref{fig6-1} 所示即\textbf{孕线形态},其中后面一根蜡烛线的实体较小,并且被前一根的实体包进去了,日本人描述前一根蜡烛线为“非常长的黑色实体或白色实体”。本形态的名字来自一个古老的日本名词,意思就是“怀孕”。在本形态中,长的蜡烛线是“母”蜡烛线,而小的蜡烛线则是“子”或“胎”蜡烛线。孕线形态的第二根线既可以是白色的,也可以是黑色的。

\figures{fig6-1}{孕线形态}

孕线形态与吞没形态相比,两根蜡烛线的顺序恰好颠倒过来。在孕线形态中,前一个是非常长的实体,它将后一个小实体包起来。而在吞没形态中,后面是一根长长的实体,它将前一个小实体覆盖进去了。孕线形态与吞没形态的另一个区别是,在吞没形态中,两根蜡烛线实体的颜色应当互不相同。而在孕线形态中,这一点倒不是必要条件。

孕线形态与西方技术分析理论中的\textbf{收缩日}概念有相似之处。按照西方的理论,如果某日的最高价和最低价均居于前一日价格区间的内部,则这一天就是一个收缩日。但是,孕线形态并没有如此严格的要求。对孕线形态来说,所要求的全部条件是,第二个实体居于第一个实体内部,即使第二根蜡烛线的影线超越了第一根线的高点或低点也无所谓。
\subsection{十字孕线形态}
作为一条普遍的经验,在孕线形态中,第二根实体越小,则整个形态越有力量。这条经验通常都是成立的,因为第二个实体越小,市场的矛盾心态就越甚,所以越有可能形成趋势的反转。在极端情况下,随着第二根蜡烛线的开市价与收市价之间的距离的收窄,其实体便越来越小,最后就形成了一根十字线。在下降行情中,前面是一根长黑色实体(或者在上涨行情中前面是一根高高的白色实体),后面紧接着一根十字线,构成了一类特殊的孕线形态,\textbf{十字孕线形态}。

\notes{十字孕线形态也可能引发底部过程,不过,当这类形态出现在市场顶部时更有效力}。

在 \autoref{fig6-5} 中,从上吊线开始出现了一轮跌落行情,终于在 11 月 4 日—5 日通过一个孕线形态探得底部。孕线形态的第二根实体很短,因此我把它视为一根十字线。于是,这是一个十字孕线形态。这个形态尤其有意义,原因在于它的出现有助于清晰地验证位于 61 美元的先前定义的支撑水平(图中用水平直线标注)。如果这是一张线图,那么基于贯穿 9 月份始终的行情,我们也将得到同一个支撑水平。虽然我们采用的是蜡烛线图,依然可以,也应该用传统的线图支撑水平或阻挡水平。由此,这里有一个东方技术信号(孕线形态)验证了传统的西方技术信号(支撑水平线)。

在本图中,更早些的 9 月 29 日和 30 日组成了另一个十字孕线形态。随着这个孕线形态的出现,短线趋势从上升转为横向延伸。这个形态强调了一个要点:\important{当趋势发生变化时,并不意味着行情必将从上升转为下降,或者从下降转为上升}。在本图所示的两例孕线形态中,在前者出现后,原来的上升趋势并没有改变。具体说来,在 11 月的孕线形态出现后,趋势从下降转为上升,而在 9 月的孕线形态出现后,趋势从上升转为中性。在上述意义上,两个孕线形态都正确地预告了趋势的转变。

\figures{fig6-5}{亚马逊(Amazon)——日蜡烛线图(十字孕线形态)}
\section{平头顶部形态和平头底部形态}
\textbf{平头形态}是由几乎具有相同水平的最高点的两根蜡烛线组成的,或者是由几乎具有相同的最低点的两根蜡烛线组成的。在上升的市场中,当几根蜡烛线的最高点的位置不相上下时,就形成了一个\textbf{平头顶部形态}。在下跌的市场中,当几根蜡烛线的最低点的位置基本一致时,就形成了一个\textbf{平头底部形态}。平头形态既可以由实体构成,也可以由影线或者十字线构成。在理想情况下,平头形态应当由前一根长实体蜡烛线与后一根小实体蜡烛线组合而成。这样就表明,无论在第一个时段市场展现了什么样的力量(如果是长白色实体,展现的便是看涨的力量;如果是长黑色实体,展现的便是看跌的力量),到了第二个时段都被瓦解了,因为第二个时段是一个小实体,且其高点与第一个时段的高点相同(在平头顶部形态中),或其低点与第一个时段的低点相同(在平头底部形态中)。如果一个看跌的(对于顶部反转)蜡烛图信号,或看涨的(对于底部反转)蜡烛图信号,同时构成了一个平头形态,该形态就多了一些额外的技术分量。

应当对不同时间框架下的平头形态区别对待,日蜡烛线、日内蜡烛线的短线图形不同于周蜡烛线图和月蜡烛线图等的长线图形。这是因为如果两个交易日或者两个日内时段具备相同的高点或低点,并没有什么大不了的。仅当这类形态同时具备其他蜡烛图的特征时(例如第一根蜡烛线为长实体,第二根为短实体;或者既符合其他蜡烛图形态的要领,又具备相同的高点或低点),才值得注意。由此可见,对于日线图或日内图表上的平头形态,必须牢记的一个主要方面是,\important{在同时具备其他特定的蜡烛线组合特征的前提下,才可以根据平头形态采取行动}。

对希望获得关于市场的长期看法的朋友来说,不妨选用周蜡烛线图和月蜡烛线图进行研究,其中由相邻的蜡烛线形成的平头顶部形态和平头底部形态可能构成重要的反转信号。这类形态甚至在没有其他蜡烛图信号相验证的条件下,也是成立的。我们不妨用下面的例子来说明其原因。在周蜡烛线图或月蜡烛线图中,如果前一个时间单位的低点在后一个时间单位内成功地经受了市场向下的试探,这个低点就可能构成重要的市场底部,引发上冲行情。在日蜡烛线图或日内蜡烛线图上,就不能这么说了。

\autoref{fig6-16} 展示了一个平头顶部形态。2 月 2 日为小实体,它不居于前一根实体范围之内,并不构成孕线形态。两根蜡烛线具有相同的高点,即 55 美元,因此这属于平头形态。不仅如此,2 月 2 日的小实体是一根上吊线(其上影线足够短,可以视之为上吊线)。当然,这根上吊线(与任何上吊线一样)需要得到看跌信号的验证,即市场收市于上吊线的实体之下。下一个时段正是如此。

从上述平头顶部形态开始,戴尔形成了下降行情,一直持续到 2 月底—3 月初的一系列锤子线,它们标志着行情进入盘整区域。2 月 26 日和 3 月 1 日是前面的两根锤子线,它们并不能构成常规的平头底部形态。为什么?虽然两者的低点确实差不多相同,但是两根锤子线不满足平头底部形态的一项常规要求——平头底部形态的第一根蜡烛线应为长实体。(\warning{我觉得应该没有这一条件才对})尽管 2 月 26 日和 3 月 1 日的两根锤子线不能归类为平头底部形态,但是由于两者看涨的长下影线突出显示了市场正在排斥位于 39 美元的低价位,我依然要强调它们的重要性。如此一来,我就把 2 月 26 日和 27 日的蜡烛线组合看成平头底部形态的变体。

\figures{fig6-16}{戴尔公司(Dell)——日蜡烛线图(平头顶部形态)}

\section{捉腰带线}
\textbf{捉腰带形态}是由单独一根蜡烛线构成的。\textbf{看涨捉腰带形态}是一根坚挺的白色蜡烛线,其开市价位于本时段的最低点(或者,这根蜡烛线只有极短的下影线),然后市场一路上扬,收市价位于或接近本时段的最高点。看涨捉腰带线又称为\textbf{开盘光脚阳线}。如果市场本来处于低价区域,这时出现了一根长长的看涨捉腰带线,则预示着上冲行情的到来。

\textbf{看跌捉腰带形态}是一根长长的黑色蜡烛线,它的开市价位于本时段的最高点(或者距离最高价只有几个最小报价单位),然后市场一路下跌。在市场处于高价区的条件下,看跌捉腰带形态的出现,就构成了顶部反转信号。看跌捉腰带线有时也称为\textbf{开盘光头阴线}。

如果市场收市于黑色的看跌捉腰带线之上,则意味着上升趋势已经恢复。如果市场收市于白色的看涨捉腰带线之下,则意味着市场的抛售压力重新积聚起来了。如果捉腰带线得到了阻挡区域的验证,或者接连出现了几根捉腰带线,或者有一阵子没有出现捉腰带线,突然来了一根,这样的捉腰带线就更加重要。

在 \autoref{fig6-22} 中,6 月初出现了一个向上跳空,很快转化为支撑区域,随后在 6 月上半个月里多次成功地经受了试探,支撑区域得到了验证。6 月 13 日的蜡烛线是看涨的捉腰带线。到 7 月底、8 月初,市场再次向下试探该窗口的支撑作用,又形成了一系列看涨的捉腰带线。后面这两根看涨捉腰带线也分别与其前一根蜡烛线组成了两个背靠背的刺透形态。从 8 月初的低点开始形成上冲行情,持续到 8 月 9 日的流星线。

\figures{fig6-22}{力博通信公司(Redback Networks)——日蜡烛线图(看涨捉腰带线)}
\section{向上跳空两只乌鸦}
如 \autoref{fig6-23} 所示,为\textbf{向上跳空两只乌鸦形态},它很罕见。“向上跳空”指的是图示的小黑色实体与它们之前的实体(第一个小黑色实体之前的实体,通常是一根长长的白色实体)之间的价格跳空。

\figures{fig6-23}{向上跳空两只乌鸦}

在理想的向上跳空两只乌鸦形态中,第二个黑色实体的开市价高于第一个黑色实体的开市价,并且它的收市价低于第一个黑色实体的收市价。

这个形态在技术上看跌的理论依据大致如下:市场本来处于上升趋势中,并且这一天的开市价同前一天的收市价相比,是向上跳空的,可是市场不能维持这个新高水平,结果当天反而形成了一根黑色蜡烛线。到此时为止,多头至少还能捞着几根救命稻草,因为这根黑色蜡烛线还能够维持在前一天的收市价之上。第三天,又为市场抹上了更深的疲软色彩:当天市场曾经再度创出新高,但是同样未能将这个新高水平维持到收市的时候。然而更糟糕的是,第三日的收市价低于第二日的收市价。如果市场果真如此坚挺,那么为什么它不能维持新高水平呢?为什么市场的收市价下降了呢?这时候,多头心中恐怕正在惴惴不安地盘算着上述两个问题。思来想去,结论往往是,市场不如他们当初指望的那样坚挺。如果次日(也就是指第四天)市场还是不能拿下前面的制高点,那么,我们可以想见,市场将会出现更低的价格。

\autoref{fig6-25} 说明了把蜡烛图形态与它贴身的周边环境相结合的重要意义。7 月中旬虽然出现了一个向上跳空两只乌鸦形态,但是它本身并不构成卖出信号。这是因为,7 月 17 日该股票向上跳空,通常向上跳空是行情坚挺的征兆——无论是在蜡烛图上还是在线图上都是如此。因而,尽管该向上跳空两只乌鸦形态发出了警告信号,我认为它没有那样疲软。

\figures{fig6-25}{康宁公司——日蜡烛线图(向上跳空两只乌鸦)}

\section{三只乌鸦}
在向上跳空两只乌鸦形态中,包含了两根黑色蜡烛线。如果连续出现了三根依次下降的黑色蜡烛线,则构成了所谓的\textbf{三只乌鸦形态}。如果三只乌鸦形态出现在高价格水平上,或者出现在经历了充分延伸的上涨行情中,就预示着价格即将下跌。有的时候,三只乌鸦形态又称作\textbf{三翅乌鸦形态}。从外形上说,这三根黑色蜡烛线的收市价都应当处于其最低点,或者接近其最低点。在理想的情形下,每根黑色蜡烛线的开市价也都应该处于前一个实体的范围之内。

如 \autoref{fig6-27} 所示,从 4 月 15 日开始形成了一个三只乌鸦形态。从三只乌鸦开始的下降行情一路几乎不受阻碍地延续到了 P 处的刺透形态。三只乌鸦形态中的第二根和第三根蜡烛线(4 月 16 日和 17 日)都开市于之前的实体之下。虽然在常规的三只乌鸦形态中后续蜡烛线的开市价居于之前的黑色实体内部,但是这两根蜡烛线的开市价低于之前的实体,可以视作更加疲软的信号。其原因在于,第二根和第三根黑色实体开市价低于前一日的收市价,之后在整个交易日里都无力夺得实质性的立足地。

三只乌鸦形态可能对长线交易者(做空)更有用处。这是因为本形态在第三根蜡烛线才能完成。显然,到了这个时候,市场已经回落了相当大的幅度。举例来说,上述三只乌鸦是从 70.75 美元处开始的。既然我们需要第三根黑色实体来完成形态,那么在得到信号的时候,股价已经跌到了 67.87 美元。

无论如何,在本例中,当三只乌鸦形态的第一根黑色蜡烛线出现的时候,我们就可以看出行情遇到麻烦的一点端倪了。个中缘由是,股票当日开市于之前 3 月份的历史高点 70 美元之上,然而牛方未能坚守上述新高,当日收市时,反而跌回到 70 美元以下。如果市场先创新高,之后却不能守住,可能带有看跌的意味。这里的情况便是这样。

\figures{fig6-27}{鹏斯公司(Pennzoil)——日蜡烛线图(三只乌鸦形态)}

我们再来看看之前 1 和 2 处的高点。在 2 月初的 1 处,鹏斯公司为当前行情创了新高,但这里的一组蜡烛线却向我们发出了强烈的“火光”警告信号,行情并不像表面上那么顺利。具体说来,在 2 月 2 日所在的一周的后几天,尽管股价不断创下更高的高点、更高的低点、更高的收市价,但是它们都是小实体,都有长长的上影线。这肯定显示出当前的行情变化其实并不是一面倒地有利于牛方。之后价格回落,直到B处的看涨吞没形态才结束。从此处开始的一轮上冲行情持续推升,持续到 3 月 2 日所在的一周,图中用 2 做了标记。2 处的上冲行情与 1 处的上涨行情有异曲同工之妙,2 处的行情也有更高的高点、更高的低点、更高的收市价,这样如果在线图上看起来,行情显得很健康。然而从蜡烛图的角度来看,3 月 4 日、5 日、6 日的价格攀升带有长长的上影线。这一点证明多方相对强势的力量正在涣散。3 月 6 日的蜡烛线是一根流星线。

\section{白色三兵挺进形态}
与三只乌鸦形态相对的形态称为\textbf{白色三兵挺进形态},或者更通俗的说法是\textbf{白三兵形态}(如 \autoref{fig6-28})。本形态是由接连出现的三根白色蜡烛线组成的,它们的收市价依次上升。当市场在某个低价位稳定了一段时间后,如果出现了这样的形态,就标志着市场即将转强。

白三兵形态表现为一个逐渐而稳定的上升过程,其中每根白色蜡烛线的开市价都处于前一天的白色实体之内,或者处在其附近的位置上。每一根白色蜡烛线的收市价都应当位于当日的最高点或接近当日的最高点。这是市场的一种很稳健的攀升方式(不过,如果这些白色蜡烛线伸展得过长,那么我们也应当对市场的超买状态有所戒备)。

\figures{fig6-28}{白色三兵挺进形态}

如果其中的第二根和第三根蜡烛线,或者仅仅是第三根蜡烛线,表现出上涨势头减弱的迹象,那就构成了一个\textbf{前方受阻形态}(如 \autoref{fig6-29} 所示)。这就意味着这轮上涨行情碰到了麻烦,持有多头头寸者应当采取一些保护性措施。特别是在上升趋势已经处于晚期阶段时,如果出现了前方受阻形态,则更得多加小心。\tips{在前方受阻形态中,作为上涨势头减弱的具体表现,既可能是其中的白色实体一个比一个小,也可能是后两根白色蜡烛线具有相对较长的上影线。}如果在后两根蜡烛线中,前一根为长长的白色实体,并且向上创出了新高,后一根只是一个小的白色蜡烛线,那么就构成了一个\textbf{停顿形态}(如 \autoref{fig6-30} 所示)。有时候,这种形态也称为\textbf{深思形态}。当这一形态出现时,说明牛方的力量至少暂时已经消耗尽了。在本形态中,最后一根小的白色蜡烛线既可能从前一根长长的白色蜡烛线向上跳空(这种情况下,该蜡烛线就变成了一根星线),或者正如日本分析师所描述的那样“骑在那根长长的白色实体的肩上”(这就是说,位于前一根长长的白色实体的上端)。这根小小的实体暴露了牛方能量的衰退。当停顿形态发生时,便构成了多头头寸平仓获利的紧要时机。

\figures{fig6-29}{前方受阻形态}

虽然前方受阻形态与停顿形态在一般情况下都不属于顶部反转形态,但是有时候,它们也能引导出不容忽视的下跌行情。\important{我们应当利用前方受阻形态和停顿形态来平仓了结已有的多头头寸,或者为多头头寸采取保护措施,但是不可据之开立空头头寸}。一般来说,如果这两类形态出现在较高的价格水平上,则更有预测意义。

前方受阻形态与停顿形态之间并没有太大差异。关于白三兵,需要考虑的主要因素是,如果三根蜡烛线的每一根的收市价都位于或接近本时段的最高价,则最具有建设性。如果后两根蜡烛线表现出犹豫的迹象,是小实体,或者是上影线,那么这些线索表明,上冲行情正变得衰弱。

\figures{fig6-30}{停顿形态}

\autoref{fig6-32} 是白三兵形态良好的实例。三根白色蜡烛线的收市价都非常接近本时段最高价,每一根的开市价都位于先前一根实体的内部或之上。关于白三兵形态需要考虑的一个方面是,等到白三兵形态完成时,市场可能已经明显脱离其低位了。在本例中,微软离开其低位差不多 4 美元,这可是较大比例的行情变化了。因此,\tips{除非交易者长线看好,在白三兵形态完成时买进或许并不具备有吸引力的风险报偿比。}

我发现,\tips{在白三兵形态出现后,一旦行情调整,则其中的第一根或第二根白色蜡烛线,即白三兵的起点处,经常构成支撑水平}。在本例中,在白三兵形态出现后股票进入整理阶段,缓缓回落,直到形成一根锤子线为止。这验证了白三兵形态中第二根蜡烛线内部形成的支撑水平。

\figures{fig6-32}{微软——日蜡烛线图(白三兵形态)}
\section{三山形态和三川形态}
与西方的三重顶形态相似,日本也有所谓的\textbf{三山顶部形态}(如 \autoref{fig6-35} 所示)。一般认为,本形态构成了一种主要顶部反转过程。如果市场先后三次均从某一个高价位上回落下来,或者市场对某一个高价位向上进行了三次尝试,但都失败了,那么一个三山顶部形态就形成了。在三山顶部形态的最后一座山的最高点,还应当出现一种看跌的蜡烛图指标(比如说,一根十字线,或者一个乌云盖顶形态等),对三山顶部形态做出确认。

在三山顶部形态中,如果中间的山峰高于两侧的山峰,则构成了一种特殊的三山形态,称为\textbf{三尊顶部形态}。

\figures{fig6-35}{三山形态}

\textbf{三川底部形态}恰巧是三山顶部形态的反面。在市场先后三度向下试探某个底部水平后,就形成了这类形态。市场必须向上突破这个底部形态的最高水平,才能证实底部过程已经完成。与西方的头肩形底部形态(也称为倒头肩形)对等的蜡烛图形态是变体三川底部形态,或者称为\textbf{倒三尊形态}。

\autoref{fig6-42} 是三尊顶部形态的又一个例子。既然本形态与头肩顶同质而异名,我们就可以转而采用西方技术分析,正如上一个图例所讨论的,以头肩形颈线的概念为基础来分析。

具体说来,一旦头肩顶颈线被向下突破,它从支撑作用转为阻挡作用。4 月 10 日 13:30,市场力图通过一个小小的看涨吞没形态来站稳脚跟,但是市场没能突破颈线的阻挡作用,将指数推升到颈线之上,说明空方保持着控制权。这反映出,关键的一点是要弄清楚蜡烛图形态到底是在什么样的位置形成的。在本例中,看涨的吞没形态是潜在的底部反转信号,不过,如果等市场以收市价突破到颈线阻挡水平之上之后,再从容买进,岂不是更有道理——即使拿到了看涨吞没形态的这张好牌?耐心等待是值得的,因为如果行情收市到颈线之上无疑有助于增强信心,表明多头更占上风。

\figures{fig6-42}{纳斯达克 100 指数——15 分钟蜡烛线图(三尊顶部形态)}

\section{反击线形态(约会线形态)}
当两根颜色相反的蜡烛线具有相同的收市价时,就形成了一个\textbf{反击线形态}(也称为\textbf{约会线形态})。

我们不妨把看涨反击线形态同看涨刺透形态做一番比较。如刺透形态与看涨反击线形态一样,也是由两根蜡烛线组成的。它们之间的主要区别是,看涨反击线通常并不把收市价向上推进到前一天的白色实体的内部,而是仅仅回升到前一天的收市价的位置。而在刺透形态中,第二根蜡烛线深深地向上穿入了前一个黑色实体之内。因此,刺透形态与看涨反击线形态相比较,是一种更为重要的底部反转信号。尽管如此,正如我们下面列举的一些实例所显示的,对看涨反击线形态还是不可小觑的,因为它的出现表明,行情流动的方向正在改变。

如果说看涨反击线形态与刺透形态有渊源的话,那么,看跌反击线形态与乌云盖顶形态也有类似的关系。在理想的看跌反击线形态中,第二天的开市价高于前一天的最高点,这一点与乌云盖顶形态是一致的。不过,与乌云盖顶形态不同的是,这一天的收市价并没有向下穿入前一天的白色蜡烛线之内。由此看来,乌云盖顶形态所发出的顶部反转信号,比看跌反击线形态更强。

在反击线形态中,一项重要的考虑因素是,\tips{第二天的开市价是否强劲地上升到较高的水平(在看跌反击线形态中),或者是否剧烈地下降到较低的水平(在看涨反击线形态中)}。这里的核心思想是,在该形态第二天开市时,市场本已经顺着既有趋势向前迈了一大步,但是后来,却发生了意想不到的变故!到当日收市时,市场竟然完全返回到前一天收市价的水平!如此一来,朝夕之间竟扭转了行情基调。

在 \autoref{fig6-48} 中的 10 月 15 日,展现了一例看跌的反击线形态。我们可以看出,该反击线的收市价并没有恰好处于前一时段白色蜡烛线的收市价,而是稍稍低于后者。在判断反击线形态时,对形态的定义应该留有适当变通的余地,这对于绝大多数蜡烛图信号都适用。举例来说,12 月 6 日有一个看涨的反击线。当日开市时,股价剧烈下跌向下跳空,当日收市时,收市价接近前一日的收市价,而不是恰好位于。即使第二天的收市价与第一天的收市价不是恰好相同,也肯定足够接近了,因此我认为这属于看涨反击线形态。判别这个看涨反击线形态的主要标准是,虽然白色蜡烛线的开市价非常疲软,却能够当日完全反弹,令人刮目相看。

\figures{fig6-48}{吉列公司(Gillette)——日蜡烛线图(看跌的和看涨的反击线形态)}
\section{圆形顶部形态和平底锅底部形态(圆形底部形态)}
在\textbf{圆形顶部形态}(如 \autoref{fig6-51} 所示)中,市场逐步形成向上凸起的圆弧状图案,在这个过程中,通常出现的是一些较小的实体。最后,当市场向下跳空时,就证明圆形顶部形态已经完成。这一形态与西方的圆形顶部形态是相同的。不同的是,在日本的圆形顶部形态中,应当包含一个向下跳空,作为市场顶部的附加验证信号。

\figures{fig6-51}{圆形顶部形态}

在平底锅底部形态中,市场从更低的低点转为相同的低点,再转为更高的低点。这个过程形象地证明了空头正在逐步丧失立足之地。在这之后,再添加一个向上跳空,带来进一步的证据,表明空头失去了对市场的控制权。

对圆形顶部形态来说,道理相同,而方向相反。也就是说,市场从更高的高点转为相同的的高点,再转为更低的高点。于是,上升行情的节奏松弛下来。在这之后,再来一个向下跳空,便完成了圆形顶部形态。跳空对多头行情做了进一步的盖棺论定。

\autoref{fig6-58} 清晰地显示了西方圆形底部形态和东方平底锅底部形态的区别。从 9 月 1 日到 9 月 14 日所在的一周里,该股票构造了一个圆形底部形态(因为它从更低的低点转向相同的低点,再转向更高的低点)。然而,既然在该圆形底部形态中没有向上跳空,那么它便不是平底锅底部形态。请记住,平底锅底部形态与西方的圆形底部形态同出一辙,只不过它附加了向上跳空作为最后的一推。与上述圆形底部形态同时出现的是数字 1 和 2 标注的两处信号,两条长长的上影线(且 2 处是一根流星线)。

现在让我们把注意力转向 10 月从头到尾的价格变化。在这期间,该股票逐步构筑了一个圆形底部形态(也就是从更低的低点转向更高的低点)。10 月 26 日出现了一个小幅的向上跳空,于是,这就完成了一个平底锅底部形态,带着其应有的全部看涨潜力。因为平底锅底部形态包含这个向上跳空(常规的圆形底部形态没有向上跳空),所以我认为平底锅底部形态比常规的圆形底部形态更重要。

\figures{fig6-58}{地壳公司(Earthshell)——日蜡烛线图(平底锅底部形态)}
\section{塔形顶部形态和塔形底部形态}
\textbf{塔形顶部形态}出现在高价格水平上。市场本来处在上升趋势中,在某一时刻,出现了一根坚挺的白色蜡烛线或者一系列白色蜡烛线,然后市场放缓了上涨的步调,接着出现了一根或者数根大的黑色蜡烛线,于是塔形顶部形态就完成了(如 \autoref{fig6-59} 所示)。在本形态中,中间有若干小实体,两侧长长的白色和黑色蜡烛线形如“高塔”。也就是一边由长蜡烛线组成下跌的一侧,另一边由长蜡烛线组成上涨的一侧。

\figures{fig6-59}{塔形顶部形态}

\textbf{塔形底部形态}发生在下降行情中(如 \autoref{fig6-60} 所示)。市场形成了一根或数根长长的黑色蜡烛线,表示空方动力丝毫不减。后来出现了几根小实体,缓和了行情看跌的气氛。最后出现了一根长长的白色蜡烛线,完成了一个塔形底部形态。

\figures{fig6-60}{塔形底部形态}

\autoref{fig6-61} 揭示了塔形顶部形态与圆形顶部形态的区别。该股票在 10 月的第一周上涨,形成了一系列白色实体,但之后开始踩水,留下了一系列小实体。10 月 15 日的向下跳空完成了一个圆形顶部形态。把注意力转向 12 月,我们看到了一系列延长的白色蜡烛线。蜡烛线1出现表明该股票继续上涨的机会不足了。长黑色蜡烛线 2 带来了第二支“塔”,完成了塔形顶部形态。

\figures{fig6-61}{CNB Bancshares——日蜡烛线图(塔形顶部形态与圆形顶部形态)}
% % \chapter{风险度量 1}
利率变化的影响可能会因标的合约和结算程序的不同而有所不同。

利率变化会以两种方式影响期权。首先,它可能会改变标的合约的远期价格。其次,它可能会改变期权的现值。在股票期权市场中,利率上升将提高远期价格,导致看涨期权价值上升,看跌期权价值下降。与此同时,更高的利率将降低两者的现值。看跌期权的价值显然会下降,因为这两种结果都会降低看跌期权的价值。但对于看涨期权而言,结果却具有相反的效果。更高的远期价格将导致看涨期权的价值增加,但更高的利率将降低看涨期权的现值。由于股票价格始终高于期权价格,因此远期价格的上涨总是比现值的降低产生更大的影响。因此,\important{股票看涨期权的价值将随着利率上升而上升,随着利率下降而下降。股票看跌期权的价值则恰恰相反,随着利率上升而下降,价值则上升}。

期权价值取决于交易者是用多头股票还是空头股票进行对冲,这一事实带来了大多数交易者希望避免的复杂情况。这为股票期权交易者提供了一条有用的规则:
\begin{tcolorbox}
    只要有可能,交易者就应该避免持有空头股票仓位。
\end{tcolorbox}
\section{Delta}
Delta($\Delta$)是衡量期权相对于标的合约变动方向的风险的指标。Delta 为正表示希望价格向上变动;Delta 为负表示希望价格向下变动。Delta 有几种不同的解释,任何一种解释都可能对交易者有用,具体取决于所执行的策略类型。
\subsection{变化率}
到期时,期权的价值恰好等于其内在价值。然而,在到期之前,期权的理论价值是一条曲线,随着期权进入价内非常深或远离价外,该曲线将接近内在价值,如 \autoref{fig7-4} 所示。当标的价格上涨时,曲线的斜率趋近于 +1;当标的价格下跌时,曲线的斜率趋近于零。任何给定标的价格的看涨期权的 delta 就是曲线的斜率--期权价值相对于标的合约变动的变化率。

\figures{fig7-4}{看涨期权的理论值。}

看跌期权的特征与看涨期权相似,只是看跌期权的价值与标的市场走势相反。在 \autoref{fig7-5} 中,我们可以看到,当标的价格上涨时,看跌期权的价值会贬值;当标的价格下跌时,看跌期权的价值会上升。因此,看跌期权的 delta 总是为负,范围从远价看跌期权的 0 到深价看跌期权的 -1。与看涨期权 delta 一样,看跌期权 delta 衡量的是看跌期权的价值相对于标的价格变化的变化率,但负号表示变化将与标的合约的方向相反。

\figures{fig7-5}{看跌期权的理论价值。}

尽管看涨期权的 delta 值范围是 0 到 1.00,看跌期权的 delta 值范围是 0 到 -1.00,但许多期权交易者通常将 delta 值表示为整数,去掉小数点,我们也将遵循这一惯例\footnote{这一惯例起源于美国股票期权市场,股票期权交易者通常将一个 delta 等同于一股股票。由于标的合约包含 100 股股票,因此交易者将 delta 指定为 100。许多期货期权交易者也使用这种整数格式来表示 delta。}。使用这种格式,看涨期权的 delta 值将在 0 到 100 的范围内,看跌期权的 delta 值将在 -100 到 0 的范围内。标的合约的 delta 值始终为 100。
% \chapter{Delta 中性交易:理论与实践}
Delta 中性交易(Delta-neutral trading)是一种非方向性交易技术,其利润、损失或盈亏平衡来自标的股票的价格波动和期权价格的时间价值衰减之间的关系。
\section{Delta 中性的定义}
Delta 中性持仓是一个组合净 Delta 为零或近似为零的多腿持仓。该持仓可以为卖权、买权与股票的多头和空头的任意组合。
\section{Delta 中性交易理论}
Delta 中性交易包括三个步骤:
\begin{enumerate}
    \item 交易员建立一个 Delta 中性持仓;
    \item 当标的股票的价格变化使得全部持仓的净 Delta 偏离零时,交易员将根据事先确定的规则进行股票持仓调整交易;
    \item 交易员平掉全部持仓,期望能获得一个净利润;
\end{enumerate}

调整股票交易是指买入或卖出特定数量的股票以使得全部持仓的净 Delta 等于零或接近零。事先确定的规则决定了调仓的时机,它可以是基于时间周期,也可以基于股价的变动。
% \chapter{风险管理}
一般人的本性都是很快就止盈,但却会一直拿着亏损的交易,希望能回本甚至盈利。当这些业余的交易者放弃了希望,在巨亏后平掉头寸的时候,他们的账户往往已经严重亏损甚至无可挽回了。
\section{情绪与概率}
固执地持有亏损的交易头寸只会加深亏损的程度。亏损会以一种滚雪球的方式增长,直到最初看起来很糟糕的亏损比例开始显得不算什么了,因为现在的亏损放大了很多。最终,绝望的失败者忍痛清仓出局,遭受了严重损失。当他一离开,市场就开始反转,强势回归。

此时的交易者恨不得拿头撞墙——如果他再坚持一下,他本可以赚钱的。

这些反转一次又一次的发生,因为大多数输家对刺激的反应是一样的。人们有相似的情绪,这与他们的种族或教育无关。提心吊胆的交易者满手是汗、心跳加速,不管他们是在美国纽约还是在中国香港长大,也不管他接受过 2 年还是 20 年学校教育,他们的感受和反应是一样的。
\subsection*{正的期望值}
大多数交易者有一个很好的交易系统,但为了改造成一个完美的系统反而毁了它。
\section{风险控制的两条主要原则}
只需要一次致命的损失,就能毁掉一个账户,使交易者退出游戏,就像是鲨鱼咬了致命的一口。市场也可以通过一串连续的损失毁掉账户,每次损失都并不致命,但汇集起来会使账户所剩无几,就像是一群食人鱼一样。资金管理的两大支柱是 2\% 原则和 6\% 原则,2\% 原则可以帮助躲避市场鲨鱼式的攻击,6\% 原则可躲避市场食人鱼式的攻击。
\subsection*{两种最糟糕的错误}
有两种方式可以快速毁掉一个账户:从不使用止损和持有相对账户来说过高比例的仓位。

没有设置止损的交易会使你暴露在无限的损失之中。

另一种致命的错误是过度交易——相对于你的账户来说持有过高的仓位。
\section{2\% 法则}
\begin{tcolorbox}
    2\% 原则会防止你的账户在单次交易中出现本金亏损 2\% 以上的风险。
\end{tcolorbox}

\subsection*{风险控制的铁三角}
事实上,交易规模应该是根据公式计算出来的,而不是随意决定的。可以使用 2\% 原则对你可以交易的最大数量做出理性的判断。这个过程叫作“风险控制的铁三角”(见 \autoref{fig50-1})。

\figures{fig50-1}{A. 你计划要进行的交易的最大风险额度(永远不能超过账户规模的 2\%)。B. 你预计的进场位和止损位之间的价差——你每股所承担的风险。C. 将 A 除以 B,得到所能交易的最大股数。你并不一定要交易这么多,但是不应该超过这个数字。}

随着你的账户增大,你可能会想让每笔交易规模差异化,比如对一般的交易是最大限额的三分之一,对比较有信心的交易使用三分之二,其他更有信心的可能就全额使用了。无论你怎么做,风险控制的铁三角总会为你设定最大允许的交易规模。
\section{6\% 法则}
6\% 原则给每一个账户都设定了一个当月最大回撤比例。如果你达到了限制,这个月接下来的时间就要停止交易。6\% 原则强制你在受到食人鱼攻击前,从水里走出来。

\begin{tcolorbox}
    当你这个月总损失和持仓头寸的风险额度之和达到账户总金额的 6\% 时,在本月剩下的时间内,6\% 原则将不允许你进行新的交易。
\end{tcolorbox}

在与市场的周旋中我们都有过连续盈利的时期,当我们的每笔交易都点石成金时,应该积极地交易。

同样,有一些时候我们的交易变得非常糟糕。交易系统与市场步调完全相反,接连亏损。在这个时候,要重新审视这段时期,不要给自己太大的压力,退后一步、冷静一下尤为重要。专业人士在赔钱的时候可能会去休息一下,但会继续盯着市场,等待与市场的节奏重新匹配上。而业余人士更可能加大交易规模,直到账户出现严重亏损。6\% 原则会使你暂停下来,这时你的账户大体上还是完整的。
\subsection*{可用风险的概念}
如果你用 2\% 原则来设定止损位和交易规模,那么 6\% 原则能给你的账户设定最大风险额度。
\begin{enumerate}
    \item 把你这个月所有的亏损加总。
    \item 把你现在所有的持仓头寸的风险额度加起来。一笔持仓交易的风险额度是你入场点位和止损点位之间的价差,乘以持仓数量。假如你以 50 美元的价格买了 200 股股票,止损价是 48.50 美元,每股承担的风险是 1.50 美元。这样,你该笔交易的风险额度是 300 美元。如果市场向有利于你的方向发展了,你把止损价位上调到盈亏平衡的价位,你的该笔交易的风险额度就会变成零。
    \item 将以上两项相加(这个月的总损失加上持仓头寸的风险额度)。如果两者之和已经超过你月初账户资产的 6\% 时,这个月剩下的时间你都不能再增加交易头寸了,除非市场顺着你持仓的方向发展了,允许你提高了止损线。
\end{enumerate}

如果你根据 6\% 原则已经不能再进行新的交易,还是要继续跟踪自己感兴趣的股票。如果你看到了一个确实想交易的机会,但没有足够的可用风险额度了,可以考虑平掉部分持仓头寸,释放出一些风险额度给它。

当你已经接近 6\% 原则的限制,但发现了一个非常有吸引力的交易机会,此时你有两种选择:你可以兑现一个盈利的持仓头寸来释放可用风险额度;也可以收紧一些持仓头寸的止损线,减小持仓的风险。\textbf{盈利的不是可以依靠提升止损价位来实现额度的释放吗?}
\section{从下降中恢复}
当风险提升时,我们的交易水平会随之下降。初学者能在小交易上赚钱,于是开始有了信心,然后提高交易规模。这往往是他们赔钱的开始。随着头寸的增加,风险也逐渐加大,使得他们的行动变得僵化,不再灵活,这也是他们赔钱的原因。
\part{多技术方法共同参照原则}
我们对相互验证的定义是“在同一个价位或接近同一个价位上,出现了一群相互验证的技术信号”。相互验证是一个关键概念。其原因在于,在某个支撑区域或阻挡区域,越多的信号汇集在一起,则出现反转的可能性越大。我们可以通过一系列蜡烛图形态来相互验证,也可以通过一系列西方技术信号来相互验证,还可以通过上述两方面的信号来相互验证。

绝不可企图将自己的意愿强加于市场。一定要做一个趋势追随者,不要做一个趋势预测者。如果您怀着看涨的预期,那么就在上升趋势中入市,如果您持有看跌的预期,那么就在下降趋势中入市。
\begin{itemize}
    \item Don't forget old support and resistance levels.——不要忘记过去的支撑水平和阻挡水平(过去的支撑水平可能演化为新的阻挡水平,反之亦然)。
    \item If...then system.——如果……那么系统(\textbf{如果市场的演变符合预期,那么继续实施预定的交易方案——否则,平仓出市})。
    \item Stops——\textbf{始终采用止损指令作为保护措施}。
    \item Consider options.——将期权市场纳入考虑的范围。
    \item Intraday technicals are important even if you are not a day or swing trader.——日内图表的技术因素也是重要的方面,即使您不做日内交易或短线交易。
    \item Pace trades to market environment.——调整交易的节奏,以适应不同的市场环境(根据市场的具体条件,改变自己的交易风格)。
    \item Locals——自营交易商。绝不可以忽视自营交易商的动向。
    \item Indicators——技术分析信号,越多越好(多技术方法共同参照原则)。Never trade in the belief the market is wrong.——绝不可带着“市场错了”的成见从事交易。
    \item Examine the market's reaction to news.——注意研究市场对基本面信息的反应。
\end{itemize}
\chapter{实践细节}
当股票创出新高后,你会买吗?在双重顶时会卖吗?在回调中会买吗?你会寻找趋势反转吗?以上这些的方法各不相同,每种方法都可能赚钱,也可能赔钱。你应该选择那些吸引你、让你很舒服、适合你能力和气质的交易方法。没有一种交易方法是适合所有人的,就像没有一种运动是适合所有人的一样。

要想成功地交易,先要选定一种交易模式。在进行行情数据扫描之前,你应该十分清楚你想要找到什么。开发你的系统,并通过一些小交易先测试一下,确定你可以遵守交易纪律。你必须确定在你看到设计好的交易信号出现时,会按计划交易。
\section{怎样设定止损线:不要异想天开}
接受止损带来的烦恼和痛苦,但要尽力使它们更加合理和少一些不愉快。
\subsection*{在“市场噪声”之外设定止损}
如果把止损线设得离成交价格太近的话,会受到市场无意义波动的影响;如果把止损线设得太远的话,起到的保护作用会减少很多。

在股票市场中,我们可以把信号定义为股票的趋势性变化;把噪声定义为上涨趋势中当日股价低于前一交易日最低价的部分,和下降趋势中当日股价高于前一交易日最高价的部分。

先度量市场噪声,再把止损位设在市场噪声区域外数倍的位置。简言之,使用 22 日 EMA 来定义为趋势线。如果趋势是向上的,标记出所有回溯期(10-20天)内向下穿透 EMA 线柱的深度值,将其加总后除以向下穿透的线柱数量,得到回溯期的平均向下穿透值。它反映了当前上升趋势中平均的噪声水平。你应该把止损位设在远离市场平均噪声水平的位置。这就是为什么你需要把平均向下穿透值乘以一个系数,通常是 2 以上的数字。如果止损位设得太近容易弄巧成拙。

当 EMA 趋势是下降的时候,我们使用前期线柱的最高价向上穿透来计算安全区域。我们数一下选定期间内线柱的向上穿透情况,计算它们的平均值,得到平均向上穿透值。选一个系数乘以它,比如可以从 3 开始选,将得到的值加到每次高点上。在高点卖空比在低点买入需要更宽的止损空间。
\subsection*{不要把止损线设在明显的位置}
从密集的价格区间向下探出显眼的新低点,最容易吸引交易者在新低点下方设置止损位。问题是太多人在这里设止损线,造成这个区域里止损的人过多。市场有一个神秘的习惯,会很快地跌穿这些明显低点,引发止损后再反转,发动新的上升趋势。

把止损位设在并不明显的位置比较好——要么更接近市场目前水平,要么离明显位置更远一点。更近的止损位可以减少亏损规模的风险但是会增加被洗盘出局的风险。更低的止损位可以躲过一些假突破,但是一旦真触及止损,亏损规模会更大。

尼克止损是把止损位设在近期的次低点,而不是设在最低点附近。当作为空方的时候这个原则也是一样的——不把止损线设在最高点之上,而是设在次高点。

\figures{fig54-1}{在可口可乐(KO)公司的图中,我们发现了一个伴随着牛市背离的向下假突破,动力系统从红色变成了蓝色——允许买入。如果我们做多买入,我们应该在哪里设定止损线呢?尼克止损会设在 37.04 美元——比近期的次低点少1美分,是线柱 B 的最低点。在直觉外科公司(ISRG)的图表中,我们可以看到一个伴随着熊市背离的向上假突破,动力系统由绿变蓝——允许卖出。如果我们做空卖出,我们应该在哪里设置止损线呢?尼克止损会设在 445.05 美元——比最近的次高点——线柱 B 的高点,高几美分。}
你可以激进,也可以保守,但是要记住最重要的原则:第一是要有止损;第二是不要把止损位设在太明显的位置,也就是图上谁都能看出来的位置。

平均真实波幅(ATR)止损是指当你在最近的一根线柱中入场时,把你的止损位设在离当前这根线柱的极值至少一倍 ATR 的位置,如果是在两倍 ATR 的位置设置止损位就更安全了。你可以把它当作一种移动止损的方法,随着线柱的转变而移动它。使用移动止损的一个优点是它们逐渐减小了所暴露的风险额度。使用移动止损时,如果交易价格朝对你有利的方向变化时,它可以逐渐释放可用风险额度,从而允许你开始做新的交易。
\subsection*{不要让盈利变为亏损}
不要让有丰厚账面浮盈的未平仓头寸变为亏损!在交易之前,就要计划在什么水平开始保护你的利润。比如有一笔交易的盈利目标是 1000 美元,那么在有 300 美元盈利的时候就需要开始保护利润。一旦你的未平仓头寸浮盈达到 300 美元,你可以将止损线调整到盈亏平衡的位置。我们称这种移动为“为交易翻边”。

当交易的发展已经兑现了你的预期,这笔交易的盈利潜力逐渐变小。而你的风险(盈利和止损线之间的距离)会不断增加。交易就是在管理风险,当盈利与风险的比例渐渐恶化时,你便需要减小承担的风险。通过提升止损线,保护一定比例的利润,可以使盈利与风险比例控制在更平衡的位置。
\subsection*{只顺着你交易的方向移动止损线}
\subsection*{灾难性止损:专业交易者的救生衣}
“硬止损”是一种给你的经纪商下达的指令,而“软止损”是你心中的止损线,当到需要的时候你才会去执行真实操作。新手或业余交易者一定要使用硬止损线;而对每天盯盘的专业交易者来说,当系统提升需要止损时,他能遵守纪律去执行,那他可以使用“软止损线”。
\subsection*{止损线和隔夜跳空:仅对专业交易者}
如果你持有的股票在休市期间出现了一个重大利空,你怎么办?在第二天早上开盘之前查看集合竞价情况,你意识到股价将大幅低开,远低于你的止损线,意味着滑点会很大。

如果你是一名新手或者业余交易者,那并没有什么可选择的,只能咬紧牙关承受损失。但对于冷静的、有纪律的专业交易者来说,还有一种方法,那就是用做日内交易的方式退出。首先,撤走止损线,开盘之后当作开盘第一秒买入了一样,后面进行日内交易的操作。

开盘跳空缺口常常伴随着反弹,这给那些机敏的交易者提供了减少损失的机会。但这样的情况并不是一定会发生,所以大多数的交易者不要轻易尝试这种技术。因为这么做可能导致亏损更多,而不是减少亏损。

记住在收盘前要及时退出——已经走坏了的股票,可能当天会反弹,但明天将会有更多的卖家进场卖出,驱使股价进一步下跌。不要让一次反弹引发你对反转的希望。
\section{这是 A 级交易吗}
一旦你结束了一笔交易,市场将对你的入场、退出和最重要的整体交易三方面做出评级。

如果你是一位使用周线图和日线图做波段交易的交易者,那就用日线图来计算你每笔交易的级别。你的买入评级取决于入场点、购买当日的最高点和最低点的情况。
\begin{equation}
    \text{买入评级}=\frac{\text{最高价}-\text{买入价}}{\text{最高价}-\text{最低价}}
\end{equation}

每笔交易计算买入评级,而且我认为大于 50\% 就是不错的成绩了,意味着我是在当日线柱的较低部分买入的。

下面是卖出评级的计算公式:
\begin{equation}
    \text{卖出评级}=\frac{\text{卖出价}-\text{最低价}}{\text{最高价}-\text{最低价}}
\end{equation}

当评估一笔交易时,大多数人认为他们在交易中挣到或赔掉的金额是交易质量的反映。资金规模对画资产曲线来说很重要,但对单笔交易来说并不是很好的评价指标。通过比较你实际获得金额和潜在可获得金额的比值来评价交易质量可能更有意义。计算交易评级的方式是,比较交易的损益与入场点当日通道线的高度。
\begin{equation}
    \text{交易评级}=\frac{\text{卖出点}-\text{买入点}}{\text{通道线高点}-\text{通道线低点}}
\end{equation}

\figures{fig55-1}{A 日——2014 年 2 月 10 日,星期一:高点是 52.49 美元,低点是 51.75 美元,上通道线是 53.87 美元,下通道线是 47.61 美元(我们需要通道高度来计算退出的交易评级)。买入价为 51.77 美元。买入评级=(52.49-51.77)/(52.49-51.75)=97\%。B 日和 C 日——星期二和星期三:继续上涨,开始向上移动止损线。D 日——星期四:高点 54.49 美元,低点 53.39 美元。卖出点为 53.78 美元。卖出评级=(53.78-53.39)/(54.49-53.39)=35\%。交易评级=(卖出点-买入点)/通道高度=(53.78-51.77)/(53.87-47.61)=32\%。}
\section{仔细搜寻可能的交易}
在寻找股票交易机会之前,必须先开发一个交易系统或是交易策略。如果没有一个清晰的交易系统,你寻找什么呢?

搜寻交易机会是指对一些交易品种进行复盘,然后聚焦到一些有潜力的品种上。可以用肉眼搜寻,也可以用计算机来扫描——你可能要翻阅很多图表,每一个都只是瞅一眼;或者用你的计算机处理清单上的图表,标记出符合你交易模式的股票。重述一遍,确定一种自己信任的交易模式是最重要的第一步,搜寻是第二步。

\figures{fig56-1}{美国有机商品超市(WFM)日线图,13 日和 26 日指数移动均线,动力系统,MACD 柱(12-26-9),红点——潜在或实际的熊市背离,绿点——潜在或实际的牛市背离。美国有机商品超市(Whole Foods Market,WFM)的线图说明了扫描程序不能成为自动的交易者。它只是一只看门狗,可以警示市场可以交易的机会——做多或做空。收到这样的信号后,交易者需要研究一下这只股票可能出现背离的价格水平,并把入场价格、目标价格和止损价格写下来。}

如果要扫描更大数量的股票,需要增加一些“负面规则”。比如,你需要剔除每日成交量少于 50 万股或 100 万股的股票。这些股票的图表通常很不规则,滑点也比其他交易活跃的股票的要大。你还可能会把高价股从买入名单或低价股从卖出名单中剔除。
\chapter{保持良好的记录习惯}
市场在分发奖惩方面并不始终一致。这样的情况时有发生,比如一笔缺乏计划的交易赚钱了,而一个计划周密、执行认真的交易却亏钱了。这种随机性使我们颠覆了本应遵守的原则,鼓励草率地进行交易。

好的记录交易日志的习惯是培养和坚持纪律性的最好工具。它将心理、市场分析、风险管理联系到了一起。

交易日志的三个核心要素:
\begin{enumerate}
    \item 纪律的第一步是完成功课
    \item 纪律的进阶是写下你的交易计划
    \item 纪律的高潮是执行这些计划并且完成交易日志
\end{enumerate}
\figures{fig57-1}{每日功课电子表格}
\begin{enumerate}
    \item 查看远东市场。
    \item 查看欧洲市场。市场闻鸡起舞,你会体会到在美国产生的风波在重新返回西海岸之前,如何波及亚洲,然后传到欧洲的。
    \item 经济日历。当一份重要的数据,比如失业率或产能利用率,低于或超出市场预期,你便可以期待市场将出现绚丽的烟花秀。
    \item \href{https://www.marketwatch.com/}{Marketwatch} 网站。这是一个大众流行的网站,通常来说它是反向指标。
    \item 欧元汇率。我会写下最活跃的期货合约的现价,后面用动力系统状态的首字母标记——绿色(G)、蓝色(B)或者红色(R)——前面是周线图的,然后是日线图的。下面提到的其他市场,我所采用的是同样的格式。我关注欧元期货走势有两个原因。第一个原因是无论与美国股市表现一致或相反,欧元期货的走势都能延续一段时间;另一个原因是欧元期货有时候能提供非常好的日内交易机会。
    \item 日元汇率。上一条所述两个原因中,第二个原因比第一个在日元汇率上更适用。
    \item 原油。它是经济的血脉,并且原油期货会随着其上涨下跌而变化,原油期货是可以用来交易的。
    \item 黄金。它是市场恐慌情绪与通胀预期的一个敏感指标,同时也是很受欢迎的交易品种。
    \item 债券。利率的上涨或下跌是股市走势的主要驱动因素之一。
    \item 波罗的海干散货运价指数(BDI)。对于世界经济而言,它是一个敏感的先行指标。BDI 表示干散货的运送成本,例如把纺织品从越南运往欧洲,或是把木料从阿拉斯加运往日本。BDI的波动非常大,没有直接基于 BDI 的交易品种,这有助于BDI更准确地反应经济活动的实际情况。如果你交易航运业的股票,这个指标格外有用。
    \item 新高-新低指数。我认为新高-新低指数是股票市场最好的先行指标。
    \item 芝加哥期货期权交易所波动率指数(VIX 指数),也被称为“恐慌指数”。人们调侃:“VIX 走高,放心买入;VIX 走低,小心慢行。”横批是,“提防 VIX 的 ETF”——VIX 的 ETF因在交易中与 VIX 指数不同步而臭名昭著。
    \item 标准普尔500指数。写下前一个交易日指数的收盘价,并且将动力系统周线和日线显示状态的首字母标写在后面。
    \item 日线的价值。转到标普指数的日线图,留意最新一根线柱是收在价值区域的上方、正中还是下方,以及它与通道线的关系。这帮助我识别现在市场是超买了还是超卖了。
    \item 强力指数指标。注意这个指标的 13 日 EMA 均线是在它中心线的上方还是下方(对应牛市或熊市)以及是否有背离。
    \item 对标普指数的预判。测验自己对市场预测的精准度:写下对今天收盘价会比开盘价高还是低的预测,如果没有观点就空着。根据自己的预测是否正确,次日我会给这一栏涂上绿色或红色。
    \item 在电子表格的最后一行,总结今天将如何交易:积极地、保守地、防御地(仅进行平仓交易),日内的交易或者完全不进行交易。
\end{enumerate}
\subsection*{今天你准备好交易了吗}
有时候你会觉得踩准了市场的节奏,但其他时候你会和市场脱节。你的情绪、健康以及时间的压力会影响你的交易操作。
\figures{fig57-2}{“我做好交易的准备了吗?”自我测试}
\section{制作并评价交易计划}
任何交易计划都要依据所采用的策略量身定做。交易计划必须能提示你检查财报期、分红派息日期,以及期货交割日期,使你避免被可预见的新闻所侵袭。它必须清楚地记录你计划好的买入价、目标价、止损价以及交易规模。

在进入交易之前,先写下计划能帮你在风暴中建立一个理智和稳定的堡垒,它能帮助你不会忽略任何必要的事情。
\subsection*{为你的交易计划评分(交易的阿氏评分)}
\figures{fig58-1}{交易阿氏评分在做多交易(此例为“假突破伴随背离”的策略)的使用}
将你对五个问题的答案按 0-2 分进行打分(\autoref{fig58-1}):
\begin{itemize}
    \item 强力系统的周线图(前面章节有描述)——周线图是红色得 0 分,周线图是绿色得 1 分,周线图是蓝色得 2 分。强力系统为红色时,是禁止交易的;绿色时还可以进行交易,但是可能有些太晚;蓝色(紧跟在红色之后)表示恐慌正在褪去,是买入的好时机。
    \item 强力系统的日线图——与上一条同样的问题、同样的评分,标记在日线图上。
    \item 日线价格——在日线图上,如果最新价格在其价值区间之上得 0 分;在价值区间范围内得 1 分;低于其价值得 2 分。价格在价值区间之上时,买入已经有些迟了;在价值区间内还可以;在价值区间之下则是一笔好买卖。
    \item 假突破——没有的话得 0 分;已经发生得 1 分;很有可能将要发生得 2 分。
    \item 完备性——没有周期符合得 0 分;有一个符合得 1 分;两个周期看起来都很完备得 2 分。我通常会用两个时间周期来分析市场。对任何策略来说必须有其中之一符合一种入场交易的完备形态。极少情况下两个时间周期的形态都是完备的——在一个完备,另一个可以接受的情况下就可以进行交易了。如果没有一个时间周期的形态看起来是完备的,则不是一笔 A 类交易——抛弃这只股票,转移到另一只上面去。
\end{itemize}

\figures{fig58-2}{交易阿氏评分在做空交易(此例为“假突破伴随背离”的策略)的使用}
\subsection*{使用交易表}
当你对某只股票产生兴趣,并且交易的阿氏评分肯定了你的交易想法,完成交易表将有助于你专注于此交易最核心的部分。

\figures{fig58-3}{交易表在做多交易(此例为“假突破伴随背离”的策略)的使用}
\begin{description}
    \item[交易鉴定] 画出大概的 K 线样式来标示出这种策略。填写股票代码、记录下一个财报披露的日期、记录除息日、做计划的日期。
    \item[交易的阿氏评分] 当你将交易阿氏评分的各项得分加总时,将下面这个重要问题的答案写下来:这会是一笔 A 级交易么?如果总分在 7 分以下,则放弃这只股票,去寻找其他的。
    \item[市场、买入点、目标价、止损点和风险控制] 最左边的五个空格要求我回答有关市场基本状况的问题。尖峰反弹信号是否有效,追踪股票均线的指标是看多还是看空,这只股票的空头净额是多少,需要多少天来补上,所有这些内容都已在本书前面描述过。最后一个空格是简短的总结。用箭头所连接的三个空格是我决策制定过程的核心部分。它们所要的是每笔交易最重要的三个数字:买入价、目标价、止损价。资金风险——这笔交易中,你愿意冒亏损多少钱的风险?这个数额永远不应该超过你账户资产的 2\%。我通常把它控制在远远低于这个门槛的位置。持仓规模——根据持仓限额和入场点与止损点的差额,可以算出你能买多少数量。
    \item[买入之后] A 级盈利目标是在买入价上加日通道线高度的 30\%。软止损是记在脑海中的指令,而硬止损或灾难性止损是实在的指令。它不应该比第三部分中所写的止损价低。记下你将把止损位移到盈亏平衡位置的价格水平。当你执行这些必要步骤时,检查右手边的方框:设置止损价,创建一个日志,下达止盈订单。
\end{description}
\section{交易日志}
A 部分:交易日志需要回答为何决定交易这只股票。

B 部分:记录下入场和退出的日期和价格。记录滑点和买入量、卖出量及交易等级。

C 部分:退出的原因,需要附上显示入场点和退出点的合成线图。

D 部分:退出策略的清单要比交易策略的清单长。退出的原因可能是到了目标价或止损价,也可能是到了价值区间或包络线。我也可能会因为股票不再延续趋势方向或者开始掉转方向而选择退出。还有两种消极的退出:已无法承受下跌的痛苦,或是买入后发现这是一笔糟糕的交易。

E 部分:交易后的回顾分析。我喜欢在退出交易两个月后回顾这笔交易。我设计了一个跟踪图表,用箭头标记出入场点和退出点,然后写下时过境迁后对这次交易的评论。这是吸取经验教训最好的方法。
\figures{fig59-1}{网页版交易日志(部分)}

在退出交易一两个月之后,回顾每笔交易是最好的学习方式之一。交易信号在图形右侧时,可能会显得模糊不定。而当你在图形中间看到它们时,已变得无比清晰。回顾你已经完成了的交易,并且加上一个“交易后”图表,能让你重新评估自己当时所做的决定。现在你可以清楚地看出自己做得对还是不对。你的日志能给你珍贵的经验和教训。
\end{document}