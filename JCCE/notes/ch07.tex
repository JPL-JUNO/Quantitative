\chapter{持续形态}
所谓的持续性形态,意味着形态完成后,市场仍将恢复先前的趋势。举例来说,如果在上涨行情之后出现了持续形态,那么我们预期上涨行情仍在发挥作用。(当然,这一点并不排除在持续形态出现后先发生调整行情,调整之后再恢复原先的涨势。)

将要讨论的持续形态有窗口(以及含有窗口的一些蜡烛图形态)、上升三法、下降三法、分手线,以及白三兵形态。

\section{窗口}
日本技术分析师一般把西方所说的价格跳空称为窗口。按照西方的表达方式,我们说“回填跳空”;在日本,人们则说“关上窗口”。这一部分,我们先来阐述窗口的基本概念,然后还要探讨包含窗口(价格跳空)的其他一些形态。

窗口有两种类别,一种是看涨的,一种是看跌的。\textbf{向上的窗口}(如 \autoref{fig7-1} 所示)是看涨信号。在前一个时段最高点(也就是其上影线的顶端)与本时段最低点(也就是其下影线的底端)之间,存在价格上的真空地带。

\figures{fig7-1}{向上的窗口}

\autoref{fig7-2} 展示了一例\textbf{向下的窗口}。它是看跌信号。在前一个时段最低点与本时段最高点之间,存在价格缺口。

\figures{fig7-2}{向下的窗口}

根据日本技术分析师的观点,市场参与者应当顺着窗口形成的方向建立头寸。这是因为窗口属于持续性质的技术信号。因此,如果出现了向上的窗口,我们就应该利用市场回落的机会逢低买进;如果出现了向下的窗口,就应该利用市场反弹的机会逢高卖出。

日本人还认为:“调整行情于窗口处终结。”这意味着窗口可能转化为支撑区域或阻挡区域。于是,如果出现了向上的窗口,则在今后市场向下回撤时,这个窗口将形成支撑区域(我们马上会看到,这是指窗口的全部空间)。如果市场在向下回撤时收市价达到了窗口下边缘之下的水平,那么,先前的上升趋势就不复成立了。请注意,在 \autoref{fig7-1} 中,市场在日内一度下跌到窗口下边缘之下,但是因为不是收市价低于该区域,所以向上的窗口所形成的支撑区域保持完好。

同样,如果出现了一个向下的窗口,则意味着市场还将进一步下降。此后形成的任何价格向上反弹,都会在这个窗口处遭遇阻挡(指窗口的全部空间)。如果多方拥有足够的推动力,将收市价推升到向下的窗口之上,那么,下降趋势就完结了。

在西方,一般认为价格跳空总要被回填。我不知道这一点是不是正确的,不过如果采用蜡烛图技术的概念“调整行情于窗口处终结”,那么当行情试图回填价格跳空时,便可以考虑买进(在向上的窗口处)或卖出(在向下的窗口处)。

对窗口最常发生的误解是,有些人误以为如果两根蜡烛线的实体之间不相接触,那么这两根蜡烛线便组成了一个窗口。正如 \autoref{fig7-1} 和 \autoref{fig7-2} 所示,组成窗口的蜡烛线必须是其影线之间互不重叠。无论两根蜡烛线实体之间的“缺口”有多大,除非在它们的影线之间存在缺口,否则不构成窗口。

如 \autoref{fig7-3} 所示,在 7 月 22 日的最高点和次日的最低点之间仅有 4 美分的空白,这是一个小型向上的窗口。无论向上的窗口多么小,它都应当构成潜在的支撑区域。对于向下的窗口,同样也应当构成阻挡区域。窗口不在乎尺寸大小。在 \autoref{fig7-3} 中,向上的窗口形成了支撑区域,之后当行情回落到接近其支撑区域时,留下了一些长长的下影线,突出显示此处需求强大。正如本图所示,虽然向上的窗口构成潜在的支撑区域,但是市场并不需要精确地回落到窗口所在的支撑区域之后才能向上反弹,有时甚至连接近它都谈不上。因此,在市场朝着向上的窗口回落的过程中,如果您积极看好,那么甚至当市场接近该窗口的上缘时,便可以考虑买进了,无须等到行情进入该窗口之内。如何运用窗口,取决于您的交易风格和交易的迫切程度。应当事先设置止损措施(做好思想准备或者其他措施),以防止行情收市于向上的窗口的下缘之下。

\figures{fig7-3}{轻质原油——日蜡烛线图(向上的窗口)}
在 \autoref{fig7-4} 中,在 20.50 美元和 22.50 美元之间存在一个幅度非常大的窗口。这么一来,它构成了一个幅度达 2 美元的支撑区域(从窗口的顶部 22.50 美元到窗口的底部 20.50 美元)。

在窗口幅度相对较大的情况下,请记住,向上的窗口的关键支撑水平位于窗口的底部(相应地,向下的窗口的关键阻挡水平位于窗口的顶部)。因此,在向上的窗口中,其支撑作用的“最后一口气”就是图上用虚线标注的窗口的底部(即价格跳空的下边缘)。

下面再看看另外两个向上的窗口,图上分别用 1 和 2 来做了标记。窗口 1 的支撑作用在其出现后三周之内一直维持良好,直到 4 月 6 日的蜡烛线才向下突破了该支撑水平。窗口 2 的支撑作用在其出现之后的第二天便被向下突破了。在窗口 2 被向下突破后,窗口 1 发挥了支撑作用。这就是我运用窗口的具体做法。举例来说,如果某个窗口的支撑被突破,那就在被突破的窗口下方寻找另一个窗口,以后者作为下一个支撑区域。在本图中,一旦窗口 2 被向下突破,下一个支撑位置便是窗口 1。

\figures{fig7-4}{网威公司(Novell)——日蜡烛线图(向上的窗口)}

单一的蜡烛图信号是否值得采信,必须首先从其所处的总体技术背景来考虑。\autoref{fig7-5} 的实例充分显示了这一点的重要性。3 月 1 日的第一根蜡烛线是看涨的锤子线。但是更重要的是,看看这根锤子线是如何形成的——一个向下的窗口。一方面,锤子线本身立即构成了支撑水平;另一方面,切不可忘记,因为这个向下的窗口,窗口的整个空间现在都成为阻挡区域。果然,从锤子线开始的反弹行情到向下的窗口顶边时渐渐熄火了。
\figures{fig7-5}{亚马逊——5 分钟蜡烛线图(向下的窗口)}

传统的日本技术分析理论认为,在三个向上的或向下的窗口之后,很可能市场已经处在过度超买状态,上升行情难以为继(在三个向上的窗口的情形下),或者处在过度超卖状态,下降行情无力维持(在三个向下的窗口的情形下)。

窗口的地位如此显要,无论已经出现了多少个窗口,当前的趋势都不受影响,直到最后的窗口被关闭为止。在上升行情中,可以有任意数目的向上的窗口。只要市场没有以收市价向下突破最高的那个向上的窗口,则趋势依然维持向上。

在 \autoref{fig7-8} 中,展示了这样的一个实例。8 月中旬于 B 处形成了一个看涨的吞没形态,由此开始形成了一轮上冲行情。最终,这轮行情总共打开了六个向上的窗口。10 月  5日和 6 日组成了一个孕线形态,这是我们得到的最早的线索,表明债券市场喘不过气来了。但是,要等到收市价向下突破了第六个向上的窗口之后,这轮上冲行情才终结。后来的结果表明,债券期货行情此处的向下反转形成了一个主要的历史高点,之后市场多年保持下滑态势。

\figures{fig7-8}{债券期货——日蜡烛线图(向上的窗口)}
\section{向上跳空和向下跳空并列阴阳线形态}
\textbf{并列阴阳线形态}是由具备特定形态的两根蜡烛线组成的,两者一起向上跳空或向下跳空。\autoref{fig7-10} 为向上跳空并列阴阳线形态,其中一根白色蜡烛线和一根黑色蜡烛线共同形成了一个向上的窗口。这根黑色蜡烛线的开市价位于前一个白色实体之内,收市价位于前一个白色实体之下。在这样的情况下,这根黑色蜡烛线的收市价,就构成了买卖双方争夺的要点。如果市场以收市价向下突破到该窗口之下,那么这个向上跳空并列阴阳线形态的看涨意义就不再成立了。在向下跳空并列阴阳线形态中,基本概念与上述形态是相同的,只不过方向相反。一根黑色蜡烛线和一根白色蜡烛线共同打开了一个向下的窗口。在向上跳空和向下跳空并列阴阳线形态中,两根蜡烛线的实体的大小应当不相上下。两种跳空并列阴阳线形态都很少见。

\figures{fig7-10}{向上跳空并列阴阳线形态}

在 \autoref{fig7-12} 中,9 月底出现了一个小型向上的窗口。在向上的窗口之后有两根蜡烛线,组成了向上跳空并列阴阳线形态。之所以称之为并列阴阳线形态,是因为在向上的窗口之后,先是一根白色蜡烛线,然后是一根黑色蜡烛线。然而上面刚刚指出,依我看来,在向上的窗口之后,到底那两根蜡烛线是什么样子并不要紧。\important{主要的考虑是把向上的窗口视为支撑区域,并根据收市价来判断其守与破}。正如图上虚线所示,根据收市价来判断,该支撑区域无论如何总算维护住了。后来,10 月底出现了一根看涨的捉腰带线,并且它包裹了之前的三根黑色实体,从而为该窗口的支撑作用给出了最终的验证信号。

\figures{fig7-12}{铂金——周蜡烛线图(向上跳空并列阴阳线)}
\subsection{高价位和低价位跳空突破形态\label{subsection7-2-1}}
在上升趋势中,当市场经历了一轮急剧的上涨后,在正常情况下都需要一个调整消化的过程。有时,这个整理过程是通过一系列小实体来完成的。如果在一根坚挺的蜡烛线之后,出现了一群小实体的蜡烛线,则表明市场已经变得犹豫不决了。虽然这群小实体表示行情趋势已经从向上转为中性,但是它们的出现在某种意义上是健康的,因为通过踩水的过程,缓解了市场所处的超买状态。一旦后来某一天的行情从这群小实体处打开了一个向上的窗口,就是看涨的信号。这就是一个\textbf{高价位跳空突破形态}(如 \autoref{fig7-13} 所示)。之所以这样称呼这类形态,是因为在这类形态中,市场先是在最近形成的高价位上徘徊,后来才下定决心向上跳空。
\figures{fig7-13}{高价位跳空突破形态}

可想而知,\textbf{低价位跳空突破形态}正是高价位跳空突破形态的反面角色,两者对等而意义相反。低价位跳空突破形态是一个向下的窗口,是从一个低价位的横向整理区间向下打开的。这个\textbf{横向整理区间}(一系列较小的实体)发生在一轮急剧下跌之后,曾经使市场稳定了下来。当初,从这群小实体蜡烛线的外观看来,似乎市场正在构筑一个底部。但是后来,市场以窗口的形式从这个密集区向下突破,打破了这种看涨的期望。

在 \autoref{fig7-15} 中,7 月 31 日是一根锤子线,结果成了之后上涨行情的低点,在这轮上涨中,8 月初曾经形成一个向上的窗口。在 8 月 7日所在的一周里,出现了一根长黑色实体,组成了乌云盖顶形态,在这轮上涨行情的前方添加了一块临时的挡板。

在 8 月 21 日所在的一周里,先是一根长长的白色蜡烛线,后面跟着一系列小实体,显示该股票行情已经进入调整阶段。8 月 28 日打开了一个小型向上的窗口,完成了一个高价位跳空突破形态,证明多头完全控制了市场。

\notes{在本形态中,或者在任何高价位跳空突破形态中,一旦市场收市于其中向上的窗口之下,则消除了形态的看涨意义。}对低价位跳空突破形态来说,道理相同而方向相反。
\figures{fig7-15}{康宁公司——日蜡烛线图(高价位跳空突破形态)}

在 \autoref{fig7-17} 中,4 月初的孕线形态有助于终结当时的上涨行情。从本形态开始市场下降并逐步加速,特别是 4 月 15 日的超长黑色实体加剧了下跌的进程。之后的两个时段都是纺锤线,有线索显示股票正在努力稳住。然而,4 月 17 日收市价再创新低,下一日又完成了一个低价位跳空突破形态,表明空头重新夺得全部控制权。

请观察 5 月初的小型窗口是如何转化为阻挡区域的。该阻挡区域很重要,值得牢记于心,因为在 1 处有一根锤子线,在 2 处又有一个看涨吞没形态,都是底部信号。但是在这两个看涨信号出现时,交易者对买进应当保持谨慎态度,因为根据该窗口的阻挡作用,潜在的利润空间有限。
\figures{fig7-17}{糖——日蜡烛线图(低价位跳空突破形态)}
\subsection{跳空并列白色蜡烛线形态}
在上涨行情中,先出现了一根向上跳空的白色蜡烛线,随后又是一根白色蜡烛线,并且后面这根线与前一根线大小相当,两者的开市价也差不多处在同样的水平上,这样就形成了一种看涨的持续形态。这种双蜡烛线形态称为\textbf{向上跳空并列白色蜡烛线形态}(或者称为\textbf{向上跳空并列阳线形态},如 \autoref{fig7-18} 所示)。

\warning{上面介绍的这种并列白色蜡烛线形态是很少见的}。不过,更少见的还有向下跳空的两根并列白色蜡烛线。这类形态称为向下跳空并列白色蜡烛线形态。在下跌行情中,尽管它们都是白色蜡烛线,因为之前向下的窗口,依然把它们归结为看跌信号。这是因为这两根白色蜡烛线被看作空头平仓的过程。一旦空头平仓的过程完成了,价格就要进一步下跌。这类向下跳空并列白色蜡烛线形态之所以特别罕见,其原因不难理解。在市场下降趋势中,当出现向下跳空时,如果形成跳空的蜡烛线是一根黑色蜡烛线,那么当然比一根白色蜡烛线自然得多。

\figures{fig7-18}{上升趋势中的向上跳空并列白色蜡烛线形态}

虽然下面也为这些形态提供了实例,但是形成跳空并列白色蜡烛线形态的蜡烛线的具体情况并不重要,无须死记硬背。重要的是形态中所包含的向上或向下的窗口。正如在 \autoref{subsection7-2-1} \nameref{subsection7-2-1} 讨论的那样,\tips{这两根蜡烛线的样式并没有多少影响,无论窗口之后的两根蜡烛线都是白色的(在跳空并列白色蜡烛线形态下),还是一根白色一根黑色的(在跳空并列阴阳线形态下)。唯有窗口本身提供了趋势信息,以及支撑区域或阻挡区域。}

\textbf{窗口,才是关键因素}。不论向上的窗口还是向下的窗口,窗口之后的两根蜡烛线的具体组合与颜色配合无关宏旨。向下的窗口推动趋势向下,向上的窗口推动趋势向上,并且窗口分别构成阻挡或支撑区域。

如 \autoref{fig7-20} 所示,5 月的头两个交易日形成了一个向上跳空并列白色线形态。前面曾经说过,在该向上跳空并列白色线形态中,两根蜡烛线的颜色并没有什么大不了的,5 月 1 日向上打开的窗口才真正决定了该形态的积极信号。

因为本图中包括许多窗口的实例,以下逐一分别讨论。
\begin{itemize}
    \item 1 处是一个向下的窗口,它维持趋势向下。尽管形成该窗口的 3 月 26 日是一根锤子线,也不影响上述判断。\tips{在窗口与其他形态之间进行取舍时,我通常选择窗口。}就本例而言,是要在看涨的锤子线和看跌的向下的窗口之间取舍。向下的窗口所造成的看跌前景优先于锤子线。市场必须收市于窗口之上,才能确认锤子线的看涨前景。
    \item 2 处是一个小型向上的窗口。在 4 月 4 日和 5 日形成的乌云盖顶形态之后,行情小幅回落,该窗口此时成为支撑。4 月 10 日的十字线(正处在该窗口的支撑水平上)反映了两个方面的情况:第一,2 处的窗口发挥了支撑作用;第二,十字线之前接连出现三根黑色实体,证明行情处在下降趋势,十字线的出现表明股票行情已经挣脱了之前的下降趋势。
    \item 经过 4 月 10 日成功捍卫窗口 2 的支撑作用后,行情开始上涨。上涨行情持续到 4 月 23 日,达到 495 美元的阻挡水平后停顿。该阻挡水平来自向下的窗口 1 的顶边。从该阻挡水平处引发的下降行情打开了 3 处向下的窗口。4 月 25 日是一根长黑色实体,之后跟着一根小实体,后者居于长黑实体内部,形成了一个孕线形态。这表明空头正在失去动力。
    \item 5 月 1 日形成了向上的窗口,之后是两根小的白色实体。(这就形成了之前介绍的向上跳空并列白色线形态。)在向上跳空并列白色线形态之后,行情演变成“窗口大战”,上有 3 处向下的窗口的阻挡作用(大约 488 美元),下有 4 处的向上的窗口的支撑水平(大约 475 美元)。我们看到,在一周多的时间里,上述支撑和阻挡区域都保持完好,直到 4 月 17 日,需求才足够强大,以收市价将市场推升到了向下的窗口 3 的阻挡水平之上。(4 月 14 日,市场曾经在日内的行情演变过程中推进到了该窗口之上,但是未能收市于之上——这样便维持了窗口阻挡作用完好无损。)自从突破该阻挡水平之后,市场快速上涨,直到 5 月 18 日和 21 日形成了孕线形态,终结了这轮上冲行情。
\end{itemize}
\figures{fig7-20}{铂金——日蜡烛线图(向上跳空并列白色线形态)}
\section{上升三法(上升三蜡烛线法)和下降三法(下降三蜡烛线法)形态}
所谓三法形态,包括看\textbf{涨的上升三(蜡烛线)法},以及\textbf{看跌的下降三(蜡烛线)法}。(请注意,这里我们又与数字 3不期而遇了。)这两类形态均属于持续形态。也就是说,一旦看涨的上升三法形态完成后,之前的趋势应当恢复,行情继续走高。相应地,在看跌的下降三法形态完成后,之前的下降趋势继续有效。

上升三法形态(如 \autoref{fig7-21} 所示)由以下几个方面组成:
\begin{enumerate}
    \item 首先出现的是一根长长的白色蜡烛线。
    \item 在这根白色蜡烛线之后,紧跟着一群依次下降的或者横向延伸的小实体蜡烛线。这群小实体蜡烛线的理想数目是 3 根,但是 2 根或者 3 根以上也是可以接受的,条件是:\important{只要这群小实体蜡烛线基本上都局限在前面长长的白色蜡烛线的高点到低点的价格范围之内}。我们不妨做这样的理解:由于这群较小的蜡烛线处于第一天的价格范围之内,它们与最前面的长蜡烛线一道构成了一种类似于三日孕线形态的价格形态。(在本形态中,所谓处于最前面的蜡烛线的价格范围之内,指的是这群小蜡烛线均处于该蜡烛线的上下影线的范围之内;而在真正的孕线形态中,仅仅是小蜡烛线的实体包含在前面那根蜡烛线的实体之内。)小蜡烛线既可以是白色的,也可以是黑色的,不过,黑色蜡烛线最常见。
    \item  最后一天应当是一根坚挺的白色实体蜡烛线,并且它的收市价高于第一天的收市价,同时其开市价应当高于前一天的收市价。
\end{enumerate}
\figures{fig7-21}{上升三法形态}

下降三法形态(如 \autoref{fig7-22} 所示)与上升三法形态完全是对等的,只不过方向相反。这类形态的形成过程如下:市场应当处在下降趋势中,首先出场的是一根长长的黑色蜡烛线。在这根黑色蜡烛线之后,跟随着大约3根依次上升的小蜡烛线,并且这群蜡烛线的实体都局限在第一根蜡烛线的范围之内(包括其上、下影线)。最后一天,开市价应低于前一天的收市价,并且收市价应低于第一根黑色蜡烛线的收市价。本形态与看跌旗形或看跌三角旗形形态相似。\notes{本形态的理想情形是,在第一根长实体之后,小实体的颜色与长实体相反。也就是说,对看涨的上升三法形态来说,应当是黑色的小实体;而对看跌的下降三法形态来说,应当是白色的小实体。虽然如此,从我的经验出发,2 根,至多 5 根小实体都可以很好地完成形态。同时,小实体可以是任意颜色的。}

\figures{fig7-22}{下降三法形态}

\tips{上升三法形态给我们带来的挑战在于风险报偿比方面。}等到上升三法形态完成之时,股票行情或许已经远远地离开它最近的低点了。在这种情况下,如果等到三法形态完成时买进,或许不能带来有吸引力的交易机会。如此一来,一旦上升三法形态完成,\tips{交易者必须首先考虑到潜在的利润空间,把它与风险权衡一下(风险从买入点算起,到上升三法形态中开始的那根白色蜡烛线的最低点为止)。}

\begin{tcolorbox}
    请记住,只要那群小实体维持在之前白色蜡烛线的全部交易范围内,即使它们的影线冒出白色蜡烛线的范围,也是可以接受的。
\end{tcolorbox}

现在我们来观察一个实例,其中展示了交易量分析与上升三法形态相结合的具体做法。在理想的上升三法形态中,第一根蜡烛线和最后一根蜡烛线,即那两根长的白色蜡烛线,伴随着在上升三法的所有交易时段里最大的交易量。这一点为每一根白色蜡烛线都提供了验证信息,表明多方对市场的控制力更强大。在 \autoref{fig7-26} 中,6 月 17 日是一根长长的白色蜡烛线,伴随着相对强大的交易量。不仅如此,这根蜡烛线与之前的两根还组成了一个十字启明星形态。

在 6 月 17 日的白色实体之后,一系列小实体向下缓缓飘落,同时伴随的交易量也逐步收缩。6 月 24 日是一根高耸的白色蜡烛线,伴随着放大的交易量,将市场推向更高处,也完成了上升三法形态。本图是一个经典实例,说明将交易量分析与蜡烛图指标结合起来,可以进一步增大蜡烛图指标的成功概率。

\figures{fig7-26}{花旗集团(Citigroup)——日蜡烛线图(上升三法形态)}

在绝大多数情况下,上升三法形态发生在上升趋势或横向延伸趋势中。不过,有时候本形态也有助于界定在抛售行情之后出现的转折点。\autoref{fig7-27} 给出了本形态的另一个实例。9 月初的上升三法形态有助于确认 7500 附近的支撑水平是稳固的。不过,在这个上升三法形态出现后,市场并没有立即开始上涨。虽然蜡烛图信号经常发出市场转折的信号,但是它们并不意味着在信号出现后市场必定立即开始上涨(在上升三法形态的情况下)。取而代之的是,许多蜡烛图形态的出现,正如本图的这个实例,可能加强了某个支撑区域。这正是道琼斯指数位于 7500 附近的支撑水平处发生的情形。9 月中旬出现了一例刺透形态,10 月 4 日出现了一根锤子线,累次证实了该支撑水平的有效性。

\figures{fig7-27}{道琼斯工业指数——日蜡烛线图(上升三法形态)}

\important{蜡烛图形态或蜡烛线在更大的市场背景下到底出现在什么样的位置,常常比蜡烛图形态本身更为重要}。举例来说,如果一个看涨吞没形态出现的位置接近某个阻挡水平,那么从风险报偿比的角度来看,在形态完成时买入并没有多大的吸引力,因为这是在阻挡水平处买进的。
\section{分手线形态}
\autoref{fig7-31} 所示的分手线形态也是由两根颜色相反的蜡烛线组成的,但是同反击线形态不同的是,\textbf{分手线形态}的两根蜡烛线具有相同的开市价。

分手线形态属于持续信号。道理很简单。在市场上涨的过程中,如果出现了一个黑色实体(尤其是相对较长的黑色实体)时,对多头来说,可能成为他们的一块心病。他们满腹狐疑,空头或许正在争得主动权。无论如何,如果后一天市场在开市时向上跳空,开市价回到了前一根黑色蜡烛线的开市价的水平,就能有力地证明空头已经失去了对市场的控制——特别是当天能够收市在较高位置,形成了一根白色蜡烛线。上述情形就是如 \autoref{fig7-31} 所示的看涨分手线形态的演变过程。在这类形态中,\notes{白色蜡烛线同时还应当是一根看涨捉腰带线(即其开市价位于或接近本时段最低点,收市价位于或接近本时段最高点)。}在 \autoref{fig7-31} 中,看跌的分手线形态与上述内容完全对应,但方向相反。一般认为,这类形态属于看跌的持续形态。\tips{分手线形态难得一见}。

\figures{fig7-31}{看涨的和看跌的分手线形态}