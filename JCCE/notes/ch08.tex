\chapter{神奇的十字线}
十字线是一种独特的趋势转折信号,特别是当它处在上涨行情时。在十字线出现后,如果发生下列情形,则增加了十字线构成反转信号的可能性。

\begin{enumerate}
    \item 后续的蜡烛线验证了十字线的反转信号。
    \item 市场正处在超买状态或超卖状态。
    \item 十字线在该市场出现得不多。如果在某个行情图上出现了许多十字线,则即使出现了新的十字线,也没有多大意义。
\end{enumerate}

在一根完美的十字线上,开市价与收市价处于同一水平,不过,这一标准也有一定程度的灵活性。如果某根蜡烛线的开市价与收市价只有几个基本价格单位的差别(举例来讲,在股票市场上其差别仅为几美分,或者在长期国债市场上仅有几个 1/32 美元的差距,等等),那么依然可以把这条蜡烛线看成一根十字线。根据什么原则来认定,一根近似于十字线的蜡烛线(也就是说,它的开市价与收市价很接近,但不是严格一致)到底算不算十字线呢?这是一个带有主观色彩的问题,我们找不到严格的标准。下面列举了若干技巧,当我们遇到近似的十字线时,可以借助它们来判断,是否可以把它归结为常规十字线并采取相应的行动。

\begin{enumerate}
    \item 看这根近似的十字线与其邻近的价格变化的相互关系是怎么样的。如果在这根近似十字线的周围,还有一系列的小实体,那么就不应该认为这根蜡烛线有多大的意义,因为在它附近有这么多的小实体蜡烛线。然而,如果在若干长蜡烛线之后出现了一根近似的十字线,那么我们就可以说这个时段的变化具备与十字线相同的含义,因为本时段的变化与之前的行情显现出本质上的区别。
    \item 如果当时市场正处在一个重要的转折点。
    \item 如果市场已经处在极度超买或超卖状态。
    \item 如果当时已经有其他技术信号发出了警告信息。
\end{enumerate}
这种做法的理论依据是,因为十字线可能构成了重要的警告信号,所以我们宁可错认也不能漏过。

\important{虽然十字线在引发市场的顶部反转方面是相当有效的,但是根据我们的经验来看,在下降趋势中十字线往往丧失了发挥反转作用的潜力}。其中的原因可能是这样的:十字线反映了买方与卖方在力量对比上处于相对平衡的状态。由于市场参与者抱着骑墙的态度,市场往往会“因为自身的重力而下坠”(这是市场参与者的行话)。这一点与下述情形有异曲同工之妙:当市场向上突破时必须伴随着重大的交易量才能有效地验证向上突破信号,而在市场向下突破时,交易量是不是重大、是不是构成验证信号,就不那么重要了。

\begin{tcolorbox}
    因此,当十字线出现时,在上升趋势中市场可能向下反转,而在下降趋势中市场则可能继续下跌。
\end{tcolorbox}

为了区分上涨行情中的十字线和下降行情中的十字线,称前者为\textbf{北方十字线},后者为\textbf{南方十字线}(\autoref{fig8-6})。

\figures{fig8-6}{北方十字线与南方十字线}
\section{北方十字线(上涨行情中的十字线)}
十字线的出现不一定意味着价格立即掉头向下。十字线向我们揭示了市场的脆弱状态,可能成为行情转变的起点。

如果出现了十字线,我不会把市场的短期趋势从上升调整为下降,但是会从上升调整为上升或中性。如果这根十字线同时与另一种技术信号相验证,就把趋势方向从上升调整为中性或下降。单单凭着一根十字线就把短期趋势从上升调整为下降,不应该这么干。

在 \autoref{fig8-7} 中,A 和 B 两处的纺锤线提供了线索,表明它们之前的趋势现在陷入僵局。这类小实体代表了买卖双方正在拔河拉锯,相比之下,十字线则代表了牛熊双方达到了完全的平衡。

从 B 处开始,上涨行情形成了一系列长白色实体,反映出市场生龙活虎。随着十字线的到来,虽然只是单一一个时段,显示了市场已经与先前的趋势脱钩。十字线表明,当日行情已经发生实质性转变,与之前一系列收市价明显高于开市价的白色蜡烛线大不相同。

在本例中,自从十字线出现后,指数从上涨转为横盘,再转为下降。然而,十字线的出现并不必然意味着市场即将下降。不过,十字线的出现,尤其是在本例中如此超买的市场状态下,依然有理由多加小心。轧平部分多头头寸,卖出看涨期权,上调止损指令的价位,这几招都属于应对本例十字线的对策,可供抉择。

\figures{fig8-7}{道琼斯工业指数——日蜡烛线图(出现在长长的白色蜡烛线之后的十字线)}
\section{长腿十字线(黄包车夫)、墓碑十字线与蜻蜓十字线}
某些十字线带有色彩鲜明的绰号,主要依据来自它的开市价和收市价(即十字线上的水平横线)位于本时段的高点还是低点,或者是否同时具备长长的上影线和长长的下影线。

如果蜡烛线具有长长的上影线和长长的下影线,并且实体较小,就称为\textbf{大风大浪线}。如果这类蜡烛线是一根十字线,而不是小实体,就成了\textbf{长腿十字线}。它还有个诨名,\textbf{黄包车夫(线)}。

在长腿十字线上,十字的部分表示市场正处在过渡点上。长长的上影线则说明,市场在本时段先是猛烈上推,后是急剧下滑,最后其收市价已经远远离开本时段的最高点了。拉长的下影线揭示了市场在本时段先是剧烈出货,后是强烈反弹,最后其收市价已经收复了相当部分的失地。换句话说,市场上冲,暴跌,再上冲,不一而足,大幅动荡。这是一个混乱的市场。对日本分析师来说,非常长的上影线或非常长的下影线的形成——借用他们的话来描述——就表示市场“失去了方向感”。如此一来,长腿十字线就成为市场与之前趋势分道扬镳的标志。

还有一种非常独特的十字蜡烛线,称为墓碑十字线(也称为灵位十字线,如 \autoref{fig8-3} 所示)。在某根蜡烛线上,当开市价和收市价都位于当日的最低点时,就形成了一根墓碑十字线。

\figures{fig8-3}{墓碑十字线}

\notes{墓碑十字线的长处在于昭示市场顶部方面。}

蜻蜓十字线是墓碑十字线的反面角色,是看涨的。在蜻蜓十字线上,开市价和收市价位于本时段的最高点。这意味着,在本时段内市场曾经下跌至很低的水平,但后来力挽狂澜,收市价已经回升到或非常接近本时段的最高点。这一点与锤子线相像,但是锤子线是小实体,而蜻蜓十字线是一根十字线,没有实体。

在运用十字线时,一条具有普遍意义的规则(实际上对所有蜡烛图信号都适用)是,\important{应当首先观察信号之前的行情演变轨迹}。举例来说,在行情上涨过程中出现十字线,属于潜在的反转信号。因此,首先必须有上涨行情可以反转。这意味着,如果十字线出现在交易区间的环境下,便没有什么预测意义了,因为没有可反转的趋势。日本人贴切地形容局限在横向区间的行情为“箱体”。

在 \autoref{fig8-13} 中,十字线从微观层面反映了当时市场宏观层面的交易区间环境:市场正处在上下两难之中。由于没有可反转的趋势, \autoref{fig8-13} 中的十字线没有预测意义,不过它揭示了一个事实,即确认了当时所处的无趋势的市场环境。上述分析有例外情形:虽然十字线处在交易区间的环境下,但是它的位置处于区间的顶部或底部。如此一来,它在验证阻挡作用或支撑作用方面,可能带来有用的信号。

\figures{fig8-13}{箱体区间中的十字线}

我们从 \autoref{fig8-14} 中观察三根十字线与之前趋势的关系。十字线 1 处于箱体区间的中部。因此,这个十字线没有任何预测意义,因为没有可反转的趋势。十字线 2 是蜻蜓十字线,它也处于同样的情形下。十字线 3 的情形很不一样,因为它出现的位置完全不同。该十字线出现在一轮上涨行情之后,这轮行情把股票带入了超买状态。如此一来,它就具备了反转意义。十字线 3 之后还有两根十字线,呼应了它们之前的十字线 3,这就告诉我们——股票已经使尽了看涨的力道。综上所述,十字线与之前趋势的关系具有决定意义。
\figures{fig8-14}{朗维尤纤维公司(Longview Fibre)——日蜡烛线图(箱体里的十字线)}
在图8.16中,有一系列十字线的实例,可以用来说明市场环境是如何影响十字线的重要性的。下面逐一分析各个十字线。
\begin{itemize}
    \item 十字线 1。“八级大地震”,日本人这样形容十字线 1 之前的行情变化。先是两根强有力的黑色实体势如劈竹般地下跌,然后两根势均力敌的白色蜡烛线将所有的下跌幅度席卷收回。十字线 1 表明股票已经和之前的趋势脱离关系(由于那两根长长的白色蜡烛线的缘故,之前的趋势为上升趋势)。我们从十字线以及之前长长的白色蜡烛线中选择最高点作为阻挡水平(这里白色蜡烛线的高点为 3745)。本阻挡水平在下一时段保持完好。
    \item 十字线 2。十字线 2 出现在一段短线下跌行情之后。(哪怕在十字线之前只有几个时段的下降行情,由于十字线紧随其后,我依然认为这属于下降趋势。)因此,作为下降趋势中的十字线,无须过多担心它会成为反转信号。
    \item 十字线 3。本十字线出现在一根长长的白色蜡烛线之后。这么一来,它的确暗示着自从 3705 点以来的上涨行情可能失去了动力。然而当这根十字线出现时,股票处在超买状态吗?据拙见,不是超买状态(可以把这里的情形与十字线 1 之前几乎垂直的强力上冲行情比较一下)。因此,本十字线相比在超买状态下的十字线,意义相对较小。一旦市场收市于该十字线之上,便消除了它可能带有的任何一点看跌意味。
    \item 十字线 4。十字线 4 发生在横向延伸的价格环境下。既然之前没有趋势可以反转(因为十字线处在箱体区间中),那么它作为反转信号的意义便较为淡薄。本十字线有一点有用的方面,它有助于增强来自几个时段之前的看跌吞没形态的阻挡水平的作用。
    \item 十字线 5。一根南方十字线。既然它没有与其他任何底部信号相互验证,也就无关紧要了。
    \item 十字线 6。与十字线 5 相同。
    \item 十字线 7。本十字线充分说明了蜡烛线和蜡烛图形态必须从它们出现之前的价格变化的大背景上来观察,才能把握分寸。本十字线出现在下降趋势之后。正如分析十字线 2、5 和 6 时曾经交代的,通常情况下,我不会把南方十字线视为底部反转的警告信号。然而,根据它所处的总体市场图像,因为本十字线验证了之前的支撑水平,所以具备更重要的意义。在 B 处曾经出现了一个看涨吞没形态,支撑水平大约在 3680。十字线7的前一个时段是一根锤子线,它告诉我们,市场正在 3680-3682 上下构建底部。这正是 7 处的南方十字线带有超越同侪的额外分量的缘故,虽然它是在下跌行情之后出现的。它验证了看涨吞没形态和锤子线的双重支撑作用。
\end{itemize}

\figures{fig8-16}{纳斯达克综合指数——5 分钟蜡烛线图(十字线)}
\section{三星形态}
三星形态是非常罕见的反转形态。如 \autoref{fig8-18} 所示,三星顶部形态是由三根十字线组成的,它们位于当前行情的新高位置。在研究蜡烛图技术的过程中,在提出某种形态或信号之前,总是遵循一个校验规则:至少需要两个方面的独立信息来源证实同一个说法。

对校验规则来说,三星形态是一个例外。

理想的三星顶部形态由三根十字线组成,中间的十字线高于前一根和后一根十字线。(这让我们回想起西方的头肩形顶部形态,其中头部高于左肩和右肩。)
\figures{fig8-18}{三星顶部形态与三星底部形态}

如 \autoref{fig8-19} 所示,在 1 月 3 日所在的一周里有两根锤子线。这为之后的上涨行情打下了基础。1 月 10 日是一根十字线,它发出信号,提示这轮上涨行情可能已经衰竭。在这根十字线之后,霍尼韦尔公司大部分时间处在横向区间中,并在这个过程中形成了一个三星顶部形态。虽然在三星顶部形态之后股票行情陡然下跌,\important{但是我们应记住,蜡烛图技术不预测价格的变化范围。于是,尽管本形态极大地增加了顶部反转的机会,但它并不预测潜在下跌行情的目标范围}。
\figures{fig8-19}{霍尼韦尔公司(Heneywell)——日蜡烛线图(三星顶部形态)}