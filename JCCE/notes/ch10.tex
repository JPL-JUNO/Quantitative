\chapter{蜡烛图信号的汇聚\label{ch10}}
如果在同一个价格区内汇聚了一群蜡烛线,或者蜡烛图形态,那么此处作为支撑区域或阻挡区域的重要性将被放大,形成重要的市场转折点的可能性将上升。

\autoref{fig10-2} 展示了一群蜡烛图信号汇聚起来,有助于确认支撑水平或阻挡水平。
\begin{itemize}
    \item 一群蜡烛图信号汇聚起来作为支撑水平。12 月 11 日是一根锤子线。尽管这根锤子线带有潜在的看涨意味,但是在出现锤子线的同时打开了一个向下的窗口,使得趋势维持向下。当市场从这根锤子线开始下跌时,下跌过程通过三根带有长下影线的蜡烛线组成的一个系列来完成。这些长下影线在一定程度上抵消了看跌的氛围。1 处的蜡烛线也是一根锤子线,但是与上面讨论的第一根锤子线不同,在之后的两天里,即 12 月 16 日和 17 日,这根锤子线成功地发挥了支撑作用。2 处的两根蜡烛线构成了一个看涨吞没形态。在 3 处,2 月初又出现了一根锤子线。这里是 1 和 2 处形成的支撑水平。4 处的蜻蜓十字线进一步证实了大约 42 美元的支撑水平。
    \item 一群蜡烛图信号汇聚起来作为阻挡水平。在 A 处股票上涨,上涨过程是通过一系列带有长上影线的蜡烛线形成的。因为这群蜡烛线具备更高的高点、更高的低点、更高的收市价,所以短期趋势保持向上。但是,那些长上影线构成了警告信号,多方并没有完全站稳立场。最后那根带有长上影线的蜡烛线出现在 1 月 6 日,这是一根流星线。几天后,在 B 处,股票形成了一个看跌吞没形态。在 C 处,小黑色实体出现在长长的白色实体之后,组成了一个孕线形态。于是,A 处的流星线、B 处的看跌吞没形态、C 处的孕线形态,三者汇聚起来,强调了位于 47.50-48 美元的天花板。
\end{itemize}

\figures{fig10-2}{百富门公司(Brown Forman)——日蜡烛线图(蜡烛图信号的汇聚)}

蜡烛图为图形分析提供了十分有效的工具。这是因为我们可以运用蜡烛图很便捷地观察图形线索,评估市场的健康状态,识别不健康的市场状态。只要简单地看一眼某根蜡烛线的形状,就能立即看出当前的需求或供给状况。