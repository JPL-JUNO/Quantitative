\chapter{蜡烛图与趋势线\label{ch11}}
如 \autoref{fig11-1} 所示,是一条向上倾斜的支撑线。至少需要两个向上反弹的低点才能连接出这样一条直线,如果通过三个或者更多向上反弹的低点,那就更好。在蜡烛图上绘制上升的支撑线时,把蜡烛线下影线的低点作为连接点。这根支撑线表明,在这段时间里,买方比卖方更为主动、积极,因为在\notes{逐渐提升的新低点}处,还能够引来新的需求。一般说来,这根线标志着市场上买方多于卖方。既然每一笔交易都同时需要一位买方和一位卖方,我更愿意认为,不是买方比卖方多,而是买方比卖方更为积极进取。

如 \autoref{fig11-2} 所示,是一根向下倾斜的支撑线。正如在讨论 \autoref{fig11-1} 时所说的,传统的支撑线是通过连接越来越高的低点得来的。不过, \autoref{fig11-2} 中的支撑线连接的则是越来越低的低点。下降的支撑线之所以有用武之地,是因为在市场上发生了许多实例,其价格是从下降的直线处向上反弹的。在缺少其他关于支撑水平的线索时,这样的直线给我们提供了潜在的支撑区域。\tips{在什么样的情形下不存在明显的支撑水平呢?当市场为当前行情创新低,特别是创纪录的新低的时候}。

\figures{fig11-1}{上升(向上倾斜)的支撑线}
\figures{fig11-2}{下降(向下倾斜)的支撑线}

常规的上升支撑线因为向上倾斜,被视为具有看涨意义。下降的支撑线因为市场正在创造更低的低点,可被当作具有看跌意义的支撑线。如此一来,从这类支撑线上引发的向上反弹可能只是有限幅度的、不持久的。虽然如此,它可能构成了考虑买进的区域,特别是在若干技术指标在这类直线上汇聚起来的时候。

在 \autoref{fig11-4} 中,整个 1 月,根据图中一系列更低的低点来评估,亚马逊始终处在下降趋势中。连接低点 L1 和 L2,提供了一条尝试性的支撑线。在 L3 处,当市场防守成功后,这条下降的支撑线的重要性得到了确认。于 L4 的低点处,市场对这条向下倾斜的支撑线试探成功了,并且形成了一个看涨的刺透形态。从本刺透形态开始的上冲行情在 2 月 2 日和 3 日之间打开了一个向上的窗口。一方面,在 2 月 9 日长长的白色蜡烛线之前,窗口的底边作为支撑水平保持完好;另一方面,在 2 月 9 日长长的白色蜡烛线之后,当前上冲行情遭遇了一根十字线(它也是一根流星线),被短路了。

\figures{fig11-4}{亚马逊——日蜡烛线图(下降的支撑线)}

\autoref{fig11-6} 展示了一条典型的下降的阻挡线。至少需要两个向下反弹的高点才能连接出这样一条直线。当然,如果有三个或更多个高点,直线就更有影响力。它表示卖方比买方更为积极进取,因为卖方愿意在更低的高点上卖出。这条阻挡线表明,在这段时间中,卖方比买方更为大胆、积极,因为在逐渐降低的新高点处,依然吸引了卖方的卖出意愿。这根直线反映出市场正处于下降趋势中。在蜡烛图上绘制阻挡线时,方法是连接蜡烛线上影线的顶点。

\figures{fig11-6}{下降的阻挡线}

常规的阻挡线是由一系列越来越低的高点连接而成的。但是,如果市场正处在历史的新高位置,不存在更早的高点可用来连成潜在的阻挡线,那怎么办呢?在这种情况下,我常常绘制上升的阻挡线。如 \autoref{fig11-7} 所示,这是连接一系列更高的高点得来的(不同于下降的阻挡线通过连接更低的高点得来)。

\figures{fig11-7}{上升的阻挡线}

在 \autoref{fig11-8} 中,我们连接区域 1、2 和 3,得到了一条经典的阻挡线。白银从区域 2 的孕线形态开始形成一轮下降行情,下降行情的底部形成了两根大风大浪蜡烛线。这是第一个征兆,表明市场向下的力量正在消散。在 3 月 6 日和 7 日之间,打开了一个向上的窗口,驱使趋势转而向上。窗口立即转化为支撑区域,随后几天的变化证明了这一点。

现在,蜡烛图上已经出现了反转信号,我们可以转向西方技术分析——这条阻挡线,来寻求价格目标。这构成了潜在的阻挡区域,3 月 13 日所在的一周,市场陷于停顿,正说明了这一点。本图强调了我们离不开西方技术分析工具的原因,哪怕我们的注意力依然主要放在蜡烛图分析上。蜡烛图能够给出早期的反转信号,而西方工具能够提供价格目标和止损区域。

\figures{fig11-8}{白银——日蜡烛线图(下降的阻挡线)}

\section{破低反涨形态与破高反跌形态}
如 \autoref{fig11-11} 所示,破低反涨形态发生在横向波动区间的支撑区域,起先市场向下突破了支撑区域,后来返回到曾经被跌破的支撑区域之上。换句话说,新低价格水平是不能维持的。一旦破低反涨形态形成,我们就能获得一个止损退出的区域,还有一个价格目标。如 \autoref{fig11-11} 所示,如果某支撑区域最近被向下突破,但市场不能维持,很快回升到支撑区域之上,就可以考虑买进。如果市场坚挺,就不应当跌回最近的低点。最近的低点可以作为止损水平(最好以收市价为标准)。该破低反涨形态的价格目标要么是形态出现之前的行情高点,要么是之前横向交易区间的顶边。这一点将在本节后面的一些示例中进行说明。

\figures{fig11-11}{破低反涨形态}

如 \autoref{fig11-12} 所示的为破高反跌形态。这种形态发生在横向区间的阻挡水平,市场起先向上突破了阻挡水平,但是多方无力维持新高价位。这其实是假突破现象换了一种说法。为了利用破高反跌形态来交易,当市场从先前的阻挡水平之上回落到其下方时,可以考虑卖出。如果市场果真疲软,它就不应该再涨回到最近的高点。下方的价格目标是市场最近的新低,或者是横向交易区间的底边。

虽然交易量并不是这里的主要议题,但是在破低反涨形态中,如果向下突破支撑水平的时候交易量较轻,随后向上反弹至最近跌破的支撑水平之上时交易量较重,就进一步增强了本形态的看涨意义。与之类似,在破高反跌形态中,如果向上突破阻挡水平时交易量较轻,随后回落至最近向上突破的阻挡水平之下时交易量较重,那么破高反跌形态成功的可能性也将进一步增加。

\figures{fig11-12}{破高反跌形态}

为什么破低反涨形态与破高反跌形态具有如此神奇的效用?要回答这个问题,就得谈到拿破仑的一段话。有人问他,他认为什么样的军队是最好的军队。他的回答简明扼要,“获胜的军队”。我们不妨把市场看作两支部队——牛方和熊方——拼杀的战场。当市场处于水平的交易区间时,双方拼力争夺的地盘特别明确,就是这块水平区间。其上方的水平阻挡线是熊方必守的最后防线,下方的水平支撑线是牛方必守的最后防线。如果空头不能守住在向下突破支撑水平后所创的新低价位,或者多头不能维护在向上突破阻挡水平后所创的新高价位,这一方就不能取得胜利。

有时候,交战的一方,比如大户交易商、商业账户经理,甚至可能是自营交易商,会派出小股的“侦察兵”(这是我的说法,不是蜡烛图的术语),前去试探对方部队的决心。举例来说,牛方可能向上推一推,企图使价格上升到一条阻挡线之上。在这样的交火中,我们就得密切关注牛方表现出的坚定程度。如果牛方这支侦察部队能够在敌方的土地上安营扎寨(也就是说,在数日内,市场的收市价都处于该阻挡线的上方),那么牛方的向上突破就成功了。牛方的新生力军将要增援这支先头部队,市场就将向上运动。只要这块滩头阵地掌握在牛方的手中(就是说,市场已经把这个旧的阻挡区转化为新的支撑区,并维持其支撑作用),那么牛方的部队就会控制着市场的局势。然而,一旦市场被打退到先前已经被向上突破的阻挡区域之下,那么牛方便失去了控制权。

在 \autoref{fig11-14} 中,显示了这种“火力侦察兵”现象,让我们从这个角度来观察这里的破高反跌形态。9 月底形成了一个阻挡区域,图中用两根水平线来做了标记。10 月 16 日是一根拉长的白色蜡烛线,将股价推升到了阻挡区域之上。因为瞻博网络以收市价创新高,向上突破了这块明显的阻挡区,所以多方的“火力侦察兵”至少此时已经夺得了一块立足地。后一天,是一根黑色蜡烛线,其收市价证明牛方底气不足,不能守住新高价位,于是整个市场的基调为之一变。如此一来,破高反跌形态就形成了。破高反跌形态的高点在下一周发挥了阻挡作用,在 10 月 23 日完成看跌吞没形态之后,市场崩溃了。

我们为这个破高反跌形态设定的价格目标是前一个低点,即之前带领我们到达破高反跌形态的上涨行情的起始点。有人可能选择区域1,也有人可能选择区域 2,这存在一定的主观性。我会选择 1 作为保守的价格目标,2 作为备用目标。这个破高反跌形态案例还说明了,在采用这种技术手段的时候,并不需要清晰定义的阻挡水平。

\figures{fig11-14}{瞻博网络公司——日蜡烛线图(破高反跌形态)}

\section{极性转换原则}
日本人有句谚语:“大红的漆盘无须另外的装饰。”这种“简单的就是美好的”的概念,道破了市场技术分析理论的真谛。在蜡烛图分析的实践中,我常常对这一原则身体力行。这一原则既简单明白,又强大有力——过去的支撑水平演化为新的阻挡水平,过去的阻挡水平演化为新的支撑水平。这就是我所说的“极性转换原则”。如 \autoref{fig11-18} 所示,就是过去的支撑水平转化为阻挡水平的情形。如 \autoref{fig11-19} 所示,是过去的阻挡水平转化为新的支撑水平的情形。

\figures{fig11-18}{极性转换原则:过去的支撑水平转化为新的阻挡水平}

\figures{fig11-19}{极性转换原则:过去的阻挡水平转化为新的支撑水平}

在 \autoref{fig11-20} 中,12 月下旬,一场陡峭的抛售行情结束于 5.35 美元的水平(在 A 处)。当市场再一次向下试探这个水平的时候,至少有三类市场参与者可能要考虑买进。

第一群市场参与者可能是那些在 12 月下旬的抛售行情中一直等待市场稳定下来的人。现在,他们发现市场在这里受到了支撑,于是得到了一个入市参考点——5.35 美元的水平(点 A 所示的低点)。几天以后,该支撑水平成功地经受住市场的试探(点 B 处),在这个过程中,很可能市场已经吸引了新的多头加盟。

第二群市场参与者可能是那些原来持有多头头寸,但是在 12 月下旬的抛售行情中被止损平仓的人。在这些被止损出市的老多头中,当他们看到 1 月中旬从点 B 到 5.60 美元的上涨行情时,可能有一部分人会觉得当初判断白银市场为牛市是正确的,只不过买进的时机没有选择好。现在,是买进的时候了。他们希望借此机会,证明自己当初的看法是有道理的。于是,等到市场再度向下回落到点 C 的时候,他们便重新买进,建立多头头寸。

第三群市场参与者可能是那些曾经在 A 处和 B 处买进的人。他们也注意到了从 B 开始的上涨行情,因此,如果有“合适的”价位,他们就可能为已有的头寸加码。在 C 处,市场返回了支撑水平,他们自然就得到了一个合适的价位。于是在 C 处,又出现了更多的买进者。依此类推,当市场再度向下撤回到 D 处的时候,自然还能吸引更多的看多者入市。

但是不久,问题就开始了。在 2 月下旬,价格向下突破了 A、B、C 和 D 处形成的支撑水平。2 月 28 日是一根锤子线,有理由感到一点乐观。但是,曾经在这些旧的支撑区域买进的人,现在无一例外地处于亏损状态。请您自问一下,在您的市场图表上,什么样的价格最重要?是当前趋势的最高价吗?是当前趋势的最低价吗?还是昨日的收市价?都不是。\textbf{在任何图表上,最重要的价格是您开立头寸时的交易水平。人们与自己曾经买进或卖出过的价格水平结下了强烈的、切身的、情绪化的不解之缘。}那些在 5.35 美元支撑区买进的多头现在“临时抱佛脚”“病急乱投医”,一心祈祷市场回到他们的盈亏平衡点。

于是,一旦市场上冲到这些多头者买进的区域(在 5.35 美元附近)附近,他们谢天谢地,赶紧乘机平回手上的多头头寸。这么一来,当初在 A、B、C、D 处买进的市场参与者,也许现在就变成了卖出者。这一点,正是过去的支撑水平演化成新的阻挡水平的主要缘由,如图上 E 和 F 处所示。

\figures{fig11-20}{白银——日蜡烛线图(极性转换原则)}