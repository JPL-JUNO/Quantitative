\chapter{反转形态}
技术分析师瞪大眼睛盯着价格的涨落,为的是及早发现市场心理变化和趋势变化的警告信号。反转价格形态就是这样的技术线索。在西方技术分析理论中,反转信号包括双重顶、双重底、反转日、头肩形、岛形反转顶和岛形反转底等各种价格形态。

然而,\important{从一定意义上来说,“反转形态”这一术语的用词是不准确的。听到反转形态这个术语,往往使人误以为现有趋势将会突然结束,立即逆转为反向新趋势。实际上,这种情况很少发生。趋势的逆转,一般都是伴随着市场心理的逐渐改变进行的,通常需要经过一个缓慢的、分阶段的演变过程。}

确切地说,趋势反转信号的出现,意味着之前的市场趋势可能发生变化,但是市场并不一定就此逆转到相反的方向。弄清楚这一点,是至关重要的。我们不妨用行驶中的汽车来打个比方。一场上升趋势就相当于一辆前进的汽车。汽车的刹车灯亮了,随后汽车停了下来。刹车灯相当于趋势反转信号,它表明先前的趋势(相当于汽车向前行驶)即将终止。现在,汽车静止不动,那么下一步,司机是打算掉头向反方向行驶,还是停在那里不动,抑或是继续向前行驶呢?如果没有更多的线索,我们根本无从知晓。

\begin{tcolorbox}
    请务必留意,当我说“反转形态”的时候,这个术语仅仅意味着之前的趋势将发生变化,但是未必一定会反转。把反转形态理解成趋势变化形态,才是慎重可取的考虑。
\end{tcolorbox}

反转形态,就是市场以其特有的方式为我们提供的一个指路牌,牌子上写着:“当心——趋势正在发生变化。”也就是说,市场的心理状态正在发生变化。那么,为了适应这种新的市场环境,我们就应当及时调整自己的交易方式。当反转形态出现时,如何建立新头寸,如何了结旧头寸,存在多种多样的选择。

这里有一条重要原则:只有当反转信号所指的方向与市场的主要趋势方向一致时,我们才能依据这个反转信号来开立新头寸。举个例子,假定在牛市的发展过程中,出现了一个顶部反转形态。虽然这是一个看跌的信号,却不能保证卖出做空是有把握的。这是因为,市场当前的主要趋势依然是上升的。无论如何,就这个反转形态的实质意义来说,它构成了了结既有多头头寸的交易信号。之后,我们将在调整过程中寻求看涨信号买进做多,因为当前主要趋势向上。
\section{伞形线}
如 \autoref{fig4-4} 所示的蜡烛线具有明显特点。它们的下影线较长,而实体(或黑或白)较小,并且在其全天价格区间里,实体所处的位置接近顶端。\autoref{fig4-4} 所列蜡烛线被称为\textbf{伞形线},因为它们的轮廓呈伞状,长长的下影线如伞柄,小实体如伞面。在本图中,我们同时列出了黑白两种蜡烛线。有趣的是,这两种蜡烛线都既可能是看涨的,也可能是看跌的,具体情况要由它们在趋势中所处的位置来决定。

\figures{fig4-4}{伞形线}

伞形线,不管是哪一种,只要它出现在下降趋势中,那么,它就是下降趋势即将结束的信号。在这种情况下,这种蜡烛线称为锤子线,意思是说“市场正用锤子夯实底部”,如 \autoref{fig4-5} 所示。

\figures{fig4-5}{锤子线}

如上所述,伞形线的性质取决于伞形线出现之前的主流趋势方向,随趋势而变。在下跌行情之后出现的伞形线是看涨信号,称为锤子线。在 \autoref{fig4-4} 所示的两种蜡烛线中,无论哪一种,如果出现在上冲行情之后,那么,它就表明之前的市场运动也许已经结束。显而易见,这类蜡烛线就称为\textbf{上吊线}。

形状相同的蜡烛线,有时是看涨的,有时又是看跌的,看起来或许有些不合常情。但是,如果熟悉西方技术分析理论中的岛形顶和岛形底,那就不难看出,在这个问题上,东西方的思路如出一辙。对岛形反转形态来说,既可以是看涨的,也可以是看跌的,\tips{这取决于它在市场趋势中所处的位置}。如果岛形反转形态出现在长期的上升趋势之后,则构成看跌信号;如果岛形反转形态出现在下降趋势之后,则构成看涨信号。

一般根据三个方面的标准来识别锤子线和上吊线。
\begin{enumerate}
    \item 实体处于整个价格区间的上端,但实体本身的颜色无所谓。
    \item 带有长长的下影线,且下影线的长度至少达到实体高度的两倍。
    \item 在这类蜡烛线中,应当没有上影线;即使有上影线,长度也是极短的。
\end{enumerate}

可以从三个方面来区分上吊线和锤子线——趋势、该蜡烛线出现之前行情运动的区间、验证信号。具体说明如下:
\begin{itemize}
    \item 趋势:锤子线只能紧接在下跌行情后出现;上吊线必须紧接在上涨行情后出现。
    \item 蜡烛线出现之前行情运动的区间:锤子线之前即便只有短线下跌也是有效的;但是,上吊线之前必须具备长足的上冲行情,最好是创下行情全历史新高后,方为有效。
    \item 验证信号:上吊线出现后得到验证方为有效;对锤子线则无此要求。
\end{itemize}

在看涨的锤子线的情况下,或者在看跌的上吊线的情况下,其下影线越长、上影线越短、实体越小,那么,这类蜡烛线就越有意义。
\subsection{锤子线}
锤子线的实体既可以是白色的,也可以是黑色的。从 \autoref{fig4-5} 可见,即使锤子线的实体是黑色的,其收市价也接近本时段最高价。但是,如果锤子线的实体是白色的,其看涨的意义则更坚挺几分(因为其收市于最高价)。如果锤子线的实体是白色的,日本人称之为“力量线”。