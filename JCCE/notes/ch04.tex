\chapter{反转形态}
技术分析师瞪大眼睛盯着价格的涨落,为的是及早发现市场心理变化和趋势变化的警告信号。反转价格形态就是这样的技术线索。在西方技术分析理论中,反转信号包括双重顶、双重底、反转日、头肩形、岛形反转顶和岛形反转底等各种价格形态。

然而,\important{从一定意义上来说,“反转形态”这一术语的用词是不准确的。听到反转形态这个术语,往往使人误以为现有趋势将会突然结束,立即逆转为反向新趋势。实际上,这种情况很少发生。趋势的逆转,一般都是伴随着市场心理的逐渐改变进行的,通常需要经过一个缓慢的、分阶段的演变过程。}

确切地说,趋势反转信号的出现,意味着之前的市场趋势可能发生变化,但是市场并不一定就此逆转到相反的方向。弄清楚这一点,是至关重要的。我们不妨用行驶中的汽车来打个比方。一场上升趋势就相当于一辆前进的汽车。汽车的刹车灯亮了,随后汽车停了下来。刹车灯相当于趋势反转信号,它表明先前的趋势(相当于汽车向前行驶)即将终止。现在,汽车静止不动,那么下一步,司机是打算掉头向反方向行驶,还是停在那里不动,抑或是继续向前行驶呢?如果没有更多的线索,我们根本无从知晓。

\begin{tcolorbox}
    请务必留意,当我说“反转形态”的时候,这个术语仅仅意味着之前的趋势将发生变化,但是未必一定会反转。把反转形态理解成趋势变化形态,才是慎重可取的考虑。
\end{tcolorbox}

反转形态,就是市场以其特有的方式为我们提供的一个指路牌,牌子上写着:“当心——趋势正在发生变化。”也就是说,市场的心理状态正在发生变化。那么,为了适应这种新的市场环境,我们就应当及时调整自己的交易方式。当反转形态出现时,如何建立新头寸,如何了结旧头寸,存在多种多样的选择。

这里有一条重要原则:只有当反转信号所指的方向与市场的主要趋势方向一致时,我们才能依据这个反转信号来开立新头寸。举个例子,假定在牛市的发展过程中,出现了一个顶部反转形态。虽然这是一个看跌的信号,却不能保证卖出做空是有把握的。这是因为,市场当前的主要趋势依然是上升的。无论如何,就这个反转形态的实质意义来说,它构成了了结既有多头头寸的交易信号。之后,我们将在调整过程中寻求看涨信号买进做多,因为当前主要趋势向上。
\section{伞形线}
如 \autoref{fig4-4} 所示的蜡烛线具有明显特点。它们的下影线较长,而实体(或黑或白)较小,并且在其全天价格区间里,实体所处的位置接近顶端。\autoref{fig4-4} 所列蜡烛线被称为\textbf{伞形线},因为它们的轮廓呈伞状,长长的下影线如伞柄,小实体如伞面。在本图中,我们同时列出了黑白两种蜡烛线。有趣的是,这两种蜡烛线都既可能是看涨的,也可能是看跌的,具体情况要由它们在趋势中所处的位置来决定。

\figures{fig4-4}{伞形线}

伞形线,不管是哪一种,只要它出现在下降趋势中,那么,它就是下降趋势即将结束的信号。在这种情况下,这种蜡烛线称为锤子线,意思是说“市场正用锤子夯实底部”,如 \autoref{fig4-5} 所示。

\figures{fig4-5}{锤子线}

如上所述,伞形线的性质取决于伞形线出现之前的主流趋势方向,随趋势而变。在下跌行情之后出现的伞形线是看涨信号,称为锤子线。在 \autoref{fig4-4} 所示的两种蜡烛线中,无论哪一种,如果出现在上冲行情之后,那么,它就表明之前的市场运动也许已经结束。显而易见,这类蜡烛线就称为\textbf{上吊线}。

形状相同的蜡烛线,有时是看涨的,有时又是看跌的,看起来或许有些不合常情。但是,如果熟悉西方技术分析理论中的岛形顶和岛形底,那就不难看出,在这个问题上,东西方的思路如出一辙。对岛形反转形态来说,既可以是看涨的,也可以是看跌的,\tips{这取决于它在市场趋势中所处的位置}。如果岛形反转形态出现在长期的上升趋势之后,则构成看跌信号;如果岛形反转形态出现在下降趋势之后,则构成看涨信号。

一般根据三个方面的标准来识别锤子线和上吊线。
\begin{enumerate}
    \item 实体处于整个价格区间的上端,但实体本身的颜色无所谓。
    \item 带有长长的下影线,且下影线的长度至少达到实体高度的两倍。
    \item 在这类蜡烛线中,应当没有上影线;即使有上影线,长度也是极短的。
\end{enumerate}

可以从三个方面来区分上吊线和锤子线——趋势、该蜡烛线出现之前行情运动的区间、验证信号。具体说明如下:
\begin{itemize}
    \item 趋势:锤子线只能紧接在下跌行情后出现;上吊线必须紧接在上涨行情后出现。
    \item 蜡烛线出现之前行情运动的区间:锤子线之前即便只有短线下跌也是有效的;但是,上吊线之前必须具备长足的上冲行情,最好是创下行情全历史新高后,方为有效。
    \item 验证信号:上吊线出现后得到验证方为有效;对锤子线则无此要求。
\end{itemize}

在看涨的锤子线的情况下,或者在看跌的上吊线的情况下,其下影线越长、上影线越短、实体越小,那么,这类蜡烛线就越有意义。
\subsection{锤子线}
锤子线的实体既可以是白色的,也可以是黑色的。从 \autoref{fig4-5} 可见,即使锤子线的实体是黑色的,其收市价也接近本时段最高价。但是,如果锤子线的实体是白色的,其看涨的意义则更坚挺几分(因为其收市于最高价)。如果锤子线的实体是白色的,日本人称之为“力量线”。

锤子线带有长长的下影线,且收市于本时段最高价,或接近最高价,意味着在当天的交易过程中,市场起先曾急剧下挫,后来却完全反弹上来,收市在当日的最高价处,或者收市在接近最高价的水平上。这一点本身就具有看涨的味道。锤子线收市价位于最高价或接近最高价,正是其没有上影线,或者上影线很小的原因。这一点是锤子线的判据。反之,如果蜡烛线带有长的上影线,则意味着收市价显著低于本时段最高价。

既然锤子线属于底部反转信号,就需要之前存在下降趋势才谈得上反转。

\figures{fig4-7}{2 月 24 日看到一根锤子线。它很经典,因为其下影线很长,而实体较小,且实体位于当日交易区间的顶端。它是在一轮下跌行情之后出现的。这正是锤子线出现的必要前提。2 月 22 日的蜡烛线不会被定义为锤子线,因为它不符合条件,下影线长度达不到实体长度的 2-3 倍。这类长下影线是必要条件,因为它表示市场在本时段曾经急剧下挫,但在收市时,行情收市价位于或接近最高价,熊方反倒一败涂地。这是一场日本人所形容的“神风反攻”。24 日的经典锤子线清晰地揭示了上述过程。我敢打赌,在出现了这样经典的锤子线之后,空头立场一定会动摇。}

\autoref{fig4-7} 揭示了蜡烛图技术至关重要的一方面特点。如果希望借助蜡烛图来成功地交易,那么不仅需要理解蜡烛图形态本身,还需要在评估交易的风险报偿比的基础上,理解蜡烛图形态所处的相对位置。\important{始终应当先考虑风险报偿比},再根据蜡烛图形态或蜡烛线信号发出交易指令。现在,让我们牢记交易之前应考虑风险报偿比,再重新审视 24 日的理想锤子线。

在该锤子线即将完成时(请记住,我们不得不等待行情收市),股价收市于接近 48 美元处。如果在锤子线完成时买进(48 美元附近),当市场再次下跌到锤子线最低点约 43 美元时止损,则上述两个价格之差便是风险,即大约 5 美元。如果您的赢利目标远大于 5 美元,在锤子线完成时买进并没有什么不对。不过,对某些活跃的短线交易者来说,5 美元或许风险太大了。

于是,为了降低潜在的交易风险,交易者或许打算等待市场回调至锤子线下影线范围内的机会(当然,在锤子线出现后,许多时候市场并不回调)。利用锤子线低点作为潜在的买进点位,就有可能在接近止损的位置上建仓。

设想交易者甲识别出 2 月 24 日的锤子线。当他看到如此精彩的锤子线时过于兴奋,以至于在锤子线的收市价附近,即接近 48 美元处立即买进了。次日,市场开市时向下跳空,开市价为 44.50 美元。交易者甲的头寸现在处在水下 3.5 美元的深处了。或许他最多承受 4.5 美元的亏损,就要采取止损措施,轧平昨日的多头头寸。如果是这样的话,最终甲或许认为这根蜡烛线不起作用。

交易者乙同样意识到锤子线是潜在的反转信号,但是她记住了风险报偿比的前提,没有在锤子线收市时买进(因为对她来说,此处买进的风险太大了)。次日,当市场开市于低价位时,深入锤子线长长下影线的低端,这是潜在的支撑区间。乙认为,股价现在接近支撑水平,决定买进。当股价从支撑位上冲后,乙就会对蜡烛图大唱赞歌。

当然,市场有时守不住潜在的支撑水平或阻挡水平,锤子线也是半斤八两,有时也守不住。可见,对蜡烛图功能的个人发挥,决定了蜡烛图贡献的多寡,运用之妙,存乎一心。

\figures{fig4-8}{锤子线有助于确认支撑水平。}

\autoref{fig4-8} 为纳斯达克 100 指数(NDX)5 分钟的蜡烛线图,从 A 点处开始,发动了一轮上冲行情。在接近 3723 的价位,当位置 1 和 2 处出现了两根小实体后,这是该轮上涨已经是强弩之末的第一个征兆。市场从此处陡然回落。当行情接近潜在支撑区域 3680 左右时,形成了一根锤子线。如果支撑水平牢靠,那么该锤子线就应该成为支撑区域。在随后的两个时段,该锤子线果然发挥了支撑作用。当然,如果纳斯达克 100 指数收市于 3680 的支撑水平以下,上述看涨预期就失效了。\tips{这是技术分析的一个重要方面——应该总是存在某个价位,表明我们是错误的。}在本例的情况下,便是市场低于 3680。
\subsection{上吊线}
上吊线的形状与锤子线相同,唯一区别是,上吊线出现在上涨行情之后。因为长下影线具有增强功效的作用,而上吊线有这样的长下影线。\important{上吊线出现后,一定要等待其他看跌信号的证实,这一点特别重要}。最低限度的验证信号是,之后时段的开市价低于上吊线的实体。但是,我总是\tips{建议以收市价低于上吊线的实体为验证信号。}

为什么要等待之后时段的收市价低于上吊线的实体呢?因为如果次日市场收市在低于上吊线实体的水平,那么凡是在上吊线当日的开市、收市时买进的交易商(大量的交易活动发生在这两个时段),会通通背上亏损的头寸,被“吊”在上面。同样道理,我还希望看到上吊线位于市场全历史最高价,或者至少处于一轮重大行情的最高位。在这种情况下,多头在上吊线之日的入市点位处在历史新高处,这就迫使他们更加紧张。结果,这批多头或许决定撤出他们新开的多头头寸。他们的头寸倾泻而至,加重了卖出压力。

\figures{fig4-10}{微软(Microsoft)——日蜡烛线图(上吊线和锤子线)}

\autoref{fig4-10} 所示的实例颇精彩,从中可以看到,同一种蜡烛线,既可以是看跌的(如 1 月 29 日的上吊线),也可以是看涨的(如 2 月 22 日的锤子线)。尽管在这个实例中,上吊线和锤子线的实体都是白色的,但是它们实体的颜色并没有太大意义。1 月 29 日的蜡烛线是上吊线,因为之前趋势是上冲行情。上吊线给本轮行情创了新高。下一日(2 月 1 日),收市价低于上吊线的实体,这让所有的新多头——在上吊线当日开市和收市时买进的——通通陷入困境。2 月 22 日的蜡烛线是锤子线,因为它出现在下降趋势之后。锤子线之前的一个交易日也是短实体。这是一条更早出现的线索,表明空头的努力遭遇了阻碍。而锤子线成为进一步的看涨证据。

\autoref{fig4-12} 揭示了上吊线出现后等待验证信号的重要性。位于 1、2 和 3 处的蜡烛线都属于上吊线(它们的上影线都很短,足以判定它们属于上吊线)。其中每一根线的收市价都为当前的上涨行情创了新高,因而维持了上升惯性的持续有效。如果要确认这些蜡烛线的看跌意味,即上涨趋势要从上涨转为不那么强势的,就必须看到之后的收市价低于上述上吊线的实体。这种情况没有发生。记住,任何人在上吊线开市和收市时买进,都是买在当前行情的最高点。如果在上吊线之后市场持续创新高(此处正如此),那些多头还会感觉难受吗?当然不会。他们正高兴呢,因为行情比他们买进的点位更高了。因此,正如本例所示,只要市场没有收市于上吊线实体的水平之下,则牛市趋势保持稳固。

\figures{fig4-12}{加贝利资产管理公司(Gabelli Asset Management)——日蜡烛线图(上吊线及其验证信号)}
\section{吞没形态(抱线形态)}
吞没形态属于主要的反转形态,由两根颜色相反的蜡烛线实体构成。

\autoref{fig4-13} 显示的是看涨吞没形态。在本图中,市场本来处于下降趋势之中,但是后来出现了一根坚挺的白色实体,这根白色实体将它前面的那根黑色实体“抱进怀里了”,或者说把它吞没了(该形态正由此得名)。它有个绰号叫“抱线形态”,道理一目了然。(这种情形说明市场上买进压力已经压倒了卖出压力。

\figures{fig4-13}{看涨吞没形态}

\autoref{fig4-14} 是看跌吞没形态的示意图。在本图中,市场原本正向着更高的价位趋升。但是,当前一个白色实体被后一个黑色实体吞没后,就构成了顶部反转信号。这种情形说明,市场上供给压倒了需求。

\figures{fig4-14}{看跌吞没形态}

关于吞没形态,我们有三条判别标准:
\begin{enumerate}
    \item 在吞没形态之前,市场必须处在明确的上升趋势(看跌吞没形态)或下降趋势(看涨吞没形态)中,哪怕这个趋势只是短期的。
    \item 吞没形态由两条蜡烛线组成。其中第二根蜡烛线的实体必须覆盖第一根蜡烛线的实体(但是不一定需要吞没前者的上下影线)。
    \item 吞没形态的第二个实体应与第一个实体的颜色相反。(这一条标准有例外的情况,条件是,吞没形态的第一条蜡烛线是一根十字线。如此一来,如果在长时间的下降趋势之后,一个小小的十字线被一个巨大的白色实体所吞没,就可能构成底部反转形态。反之,在上升趋势中,如果一个十字线被一个巨大的黑色实体所吞没,就可能构成顶部反转形态)。
\end{enumerate}

下面列出了一些参考要素,如果吞没形态具有这样的特征,那么它构成重要反转信号的可能性将大大增强:
\begin{itemize}
    \item 如果在吞没形态中,第一天的实体非常小(即纺锤线),而第二天的实体非常大。第一天蜡烛线的小实体反映出原有趋势的驱动力正在消退,而第二天蜡烛线的长实体证明新趋势的潜在力量正在壮大。
    \item 如果吞没形态出现在超长延伸的或非常快速的市场运动之后。如果存在非常快速的或超长程的行情运动,则导致市场朝一个方向走得太远(要么超买,要么超卖),容易遭受获利平仓头寸的打击。
    \item 如果在吞没形态中,第二个实体伴有超额的交易量。
\end{itemize}

吞没形态的一种主要作用是构成支撑水平或阻挡水平。我们借助 \autoref{fig4-15}。如 \autoref{fig4-15} 所示,在组成看跌吞没形态的两根蜡烛线中,选取其最高点。该高点构成了阻挡水平(以收市价来观察突破与否)。把同样的概念运用到看涨吞没形态。该形态的最低点构成了支撑水平。
\figures{fig4-15}{看跌吞没形态的阻挡水平(压力线)}

将吞没形态看作阻挡水平和支撑水平,是一个很有用的技巧,尤其当市场离开低点过远时(看涨吞没形态)或离开高点过远时(看跌吞没形态),可以借之选择更舒适的卖出或买进点。举例来说,等到看涨吞没形态完成时(请记住,我们需要一直等到本时段收市时才能确认当前蜡烛线组合属于看涨吞没形态),市场可能已经远离之前的低点了。因此,我会觉得行情已经离开了有吸引力的买进区域。在这种情况下,我们可以等待市场调整,它可能再次进入看涨吞没形态低点附近的支撑区域,然后,再考虑入市做多。在看跌吞没形态的情况下,同样的道理,而方向相反。

如 \autoref{fig4-17} 所示,连续 6 根黑色实体依次下降,之后,5 月 5 日早些时候首次出现了一根白色蜡烛线。这根白色蜡烛线完成了图示的看涨吞没形态。于是,我们利用该看涨吞没形态的低点(通过比较,选择组成该形态的两条蜡烛线的最低点)作为支撑水平,大约位于 56 美元。朗讯从该看涨吞没形态起上冲,直至遭遇一个看跌吞没形态的阻击。请注意,该看跌吞没形态的高点转化为下一时段的阻挡水平。在从该看跌吞没形态开始的下降行情中,一根十字线提供了试探性线索,显示该股票正在努力站稳脚跟,其位置便在之前的看涨吞没形态构成的支撑水平附近。在对该看涨吞没形态形成的支撑水平试探成功后,股票从此处上冲;当面临之前的看跌吞没形态所形成的阻挡水平时,市场经过了少数几个时段的犹豫;最终沿着一条向上倾斜的支撑线持续攀升。

\figures{fig4-17}{朗讯科技公司——60 分钟蜡烛线图(看涨吞没形态和看跌吞没形态)}
\subsection{设置保护性止损指令的重要性}
在初始交易时,就应该同时设定止损保护指令,此时交易者的态度最具有客观性。仅当市场表现符合预期时,方可持有头寸不动。如果后续价格变化与原本的预期相左,或者未能验证原本的预期,就是轧平头寸的退出时机。如果市场运动与所持头寸方向相反,您或许会盘算,“干吗设置止损指令呢?只是短期行情运动,暂时对我不利而已”。这么一来,您一边一厢情愿地期望市场转向对自己有利的方向,一边顽固地坚持原有头寸不动。请记住两点事实:
\begin{enumerate}
    \item 所有的长期趋势都是从短线运动开始的。
    \item 市场根本不给任何主观愿望留有余地。市场总是自顾自演变,绝不考虑你的想法或你的头寸。
\end{enumerate}
\section{乌云盖顶形态\label{sec4-3}}
这种形态(见 \autoref{fig4-22})也是由两根蜡烛线组成的,属于顶部反转形态。它一般出现在上升趋势之后,有些情况下也可能出现在水平调整区间的顶部。在这一形态中,第一天是一根坚挺的白色实体;第二天的开市价超过了第一天的最高价(就是说超过了第一天上影线的顶端),但是到了第二天收市的时候,市场却收市在接近当日最低价的水平,并且明显地向下扎入第一天白色实体的内部。第二天的黑色实体向下穿进第一天的白色实体的程度越深,则该形态构成顶部反转形态的可能性就越大。有些日本技术分析师要求,第二天黑色实体的收市价必须向下穿过前一天白色实体的 50\%。如果黑色实体的收市价没有向下穿过白色蜡烛线的中点,那么,当这类乌云盖顶形态发生后,或许我们最好等一等,看看是否还有进一步的看跌验证信号。

\figures{fig4-22}{这种看跌形态背后的道理很容易理解。在形态发生之前,市场本来处于上升趋势中。有一天出现了一根坚挺的白色蜡烛线,第二天市场在开市时便向上跳空。到此刻为止,多头完全掌握着主动权。然而,此后市场的技术景象却完全改变了!事实上,市场收市在当日的最低价处,或者在最低价附近,并且这个收市价明显地向下扎进了前一天的实体内部,消除了第一天取得的大部分进展。在这种情况下,多头头寸的持有者信心开始动摇。还有一些人一直在找机会卖出做空,那么现在他们就得到了一个设置止损指令的参考水平——乌云盖顶形态的第二日形成的新高价格水平。}

下面列出了一些参考因素,如果乌云盖顶形态具有这样的特征,则有助于增强其技术分量:
\begin{itemize}
    \item 在乌云盖顶形态中,黑色实体的收市价向下穿入前一个白色实体的程度越深,则该形态构成市场顶部的机会越大(如果黑色实体覆盖了前一天的整个白色实体,那就形成了看跌吞没形态,而不是乌云盖顶形态)。从这一点上说,作为顶部反转信号,看跌吞没形态比乌云盖顶形态具有更重要的技术意义。\important{如果在乌云盖顶形态之后,或者在看跌吞没形态之后,出现了一根长长的白色实体,而且其收市价超过了这两种形态的最高价,那么这可能预示着新一轮上冲行情的到来}。
    \item 如果乌云盖顶形态发生在一个超长程的上升趋势中,它的第一天是一根坚挺的白色实体,其开市价就是最低价(就是说,是光脚的),而且其收市价就是最高价(就是说,是光头的);它的第二天是一根长长的黑色实体,其开市价位于最高价,而且收市价位于最低价(这是一个光头光脚的黑色蜡烛线)。
    \item 在乌云盖顶形态中,如果第二个实体(即黑色实体)的开市价高于某个主要的阻挡水平,但是市场未能成功地坚守住,那么可能证明多头已经无力控制市场了。
    \item 如果在第二天开市的时候,交易量非常大,那么这里可能发生“胀爆”现象。具体说来,当日开市价创出了新高,而且开市时的成交量极大,可能意味着很多新买家终于下决心入市,跳上了牛市的“船”。随后,市场却发生了抛售行情。那么,很可能用不了太久,这群人数众多的新多头(还有那些早已在上升趋势中坐了轿子的老多头)就会认识到,他们上的这条船原来是“泰坦尼克号”。对期货交易商来说,极高的持仓量也是一种警告信号。
\end{itemize}

正如看跌吞没形态可以被视为阻挡水平,组成乌云盖顶形态的两个时段里的最高点也构成了阻挡水平。\autoref{fig4-22} 对此做了说明。

在 \autoref{fig4-25} 中,在 1 和 2 处接连出现了两例看涨吞没形态,突出地显示 3250-3275 区域具有稳固的支撑作用。一轮上冲行情从第二个看涨吞没形态开始,遇到乌云盖顶形态后开始犹豫起来。紧接着,一根白色实体稍稍向上穿越了该乌云盖顶形态的阻挡水平(图上用水平直线做了标记)。虽然这个突破信号算不上决定性的,但是它以收市价向上突破了阻挡水平,因此构成了积极信号。

一定要根据市场条件随机应变,\autoref{fig4-25} 充分说明了这一点的重要性。具体说来,虽然向上突破第一个乌云盖顶形态的阻挡水平为当前趋势开创了更高的高点,但是在下一个时段,我们对市场的观感从本来的积极转为更加谨慎。为什么?因为就在突破信号的下一个时段,出现了一根黑色蜡烛线,完成了另一个乌云盖顶形态。这根黑色蜡烛线反映了牛方没有能力坚守新高的阵地。

\figures{fig4-25}{纳斯达克综合指数——60 分钟蜡烛线图(乌云盖顶形态)}

\section{刺透形态}
对大多数蜡烛图形态来说,有一个看跌形态,就有一个相反的看涨形态。乌云盖顶形态便是这样的。有一个与乌云盖顶形态相反的形态,它的名称为\textbf{刺透形态}(如 \autoref{fig4-26} 所示)。刺透形态出现在下跌的市场上,也是由两根蜡烛线所组成的。其中第一根蜡烛线具有黑色实体,而第二根蜡烛线则具有白色实体。在白色蜡烛线这一天,市场的开市价曾跌到低位,最好下跌至前一个黑色蜡烛线的最低价之下,但是后来市场又将价格推升回来,深深刺入黑色蜡烛线的实体内部。

\figures{fig4-26}{刺透形态}

刺透形态与看涨吞没形态同属于一个家族。在看涨吞没形态中,白色实体吞没了前面的那条黑色实体。而在看涨的刺透形态中,白色实体仅仅向上刺入了前一个黑色实体的内部,但是未能完全覆盖。在刺透形态中,白色实体向上刺入黑色实体的程度越大,该形态构成底部反转信号的可能性就越大。在理想的刺透形态中,白色实体必须向上穿入前一个黑色实体的中点水平以上。关于刺透形态背后的心理过程,可以做如下理解:市场本来处于下降趋势中,刺透形态第一天疲弱的黑色实体加强了这种市场预期。第二天,市场以向下跳空的形式开市。到此刻为止,空头观察着行情的发展,感觉诸事顺遂。可是到当日收市的时候,市场却涨了回去,结果收市价不仅完全回到了前一天的水平,而且变本加厉地向上大大超越了这个水平。现在,熊方开始对手上的空头头寸忐忑不安起来。有些市场参与者一直在寻找买进的机会,他们据此推断,市场不能维持这个新低价位,或许这正是入市做多的大好时机。

关于刺透形态,也有四项参考因素,如果刺透形态兼具这些特征,那么它的技术分量将大为增强。这四项参考因素与 \autoref{sec4-3} \nameref{sec4-3} 的四项参考因素内容相同,而方向相反。

在讲述乌云盖顶中然我们更愿意看到黑色实体的收市价向下穿过了前一个白色实体的中点,但是在这一条判别准则上,还是有一定的灵活余地的。然而在刺透形态中,却没有什么灵活的余地。在刺透形态中,白色蜡烛线的实体必须向上推进到黑色蜡烛线实体的中点以上。之所以看涨的刺透形态不如乌云盖顶形态灵活,是因为日本人认为处理底部反转形态必须更加谨慎,他们对形状近似的价格形态做了进一步的区分,将它们分为三种情况,分别称为\textbf{待入线形态、切入线形态、插入线形态}。这三种形态与刺透形态在基本构造上是相似的。它们之间的区别在于,其中的白色蜡烛线实体向上穿入黑色蜡烛线实体的程度是不同的。在待入线形态中,白色蜡烛线(其外形通常是比较小的)的收市价位于前一个蜡烛线的最低价下方的附近。在切入线形态中,白色蜡烛线(它也应当是较小的白色蜡烛线)的收市价稍稍进入前一个黑色实体的范围之内。在插入线形态中,白色蜡烛线比上述两个形态的更长一些,其收市价也更多地刺入前一个黑色实体之内,但是没有达到黑色实体中点的水平。

能不能记住刺透形态的近似形态的各种具体形态并不重要。\notes{只要记住,白色实体必须推升到黑色实体的中点以上,才能构成较为有效的底部反转信号}。

\figures{fig4-31}{美国通用保险(American General)——日蜡烛线图(刺透形态)}

在 \autoref{fig4-31} 上,3 月中旬有一个看涨吞没形态,由此开始形成一轮上冲行情;3 月 24 日接近 59 美元的纺锤线显示了这轮行情可能遇到麻烦的迹象。4 月 3 日,收市价超越 59 美元,全天形成一根长白色实体,标志着多头回来了,至少到本时段为止重掌大局。在 4 月 4 日这一天,该股票形成了一个乌云盖顶形态的变体。之所以说它是乌云盖顶形态的变体,是因为在通常的情况下,我们希望该形态第二天的开市价高于第一天的最高价,而在本例中,第二天的开市价只是高于第一天的收市价。虽然如此,考虑到 4 月 4 日黑色蜡烛线向下回落到白色实体的内部如此深入的地步,增加了该形态的技术效力,该形态可能比中规中矩的乌云盖顶形态更有效。

4 月 17 日和 18 日组成的刺透形态引出了之后的上冲行情。从这个刺透形态开始的上冲行情一直持续到 4 月 24 日和 25 日组成的另一个乌云盖顶形态。这个乌云盖顶形态也可以看成经典乌云盖顶形态的变体。为什么?因为其中的黑色实体没有向下跌破白色实体的中线。无独有偶,虽然这个蜡烛线组合不属于常规的乌云盖顶形态,但是它具备两个要素,让我相信它的看跌性质与传统的乌云盖顶形态不相上下。具体来说:(1)4 月 25 日的黑色实体开市价急剧上涨,高于前一天的最高价,在这个基础上,其收市价低于前一日的收市价;(2)该形态验证了 4 月上旬乌云盖顶形态形成的阻挡水平,也标志着市场向上尝试该阻挡水平以失败收场。

本图揭示了一些具有普遍意义的概念,适用于如何对待不是很理想的蜡烛图形态。(1)形态构成的具体情况;(2)它所处的总体市场图景。当我们遇到不那么完美的蜡烛图形态时,上述两方面的因素有助于我们评估其有效性,看看它是不是具备与更准确定义的蜡烛图形态同等的技术含义。正因为需要这样的主观判断,使得蜡烛图形态的电脑识别十分困难。举例来说,本图所述的两例乌云盖顶形态都不满足乌云盖顶形态的经典定义,但是根据它们的构成情况,以及它们在行情演变过程中所处的位置,依然将它们视为乌云盖顶形态。

\begin{tcolorbox}
    如果在刺透形态的次日,实体没有深深地进入第一日实体的内部,通常建议,应等待进一步的验证信号。
\end{tcolorbox}

\figures{fig4-32}{国际商业机器公司——15 分钟蜡烛线图(刺透形态)}

假设刺透形态后一根为白色蜡烛线,下一个时段的收市价向上超越了该白色实体,就构成验证信号。在 \autoref{fig4-32} 中,3 月 31 日上午的晚些时候,一根白色蜡烛线向上推进到了前一根黑色实体的内部。既然它并没有向上超过黑色实体的中线,就不能构成刺透形态。这是一个插入线形态。在插入线形态的下一个时段,收市价达到了更高的水平,这有助于增强形态的效果,说明其可能是一个底部信号。在 3 月 31 日晚些的一个时段,在 117 美元附近,形成了另一个插入线形态。通常,在插入线形态出现后,应该等待进一步的看涨验证信号(正如在当日更早的时候看到的那样)。但是,对于本图中的第二个插入线形态,由于它验证了之前的支撑区域,就没有太大的必要来等待看涨验证信号了(也就是说,不需要那么多理由来等待下一个时段的收市价推升到更高位置)。如此一来,对活跃的动量交易者来说,不妨在第二个插入线形态的白色蜡烛线收市时,利用它作为买进机会。第二天上午的早些时候,出现了一个看跌吞没形态,给出了退出信号。

\tips{是否可以说,经过验证的插入线可以不用等待确认信号,比如说前面确定了支撑线?}