\chapter{反转形态}
技术分析师瞪大眼睛盯着价格的涨落,为的是及早发现市场心理变化和趋势变化的警告信号。反转价格形态就是这样的技术线索。在西方技术分析理论中,反转信号包括双重顶、双重底、反转日、头肩形、岛形反转顶和岛形反转底等各种价格形态。

然而,\important{从一定意义上来说,“反转形态”这一术语的用词是不准确的。听到反转形态这个术语,往往使人误以为现有趋势将会突然结束,立即逆转为反向新趋势。实际上,这种情况很少发生。趋势的逆转,一般都是伴随着市场心理的逐渐改变进行的,通常需要经过一个缓慢的、分阶段的演变过程。}

确切地说,趋势反转信号的出现,意味着之前的市场趋势可能发生变化,但是市场并不一定就此逆转到相反的方向。弄清楚这一点,是至关重要的。我们不妨用行驶中的汽车来打个比方。一场上升趋势就相当于一辆前进的汽车。汽车的刹车灯亮了,随后汽车停了下来。刹车灯相当于趋势反转信号,它表明先前的趋势(相当于汽车向前行驶)即将终止。现在,汽车静止不动,那么下一步,司机是打算掉头向反方向行驶,还是停在那里不动,抑或是继续向前行驶呢?如果没有更多的线索,我们根本无从知晓。

\begin{tcolorbox}
    请务必留意,当我说“反转形态”的时候,这个术语仅仅意味着之前的趋势将发生变化,但是未必一定会反转。把反转形态理解成趋势变化形态,才是慎重可取的考虑。
\end{tcolorbox}

反转形态,就是市场以其特有的方式为我们提供的一个指路牌,牌子上写着:“当心——趋势正在发生变化。”也就是说,市场的心理状态正在发生变化。那么,为了适应这种新的市场环境,我们就应当及时调整自己的交易方式。当反转形态出现时,如何建立新头寸,如何了结旧头寸,存在多种多样的选择。

这里有一条重要原则:只有当反转信号所指的方向与市场的主要趋势方向一致时,我们才能依据这个反转信号来开立新头寸。举个例子,假定在牛市的发展过程中,出现了一个顶部反转形态。虽然这是一个看跌的信号,却不能保证卖出做空是有把握的。这是因为,市场当前的主要趋势依然是上升的。无论如何,就这个反转形态的实质意义来说,它构成了了结既有多头头寸的交易信号。之后,我们将在调整过程中寻求看涨信号买进做多,因为当前主要趋势向上。
\section{伞形线}
如 \autoref{fig4-4} 所示的蜡烛线具有明显特点。它们的下影线较长,而实体(或黑或白)较小,并且在其全天价格区间里,实体所处的位置接近顶端。\autoref{fig4-4} 所列蜡烛线被称为\textbf{伞形线},因为它们的轮廓呈伞状,长长的下影线如伞柄,小实体如伞面。在本图中,我们同时列出了黑白两种蜡烛线。有趣的是,这两种蜡烛线都既可能是看涨的,也可能是看跌的,具体情况要由它们在趋势中所处的位置来决定。

\figures{fig4-4}{伞形线}

伞形线,不管是哪一种,只要它出现在下降趋势中,那么,它就是下降趋势即将结束的信号。在这种情况下,这种蜡烛线称为锤子线,意思是说“市场正用锤子夯实底部”,如 \autoref{fig4-5} 所示。

\figures{fig4-5}{锤子线}

如上所述,伞形线的性质取决于伞形线出现之前的主流趋势方向,随趋势而变。在下跌行情之后出现的伞形线是看涨信号,称为锤子线。在 \autoref{fig4-4} 所示的两种蜡烛线中,无论哪一种,如果出现在上冲行情之后,那么,它就表明之前的市场运动也许已经结束。显而易见,这类蜡烛线就称为\textbf{上吊线}。

形状相同的蜡烛线,有时是看涨的,有时又是看跌的,看起来或许有些不合常情。但是,如果熟悉西方技术分析理论中的岛形顶和岛形底,那就不难看出,在这个问题上,东西方的思路如出一辙。对岛形反转形态来说,既可以是看涨的,也可以是看跌的,\tips{这取决于它在市场趋势中所处的位置}。如果岛形反转形态出现在长期的上升趋势之后,则构成看跌信号;如果岛形反转形态出现在下降趋势之后,则构成看涨信号。

一般根据三个方面的标准来识别锤子线和上吊线。
\begin{enumerate}
    \item 实体处于整个价格区间的上端,但实体本身的颜色无所谓。
    \item 带有长长的下影线,且下影线的长度至少达到实体高度的两倍。
    \item 在这类蜡烛线中,应当没有上影线;即使有上影线,长度也是极短的。
\end{enumerate}

可以从三个方面来区分上吊线和锤子线——趋势、该蜡烛线出现之前行情运动的区间、验证信号。具体说明如下:
\begin{itemize}
    \item 趋势:锤子线只能紧接在下跌行情后出现;上吊线必须紧接在上涨行情后出现。
    \item 蜡烛线出现之前行情运动的区间:锤子线之前即便只有短线下跌也是有效的;但是,上吊线之前必须具备长足的上冲行情,最好是创下行情全历史新高后,方为有效。
    \item 验证信号:上吊线出现后得到验证方为有效;对锤子线则无此要求。
\end{itemize}

在看涨的锤子线的情况下,或者在看跌的上吊线的情况下,其下影线越长、上影线越短、实体越小,那么,这类蜡烛线就越有意义。
\subsection{锤子线}
锤子线的实体既可以是白色的,也可以是黑色的。从 \autoref{fig4-5} 可见,即使锤子线的实体是黑色的,其收市价也接近本时段最高价。但是,如果锤子线的实体是白色的,其看涨的意义则更坚挺几分(因为其收市于最高价)。如果锤子线的实体是白色的,日本人称之为“力量线”。

锤子线带有长长的下影线,且收市于本时段最高价,或接近最高价,意味着在当天的交易过程中,市场起先曾急剧下挫,后来却完全反弹上来,收市在当日的最高价处,或者收市在接近最高价的水平上。这一点本身就具有看涨的味道。锤子线收市价位于最高价或接近最高价,正是其没有上影线,或者上影线很小的原因。这一点是锤子线的判据。反之,如果蜡烛线带有长的上影线,则意味着收市价显著低于本时段最高价。

既然锤子线属于底部反转信号,就需要之前存在下降趋势才谈得上反转。

\figures{fig4-7}{2 月 24 日看到一根锤子线。它很经典,因为其下影线很长,而实体较小,且实体位于当日交易区间的顶端。它是在一轮下跌行情之后出现的。这正是锤子线出现的必要前提。2 月 22 日的蜡烛线不会被定义为锤子线,因为它不符合条件,下影线长度达不到实体长度的 2-3 倍。这类长下影线是必要条件,因为它表示市场在本时段曾经急剧下挫,但在收市时,行情收市价位于或接近最高价,熊方反倒一败涂地。这是一场日本人所形容的“神风反攻”。24 日的经典锤子线清晰地揭示了上述过程。我敢打赌,在出现了这样经典的锤子线之后,空头立场一定会动摇。}

\autoref{fig4-7} 揭示了蜡烛图技术至关重要的一方面特点。如果希望借助蜡烛图来成功地交易,那么不仅需要理解蜡烛图形态本身,还需要在评估交易的风险报偿比的基础上,理解蜡烛图形态所处的相对位置。\important{始终应当先考虑风险报偿比},再根据蜡烛图形态或蜡烛线信号发出交易指令。现在,让我们牢记交易之前应考虑风险报偿比,再重新审视 24 日的理想锤子线。

在该锤子线即将完成时(请记住,我们不得不等待行情收市),股价收市于接近 48 美元处。如果在锤子线完成时买进(48 美元附近),当市场再次下跌到锤子线最低点约 43 美元时止损,则上述两个价格之差便是风险,即大约 5 美元。如果您的赢利目标远大于 5 美元,在锤子线完成时买进并没有什么不对。不过,对某些活跃的短线交易者来说,5 美元或许风险太大了。

于是,为了降低潜在的交易风险,交易者或许打算等待市场回调至锤子线下影线范围内的机会(当然,在锤子线出现后,许多时候市场并不回调)。利用锤子线低点作为潜在的买进点位,就有可能在接近止损的位置上建仓。

设想交易者甲识别出 2 月 24 日的锤子线。当他看到如此精彩的锤子线时过于兴奋,以至于在锤子线的收市价附近,即接近 48 美元处立即买进了。次日,市场开市时向下跳空,开市价为 44.50 美元。交易者甲的头寸现在处在水下 3.5 美元的深处了。或许他最多承受 4.5 美元的亏损,就要采取止损措施,轧平昨日的多头头寸。如果是这样的话,最终甲或许认为这根蜡烛线不起作用。

交易者乙同样意识到锤子线是潜在的反转信号,但是她记住了风险报偿比的前提,没有在锤子线收市时买进(因为对她来说,此处买进的风险太大了)。次日,当市场开市于低价位时,深入锤子线长长下影线的低端,这是潜在的支撑区间。乙认为,股价现在接近支撑水平,决定买进。当股价从支撑位上冲后,乙就会对蜡烛图大唱赞歌。

当然,市场有时守不住潜在的支撑水平或阻挡水平,锤子线也是半斤八两,有时也守不住。可见,对蜡烛图功能的个人发挥,决定了蜡烛图贡献的多寡,运用之妙,存乎一心。

\figures{fig4-8}{锤子线有助于确认支撑水平。}

\autoref{fig4-8} 为纳斯达克 100 指数(NDX)5 分钟的蜡烛线图,从 A 点处开始,发动了一轮上冲行情。在接近 3723 的价位,当位置 1 和 2 处出现了两根小实体后,这是该轮上涨已经是强弩之末的第一个征兆。市场从此处陡然回落。当行情接近潜在支撑区域 3680 左右时,形成了一根锤子线。如果支撑水平牢靠,那么该锤子线就应该成为支撑区域。在随后的两个时段,该锤子线果然发挥了支撑作用。当然,如果纳斯达克 100 指数收市于 3680 的支撑水平以下,上述看涨预期就失效了。\tips{这是技术分析的一个重要方面——应该总是存在某个价位,表明我们是错误的。}在本例的情况下,便是市场低于 3680。
\subsection{上吊线}
上吊线的形状与锤子线相同,唯一区别是,上吊线出现在上涨行情之后。因为长下影线具有增强功效的作用,而上吊线有这样的长下影线。\important{上吊线出现后,一定要等待其他看跌信号的证实,这一点特别重要}。最低限度的验证信号是,之后时段的开市价低于上吊线的实体。但是,我总是\tips{建议以收市价低于上吊线的实体为验证信号。}

为什么要等待之后时段的收市价低于上吊线的实体呢?因为如果次日市场收市在低于上吊线实体的水平,那么凡是在上吊线当日的开市、收市时买进的交易商(大量的交易活动发生在这两个时段),会通通背上亏损的头寸,被“吊”在上面。同样道理,我还希望看到上吊线位于市场全历史最高价,或者至少处于一轮重大行情的最高位。在这种情况下,多头在上吊线之日的入市点位处在历史新高处,这就迫使他们更加紧张。结果,这批多头或许决定撤出他们新开的多头头寸。他们的头寸倾泻而至,加重了卖出压力。

\figures{fig4-10}{微软(Microsoft)——日蜡烛线图(上吊线和锤子线)}

\autoref{fig4-10} 所示的实例颇精彩,从中可以看到,同一种蜡烛线,既可以是看跌的(如 1 月 29 日的上吊线),也可以是看涨的(如 2 月 22 日的锤子线)。尽管在这个实例中,上吊线和锤子线的实体都是白色的,但是它们实体的颜色并没有太大意义。1 月 29 日的蜡烛线是上吊线,因为之前趋势是上冲行情。上吊线给本轮行情创了新高。下一日(2 月 1 日),收市价低于上吊线的实体,这让所有的新多头——在上吊线当日开市和收市时买进的——通通陷入困境。2 月 22 日的蜡烛线是锤子线,因为它出现在下降趋势之后。锤子线之前的一个交易日也是短实体。这是一条更早出现的线索,表明空头的努力遭遇了阻碍。而锤子线成为进一步的看涨证据。

\autoref{fig4-12} 揭示了上吊线出现后等待验证信号的重要性。位于 1、2 和 3 处的蜡烛线都属于上吊线(它们的上影线都很短,足以判定它们属于上吊线)。其中每一根线的收市价都为当前的上涨行情创了新高,因而维持了上升惯性的持续有效。如果要确认这些蜡烛线的看跌意味,即上涨趋势要从上涨转为不那么强势的,就必须看到之后的收市价低于上述上吊线的实体。这种情况没有发生。记住,任何人在上吊线开市和收市时买进,都是买在当前行情的最高点。如果在上吊线之后市场持续创新高(此处正如此),那些多头还会感觉难受吗?当然不会。他们正高兴呢,因为行情比他们买进的点位更高了。因此,正如本例所示,只要市场没有收市于上吊线实体的水平之下,则牛市趋势保持稳固。

\figures{fig4-12}{加贝利资产管理公司(Gabelli Asset Management)——日蜡烛线图(上吊线及其验证信号)}
\section{吞没形态(抱线形态)}
吞没形态属于主要的反转形态,由两根颜色相反的蜡烛线实体构成。

\autoref{fig4-13} 显示的是看涨吞没形态。在本图中,市场本来处于下降趋势之中,但是后来出现了一根坚挺的白色实体,这根白色实体将它前面的那根黑色实体“抱进怀里了”,或者说把它吞没了(该形态正由此得名)。它有个绰号叫“抱线形态”,道理一目了然。(这种情形说明市场上买进压力已经压倒了卖出压力。

\figures{fig4-13}{看涨吞没形态}

\autoref{fig4-14} 是看跌吞没形态的示意图。在本图中,市场原本正向着更高的价位趋升。但是,当前一个白色实体被后一个黑色实体吞没后,就构成了顶部反转信号。这种情形说明,市场上供给压倒了需求。

\figures{fig4-14}{看跌吞没形态}

关于吞没形态,我们有三条判别标准:
\begin{enumerate}
    \item 在吞没形态之前,市场必须处在明确的上升趋势(看跌吞没形态)或下降趋势(看涨吞没形态)中,哪怕这个趋势只是短期的。
    \item 吞没形态由两条蜡烛线组成。其中第二根蜡烛线的实体必须覆盖第一根蜡烛线的实体(但是不一定需要吞没前者的上下影线)。
    \item 吞没形态的第二个实体应与第一个实体的颜色相反。(这一条标准有例外的情况,条件是,吞没形态的第一条蜡烛线是一根十字线。如此一来,如果在长时间的下降趋势之后,一个小小的十字线被一个巨大的白色实体所吞没,就可能构成底部反转形态。反之,在上升趋势中,如果一个十字线被一个巨大的黑色实体所吞没,就可能构成顶部反转形态)。
\end{enumerate}

下面列出了一些参考要素,如果吞没形态具有这样的特征,那么它构成重要反转信号的可能性将大大增强:
\begin{itemize}
    \item 如果在吞没形态中,第一天的实体非常小(即纺锤线),而第二天的实体非常大。第一天蜡烛线的小实体反映出原有趋势的驱动力正在消退,而第二天蜡烛线的长实体证明新趋势的潜在力量正在壮大。
    \item 如果吞没形态出现在超长延伸的或非常快速的市场运动之后。如果存在非常快速的或超长程的行情运动,则导致市场朝一个方向走得太远(要么超买,要么超卖),容易遭受获利平仓头寸的打击。
    \item 如果在吞没形态中,第二个实体伴有超额的交易量。
\end{itemize}