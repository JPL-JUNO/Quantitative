\chapter{星线\label{ch05}}
\autoref{fig5-1} 所示,\textbf{星蜡烛线}(简称\textbf{星线})的实体较小(可以是白色,也可以是黑色),并且在它的实体与它前面较大的蜡烛线实体之间形成了价格跳空。换句话说,星线的实体可以处在前一个时段的上影线范围内;只要星线的实体与前一个实体没有任何重叠(有一些例外情形,本章后面还要讨论),那么这个星蜡烛线就是成立的。如果星线的实体已经缩小为十字线,则称之为十字星线。当星线,尤其是十字星线出现时,就是一个警告信号,表明当前的趋势或许好景不长了。

\figures{fig5-1}{上升趋势中的星线和下降趋势中的星线}
\section{启明星形态}
启明星形态属于底部反转形态(如 \autoref{fig5-3} 所示)。它的名称由来是,就像启明星(金星)预示着太阳的升起一样,这个形态预示着价格的上涨。本形态由三根蜡烛线组成:
\begin{itemize}
    \item 蜡烛线 1。一根长长的黑色实体,形象地表明空头占据主宰地位。
    \item 蜡烛线 2。一根小小的实体,并且它与前一根实体之间不相接触(这两条蜡烛线组成了基本的星线形态)。小实体意味着卖方丧失了驱动市场走低的能力。
    \item 蜡烛线 3。一根白色实体,它明显地向上推进到了第一个时段的黑色实体之内,标志着启明星形态的完成。这表明多头已经夺回了主导权。
\end{itemize}

\figures{fig5-3}{启明星形态}

如 \autoref{fig5-3} 所示,在构成本形态的三根蜡烛线中,\tips{最低的低点构成了支撑水平}。

在理想的启明星形态中,第二根蜡烛线(即星线)的实体,与第三根蜡烛线的实体之间有价格跳空。即使没有这个价格跳空,似乎也不会削减启明星形态的技术效力。\notes{其决定性因素是,第二根蜡烛线应为纺锤线,同时第三根蜡烛线应显著深入到第一根黑色蜡烛线内部}。

在理想的启明星形态和黄昏星形态中,蜡烛线 1 与 2,2 与 3 的实体之间都不应该相互触及。不过在有些市场上,上一日收市和下一日开市要么是同一个时点,要么两者时间接近,在这种情况下,对启明星形态和黄昏星形态要求的定义条件便可以更灵活一些。试举例如下:
\begin{enumerate}
    \item 外汇市场,不存在正式的开市或收市时间。
    \item 很多指数市场,例如半导体指数或药品指数等。
    \item 日内图表。举例来说,在 15 分钟的线图上,某一个 15 分钟时段的开市价与相邻的前一个 15 分钟时段的收市价通常没什么差异。
\end{enumerate}
\section{黄昏星形态}
黄昏星是启明星的反面对等形态,在顶部,是看跌的。它的名称由来也是显而易见的,因为黄昏星(太白星)恰好出现在夜幕即将降临之际。既然黄昏星是顶部反转形态,那么,就应当出现在上升趋势之后,才能发挥其技术效力。黄昏星形态是由三根蜡烛线组成的(如 \autoref{fig5-7} 所示)。在前两根蜡烛线中,前一根是长长的白色实体,后一根是星线。星线的出现,是顶部形态的第一个征兆。第三根蜡烛线证实了顶部过程的发生,完成了属于三线形态的黄昏星形态。第三根蜡烛线具有黑色实体,它剧烈地向下扎入第一天的白色实体内部。

原则上说,在黄昏星形态中,首先在第一根实体与第二根实体之间,应当形成价格跳空;然后在第二根实体与第三根实体之间,再形成另一个价格跳空。但是,正如前面关于启明星的部分曾经详细介绍的那样,上述第二个价格跳空并不常见,而且对本形态的成功来说可有可无,不是必要条件。\important{本形态的关键之处在于第三天的黑色实体向下穿入第一天的白色实体的深浅程度}。

\figures{fig5-7}{黄昏星形态}

下面列了一些参考因素,如果黄昏星形态兼具这样的特征,则有助于增加它们构成反转信号的机会。这些因素包括:
\begin{enumerate}
    \item 如果第一根与第二根蜡烛线,第二根与第三根蜡烛线的实体之间不存在重叠。
    \item 如果第三根蜡烛线的收市价向下深深扎入第一根蜡烛线的实体内部。
    \item 如果第一根蜡烛线的交易量较小,而第三根蜡烛线的交易量较大。这表明之前趋势的驱动力正在减弱,新趋势方向的驱动力正在加强。
\end{enumerate}

\tips{黄昏星形态的最高点构成阻挡水平},在 \autoref{fig5-7} 中用虚线做了标记。

有些蜡烛图形态对交易者具有挑战性,具体说来,当形态完成时,市场或许已经明显离开其高点或低点了。在黄昏星形态中,因为必须等待一根长长的黑色实体来完成形态,可能是在市场已经向下转折许久之后,它才发出反转信号。借 \autoref{fig5-10} 来观察这个方面。

正如本图所示,本轮行情的高点接近 34 美元。当黄昏星形态完成时,其第三根蜡烛线的收市价接近 31 美元。如此一来,如果交易者根据黄昏星形态的信号在 31 美元卖出,所承受的风险是 31 美元到黄昏星形态最高点 34 美元的距离。如果交易者的价格目标是 3 美元风险值的许多倍,那么这个风险并不是问题。只有在这样的前提下,这里才能成为风险报偿比具有吸引力的交易机会。

如果 3 美元的风险值过大,交易者可以等待行情反弹,利用市场接近该黄昏星形态顶部阻挡区域的机会交易(当然,不保证一定会发生反弹行情),从而改进其风险报偿比。在本例中,在黄昏星形态之后的两个时段内,得到了幅度为 2 美元的反弹行情,将股价带到了非常接近关键阻挡水平的 34 美元的位置。在这之后,股价重新开始下降,一直持续到 4-5 月的价格区间。一系列蜡烛线实体一蟹不如一蟹地不断收缩,预示着市场转折的机会增加了。

\figures{fig5-10}{罗杰通讯(Roger Communications)——周蜡烛线图(黄昏星形态)}

\section{十字启明星形态和十字黄昏星形态}
在常规的黄昏星形态中,第二根蜡烛线具有较小的实体,如果不是较小的实体,而是一个十字线,则称为\textbf{十字黄昏星形态}。十字黄昏星形态是常规黄昏星形态的一种特殊形式。

在启明星形态中,如果其星线(即三根蜡烛线中的第二根蜡烛线)是一个十字线,则成为\textbf{十字启明星形态}。这一类启明星形态能构成有意义的市场底部过程。

\figures{fig5-12}{十字启明星形态}

在十字黄昏星形态中,如果十字线的下影线既不与第一根蜡烛线的上影线重叠,也不与第三根蜡烛线的上影线重叠(即影线之间不相接触),这条十字线就构成了一个主要顶部反转信号,称为\textbf{弃婴顶部形态}。这种形态非常罕见!

在与上述形态对等的底部反转形态中,道理是一样的,只不过方向相反而已。具体说来,在下降趋势中,如果在一个十字星线的前后均发生了价格跳空(相关蜡烛线的上下影线互不接触),那么这条十字星线就构成了一个主要底部形态。人们将这种形态称为\textbf{弃婴底部形态}。这种形态也是极为少见的!弃婴形态与西方的岛形顶部形态或岛形底部形态类似,不过其中的孤岛还应当是一根十字线,不难想象,这样的情形何其稀罕。

在理想的弃婴底部形态中,第二根蜡烛线是十字线。在 \autoref{fig5-18} 中,第二根蜡烛线不是理想的弃婴底部形态所要求的十字线,而是一根微小的实体。无论如何,该实体如此细小,以至于可以把它视为十字线(本形态的第二根蜡烛线也可以归类为锤子线)。因此,这是弃婴底部形态的一种变体。从该底部反转形态开始,一轮上涨行情一直持续到一系列长长的上影线(箭头所指处),它们发出警告信号,多头现在不能完全做主了。这进一步增强了以下预期:市场已经碰到天花板了。4 月 6 日和 7 日组成了一个看涨吞没形态,之后市场开始向上反弹。

\figures{fig5-18}{豆油——日蜡烛线图(弃婴底部形态)}
\section{流星形态与倒锤子形态}
如 \autoref{fig5-19} 所示,在\textbf{流星形态}中,\textbf{流星线}具有较小的实体,而且实体处于其价格区间的下端,同时,流星线的上影线较长。我们可以看出其名称的由来,它的外观恰如其名称,像一颗流星,带着熊熊燃烧的长尾巴划过天际。日本人贴切地描绘道,流星形态预示着前方有麻烦了。既然它只是一个时段,通常不像看跌吞没形态或黄昏星形态那样构成主要反转信号。它与上述两个形态还有一点不同,我\important{不认为流星线构成了关键阻挡水平}。

与所有的星蜡烛线一样,流星线实体的颜色并不重要。流星线的形状形象地显示,当日市场开市于它的最低点附近,后来强烈地上冲,但最后却向下回落,收市于开市价附近。换句话说,这个交易时段内的上冲行情不能维持下去。

因为流星线属于看跌反转信号,它必须出现在一段上冲行情之后。在理想的流星形态中,流星线的实体与前一根蜡烛线的实体之间存在价格跳空。不过我们将从几个实例中看到,这样的价格跳空并非总是必须的。在流星形态中,没有向上跳空恰恰给它的负面意义增添了一个理由。其原因在于向上跳空本身是一个正面信号。在日本术语里,向上跳空被称为“向上的窗口”。如此一来,在流星形态中没有向上跳空的情况下,我们可以更安心地判断趋势即将转入不那么牛气的状态。

在下降趋势后,如果出现了与流星线外观一致的蜡烛线,则可能构成看涨信号。这样的蜡烛线称为\textbf{倒锤子线}。
\figures{fig5-19}{流星形态}
\subsection{流星形态(流星线)}
在 \autoref{fig5-20} 中,假如不是采用蜡烛图格式,而是线图格式,那么时段 A、B 和 C 反映出的是健康的市场环境,因为每个时段都具备更高的高点、更高的低点以及更高的收市价。可是,从蜡烛图技术的角度来观察,我们从它们三者得到的是警告性的图形信号,头顶上悬着麻烦呢。具体说来,A、B和 C 三处看跌的上影线突出地显示该股票正处在“苦恼的上涨”过程中(日本人就是这么说的)。在时段 C,关于顶部反转的最终验证信号来了,这是一根流星线。或许注意到,墓碑十字线(\autoref{ch08} 将讨论)的外形与流星线看起来相像。墓碑十字线是流星线的特殊形式。流星线具备小实体,而墓碑十字线——作为一个十字线——没有实体。因此,墓碑十字线比流星线来得更疲软。

\figures{fig5-20}{Mail Well——日蜡烛线图(流星线)}
\subsection{倒锤子形态(倒锤子线)}
虽然倒锤子线不属于星线形态,但是因为它的外形与流星线相像,所以我们把它放在这个部分来讨论。如 \autoref{fig5-23} 所示,倒锤子线看上去与流星线颇为相像,它也有较长的上影线和较小的实体,并且实体居于整个价格范围的下端。两者之间唯一不同的是,倒锤子线出现在下降行情之后。结果,流星线是一根顶部反转蜡烛线,而倒锤子线却是一根底部反转蜡烛线。倒锤子线实体的颜色无关紧要。同一种形状的蜡烛线既可以是看涨的,也可以是看跌的,取决于在其出现之前的趋势方向,在这个概念上,倒锤子线和流星线是一对,锤子线和上吊线是一对(参见 \autoref{ch04})。

\begin{tcolorbox}
    正如上吊线需要其他看跌信号的验证,倒锤子线也需要其他看涨信号的验证。验证信号既可以是次日开市价高于倒锤子线的实体,也可以是次日收市价高于倒锤子线的实体,尤其是后者更为有力。
\end{tcolorbox}

之所以倒锤子线需要看涨信号的验证,是因为它的长上影线给倒锤子线涂上了一层疲软的色彩。也就是说,在倒锤子蜡烛线当日,市场的开市价位于当日最低价处,或者接近最低价。后来市场上涨了,但是多头无力将上涨行情维持下去。最后,市场收市于当日最低价,或者在最低价的附近。为什么这样的蜡烛线竟然是潜在的看涨反转信号呢?其解答必须从后一天的行情中寻找。如果后一天市场开市于倒锤子线的实体之上,特别是收市于倒锤子线的实体之上,则意味着凡是在倒锤子线当日开市和收市时卖出做空的人现在通通处于亏损状态。市场维持在倒锤子线实体之上的时间愈久,则上述空头止损出市的可能性愈大。在这种情况下,首先可能引发空头平仓上涨行情,空头平仓上涨行情进而可能促使企图抄底做多的人跟风买入。这个过程自我循环,螺旋上升,结果就可能形成一段上冲行情。\textbf{其实是逼空头回补仓位}

\figures{fig5-23}{倒锤子线}

在 \autoref{fig5-24} 中,5 月 24 日的一根锤子线在 76 美元构成了支撑水平。次日,形成了一个倒锤子线。它为当前市场变动创下了收市价的新低,如此一来,它就延续了原来向下的短期趋势方向。无论如何,锤子线的支撑水平依然保持完好。5 月 26 日的收市价起到了一石二鸟的作用:首先,它再次确认了锤子线的支撑作用;其次,它为倒锤子线提供了验证信号,因为它的收市价高于倒锤子线的实体。如果对 76 美元的支撑水平还要求进一步的看涨验证信号,那么 6 月 2 日又来了另一个锤子线。

\figures{fig5-24}{微软——日蜡烛线图(倒锤子线)}