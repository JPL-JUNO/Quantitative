\chapter{星线}
\autoref{fig5-1} 所示,星蜡烛线(简称星线)的实体较小(可以是白色,也可以是黑色),并且在它的实体与它前面较大的蜡烛线实体之间形成了价格跳空。换句话说,星线的实体可以处在前一个时段的上影线范围内;只要星线的实体与前一个实体没有任何重叠(有一些例外情形,本章后面还要讨论),那么这个星蜡烛线就是成立的。如果星线的实体已经缩小为十字线,则称之为十字星线。当星线,尤其是十字星线出现时,就是一个警告信号,表明当前的趋势或许好景不长了。

\figures{fig5-1}{上升趋势中的星线和下降趋势中的星线}
\section{启明星形态}
启明星形态属于底部反转形态(如 \autoref{fig5-3} 所示)。它的名称由来是,就像启明星(金星)预示着太阳的升起一样,这个形态预示着价格的上涨。本形态由三根蜡烛线组成:
\begin{itemize}
    \item 蜡烛线 1。一根长长的黑色实体,形象地表明空头占据主宰地位。
    \item 蜡烛线 2。一根小小的实体,并且它与前一根实体之间不相接触(这两条蜡烛线组成了基本的星线形态)。小实体意味着卖方丧失了驱动市场走低的能力。
    \item 蜡烛线 3。一根白色实体,它明显地向上推进到了第一个时段的黑色实体之内,标志着启明星形态的完成。这表明多头已经夺回了主导权。
\end{itemize}

\figures{fig5-3}{启明星形态}

如 \autoref{fig5-3} 所示,在构成本形态的三根蜡烛线中,\tips{最低的低点构成了支撑水平}。

在理想的启明星形态中,第二根蜡烛线(即星线)的实体,与第三根蜡烛线的实体之间有价格跳空。即使没有这个价格跳空,似乎也不会削减启明星形态的技术效力。\notes{其决定性因素是,第二根蜡烛线应为纺锤线,同时第三根蜡烛线应显著深入到第一根黑色蜡烛线内部}。

在理想的启明星形态和黄昏星形态中,蜡烛线 1 与 2,2 与 3 的实体之间都不应该相互触及。不过在有些市场上,上一日收市和下一日开市要么是同一个时点,要么两者时间接近,在这种情况下,对启明星形态和黄昏星形态要求的定义条件便可以更灵活一些。试举例如下:
\begin{enumerate}
    \item 外汇市场,不存在正式的开市或收市时间。
    \item 很多指数市场,例如半导体指数或药品指数等。
    \item 日内图表。举例来说,在 15 分钟的线图上,某一个 15 分钟时段的开市价与相邻的前一个 15 分钟时段的收市价通常没什么差异。
\end{enumerate}