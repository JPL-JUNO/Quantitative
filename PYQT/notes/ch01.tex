\chapter{量化交易速览}
\section{国内常用的量化交易策略}
\subsection{期货 CTA 策略}
广义上,CTA 策略基本上能够分三大类,其中,趋势跟踪策略约占 70\%、均值回归(有时也叫价差套利)占 25\% 左右、逆趋势或趋势反转占 5\% 左右。但是由于趋势跟踪策略所占比重巨大及国内习惯把趋势跟踪策略等同于 CTA 策略(狭义理解),故在以后讲述的 CTA 策略特指趋势跟踪策略。

\paragraph{趋势跟踪策略} 趋势跟踪策略是基于市场并非有效的假设,基本面变化的信息需要一定的传递时间,资产价格不能立即反映基本面的变化,价格向合理方向逐渐变化的过程所表现出的趋势。与正态分布相比,资产收益率的分布通常具有“尖峰肥尾”的特点,“肥尾”提供了趋势跟踪策略的收益。趋势跟踪策略的赢利与市场的波动性密切相关,存在亏损的可能,因此,趋势跟踪策略交易者要注意及时止损。趋势跟踪策略在一定程度就是“追涨杀跌”的策略,通过快速止损实现“大赢小亏”,从而在整体上获利。一般情况下,国内习惯把趋势跟踪策略简称为 CTA 策略。

该策略是以某资产价格的历史信息为基础,要么设置一个价格正常波动的范围(即通道),当价格突破这个范围时采取策略;要么通过长短期均线的相对运动趋势采取策略。这类策略在本质上都是一种基于市场波动强度的投资策略,在市场波动剧烈时容易获利,在市场波动较小时,收益率较低。并非每次交易都能获利,及时止损从而实现“大赢小亏”,才是趋势跟踪策略的目标。

根据所跟踪趋势级别的不同,趋势跟踪策略可分为日内短线策略和日间中长线策略。
\begin{description}
    \item[日内短线策略] 日内短线策略要求所有开仓头寸都必须在日内交易时段结束前平仓出局,这种策略下资金暴露在风险中的时间最短,能获得稳定的利润收益,但也要求所选择的投资品种必须在日内有较大的波动和成交量,故这种策略选择的投资品种多为豆粕、螺纹钢、橡胶等商品。在之后的章节会详细地介绍由 vn.py 官方实现的几种经典日内短线策略。
    \item[日间中长线策略] 日间中长线策略主要把握市场或标的品种的中长期趋势,不同于短线策略主要依赖短期技术模型的胜率优势,建立在中长期趋势至上的量化交易策略,不仅关注技术分析,而且设定一套针对期货或其他金融品种基本面分析的基本面模型,或者两者结合。当然,这种量化交易策略的实际获利,也不会像短线策略那样被快速反映在账面之上,实际量化模型的有效性和收益率情况,也往往需要更长时间的验证。
\end{description}

\paragraph{价差套利策略} 价差套利策略,通过捕捉市场的不合理价差,买入被低估的资产,卖出被高估的资产,获得回归收益,达到资本赢利或避险的目的。套利交易风险小、回报稳定,对于大资金而言,如果单边重仓介入,将面临持仓成本较高、风险较大的不足;反之,如果单边轻仓介入,虽然可能降低风险,但其机会成本、时间成本也较高。因此整体而言,大资金单边重仓或单边轻仓介入期市,均难以获得较为稳定和理想的回报。而大资金如以多空双向持仓介入期市,也就是进行套利交易,则既可回避单边持仓所面临的风险,又可能获取较为稳定的回报。

价差套利进一步细分为\notes{跨期套利、期现套利、跨品种套利、跨市场套利这四种。}

\begin{description}
    \item[跨期套利] 跨期套利是指在同一市场上同时买入、卖出同种商品不同交割日的期货合约,以期在有利时机同时将这两个交割月份不同的合约对冲平仓而获利。跨期套利是套利交易策略中最普遍的一种,可以通过对冲和交割两种方式平仓。导致配对资产价格差的主要原因是资金的不均衡和季节性因素,两合约上资金的不平衡,使得某个合约的波动速度要明显快于其他合约,从而出现套利机会。跨期套利在同一交易所内完成配对资产的交易,不需要划转资金,容易实现账面平衡。通常情况下,跨期套利只发生在期货价格大于现货价格的情形下,因为期货价格小于现货价格时,相应操作属于投机而不是套利。以对冲进行套利时,若市场处于牛市,会导致近月合约价格上升幅度大于远月,或近月合约价格下降小于远月,此时应“买近卖远”;若市场处于熊市,会导致近月合约价格上升幅度小于远月,或下降幅度大于远月,此时应“卖近买远”。\important{跨期套利主要涉及季节性波动套利,而季节性波动主要是由供需的季节性变化导致的。只要供需结构不发生较大的变化,季节性波动套利的模式就有可操作性。}季节性波动套利的焦点在于不同月份合约的强弱变化,关注的合约组合是 1 月和 5 月组合,以及 9 月和 1 月组合。
    \item[期现套利] 期现套利是指某种商品期货合约,当期货市场与现货市场在价格上出现差距,从而利用两个市场的价格差距,低买高卖而获利,如 \autoref{fig1-1} 所示。理论上,期货价格是商品未来的价格,现货价格是商品目前的价格,按照经济学上的同一价格理论,两者间的差距,即“基差”(基差=现货价格-期货价格)应该等于该商品的持有成本。一旦基差与持有成本偏离较大,就出现了期现套利的机会。其中,期货价格要高出现货价格,并且超过用于交割的各项成本,如运输成本、质检成本、仓储成本、开具发票所增加的成本等。期现套利主要包括正向买进期现套利和反向买进期现套利两种。当期货价格大于现货价格时,称为正向市场。当期货价格对现货价格的升水大于持有成本时,套利者可以实施正向买进期现套利。即买入(持有)现货的同时卖出同等数量的期货,等待期现价差收敛时平掉套利头寸或通过交割结束套利。当期货价格小于现货价格时,称为反向市场。反向套利是构建现货空头和期货多头的套利行为(在期现套利中就是做空基差),由于现货市场上不存在做空机制,反向套利的实施会受到极大的限制。
    \item[跨品种套利] 跨品种套利是利用存在相关性的两种商品的期货合约价格差进行套利交易,即买入某一交割时间某种商品的期货合约,同时卖出另一相同交割时间、相关联的商品的期货合约,以期在合适时机将这两种合约同时对冲平仓从而获取利润。跨品种套利的本质是寻找价格差具有相对稳定关系的相关性的商品,并捕捉两者价格差偏离正常状态的情形,采取相关的反向操作获取利润。跨品种套利的品种一般有两类:一是选择产品与原材料,二是选择能互相替代的产品。具体国内市场而言,跨品种套利一般可以在以下品种中进行。
        \begin{itemize}
            \item 螺纹钢与铁矿石、焦炭。钢铁生产中最重要的原料就是铁矿石,其次是焦炭。钢铁生产的技术流程现已十分成熟,没有大的变化。生产 1 吨生铁,大约需要 1.5-2 吨的铁矿石、0.4-0.6 吨的焦炭。因此,钢铁的价格基本上取决于铁矿石与焦炭的价格。钢铁与铁矿石的相关性很强,与焦炭的相关性次之。
            \item 大豆与豆油、豆粕豆油是常用的食用油,而豆粕则可以作为动物饲料。压榨加工大豆,可以产出豆油并剩下豆粕,因此这三者之间可以进行跨品种套利。一般而言,$100\% \text{大豆}=18.5\% \text{豆油}+80\% \text{豆粕}+1.5\% \text{消耗}$。
            \item 焦煤与焦炭。焦煤是焦炭的上游产业,按照现在的生产技术,1.3 吨焦煤可以产出1吨焦炭。因此,二者价格相关性强,可以进行跨品种套利。
            \item 热轧卷板与螺纹钢。热轧卷板是一种钢板,以板坯为原料,加热之后进行粗轧和精轧后产出。热轧卷板作为一种重要的钢材,广泛应用于基建、船舶、汽车等领域。热轧卷板与螺纹钢同为钢材,原材料成本相近,因此两者价格具有较好的相关性。然而,由于下游消费市场具有差异,两者短期的供需关系会有不同,也就有了套利机会。
            \item 豆油、棕榈油与菜籽油。豆油、棕榈油与菜籽油均为食品添加剂,互为替代品。\tips{一般情况下,豆油与棕榈油、豆油与菜籽油的相关性较强,而棕榈油与菜籽油的相关性则相对弱些,因此推荐使用豆油与其他两个品种进行套利}。豆油的原料大豆主要产自美国、巴西及阿根廷,而棕榈油则一般产自印度尼西亚和马来西亚。由于不同地区的气候差异等因素,豆油与棕榈油的价差往往会出现波动,为投资者提供了套利机会。由于菜籽油营养更为丰富且原料价格高,菜籽油的价格一般高于豆油,两者的价差一般较为稳定。同样,价差受到季节性气候等的影响,会出现一些跨品种套利机会。
            \item 强麦与玉米。强麦指强筋小麦。小麦和玉米是世界范围内重要的两种农作物,在粮食和饲料市场中占据相当大的份额。两者互为替代品,价格具有同涨同跌的大趋势。但由于两者的收获季节不同,受气候等因素的影响也不同,因此价差会出现波动,提供跨品种套利机会。
            \item 沪深300 指数与上证50 指数、中证500 指数。由于沪深300 指数、上证50 指数、中证500 指数成分股之间存在差异,所表现出的市场走势特征有所不同。根据 2007 年以来市场实际走势来看,整体而言沪深300 指数及上证50 指数走势的相关性极强,无论是长周期或短周期,在上涨阶段或下行阶段,或震荡阶段,沪深300 指数及上证50 指数走势的相关性均在 95\% 以上。而中证500 指数与其他两大指数的相关性较弱,由于成份股的差异性,中证500 及上证50 指数的相关性最弱。
        \end{itemize}
    \item[跨市场套利] 跨市场套利即对同一期货品种在不同市场间进行套利。国内 3 个商品期货交易所并没有重复的品种,因此跨市场套利一般在国内和海外的期货交易所之间进行。对于同一种商品,交易所与原产地的距离会影响价格。
\end{description}
\figures{fig1-1}{期现套利}
\paragraph{反趋势策略} 趋势跟踪策略追踪趋势,反趋势策略则预测拐点。反趋势策略通常运用头肩形态、突破形态、交易量等反转指标来发现趋势的转折信号,建立头寸。

大部分反趋势模型寻找卖掉短期内超买的和买入短期内超卖的机会。这有点像在等待橡皮筋拉伸到它的极限时机,打赌它们会回弹到一个相对松弛的状态。这些行为使得反趋势交易模型在市场缺乏方向性或者波动性很大时斩获颇丰,并能够在市场拐点出现的时候迅速反应。反趋势模型的缺点是在稳定的、趋势性较强的市场环境中交易经常不顺,也就是常说的“赢小钱亏大钱”。
\subsection{股票 Alpha 策略}
国内比较常见的是 Alpha 策略,即运用复杂的量化方法从技术面、基本面角度分析未来价格变动趋势,以及不同股票间的相关性,进而买入低估值股票的同时卖出高估值股票,或者通过股指期货对现有投资组合头寸进行完全(或部分)对冲,隔离系统风险,获取 Alpha 收益。该策略的成功取决于量化选股模型的有效性、对冲的覆盖程度,选股模型越有效,系统风险对冲得越好,策略效果越好。

市场上常见的指数基金表现为:如果整个市场涨了,业绩也跟着涨,但如果整个市场跌了,业绩也跟着跌。因为它的 Beta 系数一般在1左右,它的收益主要来源于 Beta。

若把投资组合收益率分解成 Alpha 和 Beta 两部分以后,发现一个最重要的事实,这两部分的价值是不一样的。简单地说,Alpha 很难得,Beta 很容易。只要通过调节投资组合中的现金和股票指数基金(或者股指期货)的比率,就可以很容易地改变 Beta 系数,即投资组合中来自整个市场的收益。

因此 Beta 很便宜,Alpha 却很贵。指数基金和 ETF 基金是购买纯 Beta 的工具。因为只有 Beta,所以它们一般只收取很低的管理费。没有 Alpha,所以它们一定不会收取基于利润的分成费。在主动型公募基金,基金经理试图获得更好的绩效,也就是除 Beta 以外还想得到更多的 Alpha。想获得 Alpha 靠的是真本领。Beta 只是随大势,但“水可载舟,亦可覆舟”。国内的许多基金都只有 Beta,当然这在很大程度上是因为缺乏金融工具的选择,比如在融资融券出台之前不可以沽空。大盘开始暴跌的时候,也就是“股神”神话破灭的时候。业内人士有个比喻:Alpha 是肉,Beta 是面。指数基金全是 Beta,卖的是馒头;主动型公募基金卖的有肉有面,是包子;而对冲基金卖的就是纯肉。肉比包子贵,包子比馒头贵,贵表现在其收费模式是“2-20”,即 2\% 基础管理费和 20\% 业绩提成。
\subsubsection{Alpha 的含义}
\paragraph{Alpha 策略的基本思想} Alpha 策略是典型的对冲策略,通过构建相对价值策略来超越指数,然后通过指数期货或期权等风险管理工具来对冲系统性风险。Alpha 策略属于市场中性策略,但是 Alpha 策略的约束更小,其 Alpha 来源可能是行业的、风格的或者其他的。Alpha 策略注重选股,属于主动投资,相比之下,Beta 策略注重对投资时机的选择,属于被动投资。
\paragraph{Alpha 策略的分类} 在实际中经常使用的 Alpha 策略主要有多因子、风格轮动、行业轮动、资金流、动量反转等。

多因子是应用最为广泛的一种策略,该策略选择一系列因子来搭建模型。通过这些因子筛选股票,满足则买入,不满足则卖出。多因子的最大优势在于,在不同的市场和行情下,因子库中总有一些因子能够发挥作用。

风格轮动是指利用市场的风格特征进行投资。市场有时会偏好小盘股,有时偏好大盘股。通过观察某些指标来判断市场的倾向性,在风格转换的初期介入,可获得较大的超额收益。

行业轮动是指市场在经济周期的作用下对各个行业产生不同的偏好。在经济周期中,我们可以按照顺序依次对各个行业进行资产配置,从而获取相比于“买入-持有”策略的超额收益。

资金流是根据资金的流向来进行选股。对于一只股票,资金流入时,股票的价格应该会上涨;资金流出时,股票的价格应该会下跌。通过观察资金流的情况,我们可以预测未来股价的变化。

动量反转是指股票的强弱变化情况,过去一段时间强的股票,在未来一段时间继续保持强势,过去一段时间弱的股票,在未来一段时间继续弱势,这叫作动量效应。过去一段时间强的股票在未来一段时间走弱,或者过去一段时间在弱的股票在未来一段时间走强,这叫作反转效应。通过判断动量反转的有效性,筛选出应该购买的股票。
\paragraph{Alpha 策略的优势} Alpha 策略有三大优势:一是回避了择时这一难题,仅需专注于选股;二是波动较单边买入持有策略要小;三是在单边下跌的市场下也能赢利,Alpha 与市场的相关性理论值为 0。在熊市或者盘整期,可以采用“现货多头+期货空头”的方法,一方面建立能够获取超额收益的投资组合的多头头寸,另一方面建立股指期货的空头头寸以对冲现货组合的系统风险,从而获取正的绝对收益。

\subsubsection{因子的分类}
多因子策略对于因子的分类方法很多,整体而言,因子可以被分为基本面因子和技术面因子。基于对一只股票的不同特征的刻画,一般而言,可以将因子更加细致地分为:赢利性、估值、现金流、成长性、资产配置、价格动量和技术面因子。

\paragraph{赢利性} 与赢利性相关的因子主要反映了公司利用现有资源实现收益的能力。公司的赢利能力可以通过许多方法来衡量,例如投入资本回报率(ROIC)、已利用资本回报率(ROCE)、净资产收益率(ROE)、总资产收益率(ROA)、边际利润、人均收入、经济利润、投资增额收益率。整体而言,赢利性因子是一类效果较好的因子,即赢利性好的公司股票具有显著的正超额收益,而赢利性差的公司股票具有显著的负超额收益。

\paragraph{估值} 估值因子主要反映了股票作为一种资产的价值与其价格的相关性,但其决定性因素是该公司为其客户创造价值的多少。估值可以通过许多方法得到,但都包括了一定的定性分析和对未来的预测。常见的估值因子有:自由现金流价格比、外部融资总资产比、企业价值与 EBITDA 比(EV/EBITDA)、市盈率、股息率等。市销率可以说是美国股市最有效的因子,但是在中国股市却失效。国内分析师团体更倾向于用市盈率来进行估值,主流研报上市盈率也更有市场,故市盈率可以说是中国最有效的估值因子。

\paragraph{现金流} 现金流可以分为经营性、投资性和融资活动三类。其中,经营性现金流,包括从商品销售和服务得到的现金减去生产这些产品和提供这些服务需要支付的现金流出,包括为利润支付的现金税和为债务融资支付的利息。一个公司产生的经营性现金流水平是衡量未来股市回报的一个重要指标。常见的现金流因子有:自由现金流和营业收入之比、投入资本现金回报率等。

\paragraph{成长性} 成长性因子在市场中通常获得的超额收益较为微弱。这主要因为成长性投资更多是定性而非定量的,更加依赖投资者独到且有前瞻性的眼光而非精确的数据分析,更加偏向于“艺术”而非“量化”。尽管如此,成长性仍然是我们因子库中重要的一部分。正如成长性投资者们所说的:“我所知道的投资中最大的一个错误,就是对那些最伟大公司和其他普通公司一视同仁。”在实际使用成长性因子的过程中,我们常常和其他因子结合使用,以弥补其预测性不足的劣势。常见的成长性因子有:每股自由现金流、每股盈余等。

\paragraph{资产配置} 资产配置主要涉及一家公司资本资源的使用情况,主要考虑现金来源和现金使用两方面的内容。现金来源主要包括业务经营、资产和投销售收入、发行股票和发行债券等。现金使用主要包括经营费用、业务投资、业务收购、项目或证券投资、支付现金股利、偿还债务及回购股份等。常见的资产配置因子有:净回购股份与投入资本比、一年流通股减少量、一年长期债务减少量、外部融资和总资产比、三年平均资本支出和投入资本比、收购与投入资产比等。

\paragraph{价格动量} 价格动量因子能够在一定意义上反映市场周期和投资者情绪,并依此对未来进行预测。衡量价格动量的一般指标是价格变化的速度,或一段时间内价格的变化率。正的价格动量意味着某只股票的买家数量正在不断增加,需求大于供给;负的价格动量则意味着供需平衡向卖家倾斜,供给大于需求。常见的价格动量因子有:相对强弱、价格范围、相对强弱指数等。

\paragraph{技术面因子} 技术面因子相比于基本面因子,数据更新时间快,更加注重市场的微观结构,而非股票的价值。常见的技术面因子有:强弱指标(RSI)、随机指标(KD)、趋向指标(DMI)、平滑异同平均线(MACD)、能量潮(OBV)等。由于技术面因子的Alpha往往不稳定,所以在实际应用中较为少见。
\subsubsection*{因子的筛选和评价}
因子筛选的前提是获取充足的历史数据,包括基本的股价历史行情、基本面数据、分析师情绪指数、宏观经济数据等。

\paragraph{因子筛选的整体思路} 筛选因子的主要原则有:
\begin{itemize}
    \item 数据的准确性和真实性;
    \item 数据的完整性;
    \item 数据来源的稳定性。
\end{itemize}

\paragraph{因子评价的整体思路} 我们试图找出这样的因子:对于绝大多数股票而言,当该因子参数越大/越小时,超额收益越大/越小,或者恰好相反。研究股票超额收益和因子参数之间关系的方法主要有两种:
\begin{itemize}
    \item 根据因子参数的大小对股票进行分组,计算每组的平均超额收益,并依次进行因子胜率、$t$ 检验。
    \item 在每一个时间点上,计算全体股票截面上的超额收益率和因子参数大小的相关系数,以及信息比率。
\end{itemize}

\subsubsection*{因子的组合}
\paragraph{冗余因子的剔除} 需要剔除掉一些有效但是信息冗余的因子,即在同类的因子中只需要保留收益最好、区分度最高的那一个。

剔除冗余因子的一般方法如下:
\begin{itemize}
    \item 取出各个有效因子在各个时间点上关于分组的序列;
    \item 计算这些序列的相关性矩阵;
    \item 得到相关性矩阵的时间序列,并求该时间序列的均值;
    \item 通过这个均值矩阵挑出相关性较大的因子组;
    \item 对于每个因子组,挑选其中有效性最好或者收益最好的一个因子作为最终的因子。
\end{itemize}
\paragraph{因子的降维} 在多因子模型的实际应用中,希望将有效的因子加以组合和处理,减少模型中变量的个数,这种减少自变量的过程叫作降维。

降维的主要方法有:因子简单平均降维法、因子历史平均收益率加权平均降维法、逐步回归分析、主成分分析等。
\begin{description}
    \item[因子简单平均降维法] 因子简单平均降维法就是对同类的因子进行简单的等权平均处理,对因子参数求平均,作为新的复合因子。
    \item[因子历史平均收益率加权平均降维法] 加权平均降维法就是对同类的因子按照历史平均收益求加权平均,因子的历史平均收益取各个时间点分组的第一组的收益。
    \item[逐步回归分析] 在实际的多元回归问题中,我们总试图找到所谓“最优”回归方程,主要是指希望在回归方程中包含所有对因变量 $y$ 影响显著的自变量而不包含对 $y$ 影响不显著的自变量的回归方程。逐步回归分析正是根据这种原则提出来的一种回归分析方法。它的主要思路是在考虑的全部自变量中按其对 $y$ 的作用大小、显著程度大小或者说贡献大小,由大到小地逐个引入回归方程,而那些对 $y$ 作用不显著的变量可能始终不被引入回归方程。另外,已被引入回归方程的变量在引入新变量后也可能失去重要性,而需要从回归方程中剔除出去。引入一个变量或者从回归方程中剔除一个变量都是逐步回归的一步,每一步都要进行 $F$ 检验,以保证在引入新变量前回归方程中只含对 $y$ 影响显著的变量,而不显著的变量已被剔除。
    \item[主成分分析] 主成分分析的基本思路是将原来具有相关性的一些指标组合成一组新的互相无关的综合指数来代替原来的指标。一般情况下,用原来指标的线性组合作为新的综合指标。我们认为一个综合指标的方差越大,其包含的信息也就越多。因此,在所有线性组合中,用方差最大的那一个作为第一主成分。如果认为第一主成分不能有效地反映原来的信息,我们就取另一个和第一主成分相关系数为 0 的线性组合作为第二主成分,依此类推。
\end{description}
\paragraph{因子权重的确定} 对因子赋权的方法有很多,在此简要介绍三种:等权赋值、回归赋值、IC-IR 因子赋值。
\begin{itemize}
    \item 等权赋值:等权赋值是指在组合各个因子时对各个因子赋以相等的权重。
    \item 回归赋值:回归赋值是指在组合各个因子时,我们对某个时间区间上的收益率和参数因子进行最小二乘法回归,回归所得的系数向量即为各个因子的权重向量。
    \item IC-IR 因子赋值:IC-IR 因子赋值是指在组合各个因子时,考虑因子的 IC 序列,优化因子组合的 IR 值,取使得 IR 值最大的组合权重为最终的权重。
\end{itemize}

\paragraph{基于因子库选股} 在完成了因子的筛选和组合之后,就基本建立起了自己的 Alpha 因子库。基于这个因子库,可以筛选出这些因子较为突出的股票,并通过这些股票实现因子的超额收益。常见的选股方法有两种,分别是打分法和回归法。
\begin{description}
    \item[打分法] 打分法就是根据各个因子的大小对在一定时间内(如每 2 周)对一篮子股票进行打分,按照一定的权重相加得到一个总分,通过分数的高低进行股票的筛选,如购买前 50 名股票。基于周期打分循环,每 2 周调一次仓位。打分法的特点是比较稳健,不易受到特殊值的影响。
    \item[回归法] 回归法就是用过去的股票收益率对多因子模型进行回归,得到回归方程,把最新的因子值代入回归方程中得到一个对于未来股票值的预测,根据这个预测来进行股票的筛选。回归法的优点是能够比较及时地调整股票对各个因子的敏感性,但是回归法比较容易受到极端值的影响,导致选股失败。
\end{description}
\subsection{期权波动率套利策略}
期权的套利可分为无风险套利和风险套利。

无风险套利以平价公式套利为核心,辅以贴现套利、盒式套利等,其原理主要是捕捉市场交易价格与其理论价值之间差异的交易机会,并通过行权机制予以套利空间锁定的保障。由于机构投资者的大量参与及市场交易机制效率的提升,无风险套利的应用空间已大幅缩窄,策略市场容量也相对较少。

风险套利原则上都是在试图尽可能剥离掉其他因子的影响后,对期权组合中的某一风险因子进行“低买高卖”实现获利目的,如波动率、相关性及时间价值等,因此对于风险因子高低程度及未来变化趋势判断正确与否决定了此类策略的最终损益,其中要数期权波动率套利策略最为常见。
\subsection{高频交易策略}
\subsubsection*{交易波动率的优势}

交易期权的策略主要可以分为两大类:交易标的方向和交易波动率,也就是常说的方向性交易和波动率交易。

在方向性交易中一般是不用考虑希腊值(Delta, Gamma, Theta 和 Vega)的,但是会暴露更大的风险。

波动率交易对于交易方向的优势就是更低的风险和更大的收益。大量的学术研究表明,股票的价格基本属于随机游走的状态,波动率是标的物(即期货)对数收益率的方差,而且存在均值回归,即可以通过研究历史隐含波动率来预测未来波动率的大小。一句话总结:标的物是涨是跌太难猜了,波动率变化更好猜。从概率上看,波动率交易比方向性交易的胜率更大。

\subsubsection*{波动率套利}
波动率套利的收益不依赖于标的资产的价格变动方向,而依赖于标的资产的价格波动情况,其核心是寻找期权的隐含波动率和市场的实际波动率的价差,并对其进行相应交易。换句话说,如果预测的波动率与期权的隐含波动率有显著不同,就可以通过相应的期权交易进行获利。需要补充的是,预期波动率指期权交易者根据市场情况与历史数据对未来的价格波动率做出的一种预测,是对未来波动率的一种估量;隐含波动率是指实际期权价格所隐含的波动率。

在众多波动率套利策略中,又以 Gamma Scalping 策略最为常见(简单):通过对波动率的预测,每天对冲 Gamma 以获得高抛低吸的利润。

首先需要用到当月或者近月的平价期权(ATM)通过对冲 Delta 的方式构造跨式期权。平价期权拥有更大的 Gamma 值,但是近月合约由于有更好的流动性,其敏感程度相对于远月合约要小得多,即 Vega 相对较小,即 Gamma 负责赢利,Theta 负责亏损。跨式期权具有 0 Delta、正 Gamma、负 Theta 和正 Vega 的特点。

若预测近期股市会出现暴涨暴跌行情,即预期波动率大于隐含波动率,则买入跨式期权。该策略的利润源自 Gamma 贡献的价值足够大来覆盖 Theta 的时间成本,也就是说 $(0.5\times s2\times Gamma-Theta)>0$。当行情真的有大波动时,其实现波动率要大于隐含波动率,Gamma Escaping 策略就会赢利。反之,若预测错误,Gamma 的赢利覆盖不了 Theta 的时间成本,策略就会出现亏损。

总的来说,就是:
\begin{itemize}
    \item 预期波动率大于隐含波动率,做多 Gamma。预测对了,当日赢利,反之亏损。
    \item 预期波动率小于隐含波动率,做空 Gamma。预测对了,当日赢利,反之亏损。
\end{itemize}
\subsubsection{动态对冲}
与买股票价格上涨,平仓获利离场一样,当 Gamma 的贡献足够大时,也需要对冲 Delta,获利离场(假设当天建仓 Delta 为 0,Gamma 会在第二天产生新的 Delta)。相对于 50ETF 期权,对冲物有 3 种,分别是 50ETF、上证50 股指期货和看涨看跌期权合成期货。效果最好的是用期权合成期货来对冲,其优势是杠杆高、成本低、流动性好。\tips{我觉得如何可以,应该使用 50ETF 对冲,期权的活跃度比较低}

有了对冲物之后,可以考虑如何进行动态对冲了。常见的有以下 3 种动态对冲方法。
\begin{itemize}
    \item 定时对冲:在一定时间周期内进行对冲,可以是每天收盘前对冲,也可以每隔 15 分钟或 30 分钟对冲来锁定部分日内波动。
    \item 阈值对冲:通过自动对冲算法,设定 Delta 阈值,突破时瞬间对冲锁定短时间波动赢利,例如等 Delta 冲到5000 时全部对冲,让 Delta 归零、再次积累。
    \item 智能对冲:通过 CTA 信号或高频信号来实时判断标的物的涨跌方向,当仍然有趋势时智能判断对冲 Delta 的数量,这样可以捕捉到更大赢利机会,而且节省手续费。
\end{itemize}

\subsubsection*{高频交易策略的类型}
\paragraph{高频做市商} 高频做市商策略是在交易所挂限价单进行双边交易以提供流动性。所谓双边交易,是指做市商手中持有一定存货,同时进行买和卖两方交易。这种策略的收入包括买卖价差、交易所提供的返佣和固定佣金。

每交易一笔都有返佣,返佣的数值一般很小(远远小于价格最小变动单位,比如若设价格最小变动单位为一分,那么报价只能取 100.1 元或 100.2 元),但如果交易笔数巨大,积少成多,便可以成为不菲的收入。通过赚取返佣,做市商只需要保证每笔交易不赔即可,并非一定要追求低买高卖,反而要保证自己的委托单尽可能多地被执行,以争取更大的流量。为了做到这一点,下单和改单的速度是个关键,这也是为什么这一行如今被以速度见长的高频交易商把持的原因。

佣金比返佣更吸引人。做市商只需保证每月或每天参与一定规模的交易,就可以再额外从交易所处获取一笔不小的收入。这种模式的好处在于,做市商不仅不用追求买卖价差,甚至连流量也不需争抢,只要完成限定的额度即可,难度大大降低。

\paragraph{高频套利} 套利策略注重两种高相关性的产品之间的价差。比如说一个股指 ETF 的价格,理论上应该等于组成该 ETF 的股票价格的加权平均。但因为种种原因,有时会发现市场上这两种价格并不一致,此时即产生套利机会,可以买入价低一方,同时卖出价高一方,以赚取差价。随着市场流动性的增强,这种机会发生的次数越来越少,规模越来越小,并且机会经常转瞬即逝,因此往往需要借助高频交易的技术来加大搜寻的规模和把握交易时机。例如,其应用场景可以是跨市套利。

\paragraph{短趋势策略} 短趋势策略即意味着预测一定时间内的价格走势。相对于低频的趋势策略,高频交易的主要数据源是比 Tick 级别数据更精确的交易委托账本(Order Book Events),所以可以在委托单的粒度上进行分析和预测来抓大单的动向。Tick 级别数据其实就是一种对交易委托账本的降采样,其前提假设是:最佳买卖价是最重要的信息,以丢弃其他相对不如这个重要的信息为代价,缩减数据规模,让数据处理变得更容易。
\section{宽客的两大阵形:P 宗与 Q 宗}
Q 宗根本在于风险中性测度,即不存在无风险套利机会,换句话说可以完美对冲各种风险。因为其定价是基于不存在无风险套利的,这是一个非常虚幻的假设,其模型要求得到一个很好的数学解析解,必定很注重模型,而对实盘上的历史数据不太重视。所以 Q 宗宽客“重模型而轻数据”,涉及的数学知识主要是随机过程、偏微分方程等分支。

P 宗的“P”是指真实概率测度。所谓真实,主要指模型依赖的概率分布是从历史数据上估算出来的。最多只能说是从真实数据上估算出来的,显然没有什么东西保证历史一定会重演(比如黑天鹅)。从定义可以看出这套方法主要依赖数据,数据量越大估算的效果越好,表现为“重数据而轻模型”。在实践中,对冲基金或者各大投行券商的自营部门拿到一组数据时,会用若干备选模型来“跑”,由计算结果来选择最佳的模型,涉及的技术主要是计量、时间序列、更加复杂的统计学习/机器学习。不难看出,为了倒腾数据,这套方法练到上层就要开始“刷装备”。在电子化时代这最终演化为拼机房的“军备”竞赛。

从应用上来讲,Q 宗是模型固定,用数据来精化模型的参数;而P宗
则可以有若干备选模型,由数据的计算结果来选择最佳的模型。

Q 宗可以让你在缺少数据的情况得出一些结论,从而可以凭空制造一些东西出来,所以卖方(投行)用来做衍生品定价,业务模式是开发新的衍生品出来卖出去。P 宗则喜欢数据量大,这天然就是买方所需要的技术,因为他们本来就需要针对大量证券做出筛选和投资决策,业务是数据驱动的。

从区别就可以看出两者在发展方向上的不同。本质上说,Q 宗属于“制造业”,大家比的就是造出更多更好的衍生品来卖,但如果生产出来的东西没人买,生意显然就做不下去。而 P 宗其实属于“服务业”,那些数据技术不会给你创造出什么新产品,而是通过对本来就存在的业务(比如投资决策)进行精细加工来达到优化的目的。