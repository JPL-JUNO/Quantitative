\chapter{量化股票组合管理的基本模型}
\begin{table}
    \begin{threeparttable}

        \centering
        \caption{量化股票组合管理的基本模型如何确定股票的期望收益率与风险}
        \begin{tabular}{lll}
            \hline
                          & 基本面因子模型                   & 经济因子模型                    \\
            因子暴露($\beta$) & 可直接观察到                    & 通过时间序列回归来估计               \\
            因子溢价($f$)     & 通过截面回归来估计                 & 可直接观察到\tnote{1}           \\
            期望收益率         & 因子暴露$\times$因子溢价          & 因子暴露$\times$因子溢价          \\
            风险            & 因子暴露$\times$因子溢价风险$+$特定风险 & 因子暴露$\times$因子溢价风险$+$特定风险 \\
            \hline
        \end{tabular}
        \begin{tablenotes}
            \footnotesize
            \item[1] 占一定比例的因子溢价可直接观察到。取决于所使用的因子,因子溢价可以通过构建零投资组合或主成分分析来确定
        \end{tablenotes}
    \end{threeparttable}
\end{table}
\section{基本模型的等价}
\begin{lemma}
    如果股票的期望收益率是基本因子暴露的线性函数,那么基本面因子模型与经济因子模型等价。
\end{lemma}
\section{选择正确的模型}
\begin{table}
    \centering\caption{选择正确模型的标准}
    \begin{tabular}{lll}
        \hline
        标准     & 基本面因子模型 & 经济因子模型 \\
        \hline
        理论基础   &         & 更好     \\
        因子的适用性 &         & 更好     \\
        易于实现   & 更好      &        \\
        数据要求   & 更好      &        \\
        直观吸引力  & 视情况而定   & 视情况而定  \\
        \hline
    \end{tabular}
\end{table}