\chapter{保持良好的记录习惯}
市场在分发奖惩方面并不始终一致。这样的情况时有发生,比如一笔缺乏计划的交易赚钱了,而一个计划周密、执行认真的交易却亏钱了。这种随机性使我们颠覆了本应遵守的原则,鼓励草率地进行交易。

好的记录交易日志的习惯是培养和坚持纪律性的最好工具。它将心理、市场分析、风险管理联系到了一起。

交易日志的三个核心要素:
\begin{enumerate}
    \item 纪律的第一步是完成功课
    \item 纪律的进阶是写下你的交易计划
    \item 纪律的高潮是执行这些计划并且完成交易日志
\end{enumerate}
\figures{fig57-1}{每日功课电子表格}
\begin{enumerate}
    \item 查看远东市场。
    \item 查看欧洲市场。市场闻鸡起舞,你会体会到在美国产生的风波在重新返回西海岸之前,如何波及亚洲,然后传到欧洲的。
    \item 经济日历。当一份重要的数据,比如失业率或产能利用率,低于或超出市场预期,你便可以期待市场将出现绚丽的烟花秀。
    \item \href{https://www.marketwatch.com/}{Marketwatch} 网站。这是一个大众流行的网站,通常来说它是反向指标。
    \item 欧元汇率。我会写下最活跃的期货合约的现价,后面用动力系统状态的首字母标记——绿色(G)、蓝色(B)或者红色(R)——前面是周线图的,然后是日线图的。下面提到的其他市场,我所采用的是同样的格式。我关注欧元期货走势有两个原因。第一个原因是无论与美国股市表现一致或相反,欧元期货的走势都能延续一段时间;另一个原因是欧元期货有时候能提供非常好的日内交易机会。
    \item 日元汇率。上一条所述两个原因中,第二个原因比第一个在日元汇率上更适用。
    \item 原油。它是经济的血脉,并且原油期货会随着其上涨下跌而变化,原油期货是可以用来交易的。
    \item 黄金。它是市场恐慌情绪与通胀预期的一个敏感指标,同时也是很受欢迎的交易品种。
    \item 债券。利率的上涨或下跌是股市走势的主要驱动因素之一。
    \item 波罗的海干散货运价指数(BDI)。对于世界经济而言,它是一个敏感的先行指标。BDI 表示干散货的运送成本,例如把纺织品从越南运往欧洲,或是把木料从阿拉斯加运往日本。BDI的波动非常大,没有直接基于 BDI 的交易品种,这有助于BDI更准确地反应经济活动的实际情况。如果你交易航运业的股票,这个指标格外有用。
    \item 新高-新低指数。我认为新高-新低指数是股票市场最好的先行指标。
    \item 芝加哥期货期权交易所波动率指数(VIX 指数),也被称为“恐慌指数”。人们调侃:“VIX 走高,放心买入;VIX 走低,小心慢行。”横批是,“提防 VIX 的 ETF”——VIX 的 ETF因在交易中与 VIX 指数不同步而臭名昭著。
    \item 标准普尔500指数。写下前一个交易日指数的收盘价,并且将动力系统周线和日线显示状态的首字母标写在后面。
    \item 日线的价值。转到标普指数的日线图,留意最新一根线柱是收在价值区域的上方、正中还是下方,以及它与通道线的关系。这帮助我识别现在市场是超买了还是超卖了。
    \item 强力指数指标。注意这个指标的 13 日 EMA 均线是在它中心线的上方还是下方(对应牛市或熊市)以及是否有背离。
    \item 对标普指数的预判。测验自己对市场预测的精准度:写下对今天收盘价会比开盘价高还是低的预测,如果没有观点就空着。根据自己的预测是否正确,次日我会给这一栏涂上绿色或红色。
    \item 在电子表格的最后一行,总结今天将如何交易:积极地、保守地、防御地(仅进行平仓交易),日内的交易或者完全不进行交易。
\end{enumerate}
\subsection*{今天你准备好交易了吗}
有时候你会觉得踩准了市场的节奏,但其他时候你会和市场脱节。你的情绪、健康以及时间的压力会影响你的交易操作。
\figures{fig57-2}{“我做好交易的准备了吗?”自我测试}
\section{制作并评价交易计划}
任何交易计划都要依据所采用的策略量身定做。交易计划必须能提示你检查财报期、分红派息日期,以及期货交割日期,使你避免被可预见的新闻所侵袭。它必须清楚地记录你计划好的买入价、目标价、止损价以及交易规模。

在进入交易之前,先写下计划能帮你在风暴中建立一个理智和稳定的堡垒,它能帮助你不会忽略任何必要的事情。
\subsection*{为你的交易计划评分(交易的阿氏评分)}
\figures{fig58-1}{交易阿氏评分在做多交易(此例为“假突破伴随背离”的策略)的使用}
将你对五个问题的答案按 0-2 分进行打分(\autoref{fig58-1}):
\begin{itemize}
    \item 强力系统的周线图(前面章节有描述)——周线图是红色得 0 分,周线图是绿色得 1 分,周线图是蓝色得 2 分。强力系统为红色时,是禁止交易的;绿色时还可以进行交易,但是可能有些太晚;蓝色(紧跟在红色之后)表示恐慌正在褪去,是买入的好时机。
    \item 强力系统的日线图——与上一条同样的问题、同样的评分,标记在日线图上。
    \item 日线价格——在日线图上,如果最新价格在其价值区间之上得 0 分;在价值区间范围内得 1 分;低于其价值得 2 分。价格在价值区间之上时,买入已经有些迟了;在价值区间内还可以;在价值区间之下则是一笔好买卖。
    \item 假突破——没有的话得 0 分;已经发生得 1 分;很有可能将要发生得 2 分。
    \item 完备性——没有周期符合得 0 分;有一个符合得 1 分;两个周期看起来都很完备得 2 分。我通常会用两个时间周期来分析市场。对任何策略来说必须有其中之一符合一种入场交易的完备形态。极少情况下两个时间周期的形态都是完备的——在一个完备,另一个可以接受的情况下就可以进行交易了。如果没有一个时间周期的形态看起来是完备的,则不是一笔 A 类交易——抛弃这只股票,转移到另一只上面去。
\end{itemize}

\figures{fig58-2}{交易阿氏评分在做空交易(此例为“假突破伴随背离”的策略)的使用}
\subsection*{使用交易表}
当你对某只股票产生兴趣,并且交易的阿氏评分肯定了你的交易想法,完成交易表将有助于你专注于此交易最核心的部分。

\figures{fig58-3}{交易表在做多交易(此例为“假突破伴随背离”的策略)的使用}
\begin{description}
    \item[交易鉴定] 画出大概的 K 线样式来标示出这种策略。填写股票代码、记录下一个财报披露的日期、记录除息日、做计划的日期。
    \item[交易的阿氏评分] 当你将交易阿氏评分的各项得分加总时,将下面这个重要问题的答案写下来:这会是一笔 A 级交易么?如果总分在 7 分以下,则放弃这只股票,去寻找其他的。
    \item[市场、买入点、目标价、止损点和风险控制] 最左边的五个空格要求我回答有关市场基本状况的问题。尖峰反弹信号是否有效,追踪股票均线的指标是看多还是看空,这只股票的空头净额是多少,需要多少天来补上,所有这些内容都已在本书前面描述过。最后一个空格是简短的总结。用箭头所连接的三个空格是我决策制定过程的核心部分。它们所要的是每笔交易最重要的三个数字:买入价、目标价、止损价。资金风险——这笔交易中,你愿意冒亏损多少钱的风险?这个数额永远不应该超过你账户资产的 2\%。我通常把它控制在远远低于这个门槛的位置。持仓规模——根据持仓限额和入场点与止损点的差额,可以算出你能买多少数量。
    \item[买入之后] A 级盈利目标是在买入价上加日通道线高度的 30\%。软止损是记在脑海中的指令,而硬止损或灾难性止损是实在的指令。它不应该比第三部分中所写的止损价低。记下你将把止损位移到盈亏平衡位置的价格水平。当你执行这些必要步骤时,检查右手边的方框:设置止损价,创建一个日志,下达止盈订单。
\end{description}
\section{交易日志}
A 部分:交易日志需要回答为何决定交易这只股票。

B 部分:记录下入场和退出的日期和价格。记录滑点和买入量、卖出量及交易等级。

C 部分:退出的原因,需要附上显示入场点和退出点的合成线图。

D 部分:退出策略的清单要比交易策略的清单长。退出的原因可能是到了目标价或止损价,也可能是到了价值区间或包络线。我也可能会因为股票不再延续趋势方向或者开始掉转方向而选择退出。还有两种消极的退出:已无法承受下跌的痛苦,或是买入后发现这是一笔糟糕的交易。

E 部分:交易后的回顾分析。我喜欢在退出交易两个月后回顾这笔交易。我设计了一个跟踪图表,用箭头标记出入场点和退出点,然后写下时过境迁后对这次交易的评论。这是吸取经验教训最好的方法。
\figures{fig59-1}{网页版交易日志(部分)}

在退出交易一两个月之后,回顾每笔交易是最好的学习方式之一。交易信号在图形右侧时,可能会显得模糊不定。而当你在图形中间看到它们时,已变得无比清晰。回顾你已经完成了的交易,并且加上一个“交易后”图表,能让你重新评估自己当时所做的决定。现在你可以清楚地看出自己做得对还是不对。你的日志能给你珍贵的经验和教训。