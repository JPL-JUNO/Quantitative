\chapter{群体心理}
四种动物在华尔街被提及的次数尤其多,分别是牛、熊、猪和羊。交易者们常这样调侃:“牛赚钱,熊赚钱,唯有猪羊被屠杀。”牛在战斗时总会扬起自己的角,因此象征多头。多头也就是买方——判断价格会上升,从而买入来获取利润。熊在战斗时则用其掌从上往下攻击敌人,因此象征空头。空头也就是卖方——判断价格会下跌,从而卖出来获取利润。

猪象征着这样一类贪婪者:他们或持有超出他们承受水平的头寸,只要价格发生一丁点与他们所持头寸相反的波动,他们就难以承受;或持有头寸的时间过久,当趋势已发生逆转,他们却依旧视若无睹地等待着利润增长。羊则象征着那些被动且胆小的从众者,他们跟从着趋势、小道消息和各种所谓的专家大师,他们时而受困于牛市,时而被围于熊市,你可以根据市场波动时那些满是懊悔的控诉来感受到他们的存在。

只要市场在运行,牛就会买,熊就会卖,而那些猪羊则被滚滚人流践踏于脚下,犹疑不决的交易者则在买卖的边界上观望等待。世界各地的报价屏幕上,交易标的物的最新价格信息不停地滚动,无数双眼睛紧紧地盯着它们,人们需要根据这些价格信息做出自己的交易决策。

\section{市场是什么}
\subsection*{群体,并非个体}
千万不要逆趋势交易。一旦上涨趋势得以确立,你能做的只有买进或者在一边观望。千万不要仅仅是因为感觉价格过高而去卖空——再次强调,不要和趋势作对。没有哪个规定说你一定要和人群一同奔跑,但你至少不应该逆着人群行进。

\section{交易情景}
\subsection*{个人投资者}
学习交易会耗费很多时间、金钱和精力。极少数个人交易者能达到职业水准,并靠交易养活自己。职业交易者对自己的行为极为严肃,他们在市场之外去满足自己的非理性目标,而业余的交易者则把非理性表现在市场中。

交易者的主要经济角色是供养他的经纪人——帮助他们支付抵押贷款和使他们的孩子能够在私人学校中读书。另外,投机者的角色是帮助企业在股市中筹集资金,承担商品市场上的价格波动风险,使生产者可以专注于生产。然而,当投机者下单时他们根本不会考虑到这些高尚的经济目标。
\section{你与市场群体}
看涨者与看跌者共同组成了一个松散的市场,并在其中赌未来价格的上涨和下跌。每个价格都代表市场某个时点的群体共识,因此我们也可以认为——交易者实际上都是在赌群体的未来观点和情绪。群体情绪游弋在乐观与悲观、希望与恐惧之间。大多数人无法完全贯彻执行他们的交易计划,因为很容易便在群体情绪和行为中失去了自我。

你的投资行为随着市场上的多空博弈战况不断变化。你无法控制市场,你能自主决定的只有你的头寸规模、入场与离场的选择。

大部分交易者入市之初会感到忐忑不安。一旦加入某种群体,交易者的大脑就会被一片迷雾所笼罩。受影响于群体情绪,很多交易者背离了他们的交易计划,最终只能被迫蒙受巨大的损失。

\subsection*{群体研究}
只要加入了某一群体,人就会发生变化。他们会变得更容易轻信别人,更冲动,更急切地寻找领导者,他们的行为将倾向于情绪化,而不是理智。换句话说,个人加入某一群体之后,独立思考的能力将会显著降低。

交易领域,一个群体中的交易者偶尔会抓住一些趋势,并短暂获利,但当趋势反转时他们就很可能蒙受巨大的损失。成功的交易者必须独立思考。
\subsection*{独立}
在进行交易之前,你需要认真地准备好你的交易计划,切忌随着价格的瞬息万变而情绪化交易。最好是可以亲手将你的计划书写在纸面上,这样可以更确切地认识到自己应该在什么条件下开始交易或者退出交易。交易的过程中不要肆意制订交易计划,否则你就会很容易被群体情绪同化。

唯有坚持长期以独立的个体角度进行思考和操作,你才能成为一个成功的交易者。
\begin{tcolorbox}
    在交易的过程中,你必须时刻留意自己,关注自己精神状态的变化。写出自己入场与出场的条件,包括资金管理规则。只要手上还有仓位头寸,就坚决不能修改计划。
\end{tcolorbox}
\subsection*{积极的群体}
维持一个积极的群体,最关键的是要做到保证所有的选股和交易方向都必须在独立的情况下做出,不能窥探团队领导和其他成员们的工作。分享交流可以在所有参选结果都收齐之后进行。