\chapter{成交量和时间}
很多交易者只把注意力放在价格行情上,尽管价格行情十分重要,但市场所包含的远远不只有价格一个维度,成交量给我们提供了另一种极具价值的参考维度。

另一个在市场分析中十分重要的因素是时间。市场在同一时刻按照不同的时间周期同时存在并发展着。不管你多么仔细地分析某个日线图,它的趋势都很可能被另一个时间周期的运动所颠覆。
\section{成交量}
成交量的变化体现出多空双方对价格波动的反应,并且蕴含着趋势是持续下去还是发生反转的线索。

这里有三种测量成交量的方法:
\begin{enumerate}
    \item 股票或合约的实际成交量。
    \item 发生的交易笔数。
    \item 跳动量是指在某一选定时段内——例如10分钟或者1小时内,价格的改变量。它之所以被称为跳动量是因为大部分改变是在跳动的一瞬完成的。一些交易场所并不报告一天内的成交量,这样便驱使交易者使用跳动量作为真实成交量的近似值。
\end{enumerate}
\subsection*{群体心理学}
成交量还反映了众多市场参与者之间金钱和情感的卷入程度,就像疼痛一样。交易开始于两个人之间财务上的投入。买或者卖的决策或许是理性的,但是买或卖的行为却引起了大部分人情感上的投入,买方和卖方都渴望自己是正确的,他们对着市场尖叫、祈祷或者运用幸运符,成交量水平也反映了交易者们的情感投入程度。

每一次价格的跳动都夺走了失败者的金钱并将其转移到了胜利者的腰包。成交量越大,市场里包含的痛苦就越大。

如果一个突然的价格巨变袭击了交易者,他们会立马远离危险并斩仓止损。反之,失败者会对于损失的缓慢增加很有耐心。

成交量降低意味着输家供应变少了,趋势即将发生反转,这经常发生在足够多数量的输家知道他们判断错误了之后。老的输家持续出局,但是越来越少的新的输家加入他们。成交量的降低是趋势即将发生反转的信号。(\textbf{成交量低会延续很久})

在下降趋势中的成交量迅速上升相比于上升趋势中的成交量迅速上升更有可能意味着趋势即将反转。下降趋势中的成交量骤增反映出恐惧情绪的爆发。恐惧是一种极具力量但持续时间又较短的情绪——人们飞快逃离,丢弃手中的筹码,然而这时趋势很可能反转。上升趋势中的成交量骤增往往由贪婪驱动,而贪婪是一种缓慢变化并且带来快感的情绪。上升趋势中的成交量骤增或许会使得趋势短暂停止,但之后趋势更可能继续上涨。

当成交量在上涨过程中缩水时,这意味着多头的买入意愿不再强烈,同时空头也不再逃跑。那些精明的空头们很早之前就逃离了市场,跟在他们后面的是那些无法忍受亏损痛苦的脆弱空头。成交量的降低表明支撑上升趋势的燃料不足,趋势将发生反转。

若在下行趋势成交量中枯竭,这表明空头不再急切地卖空,同时多头也不再仓促逃离市场。精明的多头们很早之前就已清仓走人,脆弱的多头们也已被消灭。成交量的降低表明留下来的多头对于亏损的容忍力极高,或许是因为他们不缺钱,或许是因为他们买入的价位较低,或者二者都有。萎缩的成交量暗示着下降趋势将有可能得到反转。

\subsection*{交易指针}
“高成交量”和“低成交量”这两个概念都是相对而言的。大部分时间,我们将股票当下的成交量与其历史平均成交量来进行比较。作为经验法则,在任何一个市场上,“高成交量”意味着高于其过去两周成交量平均值 25\% 以上的成交量,而“低成交量”意味着低于其过去两周成交量平均值 25\% 以上的成交量。
\begin{enumerate}
    \item 高成交量可以确认趋势。如果价格和成交量同时达到新的顶峰水平,价格将很可能保持高位或超过前期顶部后再创新高。
    \item 如果市场价格创下新低但成交量却创下新高,则该底部将再次确认或者创新低。一个极高成交量的底部后面常常会跟着一个较小成交量的底部,这时便是绝佳的买入机会。
    \item 如果在趋势持续的过程中成交量缩水,那该趋势将发生反转。市场达到新顶峰时,其对应的成交量却不及达到上一个顶峰时的成交量,你就应该兑现多头头寸上的盈利或者抓住做空的机会。但这一技巧并不一定在市场处于下降趋势中时管用,因为下降趋势可能以一个很低的成交量来持续下去。华尔街流传着这样一句话:“将价格拉上去需要大家来买入,但价格会自行下降。”
    \item 观察趋势中反弹的成交量情况。当上升趋势出现回落时,由于慌张的获利了结盘,成交量会增加。当这种回落持续但成交量缩水时,表明多头不再逃跑或者抛压被消耗。当成交量耗尽时,下跌带来的抛售效应已经接近其尽头,上升趋势将重新开始。此时便是一个很好的买入机会。主要下降趋势也常常被高成交量的反弹所打断,一旦脆弱的空头被消灭干净,随后成交量缩水便是卖空的信号。
\end{enumerate}
\section{以成交量为基础的指标}
种种基于成交量的指标相对于成交量柱来说提供了更加精确的时机信号。这些指标包括下面将介绍的能量潮指标(OBV)和集散指标(A/D)。强力指数指标将价格和成交量数据结合起来,帮助我们发现在哪个区域价格可能发生反转。
\subsection{能量潮指标}
能量潮指标将成交量进行滚动加总,每日的成交量会被加入或减去,这取决于当日价格比昨天价格高还是低。若股票当日收盘价高于前一日收盘价,这意味着当日多头在与空头的交战中取得了胜利,那么当日的成交量就会加到 OBV 上;若股票当日收盘价低于前一日收盘价,这意味着当日空头在与多头的交战中取得了胜利,那么当日的成交量就会从 OBV 中减去;若股票当日收盘价与前一日收盘价持平,那么 OBV 保持不变。OBV 常常在价格变化之前先行变化,起到领先指标的作用。
\subsubsection*{群体心理}
一个创新高的 OBV 表明多头势力强大,空头受到重创,价格将会上升;一个创新低的 OBV 表明空头势力强大,多头受到重创,价格将会下降。当 OBV 的走势偏离了价格的走势时,这意味着大众的情绪和大众对价格的共识相偏离,人们将更可能跟随直觉而不是理性的思考,这也是为什么成交量经常先于价格变化而变化。
\subsubsection*{交易信号}
OBV 顶峰或者底部的形态比其绝对值重要,因为 OBV 的大小取决于计算
的基准日。根据 OBV 确认的趋势方向进行交易会更加安全。
\figures{fig29-1}{麦当劳(MCD)是一只变化缓慢且走势稳定的股票。你可以看到它的交易价格波动区间很窄,该区间被几条虚线标示了出来。(底部有两条虚线,一条低一些,一条高一些)注意到 MCD 的假突破倾向(图中底部的 A、C 区域以及顶部的 B、D 区域),A 区域形成了袋鼠尾。在图形的右边,股票市场处于自由下跌阶段,然而当 MCD 股票在其近期的低价区域交易,其能量潮指标却达到高水平时,此时你应该选择买入而不是卖出。}
\begin{enumerate}
    \item 当 OBV 达到新高时,便确认了多头的力量,表明价格将很可能持续上涨,并给出买入的信号;当 OBV 跌穿上一个底部值时,便确认了空头的力量,价格将可能继续创新低,给出了卖出信号。
    \item 当 OBV 与价格相背离时,便是强烈的买入或卖出信号。如果价格上涨趋势出现了回调,随后反弹创新高,但 OBV 却没有创新高,这便产生了熊市背离,是卖出信号;如果价格在下降趋势中出现了反弹,之后再创新低,但 OBV 却没有创新低,这便产生了牛市背离,是买入信号。长期背离比短期背离更加重要,一个长达几周的熊市背离比起短短数日的熊市背离来说给出的卖空信号强度更大。
    \item 当价格在震荡区间,但 OBV 却突破新高时,便出现了买入信号;反之,当价格在震荡区间内,但 OBV 却创新低时,便出现了卖出信号。
\end{enumerate}
\subsubsection*{更多关于 OBV 的信息}
格兰维尔之所以能在股市中成功择时的另一个原因是他将能量潮指标(OBV)和净趋势指标(net field trend indicator)以及峰值指标(climax indicator)结合了起来,他将道琼斯工业指数里每一只成分股的OBV计算出来并将这些指标划分为上升、下降、持平三类,他将其称为个股的净趋势值,该值可取 1,0,-1。峰值指标是所有 30 只道琼斯工业指数成分股的净趋势值之和。

当股市反弹并且峰值指标达到新高时,此时便出现了很强的买入信号;当股市反弹但峰值指标却没有创新高时,便出现了卖出信号。
\subsection{集散指标(A/D)}
集散指标的独特之处在于,除了成交量之外,它追踪的是开盘价和收盘价之间的关系。

集散指标比 OBV 指标得到更好的调整,因为它只用当天交易量的一定比例归入到多头或空头势力中去,按照双方当日获胜的程度作为其比例。其公式如下:
\begin{equation}
    A/D=\frac{\text{Close}-\text{Open}}{\text{High}-\text{Low}+\epsilon}\times \text{Volume}
\end{equation}
如果收盘价高于开盘价,当日多头获胜,则 A/D 值为正;如果收盘价低于开盘价,则当日空头获胜,A/D 值为负;如果收盘价和开盘价一致,则当日多空双方打成平手,A/D 值为0。所有日 A/D 值的滚动计算创造了累计 A/D 的指标。
\subsubsection*{群体行为}
开盘价反映了市场闭市期间在价格上积累的压力,开盘价更有可能被那些晚上读新闻、早上交易的业余投资者们所主导。

A/D 值追踪了业余投资者和专业投资者当日对战的结果,当收盘价高于开盘价时,即专业投资者比业余投资者更倾向于看多时,A/D 值将上升;当收盘价低于开盘价时,即专业投资者比业余投资者更倾向于看空时,A/D 值将下降。与专业投资者同方向下注、与业余者反方向下注比较有利。

\subsubsection*{交易准则}
当市场中开盘价低于收盘价时,表明市场由弱转强,这时候 A/D 值上升,并给出专业投资者比业余投资者更倾向于看多的信号,此时上升将很可能持续;当 A/D 值下降时,表明专业投资者比业余投资者更倾向于看空,市场将在当日持续走低,第二天将很可能继续走低。

最好的交易信号出现在 A/D 值背离价格走向时。
\begin{enumerate}
    \item 如果价格上涨创新高但 A/D 值却没有随之创新高,则出现了卖出信号。这种熊市背离表明市场上专业投资者在上涨过程中卖出。
    \item 牛市背离发生价格创新低但 A/D 值却没有创新低时,表明了市场上专业投资者在下跌过程不断买入并逐步建仓,反弹即将到来(见 \autoref{fig29-2})。
\end{enumerate}
\figures{fig29-2}{谷歌公司的股票 GOOG 持续下跌已经数月了,但是其上升的 A/D 值表明大资金正在不断地买入。该股票在B处创造了比 A 处更低的底部价格,但是其 A/D 值的底部却抬高了。同样重要的是,A/D 值在价格暴涨之前还创了新高(在图中由垂直箭头标出),紧接着公司就公布了好得出人意料的业绩报告。一些人之前便知道了接下来将要发生的事,他们大规模的买入被累积 A/D 值形态和向上的突破所确认。技术分析有助于减少外部投资者和内幕信息知情人之间的信息地位差距。}
\subsubsection*{更多关于 A/D 值}
当你根据A/D值与价格之间的背离为依据,进行做多或做空的时候,记住就算是市场上专业的投资者也会犯错。你需要设置止损单并用巴斯克维尔的猎犬法则来保护自己。
\section{强力指数指标}
强力指数指标将价格和成交量相结合来发掘每一次上涨或下降过程中多头或空头的力量。强力指数指标可以应用于所有有成交量数据的价格图中,包括周度、日度或日内数据。它将三条至关重要的信息汇集到一块:价格改变的方向、程度以及在其改变过程中所对应的成交量。

强力指数指标可以以其最原始的形式使用,但若我们将其用移动均线平滑化,其信号将会显示得更加清晰。
\subsubsection*{如何构造强力指数指标}
市场每一次变化的力量都被划分为三个因素:方向、距离以及成交量。
\begin{itemize}
    \item 如果当日收盘价比前一日收盘价高,力量值便为正;如果收盘价比先前的收盘价低,力量值便为负。
    \item 价格变动幅度越大,力量值越大。
    \item 成交量越大,力量值越大。
\end{itemize}
\begin{equation}
    \text{强力指数指标}=\text{今日成交量}\times(\text{今日收盘价}-\text{昨日收盘价})
\end{equation}
\subsubsection*{交易心理学}
当市场收涨,表明多头在当日的多空之战中取得了胜利;当市场收跌,则表明空头赢取了当天的胜利。今天与昨天收盘价之间的差距表明了多头或空头的获胜程度,差距越大说明胜利的程度越大。

成交量反映了市场参与者的情绪卷入程度。高成交量支撑下的价格上涨和下降有着更大的惯性并且更可能持续。高成交量下的价格走势就像在雪崩中不断积累速度的雪球一样。另外,低成交量表明输家的数量很少,意味着趋势将要走到其尽头。

当强力指数指标上涨到新高时,表明多头的力量很强大,上升趋势将得到持续;当强力指数指标下降到新低时,表明空头的力量很强大,下降趋势将继续维持;若价格的改变并没有得到成交量的确认,强力指数指标将变得平坦,从而给出趋势将要反转的信号。如果价格波动很小,但成交量十分巨大,强力指数指标也将变得平坦,同样意味着反转即将到来。
\subsubsection*{短期强力指数指标}
2 日 EMA 是追踪短期多空头力量的高敏感性指标,当其向上击穿中心线时,表明多头力量更为强大;当其向下击穿中心线时,表明空头力量更为强大。

因为 2 日 EMA 是一个敏感的指标,我们可以用其来对其他指标给出的信号进行微调。当趋势跟随指标确认了上升趋势时,同时 2 日 EMA 值下降到 0 以下,此时便是一个绝佳的买点——在长期上涨趋势中的回调期进行买入(见 \autoref{fig30-1})。当趋势跟随指标确认了下降趋势,2 日 EMA 的上升给出绝佳的卖出区域。

\figures{fig30-1}{以奥多比公司(Adobe Systems,Inc.,ADBE)的股票为例,上升的周线 EMA(在图中未显示)确认了稳定上升的趋势。当周线向上走时,日线图上的 2 日强力指数不断地提供了一系列确认买点的信号。比起追涨并买在高位,在短暂的回调期买入会更好,此时股票走势与其长期趋势相反。这些与趋势相反的波动坑,将在 2 日强力指数为负时被标记出来。一旦 2 日强力指数下降到 0 值以下,此时在最近一条线柱的高点上方设置限价买入指令,一旦下行浪失去势头时,能确保你买入股票,并在上升浪中获得收益。}

\begin{enumerate}
    \item 市场处于上升趋势时,在 2 日 EMA 变负时买入。

          就算是疯狂的上升趋势中也会有偶尔的回调,如果你能忍住冲动,在 2 日 EMA 值变负时买入,你将买在接近短期底部的位置。大部分人喜欢追涨,他们在回调的时候倍受打击并无法容忍。强力指数指标帮助我们发现低风险的买入机会。

          上升趋势中,当 2 日 EMA 值变负时,你可以在比当日最高价还要高一点的位置下一个买单,若上升趋势持续并且价格回升,你的限价买入指令将会成交。如果价格继续下降,你的指令则不会被执行。持续降低你的买入价格直到接近前一根线柱的高点。一旦你的限价买入指令被触发,在比最近低点还稍微低的价位设置一个保护性止损单,这个止损单在强上升趋势中不会被触及,但它可以让你在趋势走弱时及早退出。
    \item 市场处于下降趋势时,当 2 日 EMA 值变正时卖出。

          当趋势跟随指标确认了下降趋势时,你需要等到 2 日 EMA 值变正,此时是下跌过程中的短暂反弹,为卖空的好机会。在比近期低点价格稍低的价位设置卖空指令。

          如果 2 日 EMA 在你设置了卖出指令后依然持续反弹向上,下一个交易日将你的卖出指令价格提升到前一交易日的低点附近。一旦价格下降你的卖空指令将成交,在最近的高点上方设置一个止损单,尽早将你的止损价逐步调整到盈亏平衡的水平。
    \item 2 日 EMA 与价格走势出现牛市背离时便出现了强烈的买入信号。即当价格降到了新低,而 2 日 EMA 却没有再往下降到新低时,便发生牛市背离。
    \item 2 日 EMA 与价格走向出现熊市背离时便给出了强烈的卖出信号。即当价格上升到新高,而 2 日 EMA 却没有再往上冲到新高时,便发生熊市背离。
    \item 不管何时,2 日 EMA 若下跌深度比正常的深 5 倍以上,并从该区域反弹起来时,则可以预期在不久的将来价格将回升。
\end{enumerate}

此外,2 日 EMA 可以帮助我们决定什么时候建仓,你可以在上升趋势中每次强力指数指标变负时加仓,你也可以在下降趋势中每次强力指数指标变正时减仓。

强力指数指标甚至还可以让我们瞥见未来的一角。当 2 日 EMA 值在当月内降低到最低点,说明了空头力量十分强大,并且价格将降得更低;当 2 日 EMA 在当月内升高到最高点,这说明了多头力量十分强大,并且价格将升到更高水平。

2 日 EMA 也能帮助我们决定何时清仓。它通过确认短期反弹或回调来实现该作用。在 2 日 EMA 值为负的时候买入的短线交易者可以在该值变正的时候卖出;在该指标为正的时候卖出的短线交易者可以在该指标为负的时候买回。长线交易者只应在趋势改变的时候(如 13 日价格的 EMA 线的斜率所确认的价格方向)选择清仓,或者在 2 日 EMA 与价格走势出现背离时离场。
\subsubsection*{中期强力指数指标}
13 日强力指标 EMA 可以确认较长时间段里多空方力量平衡的变化。
\begin{itemize}
    \item 当 13 日 EMA高于中心线时,多头掌管了市场,当其低于中心线时,空头重占上风。

          当上涨开始时,价格通常伴随巨大的成交量而上升。当 13 日 EMA 达到新高,便确认了上升的趋势。当上升趋势老化时,价格上升越来越慢,成交量也开始缩减。此时,13 日 EMA 只能上升到较低的高点。当其下降到 0 值以下时,便意味着牛市的尾声已经过去了。
    \item 当 13 日 EMA 创新高时,表明多头力量十分强大,上升过程将很可能持续下去。13 日 EMA 与价格走向发生熊市背离时便给出了强烈的卖空信号,若价格达到新高但该指标却只能达到较低的高点,这便在警告多头已经丧失力量,空头正在逐渐掌管市场。我们需要注意,为了使背离有效,EMA 在创新高之后必须降到 0 值线以下,然后再回升到 0 值线以上,但是却无法突破前期高点创出新高,这便产生了背离。如果没有和 0 值线的交叉,就不是有效的背离。
    \item 13 日 EMA 创新低表明下降趋势将持续。如果价格创新低了,但该指标却回升到 0 值线以上,然后再次回落到 0 值线以下,但却无法跌穿前期的底部,这便形成了牛市背离,表明空头的力量在减弱,是买入的信号。当下降趋势开始时,价格的下降通常伴随着巨大的成交量,当 13 日 EMA 创新低时,便确认了下降趋势。当下降趋势老化时,价格下降得越来越慢,相应的成交量也逐渐缩小,这时随时都可能出现反转。
\end{itemize}
\section{持仓量 Open Interest}
持仓量是指在诸如期货和期权等衍生品市场上由买方或卖方持有的合约数量。

技术分析师通常将持仓量画成位于价格柱之下的折线图。任何市场上的持仓量都随着季节交替而不断变化,因为位于年度生产周期不同阶段,工业产品使用者和生产者会进行大量的对冲。当持仓量与其季节性常态相偏离时便向我们透露出某种重要的信息。
\subsection*{群体心理}
持仓量反映了多空双方交战的激烈程度。是否继续维持多头或空头头寸取决于他们的意志。当多空双方认为市场不会向有利于他们的方向变化时,他们将会选择减仓,使得持仓量减少。

持仓量的增加表明一群自信的多头在和另一群同样自信的空头对峙,说明了多空双方对价格走向判断的分歧加大。总有一方会沦为输家,但只要潜在的输家不断涌入阵营,趋势将持续下去。

在上升趋势中,持仓量增加代表多头买进而空头卖出,多头认为价格将继续上涨而空头认为价格已经太高。一旦空头受到上升趋势的挤压,他们将被迫回补——回补的买盘将进一步推高价格。

在下降趋势中,持仓量增加代表空头积极抛空而多头在底部承接,如果价格继续下跌,这些做投机交易的多头将被迫认赔退出,他们的卖压将驱使价格进一步下跌。

如果多头相信价格将走高而买进,但空头因为害怕而不愿意抛空,则多头仅能够从另一些持有合约的多头手中买入合约。他们之间的交易不会构成新合约,持仓量维持不变。如果持仓量在上涨过程中不再增加,代表输家的供给已经不再增长。

如果空头相信价格将走低而放空,但多方因为害怕而不敢接手,则空头仅能够将合约卖给另一些希望获利了结的空头。他们之间的交易不会构成新合约,持仓量维持不变。如果持仓量在跌势中不再增加,代表抄底者的供给已经不再增长。如果持仓量的走势趋于平坦,警告的黄灯已经亮起——趋势的发展已经进入末期。

当多头决定将自己的多头头寸平掉,同时空头也决定将自己的空头头寸平掉时,如果双方达成交易,那么成交的合约就从市场上消失,持仓量下降。持仓量的下降,预示着输家认赔出场的同时赢家获利了结,他们的分歧程度下降透露着趋势即将反转的征兆。持仓量减少,代表赢家把筹码兑换为现金,输家也放弃了希望,标志着趋势即将结束。
\subsection*{交易准则}

\begin{itemize}
    \item 当持仓量在价格涨势中增加,便确认了上升趋势,并确认多头增持筹码是安全的。这代表更多的空头在持续进场。一旦空头们认输回补,他们的买盘将进一步推升价格上涨。持仓量在价格跌势中增加,显示低位承接者相当活跃。空头可以继续加码,因为当低位接盘者认赔出场时,他们的卖压将进一步迫使价格下滑。

          持仓量在价格横向走势中增加,这是空头的征兆。在这种价格走势中,空头部位大多来自商业避险者而不是投机客。如果价格没有明显的趋势而持仓量暴增,代表着精明的避险者正在抛空行情,此时你应该避免和这些比你更有信息优势的避险者进行交易。
    \item 持仓量在价格横向走势中减少,代表商业避险者正在回补,这是买进信号。当商业使用者开始进行回补时,显示他们看多后市。

          持仓量在价格上涨趋势中减少,表明赢家与输家都变得小心谨慎起来。多头获利了结,空头认赔回补。市场反映的是未来,如果绝大多数人都已经接受了某个趋势,代表该市场趋势即将反转。如果持仓量在价格上涨趋势中减少,可以考虑了结多头头寸,准备做空。

          持仓量在价格下跌趋势中下降,代表空头获利了结,多头认赔出场。在这种情况下,应该回补空头仓位。
    \item 持仓量在价格上涨趋势中走平,说明涨势已经缺乏后劲,这是上涨趋势老化的警报。在这种情况下,多头仓位应该收紧卖出止损指令的触发价格,同时避免开新的多仓。反之,如果持仓量在价格下跌趋势中走平,显示下降趋势即将完结,最好收紧空头仓位的止损指令的触发价格。如果价格与持仓量都持平,此时没有显著的意义。
\end{itemize}
\section{时间}
\subsection*{周期}
长期价格周期在经济生活中是实际存在的。例如,美国股市往往是 4 年一个周期。之所以会这样,是因为每 4 年一次的总统选举使得执政党必须提振经济。已经赢得选举的政党不再需要选民的支持,因此就会打击经济。

农产品的主要周期取决于季节因素、基本生产因素和生产商的群体心理。

长周期有助于交易者识别市场潮流。相反,大多数交易者忙于用短周期来精确把握时点、预测细微的转折点,结果麻烦不断。

从图形上看,价格高低点似乎以某种秩序排列。图形上看起来像是周期的东西往往只是人们的幻想。如果你用严格的数学程序来分析价格数据,比如用约翰·埃勒斯(John Ehlers)发明的最大熵谱分析法(maximum entropy spectral analysis)来分析,你就会发现大约 80\% 的看似周期的东西都只是市场噪音。人类需要秩序,对于大多数人来说,即使只是幻想的秩序也是好的。
\subsection*{指标的季节性}
我们可以用两个要素来定义指标的季节,即它的倾斜方向和它相对于中线上方或下方的位置。例如,我们可以将指标的季节性用于 MACD 柱状线。我们将 MACD 柱状线的倾斜方向定义为两根相邻 K 线之间的关系。当 MACD 柱状线自中线下方上升时,则是春季;当它上穿中线时,进入夏季;当它从中线上方下降时,为秋季;当它下穿中线时为冬季。春季是做多的最佳时机,秋季是卖空的最佳时机。
\subsection*{市场时间}
我们要牢牢记住,相对于时间在我们个人身上的流动速度来说,时间在市场上的流动速度是不同的。市场由大量的人群构成,移动的速度更慢。你从自己绘制的数据图表上发现的模式可能具有一定的预测价值,但你预测结果的实际发生时间可能会比你预测的时间晚很多。

我们常常在市场还没彻底触底时买入,或者在市场还没完全封顶时卖出。行动的过早会让我们在发展缓慢的趋势当中以亏损出局。

我们应该怎么做呢?首先,我们应该认识到市场所经历的时间比我们自己所经历的时间要慢很多;其次,我们不应在刚发现早期的反转信号时就急于进场。在这个信号之后会有一个更好的反转信号出现来提示行动时机,尤其是在市场顶部的时候,相比于市场底部,反转需要更长的时间来形成。
\subsection*{数字 5}
分析市场的恰当方法是至少从两个时间周期来分析它。你在形成长远的价格走势判断时必须先根据长时间周期来进行分析,然后在选择市场买卖时点时,再参照短时间周期。如果你喜欢用日线图分析,你必须先检查一下周线图;如果你想要进行超短线交易从而运用 10 分钟图,你也得先分析一下小时图。
\section{交易的时间周期}
我们可以粗略地将所有的交易划分成如下三类。
\begin{itemize}
    \item 长期交易或长期投资。即持有仓位的时间以月为单位,有时甚至以年为单位。优点:可以获得可观的长期投资收益,同时避免每天盯盘对精力的消耗。缺点:体现在跌势中持有时间过长亏损可能会很严重,让人无法忍受。
    \item 波段交易。即持有仓位的时间以日为单位,有时候以周为单位。优点:有较多的交易机会,能做到严格的风险控制。缺点:可能会错过趋势中的主升浪。
    \item 日内交易。即持有仓位的时间以分钟为单位,也有以小时为单位的。优点:有很多的短线交易机会,不存在隔夜风险。缺点:需要对市场的快速反应能力,并且频繁地交易会带来较大的交易成本。
\end{itemize}
\subsection*{投资}
投资需要具有坚定的信心和强大的耐心,尤其是如果你在持有期中要经历漫长的回调和盘整时间时。这些艰巨的挑战将使得成功的长期投资变得很难。

应对长期投资带来的挑战的一种明智的方法就是在技术分析交易工具的帮助下坚持并执行你对基本面的判断。当你决定买入时,可以参考技术指标来确保你买入的价格相对市价买入要便宜一些。如果你的投资很成功,资产价格飞涨,可以使用技术分析工具来确认被过度估值的价格区域,在该区域内兑现你的盈利并且准备好在不可避免的下跌回调过程中再次买入。这种投资计划需要高强度的专注以及坚持不懈的精神。
\subsection*{波段交易}
大趋势和震荡区间都可以持续几年时间,其中会被短期上下波动所打断。这些波动给我们提供了很多交易机会,我们可以对此加以利用。

最好的学习技巧之一就是在一笔交易完成的两个月后重新回到这笔交易,回顾它的图形。交易信号在屏幕图形右边缘上很朦胧,然而到了图形中部时就变得很清晰了。现在,随着时间的流逝,你可以清楚地看到哪些信号是起作用的以及自己在整个过程中到底犯了哪些错误。从绘制这些图形的过程中你可以学到在未来的交易中,哪些成功的操作是可以重复的,哪些失败是可以避免的。不断更新绘制历史交易记录的图形可以让你成为自己的老师。
\subsection*{日内交易}
日内交易意味着在一个交易日内开始并结束一笔交易。在数字跳动闪烁的屏幕前迅速地买进和卖出需要最高水平的精力集中以及操作纪律。与其相矛盾的是,这种交易却吸引了最冲动和赌性最高的人们。