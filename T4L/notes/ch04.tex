\chapter{}
\section{指数平滑异同移动平均线:MACD 线和 MACD 柱状线}
指数平滑异同移动平均线(moving average convergence-divergence,MACD)是以三个指数移动平均为基础,以两条曲线的形式出现在图表中,其两条线的交叉点,是一种交易信号。
\subsection{如何画出 MACD}
最初的 MACD 指标由两条线组成:一条实线(叫作 MACD 线)和一条虚线(叫作信号线)。MACD 线由两个指数移动平均(EMA)计算而来,其对价格的反应相对较快。信号线是以 MACD 线为基础,通过对 MACD 线以 EMA 的方式进行运算,实现对 MACD 线的平滑,其对价格变动的反应相对较慢。

手工做出 MACD 指标的步骤如下:
\begin{enumerate}
    \item 计算 12 日收盘价的 EMA
    \item 计算 26 日收盘价的 EMA
    \item 用 12 日收盘价的 EMA 减去 26 日收盘价的 EMA,将其差值画成一条实线,这就是较快的 MACD 线;
    \item 计算这条实线的 9 日 EMA,将其结果画成一条虚线,这就是较慢的信号线。
\end{enumerate}
\subsection*{市场心理}
MACD 线和信号线的交点表明了市场中空方和多方实力变换的平衡点。较快的 MACD 线反映的是短期内大众的心理变化,而较慢的信号线则反映了大众心理在较长期的变化。当较快的 MACD 线上升超过信号线时,表示多方主导了市场,这时候最好做多方;当较快的线落到较慢的信号线下面时,表示空方主导了市场,做空方比较有利。
\subsection*{MACD 的交易规则}
MACD 线和信号线的交叉意味着市场趋势发生了变化。顺势的方向是沿着交点的方向进行交易。这个方法产生的假突破比基于简单移动平均的方法产生假突破要少很多。

\subsection{MACD 柱状线}
相比原始的 MACD 线,MACD 柱状线能够提供更深刻的关于多空力量均衡的信息。它不仅能分辨出哪种力量处于主导地位,而且能够分辨其力量是在逐渐增强还是在减弱。MACD 柱状线是技术分析师最好用的工具之一。
\begin{equation}
    \text{MACD 柱状线}=\text{MACD 线}-\text{信号线}
\end{equation}

MACD 柱状线测量的是 MACD 线和信号线之间的差值。它将差值画为一根根柱状线——为一系列垂直的线条。

MACD 柱状线的斜率取决于相邻的两根柱状线之间的高低关系。如果后面一根柱状线比较高(就像字母“m-M”的高度关系一样),MACD 柱状线就是向上倾斜的。如果后面一根柱线图比较低(就像字母“P-p”的深度关系一样),MACD 柱状线就是向下倾斜的。
\subsubsection*{市场心理}
当较快的 MACD 线上升得比较慢的信号线快时,MACD 柱状线会上升。说明多方的力量比之前更强——这是做多的好时机。当较快的 MACD 线下降得比较慢的信号线快时,MACD 柱状线会下降,说明空方的力量在增强——这是做空的好时机。

当MACD柱状线的斜率方向和价格的变动同向时,趋势就是稳定的。\textbf{当 MACD 柱状线的斜率方向与价格的变动方向相反时,趋势的稳定程度就值得怀疑了}。

\textbf{MACD 柱状线的斜率方向比柱状线的正负重要得多}。最好是根据 MACD 柱状线的斜率方向来进行交易,因为它能告诉你在空方和多方中,到底是谁在主导市场。最好的买入信号是当 MACD 柱状线低于 0 值,而它的斜率方向是朝上的,表明空方的力量已经是强弩之末了;最好的卖出信号发生在 MACD 柱状线高于 0 值,而它的斜率方向是朝下的,表明多方已经耗尽了最后的力量。
\subsubsection*{交易规则}
\begin{itemize}
    \item 当 MACD 柱状线停止下跌开始上升时就买入,在近期的次低点下方设置保护性止损单;
    \item 当 MACD 柱状线停止上升开始下跌时就卖出,在近期的次高点上方设置保护性止损单。
\end{itemize}

MACD 柱状线在日线图中频繁地上升下降,所以每次转向都进行交易是不切实际的。在周线图上 MACD 柱状线斜率的变动更有意义。
\subsubsection*{什么时候预期市场探出新高或新低}
如果日线的 MACD 柱状线创出了三个月内的新高,说明多方的力量很强,价格还可以再创新高。如果日线的 MACD 柱状线创出了三个月内的新低,说明空方力量很强,价格可能会再次试探甚至突破之前的低点。

如果在价格上升的过程中,MACD 柱状线创出新高,说明上升的趋势是健康的,可以预期市场会继续上涨,再次试探甚至超过之前的高点;如果在价格下降的过程中,MACD 柱状线创出新低,说明空方力量很强,价格可能会再次试探甚至突破之前的低点。

MACD 柱状线就像汽车的车头灯一样——能让你看清前面的道路。需要提醒的是,虽然 MACD 柱状线不能照亮你回家的全部的路,但足以让你以适当的速度安全行驶了。
\subsubsection*{背离}
MACD 柱状线和价格出现背离的情形并不常见,但是它们却传达出了某些最强有力的信号。它们往往标志着重要的转折点。它们并不一定会出现在每个重要的顶部或底部,但是一旦你看到一个,你就知道一次大的反转可能即将到来。

\textbf{牛市背离}发生在下降趋势的终止阶段——它标志着市场底部。经典的牛市背离发生在价格和震荡指标都创出新低,开始回升,接着震荡指标穿过零点,接着价格和震荡指标又再次下降。这一次,价格跌到新的低点,但是震荡指标的底部则比前一次下跌的底部高。这样的背离经常发生在猛烈的上涨之前。

\begin{tcolorbox}
    要注意,两个底部之间有一个穿越回 0 值线的部分是真正的背离所必须具备的因素。在第二次探底之前,MACD 柱状线必须穿越回 0 值线。如果没有与 0 值线的交点,那就不是真正的背离。
\end{tcolorbox}
\figures{fig23-3}{道琼斯工业平均指数(DJIA)周线图,26 日和 13 日指数移动平均线,12-26-9MACD 线和柱状图。这里你看到的背离信号标志的是 2007~2009 年期间熊市的底部。这个背离信号在低点附近给出了非常强烈的买入信号。在 A 区域,当时雷曼兄弟破产了,一浪接一浪的卖出冲击着市场,道琼斯指数像自由落体一样下跌。MACD 线柱图创出历史新低,说明空头极端强大,A 区域的底部价格很可能会被重新试探甚至突破。在 B 区域,MACD 线柱图反弹到 0 值线之上,“打破了这个熊市”。要注意这个短暂的反弹触及到了两条移动平均线之间的“价值区间”——这是熊市反弹时一个比较常见的目标。在区域 C,道琼斯指数滑向了一个新的熊市低点,但 MACD 柱状线的底部则浅得多。而之后的回升,完成了一次牛市背离,这是非常强烈的买入信号。}

\textbf{熊市背离}发生在上涨趋势中——意味着市场的顶部。经典的熊市背离发生在价格创出新高后回落的时候,同时震荡指标落到0值以下。价格逐渐平稳,然后上升到新高,但是震荡指标仅上升到比之前的峰值要低的高点。这样的熊市背离通常预示着剧烈的下跌。

熊市背离显示出多方的能量在耗尽,价格还在惯性上涨,但空方已经准备入场接手。有效的背离很容易被看出来——它们就像从图表中跳到你眼前一样。如果你需要用尺子来量一量看这是不是一个背离,那么你可以假定它不是。

\figures{fig23-4}{道琼斯工业平均指数(DJIA)周线图,26 日和 13 日指数移动平均线,12-26-9MACD 线和柱状图。在 X 区域,道琼斯指数和其 MACD 柱状线同时上升到牛市的新高点,说明多头的力量十分强大。这意味着未来很有可能会再次试探甚至突破顶部 X 点的价格。注意,MACD 柱状线的X部分,其形态很复杂,但并不是一个背离,因为它的中间部分并未沉到 0 值线之下去。在 Y 区域,MACD 柱状线跌落到 0 值线之下了,“打破了这个牛市”。要注意,价格穿透到了两条移动均线之间的“价值区间”的下方。这是牛市中断时一个相当普遍的信号。同样要注意到在底部 Y 处,有一个“袋鼠尾”。在 Z 区域,道琼斯指数上升到一个牛市新高,但 MACD 柱状线的上升有点缺乏活力,反映出牛市的虚弱。在峰值 C 处开始的跳水,完成了熊市背离,给出了强烈的卖出信号,预示着近 30 年最严酷的一个熊市。}

\textbf{这里的新高是历史新高还是阶段新高?}

“无右肩”背离(“missing right shoulder”divergences)是指,当第二次探新高时MACD柱状线还没有穿过0值。这种情况很罕见,但是是非常
强烈的信号。这在一本叫作《未选择的路:背离交易》(Two Roads Diverged:Trading Divergences)的电子书中有详细的描述和说明。

\textbf{三重牛市或熊市背离}由三组价格和震荡指标的底部或者三组价格和震荡指标的顶部组成。它们比普通的背离更加强烈。要产生三重背离,普通的牛市背离或者熊市背离首先要出现失效。这也是需要做好审慎的资金管理的又一个理由!如果你在假突破时只损失了一小部分钱,那你能保持充足的资金和良好的心态再次入场交易。震荡指标第三次探顶或者探底一定比第一次浅,但并不一定要比第二次浅。
\subsubsection*{巴斯克维尔的猎犬}
这个信号产生在,当可信赖的图表或者指标的模式出现,但价格并没有走向你期望的方向时。比如,当背离模式出现,显示上升的趋势可能要结束了,但是价格却还在持续上涨,这种情形称为“巴斯克维尔的猎犬”。

当市场对一个完美的信号无动于衷时,那就是“巴斯克维尔的猎犬”的信号。这就表示在表象之下有更基础的东西在发生变化。这时候要做好准备,迎接一波新的大趋势。

我并不是抛物线止损指标(stop and reverse)的拥趸,但“巴斯克维尔的猎犬”是个例外。在某些罕见的场合中,当熊市背离失效时,我会做多。同样,在牛市背离失效的罕见机会里,我会卖空。