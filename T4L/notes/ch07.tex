\chapter{交易系统}
一些人有着严格定制化的交易系统,给个人留有的决策空间很小——我们可以称他们为机械化的交易者。而另一些人给自我判断留下巨大的空间——我们称他们为自由决策的交易者。不管你采用什么样的交易方法,交易系统的重要优势是你可以在闭市的时间并且当你处于冷静的状态时,去设计它。交易系统会成为你在市场波动中的理性行为之锚。

机械交易者建立了一整套准则,用历史数据进行测试,然后把它们放到系统中进行自动交易。再进一步,他的软件开始提示入场、目标和退出指令。一个机械交易者应该准确地按系统提示的进行操作。他们是否要遵循他的计划或者尝试扭曲、忽视这些信号便是另一番情况了,但是以上这些是系统应当运作的方式。

使用机械交易策略的专业投资者须持续地像鹰一样不断盯着系统的运行情况。他知道正常的回撤和交易系统阶段性失灵——需要被抛弃,之间的区别。专业的投资者能够准确地使用机械性交易系统因为他有能力进行自由性的交易!机械交易系统是一种行动计划,但是一定程度的人为判断始终是必要的,即使是用最好的、最可靠的计划。

自主决策的交易者在市场中每天都会更新交易策略。他会更倾向于比机械性交易者检查更多的因素,在不同时段给予它们不同的权重,并且更协调地跟随着最新市场行为的变化。一个好的自主决策交易系统,在给你足够多的自由的同时,包含多种不能违反的准则,特别是在风险管理领域的规则。

机械性交易一般成绩会更稳定,但是最成功的交易者使用自主决策的方式。你的决策有可能依赖于你的性格。那也是我们在生活中做出重大的决策的方式,这包括居住地的选择、想追求的事业的选择以及结婚对象的选择。我们关键的选择来源于我们性格最内在的核心处,而不是理性的思考。在交易过程中,更冷静和更执着的交易者会倾向于机械性交易,而更虚张声势的人会倾向于自主决策的交易方式。
\section{系统测试、模拟交易和逐笔交易的三个关键要求}
在使用系统进行真枪实弹的交易之前,无论你是自己开发的还是从零售商那里购买的,你都需要对这个系统进行测试。这里有两种方法可以选择。其中之一是\textbf{模型回测}:将你的系统准则在一组历史数据中进行测试,通常用长达几年时间的数据。另一个方法是\textbf{实盘测试}:用真金白银进行小规模的交易。严谨的交易者首先会使用模型回测,如果获得的结果还不错,再将其转为实盘测试。如果系统运行也是良好的,他们再不断地增加头寸规模。

用历史数据进行测试是良好的开端,但不要让漂亮的数据结果哄骗你产生错误的安全感。
\subsection*{模拟交易}
谈及交易设置,在你交易操作之前,计划好所有相关的目标数值是非常重要的。

这里有三条投资的策略:我最喜欢的是背离时的假突破;第二条是股价在强势趋势中回调到价值区间时(\autoref{fig38-1});最后一条,我偶尔退到另一个极端——会在已过度扩张的趋势反转时下筹码。每一个交易策略都有其准则,但是核心是——我只会进行符合其中之一策略的交易。对老手来说绝不会做过于随机的事情。
\figures{fig38-1}{A.交易设置——为每一笔交易计划好三个关键的数字:买入价位、盈利目标价位和止损价位。在进入市场进行交易之前,你需要决定你将愿意以什么价格成交,你期望从市场中获得的收益和你愿意遭受的风险。潜在收益与风险的比例通常最好要超过 2:1。唯一可以偏离这条准则的交易时机是,当技术指标的信号特别强烈的时候。当然,不要捏造你的目标来把一个不是很确定的交易转为一个可以接受的交易。你的交易目标需要贴近实际。B.风险管理——事先确定你能承受在这笔交易中损失多少钱。用最大损失金额除以股数,得到每一股的可承受的损失——也就是你的买入价格和止损价格之间的差额。这将限制你交易的股数。C.最后同样重要的是,每一笔单独的交易都必须建立在特定的系统和战略之上。“在我看来很不错”这种想法并不是一个系统!当你听到一些股票提示或者看到脱缰而上的趋势时,会很容易变得激动。但是要告别像小狗追逐汽车那样去追逐股票了。如果你想以交易为生,你需要确定清楚自己的交易计划、战略或系统——用自己喜欢的方式来称它们——仅仅参与那些符合它们标准的交易机会。}
\subsection*{逐笔交易的三个关键要求}
对每个计划的交易都要从三个重要的角度去考虑。这三条要求对于严肃认真进行交易的任何人来说都是至关重要的。
\section{三重滤网交易系统}
三重滤网交易系统对每次操作进行三重测试或过滤。许多交易机会在一个蜡烛图界面看起来非常有吸引力,但在另一个时间周期的界面却看起来情况相反。通过三重滤网交易系统筛选的交易机会成功可能性要大很多。
\subsection*{趋势跟随指标和震荡指标}
不同的指标在相同的市场之中给出相互矛盾的信号。趋势跟随指标随着股价的上升而上升并给出买入信号,然而此时震荡指标会显示为超买,而给出卖出信号。趋势跟随指标在股价下降的趋势中同步向下,并给出卖出的信号;但此时震荡指标却显示为超卖,并给出买入的信号。

趋势跟随指标在市场做趋势运动时能盈利,但在市场区间震荡时会导致受双倍的损失。震荡指标在震荡区间内能盈利,但是当市场开始形成趋势的时候,会给出过早的、危险的信号。交易者常说“学会和趋势做朋友”和“让你的盈利跑起来”。他们有时候也说“低买,高卖”。但是为什么要在趋势上升的时候卖?还有涨得多高算高呢?
\subsection*{选择时间周期——因素 5}