\chapter{交易系统}
一些人有着严格定制化的交易系统,给个人留有的决策空间很小——我们可以称他们为机械化的交易者。而另一些人给自我判断留下巨大的空间——我们称他们为自由决策的交易者。不管你采用什么样的交易方法,交易系统的重要优势是你可以在闭市的时间并且当你处于冷静的状态时,去设计它。交易系统会成为你在市场波动中的理性行为之锚。

机械交易者建立了一整套准则,用历史数据进行测试,然后把它们放到系统中进行自动交易。再进一步,他的软件开始提示入场、目标和退出指令。一个机械交易者应该准确地按系统提示的进行操作。他们是否要遵循他的计划或者尝试扭曲、忽视这些信号便是另一番情况了,但是以上这些是系统应当运作的方式。

使用机械交易策略的专业投资者须持续地像鹰一样不断盯着系统的运行情况。他知道正常的回撤和交易系统阶段性失灵——需要被抛弃,之间的区别。专业的投资者能够准确地使用机械性交易系统因为他有能力进行自由性的交易!机械交易系统是一种行动计划,但是一定程度的人为判断始终是必要的,即使是用最好的、最可靠的计划。

自主决策的交易者在市场中每天都会更新交易策略。他会更倾向于比机械性交易者检查更多的因素,在不同时段给予它们不同的权重,并且更协调地跟随着最新市场行为的变化。一个好的自主决策交易系统,在给你足够多的自由的同时,包含多种不能违反的准则,特别是在风险管理领域的规则。

机械性交易一般成绩会更稳定,但是最成功的交易者使用自主决策的方式。你的决策有可能依赖于你的性格。那也是我们在生活中做出重大的决策的方式,这包括居住地的选择、想追求的事业的选择以及结婚对象的选择。我们关键的选择来源于我们性格最内在的核心处,而不是理性的思考。在交易过程中,更冷静和更执着的交易者会倾向于机械性交易,而更虚张声势的人会倾向于自主决策的交易方式。
\section{系统测试、模拟交易和逐笔交易的三个关键要求}
在使用系统进行真枪实弹的交易之前,无论你是自己开发的还是从零售商那里购买的,你都需要对这个系统进行测试。这里有两种方法可以选择。其中之一是\textbf{模型回测}:将你的系统准则在一组历史数据中进行测试,通常用长达几年时间的数据。另一个方法是\textbf{实盘测试}:用真金白银进行小规模的交易。严谨的交易者首先会使用模型回测,如果获得的结果还不错,再将其转为实盘测试。如果系统运行也是良好的,他们再不断地增加头寸规模。

用历史数据进行测试是良好的开端,但不要让漂亮的数据结果哄骗你产生错误的安全感。
\subsection*{模拟交易}
谈及交易设置,在你交易操作之前,计划好所有相关的目标数值是非常重要的。

这里有三条投资的策略:我最喜欢的是背离时的假突破;第二条是股价在强势趋势中回调到价值区间时(\autoref{fig38-1});最后一条,我偶尔退到另一个极端——会在已过度扩张的趋势反转时下筹码。每一个交易策略都有其准则,但是核心是——我只会进行符合其中之一策略的交易。对老手来说绝不会做过于随机的事情。
\figures{fig38-1}{A.交易设置——为每一笔交易计划好三个关键的数字:买入价位、盈利目标价位和止损价位。在进入市场进行交易之前,你需要决定你将愿意以什么价格成交,你期望从市场中获得的收益和你愿意遭受的风险。潜在收益与风险的比例通常最好要超过 2:1。唯一可以偏离这条准则的交易时机是,当技术指标的信号特别强烈的时候。当然,不要捏造你的目标来把一个不是很确定的交易转为一个可以接受的交易。你的交易目标需要贴近实际。B.风险管理——事先确定你能承受在这笔交易中损失多少钱。用最大损失金额除以股数,得到每一股的可承受的损失——也就是你的买入价格和止损价格之间的差额。这将限制你交易的股数。C.最后同样重要的是,每一笔单独的交易都必须建立在特定的系统和战略之上。“在我看来很不错”这种想法并不是一个系统!当你听到一些股票提示或者看到脱缰而上的趋势时,会很容易变得激动。但是要告别像小狗追逐汽车那样去追逐股票了。如果你想以交易为生,你需要确定清楚自己的交易计划、战略或系统——用自己喜欢的方式来称它们——仅仅参与那些符合它们标准的交易机会。}
\subsection*{逐笔交易的三个关键要求}
对每个计划的交易都要从三个重要的角度去考虑。这三条要求对于严肃认真进行交易的任何人来说都是至关重要的。
\section{三重滤网交易系统 Triple Screen Trading System}
三重滤网交易系统对每次操作进行三重测试或过滤。许多交易机会在一个蜡烛图界面看起来非常有吸引力,但在另一个时间周期的界面却看起来情况相反。通过三重滤网交易系统筛选的交易机会成功可能性要大很多。
\subsection*{趋势跟随指标和震荡指标}
不同的指标在相同的市场之中给出相互矛盾的信号。趋势跟随指标随着股价的上升而上升并给出买入信号,然而此时震荡指标会显示为超买,而给出卖出信号。趋势跟随指标在股价下降的趋势中同步向下,并给出卖出的信号;但此时震荡指标却显示为超卖,并给出买入的信号。

趋势跟随指标在市场做趋势运动时能盈利,但在市场区间震荡时会导致受双倍的损失。震荡指标在震荡区间内能盈利,但是当市场开始形成趋势的时候,会给出过早的、危险的信号。交易者常说“学会和趋势做朋友”和“让你的盈利跑起来”。他们有时候也说“低买,高卖”。但是为什么要在趋势上升的时候卖?还有涨得多高算高呢?
\subsection*{选择时间周期——因素 5}
查尔斯·道,受人景仰的道氏理论的创立者,曾说在 20 世纪之交的股票市场存在着三种趋势,分别是持续好几年的长期趋势、持续几个月的中期趋势、和更短一些的短期趋势。20 世纪 30 年代一位伟大的市场技术分析大师,罗伯特·雷亚,将这三种趋势分别比之为潮流、波浪和浪花。他建议沿着潮流的方向交易,学会利用波浪,而去忽略一些小的浪花。

无论你喜欢用哪一个时间周期,三重滤网交易系统称之为中期时间周期。长其一号的是长期时间周期。短其一号的是短期时间周期。一旦你选定了中期时间周期,你先不用去看它,而是先去查看大一号的长期时间周期的线图,并且在长期时间周期线图里做好战略决策,然后再回到中期时间周期的线图中去。

举个例子,如果有一笔交易你想持有几天或者几个星期,你的中期时间周期很可能被定义为日线图。周线图是尺度大一号的时间周期,即长期时间周期。1 小时线图是尺度小一号的时间周期,因此是短期时间周期。

三重滤网交易系统首先要求你去检查长期时间周期图表,找出长期图表中的大趋势。它仅允许你顺着大趋势的方向进行交易。它使用中期图表中的趋势与长期趋势方向相反时的机会建立头寸。举个例子来说,就是当周趋势是上升的时候,则在日趋势下降时买入。当周趋势是下降的时候,则日趋势上升时是卖出的机会。
\subsection*{第一重滤网——市场潮流}
在牛市行情中时,动力系统会把每根线柱都标成绿色;当在熊市时,则都标成红色;当市场为中性时,则都标成蓝色。动力系统并不能告诉你应该做什么。它是一种发出警告信号,禁止你做某些事情的检查系统。当动力系统显示为红色的时候,它禁止你的买入行为;当显示是绿色的时候,它会禁止你的卖空行为。当你想买入的时候看一下周线图,你需要等到它不再是红色为止;想卖空的时候,你也需要看一下周线图,要确保它不是绿色的;动力系统为蓝色时,允许你在任意方向进行交易。

第一重滤网总结:使用趋势跟随指标识别周趋势并随着趋势的方向交易。

交易者有三种选择:买、卖或者观望。三重滤网交易系统的第一重会帮你排除其中一个选项。它的作用像监察员一样,在上升的趋势里只允许你选择买入或者观望;在下行的趋势里只允许你卖空或观望。你必须顺应潮流的方向,否则它会禁止你下水。
\subsection*{第二重滤网——市场波浪}
第二重滤网:将震荡指标应用于日线图之中。在周线的上升趋势中,利用日线的回调来寻找买入机会;在周线的下降趋势中,利用日线的反弹来寻找卖空机会。

当周趋势是上升的,三重滤网仅允许采用日线图震荡指标发出的买入信号,而不会允许采用其发出的卖空信号。2 日强力指数指标 EMA 在降到 0 值以下的时候,只要它不是下降到了几周内的新低点,就会发出买入信号。当周趋势是下降的,强力指数指标会在上升到中心线上方的时候,只要它不是上升到几周内的新高点,就发出卖空的信号(见 \autoref{fig39-2})。

\figures{fig39-2}{2 日强力指数指标 EMA 可以用作三重滤网交易系统第二重的众多震荡指标之一。当它下降到其中心线之下的时候,强力指数标记出买入的机会。当它上升到其中心线之上的时候,它会标记出卖出的机会。当周趋势是上升的(这里用绿色的水平线标出),在日线震荡指标中仅采用买入信号,以建立多头头寸。当周趋势是下降的(这里用红色的水平线标出),在日线震荡指标中仅采用卖出的信号,以建立空头头寸。}

其他的震荡指标,比如随机指标和相对强弱指标,当它们进入各自买卖区域的时候会发出交易信号。举例来说,当周线 MACD上升,但日线随机指标下降到 30 以下时,它确认为超卖的区域,是一个买入的机会。当周线 MACD 下降,但日线随机指标上升到 70 以上时,它确认为超买的区域,是一个卖空的机会。
\begin{tcolorbox}
    我感觉有点扯淡,这日线降低到 30 以下,周线还没变,保持上涨这不是有点不可能吗?而且降低到三十以下,不会判断为打破原有的上涨趋势?
\end{tcolorbox}
\subsection*{第三重滤网——买入技术}
做多时,在日线向上突破前一交易日高点的时候买入;卖空时,在日线向下突破前一交易日低点的时候卖出。

这种方法的缺点是止损点的位置太远了。在突破前一交易日高点的位置买入,同时在前一日的低点位置设置止损,意味着如果前一日的价格振幅很大,则止损价离最新的股价很远。这样要承担很大的风险,否则只能用很小的头寸进行交易。另一种风险是,当突破前一交易日振幅很窄,将止损点刚好设在前一交易日的低点之下时,当日的市场噪声就可能会触发止损。
\figures{fig39-3}{们可以使三重滤网系统的买入信号更敏锐,而不必等到 2 日强力指标反弹到 0 以上。我们可以利用其下降到 0,作为一个信号,并把我们的买单设置在短期 EMA 值之下,这就是平均 EMA 下跌穿透。计算平均穿透值,用今日的 EMA 值减去昨日的 EMA 值,将其结果加回今日的 EMA 值:这是对明日EMA值的一个估算。用估算的明日 EMA 值减去你计算的平均穿透值,作为明日设置买入订单的触发价位。你将利用回调以折扣价完成买入交易——避免了在突破时买入须支付的溢价。}

在 \autoref{fig39-3} 的四种情形中,价格下降到快速 EMA(红色)线之下。一个平均下跌穿透值是 9.6 美元。在屏幕的右下角,13 日 EMA 值位于 1266 美元水平。如果今天看到恐慌性的抛售盘,则从此价格中减去最近的平均下跌穿透值。我们应该把买入价设置在比最新 EMA 值低大约 9 美元的水平,这个价格在慢速 EMA 线之下。我们应该坚持每天进行这个计算,直到我们最终能够以低价完成买入。这是比追高更为平和的交易方式。
\subsection*{三重滤网交易总结}
\begin{table}
    \centering
    \caption{三重滤网交易总结}
    \begin{tabular}{rrrl}
        \hline
        周趋势 & 日趋势 & 行动 & 指令           \\
        \hline
        上行  & 上行  & 观望 & 无            \\
        上行  & 下行  & 买入 & EMA 穿透或者向上突破 \\
        下行  & 下行  & 观望 & 无            \\
        下行  & 上行  & 卖出 & EMA 穿透或者向下突破 \\
        \hline
    \end{tabular}
\end{table}
当周线趋势是上升的而日线震荡指标是下降的,将买单设置在日线短期 EMA 值减去一个平均下跌穿透值的价位上。可以替代的方案是,在前一日最高价上加上一个最小申报单位的价格位置设置买入指令。如果价格上涨,并突破前一日的高点,你的买入限价指令将会被触发。如果价格持续下降,则你的买入限价指令将不会被触发。第二天则相应降低买入限价指令的价格,仍然是前一日最高价加上一个最小申报单位的价格位置处。每天持续降低买入限价指令的价格,直到指令触发或者周线指示指标反转了,从而不再显示买入信号。

当周线趋势是下降的,等待日线震荡指标的反弹,并且在日线短期 EMA 值加上一个平均上涨穿透值的价位上设置卖空指令。或者是,在前一日最低价减去一个最小申报单位的价格位置设置卖空指令。只要市场开始转而向下,你的卖空指令将会被触发。如果当日反弹持续,则继续每天提高卖出的价格。这种跟踪式的卖出限价技巧的目的是在周线级别为下跌趋势而日线级别为反弹时,抓住日内的向下突破机会。
\subsection*{日内交易的三重滤网交易系统}
如果你是做日内交易的交易者,你很可能会选择 5 分钟线图作为你的中期时间周期。再一次提醒,不要先看 5 分钟线图,而是首先去分析 25 分钟或 30 分钟线图,也就是你的长期时间周期线图。在长期线图中,先做出是看空还是看多的战略决策,然后再回到中期线图来寻找买入价位和止损位(见 \autoref{fig39-4})。

\figures{fig39-4}{这里是日内交易,其长期时间周期线选择的是 30 分钟线图。观察到它显示为上升趋势。此时,我们转向稍短期的图表,5 分钟线图。当 5 分钟线图的两线柱强力指标下降到 0 之下时,标志着这是与长期趋势相违背的一个波浪——一个可以在更低价位买入的好机会。画一个覆盖 95\% 的价格区间的通道,可以用通道线来设定盈利目标。}
\subsection*{止损和止盈目标}
在你进入一个交易之前,写下三个数字:买入价位、盈利目标价位和止损价位。没有定好这三个数字的交易便是赌博。

三重滤网交易系统要求使用长期图表来设置止盈点,使用中期图表来设置止损点。如果你是使用周线图和日线图搭配的,则在周线图上设好盈利目标,在日线图上设好止损目标。当在日线图的回调中买入时,用周线图的价值区间作为盈利目标是一个好的选择。

三重滤网交易系统需要设置相对保守的止损价位。它既然让你按照市场大趋势的方向进行交易,就不允许为损失留太多空间。跟上潮流或者马上退出。
\section{动力系统 The Impulse System}
假设惯性和能量可以描述任何市场的任何时间周期的运动。要度量任意交易品种的惯性,一个好的指标是\textbf{短期 EMA 的斜率}。上升的 EMA 意味着具有牛市惯性,而下降的 EMA 则说明具有熊市惯性。任意趋势的能量可以用 \textbf{MACD 柱状线的斜率}来表示。

当两个指标都是上升的时,代表牛市,如果都是下降的,代表熊市;当两个指标相互方向相反时,代表市场是中性的(\autoref{fig40-1})。

\figures{fig40-1}{EMA 上升、MACD 上升(特别是小于 0 时),动力系统显示是绿色,牛市,禁止卖空,允许买入或观望。EMA 下降、MACD 下降(特别是大于 0 时),动力系统显示是红色,熊市,禁止买入,允许卖出或观望。EMA 上升、MACD 下降,动力系统显示是蓝色,中性,无禁止事项。EMA 下降、MACD 上升,动力系统显示是蓝色,中性,无禁止事项。}
\begin{tcolorbox}
    能不能回测试试这两个指标的买入信号,但是 MACD 的变化有时候太小了,容易平仓再开仓,有点两头挨打赚不到钱。
\end{tcolorbox}

动力系统不是一个自动交易系统,而是一个监测系统!它并没有告诉我应该做什么——\textbf{它告诉我的是不应该做什么}。如果周线图表和日线图表中有任意一个是红色的——不允许买入;如果周线图表和日线图表中有任意一个是绿色的——不允许卖空。

\figures{fig40-2}{垂直的绿色箭头标志的柱线后面紧跟着红色的柱线。红色禁止你买入。最好的买入时机是当红色消失的时候。你能看到这些绿色箭头指示出一个接一个的中期底部,包括在图表右边界的买入信号。拥有一个客观的方法能让你在市场下降停止的时候有买入的信心。动力系统也会对兑现利润的好时机给出建议。倾斜的红色箭头指向蓝色柱线,蓝色柱线出现在一系列远离价值区域的绿色柱线之后。它们显示牛市上行受阻的位置——兑现盈利的好时机,并等待下一个买入机会。}

\begin{itemize}
    \item 买入点位:红色消失之后;
    \item 卖出点位:蓝色线出现在原理价值区域处;
\end{itemize}

\subsection*{入场}
记住三重滤网系统要求在多于一个时间周期内进行分析。选择你最喜欢的时间周期,并将之定为中期时间周期。将其周期乘以5倍,找到长期时间周期。如果你最喜欢的图表是日线图,首先去分析周线图,并做出看多或看空的战略决策。使用动力系统来决定何时允许进行买入或者卖空。
\begin{itemize}
    \item 如果你是日内动量交易者,只要两个时间周期都是绿色的你便可以买入,一旦其中之一变蓝或红,你便兑现收益;
    \item 当尝试抓住市场拐点时,最好的交易信号不是绿色或者红色,而是红色或绿色开始消失时。
\end{itemize}

时间周期越短,它的信号就会越敏感:日线图上的动力系统开始改变颜色总会先于周线图。当做日内交易时,5 分钟线图改变颜色要比 25 分钟线图早。如果我的分析表明市场正在筑底,即将开始反转,我会等到日线图不再显示变红,开始变蓝,甚至变绿,然后我再去观察周线图,这时它仍然是红色的。一旦周线图从红色变蓝,系统会开始允许买入。这种技术防止我当市场仍在下降的时候过早买入。

记住,动力交易系统是一个监测系统。它不会告诉你应该做什么——但是它会很明确地告诉你不该做什么。不要违背这个监测。
\subsection*{退出}
如果你是个短期动量交易者,一旦动力系统显示的颜色不再支持你的交易方向,则马上了结你的交易,即使在两个时间周期中只有其中一个改变了颜色。通常,日线MACD的反转要快于周线 MACD。当它在上升趋势中下降,表明上升的动量正在减弱。当买入的信号消失时,马上兑现收益,而不是等待出现卖出的信号。

在下降的趋势中将这个做法反转一下。一旦动力系统不再显示红色,即使两个时间周期中只有其中一个改变了颜色,也马上清空你的空头头寸。最有效的下降部分已经结束,动力系统已经完成了它的使命。

果两个时间周期中的任意一个变成了蓝色,波段交易者或许仍会持有交易头寸。波段交易者要避免,时间周期中任意一个的颜色与交易的方向变得相反。如果你是多头,时间周期中的一个变成红色,则是时候卖出并空仓观望了;如果你是空头,当动力系统开始变绿,它就发出了让你平掉空头头寸的信号。\textbf{那么我感觉如果是日频的数据,信号变化的太快了}
\section{通道交易系统 Channel Trading Systems}
市场价格倾向于在通道中波动,就像河流在河谷中一样。当河流碰到了它的右边河岸,它会转而向左;当流到了左边河岸,它会转而向右。当价格开始上升的时候,它经常在碰到隐形的天花板时停止上涨。它的下降似乎也在碰到隐形的地板时停止下跌。通道帮助我们预测未来在哪儿最可能遇到支撑线和阻力线。

支撑线是买单比卖单密集的地方,而阻力线是卖单比买单密集的地方,通道展示了未来哪些地方有可能出现支撑或者阻力。
\subsection*{构建通道的两种方法}
我们可以通过绘制两条平行于一条移动均线的线组,来构建一个通道:一条线在移动平均线上方,另一条线在移动平均线下方。我们可以根据市场的波动性来改变两条通道线之间的距离(标准背离通道)。

以移动均线为对称中心的通道线对股票和期货交易都很有用。标准背离通道(有时也叫作布林通道)对于期权交易十分有用。

通道界定了价格正常波动和不正常波动之间的界限。价格在通道中运行是正常的,只有非正常的时间驱动才会使价格波动到通道之外。价格在通道的下轨线之下的时候,是被低估了;价格在通道的上轨线之上的时候,是被高估了。
\subsection*{对称的通道}
使用这组移动均线,并将长期移动均线作为通道线的核心。举个例子来说,如果你使用 13 日和 26 日 EMA 这组线,则通道线是平行于 26 日 EMA 线的。

通道的宽度依赖于交易者选择的系数。这个系数通常用 EMA 的百分比来表示。
\begin{equation}
    \begin{aligned}
        \text{上通道线} & =EMA+\text{通道系数}\times EMA \\
        \text{下通道线} & =EMA-\text{通道系数}\times EMA \\
    \end{aligned}
\end{equation}

\textbf{我现在的想象是:一段时间全部处于下轨下面,怎么办?又不是刚下去就价值回归了。}

当为任意市场设置通道线的时候,开始通道系数可以设置为 3\% 或者 5\%,然后不断调整通道系数值,直到通道线把最近 100 根线柱的所有价格数据中大约 95\% 的包含在内,在日线图上大约是 5 个月时间长度。

波动较大的市场需要较宽的通道,而沉寂的市场需要较窄的通道。便宜的股票通常比高价股票有更高的波动率。长时间周期的线图需要的通道更宽。根据经验,周线图的通道宽度会是日线图的两倍 $\sqrt{5}$。

\subsection*{群体心理}