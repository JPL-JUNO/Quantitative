\chapter{实践细节}
当股票创出新高后,你会买吗?在双重顶时会卖吗?在回调中会买吗?你会寻找趋势反转吗?以上这些的方法各不相同,每种方法都可能赚钱,也可能赔钱。你应该选择那些吸引你、让你很舒服、适合你能力和气质的交易方法。没有一种交易方法是适合所有人的,就像没有一种运动是适合所有人的一样。

要想成功地交易,先要选定一种交易模式。在进行行情数据扫描之前,你应该十分清楚你想要找到什么。开发你的系统,并通过一些小交易先测试一下,确定你可以遵守交易纪律。你必须确定在你看到设计好的交易信号出现时,会按计划交易。
\section{怎样设定止损线:不要异想天开}
接受止损带来的烦恼和痛苦,但要尽力使它们更加合理和少一些不愉快。
\subsection*{在“市场噪声”之外设定止损}
如果把止损线设得离成交价格太近的话,会受到市场无意义波动的影响;如果把止损线设得太远的话,起到的保护作用会减少很多。

在股票市场中,我们可以把信号定义为股票的趋势性变化;把噪声定义为上涨趋势中当日股价低于前一交易日最低价的部分,和下降趋势中当日股价高于前一交易日最高价的部分。

先度量市场噪声,再把止损位设在市场噪声区域外数倍的位置。简言之,使用 22 日 EMA 来定义为趋势线。如果趋势是向上的,标记出所有回溯期(10-20天)内向下穿透 EMA 线柱的深度值,将其加总后除以向下穿透的线柱数量,得到回溯期的平均向下穿透值。它反映了当前上升趋势中平均的噪声水平。你应该把止损位设在远离市场平均噪声水平的位置。这就是为什么你需要把平均向下穿透值乘以一个系数,通常是 2 以上的数字。如果止损位设得太近容易弄巧成拙。

当 EMA 趋势是下降的时候,我们使用前期线柱的最高价向上穿透来计算安全区域。我们数一下选定期间内线柱的向上穿透情况,计算它们的平均值,得到平均向上穿透值。选一个系数乘以它,比如可以从 3 开始选,将得到的值加到每次高点上。在高点卖空比在低点买入需要更宽的止损空间。
\subsection*{不要把止损线设在明显的位置}
从密集的价格区间向下探出显眼的新低点,最容易吸引交易者在新低点下方设置止损位。问题是太多人在这里设止损线,造成这个区域里止损的人过多。市场有一个神秘的习惯,会很快地跌穿这些明显低点,引发止损后再反转,发动新的上升趋势。

把止损位设在并不明显的位置比较好——要么更接近市场目前水平,要么离明显位置更远一点。更近的止损位可以减少亏损规模的风险但是会增加被洗盘出局的风险。更低的止损位可以躲过一些假突破,但是一旦真触及止损,亏损规模会更大。

尼克止损是把止损位设在近期的次低点,而不是设在最低点附近。当作为空方的时候这个原则也是一样的——不把止损线设在最高点之上,而是设在次高点。

\figures{fig54-1}{在可口可乐(KO)公司的图中,我们发现了一个伴随着牛市背离的向下假突破,动力系统从红色变成了蓝色——允许买入。如果我们做多买入,我们应该在哪里设定止损线呢?尼克止损会设在 37.04 美元——比近期的次低点少1美分,是线柱 B 的最低点。在直觉外科公司(ISRG)的图表中,我们可以看到一个伴随着熊市背离的向上假突破,动力系统由绿变蓝——允许卖出。如果我们做空卖出,我们应该在哪里设置止损线呢?尼克止损会设在 445.05 美元——比最近的次高点——线柱 B 的高点,高几美分。}
你可以激进,也可以保守,但是要记住最重要的原则:第一是要有止损;第二是不要把止损位设在太明显的位置,也就是图上谁都能看出来的位置。

平均真实波幅(ATR)止损是指当你在最近的一根线柱中入场时,把你的止损位设在离当前这根线柱的极值至少一倍 ATR 的位置,如果是在两倍 ATR 的位置设置止损位就更安全了。你可以把它当作一种移动止损的方法,随着线柱的转变而移动它。使用移动止损的一个优点是它们逐渐减小了所暴露的风险额度。使用移动止损时,如果交易价格朝对你有利的方向变化时,它可以逐渐释放可用风险额度,从而允许你开始做新的交易。
\subsection*{不要让盈利变为亏损}
不要让有丰厚账面浮盈的未平仓头寸变为亏损!在交易之前,就要计划在什么水平开始保护你的利润。比如有一笔交易的盈利目标是 1000 美元,那么在有 300 美元盈利的时候就需要开始保护利润。一旦你的未平仓头寸浮盈达到 300 美元,你可以将止损线调整到盈亏平衡的位置。我们称这种移动为“为交易翻边”。

当交易的发展已经兑现了你的预期,这笔交易的盈利潜力逐渐变小。而你的风险(盈利和止损线之间的距离)会不断增加。交易就是在管理风险,当盈利与风险的比例渐渐恶化时,你便需要减小承担的风险。通过提升止损线,保护一定比例的利润,可以使盈利与风险比例控制在更平衡的位置。
\subsection*{只顺着你交易的方向移动止损线}
\subsection*{灾难性止损:专业交易者的救生衣}
“硬止损”是一种给你的经纪商下达的指令,而“软止损”是你心中的止损线,当到需要的时候你才会去执行真实操作。新手或业余交易者一定要使用硬止损线;而对每天盯盘的专业交易者来说,当系统提升需要止损时,他能遵守纪律去执行,那他可以使用“软止损线”。
\subsection*{止损线和隔夜跳空:仅对专业交易者}
如果你持有的股票在休市期间出现了一个重大利空,你怎么办?在第二天早上开盘之前查看集合竞价情况,你意识到股价将大幅低开,远低于你的止损线,意味着滑点会很大。

如果你是一名新手或者业余交易者,那并没有什么可选择的,只能咬紧牙关承受损失。但对于冷静的、有纪律的专业交易者来说,还有一种方法,那就是用做日内交易的方式退出。首先,撤走止损线,开盘之后当作开盘第一秒买入了一样,后面进行日内交易的操作。

开盘跳空缺口常常伴随着反弹,这给那些机敏的交易者提供了减少损失的机会。但这样的情况并不是一定会发生,所以大多数的交易者不要轻易尝试这种技术。因为这么做可能导致亏损更多,而不是减少亏损。

记住在收盘前要及时退出——已经走坏了的股票,可能当天会反弹,但明天将会有更多的卖家进场卖出,驱使股价进一步下跌。不要让一次反弹引发你对反转的希望。
\section{这是 A 级交易吗}
一旦你结束了一笔交易,市场将对你的入场、退出和最重要的整体交易三方面做出评级。

如果你是一位使用周线图和日线图做波段交易的交易者,那就用日线图来计算你每笔交易的级别。你的买入评级取决于入场点、购买当日的最高点和最低点的情况。
\begin{equation}
    \text{买入评级}=\frac{\text{最高价}-\text{买入价}}{\text{最高价}-\text{最低价}}
\end{equation}

每笔交易计算买入评级,而且我认为大于 50\% 就是不错的成绩了,意味着我是在当日线柱的较低部分买入的。

下面是卖出评级的计算公式:
\begin{equation}
    \text{卖出评级}=\frac{\text{卖出价}-\text{最低价}}{\text{最高价}-\text{最低价}}
\end{equation}

当评估一笔交易时,大多数人认为他们在交易中挣到或赔掉的金额是交易质量的反映。资金规模对画资产曲线来说很重要,但对单笔交易来说并不是很好的评价指标。通过比较你实际获得金额和潜在可获得金额的比值来评价交易质量可能更有意义。计算交易评级的方式是,比较交易的损益与入场点当日通道线的高度。
\begin{equation}
    \text{交易评级}=\frac{\text{卖出点}-\text{买入点}}{\text{通道线高点}-\text{通道线低点}}
\end{equation}

\figures{fig55-1}{A 日——2014 年 2 月 10 日,星期一:高点是 52.49 美元,低点是 51.75 美元,上通道线是 53.87 美元,下通道线是 47.61 美元(我们需要通道高度来计算退出的交易评级)。买入价为 51.77 美元。买入评级=(52.49-51.77)/(52.49-51.75)=97\%。B 日和 C 日——星期二和星期三:继续上涨,开始向上移动止损线。D 日——星期四:高点 54.49 美元,低点 53.39 美元。卖出点为 53.78 美元。卖出评级=(53.78-53.39)/(54.49-53.39)=35\%。交易评级=(卖出点-买入点)/通道高度=(53.78-51.77)/(53.87-47.61)=32\%。}
\section{仔细搜寻可能的交易}
在寻找股票交易机会之前,必须先开发一个交易系统或是交易策略。如果没有一个清晰的交易系统,你寻找什么呢?

搜寻交易机会是指对一些交易品种进行复盘,然后聚焦到一些有潜力的品种上。可以用肉眼搜寻,也可以用计算机来扫描——你可能要翻阅很多图表,每一个都只是瞅一眼;或者用你的计算机处理清单上的图表,标记出符合你交易模式的股票。重述一遍,确定一种自己信任的交易模式是最重要的第一步,搜寻是第二步。

\figures{fig56-1}{美国有机商品超市(WFM)日线图,13 日和 26 日指数移动均线,动力系统,MACD 柱(12-26-9),红点——潜在或实际的熊市背离,绿点——潜在或实际的牛市背离。美国有机商品超市(Whole Foods Market,WFM)的线图说明了扫描程序不能成为自动的交易者。它只是一只看门狗,可以警示市场可以交易的机会——做多或做空。收到这样的信号后,交易者需要研究一下这只股票可能出现背离的价格水平,并把入场价格、目标价格和止损价格写下来。}

如果要扫描更大数量的股票,需要增加一些“负面规则”。比如,你需要剔除每日成交量少于 50 万股或 100 万股的股票。这些股票的图表通常很不规则,滑点也比其他交易活跃的股票的要大。你还可能会把高价股从买入名单或低价股从卖出名单中剔除。