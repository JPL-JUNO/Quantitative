\chapter{经典图表分析法}
\section{袋鼠尾}
当你还以为正在运行中的趋势将会继续存在时——“砰!”——三条蜡烛线形成了一个袋鼠尾,标志着市场猝不及防地发生了反转。\textbf{袋鼠尾由单根长线柱及分布其两边的普通线柱组成。}长线柱从紧密交织的震荡区间中突破出来。向上指的袋鼠尾尖端是要在此时市场顶部卖出的信号,而向下指的袋鼠尾尖则是要在市场触底时买入的信号。
\figures{fig20-1}{百健艾迪公司(BIIB)日线图}
\subsection*{袋鼠尾}
\autoref{fig20-1} 中显示的是日线图中的袋鼠尾,你可以在所有的时间周期内找出袋鼠尾。时间周期越长,其信号就越有用:周线图里的袋鼠尾比五分钟线里的更有用。

袋鼠尾,也称“手指线”,它们很有吸引力,也很容易辨认。如果你不能确定这是不是一个真正的袋鼠尾,那你就假定它不是,因为真正的袋鼠尾是不会出错的。它们会出现在大盘指标中也会出现在个股、期货和其他交易品种的图表中。

市场总是在波动,寻找产生最大成交量的价位。如果价格上涨却没伴随着足够的成交量,市场很快就会反转,在更低的价格上寻找更多的交易。如果成交量在下降过程中萎缩,价格就会上涨,在更高的价位上寻找成交量。

袋鼠尾反映了这种失败的突破。

朝上的袋鼠尾反映了多方推高价格的努力失败。他们就像一队士兵,想要从敌人手里夺取一座山头,结果发现主力没有跟上来。所以他们就逃到半山腰以求保命了。而一旦丢失了山头,军队就会撤退到别处。

朝下的袋鼠尾反映了一次失败的空方突袭。空方激进地做空,打压价格——但是低价并没有吸引足够多的成交量,于是只能撤回震荡区间。在继续下跌的尝试失败之后,你觉得市场下一步会往哪儿走呢?既然低价位附近没有什么成交量,那么就上升回调吧。

袋鼠尾意味着市场拒绝了某一价格。它通常会引导趋势的反转。一旦你发现了袋鼠尾,就马上朝相反方向操作吧。
\subsection*{袋鼠尾交易方式}
一个有经验的交易者能在第三条蜡烛线结束前辨认出袋鼠尾。比如说从日线图来看,可能你会发现价格已经波动了几天,但是在下周一,股票爆发出一根很长的阳线。如果周二股票开在周一线柱的底部位置,并且没有上涨,那你就应该考虑在周二收盘之前做空股票。如果市场已经在震荡区间内持续了一周并在周三的时候收出了一根长阴线,那你就要在周四做好准备:如果交易价格在周三线柱的顶部附近小幅震荡,那就在周四收盘前大胆买入。

\textbf{记住逆袋鼠尾方向操作是一种短期策略}。在日线图中,这些信号往往几天之后就会消失。你要在当前的市场背景下正确评估袋鼠尾。比如当你长期看多一只股票时,一定要警惕袋鼠尾。向上的袋鼠尾意味着应该在当前的价位上兑现盈利,而向下的袋鼠尾则是一个加仓的好机会。

在市场中使用保护性止损单来防止损失和取得盈利是非常重要的。在袋鼠尾的末端设置保护性止损单会让保护性止损单太宽,承担过多风险。当你逆袋鼠尾方向交易时,将你的保护性止损单设置在袋鼠尾的中间价位。如果市场开始逼近这个价位,你就该离场了。
\section{回测效果}