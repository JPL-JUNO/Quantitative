\chapter{风险管理}
一般人的本性都是很快就止盈,但却会一直拿着亏损的交易,希望能回本甚至盈利。当这些业余的交易者放弃了希望,在巨亏后平掉头寸的时候,他们的账户往往已经严重亏损甚至无可挽回了。
\section{情绪与概率}
固执地持有亏损的交易头寸只会加深亏损的程度。亏损会以一种滚雪球的方式增长,直到最初看起来很糟糕的亏损比例开始显得不算什么了,因为现在的亏损放大了很多。最终,绝望的失败者忍痛清仓出局,遭受了严重损失。当他一离开,市场就开始反转,强势回归。

此时的交易者恨不得拿头撞墙——如果他再坚持一下,他本可以赚钱的。

这些反转一次又一次的发生,因为大多数输家对刺激的反应是一样的。人们有相似的情绪,这与他们的种族或教育无关。提心吊胆的交易者满手是汗、心跳加速,不管他们是在美国纽约还是在中国香港长大,也不管他接受过 2 年还是 20 年学校教育,他们的感受和反应是一样的。
\subsection*{正的期望值}
大多数交易者有一个很好的交易系统,但为了改造成一个完美的系统反而毁了它。
\section{风险控制的两条主要原则}
只需要一次致命的损失,就能毁掉一个账户,使交易者退出游戏,就像是鲨鱼咬了致命的一口。市场也可以通过一串连续的损失毁掉账户,每次损失都并不致命,但汇集起来会使账户所剩无几,就像是一群食人鱼一样。资金管理的两大支柱是 2\% 原则和 6\% 原则,2\% 原则可以帮助躲避市场鲨鱼式的攻击,6\% 原则可躲避市场食人鱼式的攻击。
\subsection*{两种最糟糕的错误}
有两种方式可以快速毁掉一个账户:从不使用止损和持有相对账户来说过高比例的仓位。

没有设置止损的交易会使你暴露在无限的损失之中。

另一种致命的错误是过度交易——相对于你的账户来说持有过高的仓位。
\section{2\% 法则}
\begin{tcolorbox}
    2\% 原则会防止你的账户在单次交易中出现本金亏损 2\% 以上的风险。
\end{tcolorbox}

\subsection*{风险控制的铁三角}
事实上,交易规模应该是根据公式计算出来的,而不是随意决定的。可以使用 2\% 原则对你可以交易的最大数量做出理性的判断。这个过程叫作“风险控制的铁三角”(见 \autoref{fig50-1})。

\figures{fig50-1}{A. 你计划要进行的交易的最大风险额度(永远不能超过账户规模的 2\%)。B. 你预计的进场位和止损位之间的价差——你每股所承担的风险。C. 将 A 除以 B,得到所能交易的最大股数。你并不一定要交易这么多,但是不应该超过这个数字。}

随着你的账户增大,你可能会想让每笔交易规模差异化,比如对一般的交易是最大限额的三分之一,对比较有信心的交易使用三分之二,其他更有信心的可能就全额使用了。无论你怎么做,风险控制的铁三角总会为你设定最大允许的交易规模。
\section{6\% 法则}
6\% 原则给每一个账户都设定了一个当月最大回撤比例。如果你达到了限制,这个月接下来的时间就要停止交易。6\% 原则强制你在受到食人鱼攻击前,从水里走出来。

\begin{tcolorbox}
    当你这个月总损失和持仓头寸的风险额度之和达到账户总金额的 6\% 时,在本月剩下的时间内,6\% 原则将不允许你进行新的交易。
\end{tcolorbox}

在与市场的周旋中我们都有过连续盈利的时期,当我们的每笔交易都点石成金时,应该积极地交易。

同样,有一些时候我们的交易变得非常糟糕。交易系统与市场步调完全相反,接连亏损。在这个时候,要重新审视这段时期,不要给自己太大的压力,退后一步、冷静一下尤为重要。专业人士在赔钱的时候可能会去休息一下,但会继续盯着市场,等待与市场的节奏重新匹配上。而业余人士更可能加大交易规模,直到账户出现严重亏损。6\% 原则会使你暂停下来,这时你的账户大体上还是完整的。
\subsection*{可用风险的概念}
如果你用 2\% 原则来设定止损位和交易规模,那么 6\% 原则能给你的账户设定最大风险额度。
\begin{enumerate}
    \item 把你这个月所有的亏损加总。
    \item 把你现在所有的持仓头寸的风险额度加起来。一笔持仓交易的风险额度是你入场点位和止损点位之间的价差,乘以持仓数量。假如你以 50 美元的价格买了 200 股股票,止损价是 48.50 美元,每股承担的风险是 1.50 美元。这样,你该笔交易的风险额度是 300 美元。如果市场向有利于你的方向发展了,你把止损价位上调到盈亏平衡的价位,你的该笔交易的风险额度就会变成零。
    \item 将以上两项相加(这个月的总损失加上持仓头寸的风险额度)。如果两者之和已经超过你月初账户资产的 6\% 时,这个月剩下的时间你都不能再增加交易头寸了,除非市场顺着你持仓的方向发展了,允许你提高了止损线。
\end{enumerate}

如果你根据 6\% 原则已经不能再进行新的交易,还是要继续跟踪自己感兴趣的股票。如果你看到了一个确实想交易的机会,但没有足够的可用风险额度了,可以考虑平掉部分持仓头寸,释放出一些风险额度给它。

当你已经接近 6\% 原则的限制,但发现了一个非常有吸引力的交易机会,此时你有两种选择:你可以兑现一个盈利的持仓头寸来释放可用风险额度;也可以收紧一些持仓头寸的止损线,减小持仓的风险。\textbf{盈利的不是可以依靠提升止损价位来实现额度的释放吗?}
\section{从下降中恢复}
当风险提升时,我们的交易水平会随之下降。初学者能在小交易上赚钱,于是开始有了信心,然后提高交易规模。这往往是他们赔钱的开始。随着头寸的增加,风险也逐渐加大,使得他们的行动变得僵化,不再灵活,这也是他们赔钱的原因。