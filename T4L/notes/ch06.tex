\chapter{市场指标}
现在我们来看看一类与之前不同的工具:市场指标。这类指标分析的是市场的整体,而不是其中具体的某一只个股,市场指标值得交易者参照追踪,因为个股一半的波动是受市场整体趋势影响的。
\section{新高-新低指数(the new high-new low index)}
在任何交易日内股价达到近一年内最高价格的股票就是多方市场多头的领导者,在任何交易日内达到近一年内最低价格的股票便是空方市场空头的领导者。新高-新低指数(NH-NL)通过衡量任意一天中股价创出年内新高的股票数除以创出年内新低的股票数来追踪市场领导的行为。
\subsection*{如何构造新高-新低指数}
\begin{equation}
    \text{新高-新低指数}=\text{创出年内新高的股票数}-\text{创出年内新低的股票数}
\end{equation}
\subsection*{群体心理}
一只股票只有在年内最强势时才会出现在新高名单上。这表明一群焦急的多头正在追逐买入它的股份;一只股票只有在年内最弱势时才会出现在新低名单上,这表明一群激进的空方正在抛售它的股份。

新高-新低指数跟踪交易所里最强势的股票和最弱势的股票,并比较二者的数量,它揭示了领涨个股和领跌个股的力量对比。
\figures{fig34-1}{追踪的是股市大部分时间处于牛市中的某年,其新高-新低指数日线图。每一个上升趋势都难免会被短暂的回调所打断,当新高-新低指数与价格走势发生牛市背离时,该点在上图中被红色斜线箭头标出,便警告人们下跌即将到来。当新高-新低指数从负值上升到正值,该区域被图中紫色圆圈标出,表明下跌趋势已经终止,上升即将开始。尤其是当标普指数超卖时,即下跌到接近其价格运行通道下线时,上述信号极其有效。一如既往地,交易信号在各相互独立的指标结果之间相互支持时尤其有效。}
\subsection*{新高-新低指数交易法则}
交易者需要关注新高-新低指数的三个方面:新高-新低指数位于中线之上或之下的位置、新高-新低指数的趋势以及新高-新低指数与价格走向之间的背离。
\subsubsection*{新高-新低 0 值线}
新高-新低指数与 0 值中线的相对位置关系表明了是空头占据上风还是多头掌控市场。当新高-新低指数在 0 值中线上方,表明多方领导者比空方领导者更多,你最好是加入多方的阵营;当新高-新低指数在 0 值中线下方,表明空方领导者的势力更强大,你最好加入空方的阵营。新高-新低指数在牛市中可以保持在 0 值中线以上几个月的时间,在熊市中也可以持续在 0 值中线以下几个月。如果新高-新低指数为负,并且一直持续了几个月,突然某天上升到中线以上,便是涨势开始启动的信号,此时我们可以使用震荡指标来精确判断时机并寻找买入机会。如果新高-新低指数为正,并且一直持续了几个月,突然某天下降到中线以下,则是跌势开始启动的信号,此时我们可以使用震荡指标来精确判断时机并寻找卖出机会。
\subsubsection*{新高-新低指数趋势}
当市场上涨并且新高-新低指数也上升时,便确认了上升趋势;当新高-新低指数随同市场一起下降时,便确认了下降趋势。
\begin{itemize}
    \item 新高-新低指数的上升表明继续持有多头是安全的,并且还可以加仓。如果当整体市场走平或上升的时候,新高-新低指数却在下降,那便是将多头头寸利润兑现的时候了。新高-新低指数降到 0 值以下,表明空方领导者十分强大,此时继续持有空头头寸是安全的,并且可以继续卖空。如果在市场持续下跌时新高-新低指数不降反升,则表明跌势已经开始慢慢丧失领导地位,是时候回补以降低空头仓位了。
    \item 如果在市场走平的日子里新高-新低指数上升,则显现出了牛市的预兆,是买入的信号。这也表明了长官开始身先士卒,向上突围,而士兵们仍在散兵坑里蹲守。如果在市场走平的日子里新高-新低指数下降,则显现出了卖出的信号,表明长官已经提前放弃抵抗,然而士兵们还在坚守。士兵们也不笨,如果他们发现长官开始逃跑了,他们也不会继续恋战。
\end{itemize}
\subsubsection*{新高-新低指数背离}
如果近期市场的新高伴随着新高-新低指数同时创新高,则即使上升过程受到短暂回调的阻碍,涨势也将很可能继续持续下去。当市场新低伴随着新高-新低指数达到新低,表明空头很好地主宰了趋势,跌势将继续持续下去。反之,新高-新低指数与市场指数的背离则表明此时的市场领导者在撤退,趋势可能将发生反转。
\begin{itemize}
    \item 如果市场创出新高而新高-新低指数只达到了一个次高点,就产生了熊市背离,这表明虽然市场整体正在走高,但多方领导力正在弱化。熊市背离表明上升趋势结束,但是同时我们需要关注次高点的高度。如果新高-新低指数的最近峰值只是稍微比 0 值高一些,在 +100 以内,那么随后可能会出现一次重大回调,此时应该卖空。另一方面,如果最近峰值高于 +100,则表明向上的领导力还很强大,足以阻止市场下跌。
    \item 如果市场创出新低而新高-新低指数只达到了一个次低点,就产生了牛市背离。这表明尽管市场在走低,但空方领导力正在弱化。如果近期低点在 -100 以内,表明空方领导力耗尽,一个重要的向上反转即将来临。如果近期低点较深,则表明空方还有些力量,下跌趋势可能会稍有停顿,但趋势不会反转。要记住,股市底部的牛市背离比股市顶部的熊市背离发展得快,所以要快买慢卖。
\end{itemize}
\textbf{这个 +100 和 -100 是不是太绝对了?}
\subsubsection*{65 日和 20 日新高-新低指数}
最近这些年在新高-新低指数分析中最大的创新点就是两个新的历史追溯时间周期的加入:20 日和 65 日。常规的日度新高-新低指数是每日的最高价和最低价与过去一年内的最高价、最低价进行比较。20 日新高-新低指数比较的是过去一个月的高低价数据,而 65 日新高-新低指数比较的是过去一个季度的高低价数据。较短周期的新高-新低指数的分析对于短期择时很有帮助。

这两种新的时间周期比标准的年度新高-新低指数传递了更加敏锐的信号。其中的逻辑很简单:在某只股票达到近一年内的新高之前,它必须先达到月内新高,然后再达到季度内新高才行。如果某只股票正处于下降趋势中,此时需要很长的一段时间才能让价格恢复并创造近一年来的新高,但创造月内新高和季度内新高需要的时间就没那么长了。

根据新高-新低指数来追踪市场领导者将帮助我们改进市场择时的准确度。有两种方法来使用新高-新低指数:首先,由于个股走势在很大程度上依赖于市场整体趋势,我们可以使用新高-新低指数来决定个股何时买入或卖出;其次,我们可以使用新高-新低指数来交易那些跟随市场整体走势的投资标的,比如标普 500 电子迷你股指期货。
\section{50 日均线上的股票数占比}
每一个价格都代表着市场参与者对价格达成的暂时性共识,而移动均线则代表着一段时期内对价值达成的暂时性共识的平均值。这也意味着当一只股票股价处于其均线以上,即当下对价格达成的共识高于其过去共识的平均值时,这就是牛市;当一只股票股价处于其均线以下时,也就意味着熊市。

当市场向上走时,股价处于均线之上的股票数占总数的百分比也会增加。在市场下降过程中,股价处于均线之上的股票数占总数的百分比则会持续缩水。
\figures{fig35-1}{50 日均线上的股票数占比指标达到一个极端值——75\% 以上或者 25\% 以下,然后又从该水平反弹,说明中期趋势很可能走到了一个转折点。该指标的反转闪现出了整个市场的一个信号:在该指标上升的时候买入并在该指标下降的时候卖出。在 2013 年的年末,此时市场开始了一轮几乎没有回调的持续上涨,在该指标达到 25\% 以上时,上升过程中的买入信号就开始显现。这些指标并不能标示出每一次反转——没有任何一个指标能做到这样,但一旦该信号显现出来时,我们最好要注意它。}

在周线图上来分析该指标,这可以捕捉到中期的反转点——当市场出现这种情况时通常预示着该反转趋势将持续数周或者数月。
\begin{tcolorbox}
    交易价格在 50 日均线以上的股票数占总股票数的百分比这一指标并不是通过其百分比达到某一特定水平来发出交易信号,而是通过接近某一水平之后发生反转来发出交易信号。该指标通过上升到上限参照线之上再下跌到该参照线之下来显示顶部即将完成的信号;同样,该指标通过下降到下限参照线之下再回升到该参照线之上来暗示底部已经形成。
\end{tcolorbox}

注意到该指标的顶部通常是很宽的,同时其底部却是很陡峭的。顶部的形成是由于贪婪,这是一种让人感到快乐并且能够持续很久的情绪;底部的形成是由于恐惧,这是一种让人反应剧烈但是持续时间不长的情绪。

有一些指标在捕捉反转时机时刚刚好,而其他部分指标可能只能捕捉到主要趋势持续过程中的短暂反弹。我们可以让这些指标成为一种参考,而不是仅仅使用其中单一的某个指标来做交易决策。在做决策的时候参考多重指标,当它们的结果相互支持时,它们之间的可靠度将得到相互加强。
\section{其他市场指标}
\subsection*{腾落指数(advance/decline,A/D)}
腾落指数(A/D)测算大众参与上涨和下跌的程度。该值是将每天收盘价上涨股票家数减去收盘价下跌股票家数后的值。如果道琼斯工业指数追踪的是整体的表现,新高-新低指数关注市场中的长官们,则 A/D 线显示的是士兵们是否在跟随长官。当 A/D 线与道琼斯指数同步创出新高或者新低时,上涨或下跌趋势延续的可能性更大。

交易者应当关注 A/D 线的新高和新低,而不是其绝对值的大小,因为它的绝对值主要取决于起止日期。如果股市创出新高的同时 A/D 线也创出新高,表明上涨有广泛的基础,上升趋势有望得以延续。相对于没有广泛基础的上涨和下跌来说,有着广泛基础的上涨和下跌趋势的持久力更强。如果股市创出新高,而 A/D 线只达到了一个低于前期上涨高点的次高点,这便表明参与上涨的股票数在减少,上涨趋势可能已经接近尾声。当市场创出新低,而 A/D 达到了一个高于前期低点的次低点时,表明参与下跌的股票数在减少,下跌趋势正在接近尾声。这些信号如果不是比市场反转提前几个月发出,那么至少也要提前几周发出。
\begin{tcolorbox}
    比如说大权重的股票在上涨,推动指数不断上涨,但是腾落指数没有新高,这是应该去 Short?
\end{tcolorbox}

最活跃的股票指数(Most Active Stocks, MAS)是纽约交易所中 15 只最活跃股票的 A/D 线。
许多报纸每天都会把这些股票列出来,只有当某只股票吸引了众多投资者眼球
并极具市场人气时才会出现在该列表上。最活跃的股票这一指标(MAS)是
一个大资金指标,它显示大资金是看多还是看空。当 MAS 走向与市场趋势发
生背离时,市场将极有可能发生反转。(A 股的狗屎之处在于,这些都是游资疯狂炒作的垃圾股)

\figures{fig36-1}{腾落指数线走势的转折经常和价格发生的转折相一致,有时候它会发生在价格反转之前。这种能够提前给人警告的特点使得腾落指数线的走势值得交易者去追踪。在 A 区域,价格在底部徘徊并创新低,然而此时的腾落指数线走势却暗示出了反弹的到来。在 B 区域则恰好相反——价格不断走高,然而此时腾落指数线的走低表明下跌即将来临。在 C 区域,价格持续走低,然而腾落指数线上升,预示着将要发生反弹。这些警告信号并不是在每次转折点中都出现。}

基于低价股的成交量来构造的指标已经由于美国股市成交量总体的增长以及道琼斯指数 10 倍的增幅而失效。在期权交易流行起来后,成员卖空率(member short sale ratio)和专家卖空比例(special short sale ratio)也退出了历史舞台。成员卖空和专家卖空现在只是和跨市场套利结合在一块儿。当保守的零星股票交易者转向购买共同基金时,零星交易统计数据(odd-lot statistics)也就失去了其价值。在投资者有卖出期权后,零股融券比(odd-lot short sale ratio)便失去作用。
\section{一致性指标(consensus indicator)和投入指标(commitment indicator)}
财经记者和投资顾问对趋势变化的反应总是慢一拍,错过了重大转折点。当这些人转向强烈的看多或看空时,进行反向操作将有利可图。当财经记者和投资顾问对牛市或者熊市的判断达成高度共识时,便是趋势已经持续很长时间并即将迎来反转的信号。

一致性指标,又以反向指标著称,它不适合用作精确的择时。它们只是让你注意到趋势已经衰竭的事实。当你得到这一信息时,要用技术指标来把握更精确的时点。

只要多空双方还有冲突,趋势就会持续下去。当群体达成强烈的共识时,趋势将会反转。当群体高度一致地看多时,你就可以做好卖出的准备了。当群体高度一致地看空时,则准备买进。因为价格是由群体决定的,当大多数人转而看多时,剩下的能支撑牛市的买家就不多了。
\subsection*{跟踪投资顾问的观点}
趋势持续的时间越长,投资顾问们宣称趋势将要持续的声音就越大。投资顾问在市场顶部时最乐观,在市场底部时最悲观。当大多数投资顾问变得强烈看多或看空时,采取反向操作的策略不失为明智之举。

有些投资顾问很擅长将话说得模棱两可。那些说两面话的人不管市场走向如何,事后都会宣称自己的判断是正确的,然而提供跟踪市场服务的杂志编辑拥有丰富的经验来看穿这种两面话。
\subsection*{来自媒体的信号}
为了理解任意一组人群,你必须知道其成员渴望什么,害怕什么。财经记者需要显得严肃、聪明、信息灵通;他们害怕显得无知或者不切实际。这也是为什么他们会常常表现得像墙头草,对每一个问题都提出好几个不同的方面。只要财经记者写一些类似于“除非有某些不可预见的因素导致市场下跌,否则货币政策有望助推市场上扬”的东西,他们就很安全。
\subsection*{来自广告的信号}
如果某一主流报纸或杂志刊登了 3 个或更多的广告忽悠同样的“机会”,往往表明顶部即将来了。这是因为只有已经深入人心的上升趋势才能突破几家经纪公司的惯性。当他们一致认同这一趋势,并提供操作建议、进行宣传、在报纸上登广告时,说明趋势实际已经非常过时了。
\subsection*{期货交易者的投入}
政府机构和交易所收集不同类型交易者的买卖情况,发布汇总的持仓情况数据。跟随那些有成功投资记录的交易者操作,采取与那些操作记录持续糟糕的交易者反向操作不失为良策。

商品期货交易委员会将所有市场参与者分成三类:贸易商、小投机者、大投机者。贸易商,又称为套期保值者,是在正常生产经营中需要用到实际商品的企业或个人。从理论上来讲,他们参与期货交易是为了规避经营风险。例如银行参与利率期货交易以规避贷款组合的利率风险,或者是食品加工企业参与小麦期货交易以对冲购买谷物的价格波动风险。套期保值者只需交一小部分保证金,而且不受投机仓位的限制。

大投机者是那些仓位达到了申报水平的交易者。商品期货交易委员会报告贸易商和大投机者的买卖情况。为了了解小投机者的仓位情况,你必须用总持仓量减去前两类交易者的持仓量。

知道了某一群体是做多还是做空还不够。贸易商往往在期货市场上做空,因为他们中的许多人拥有商品现货。小交易者往往做多,这反映了他们多年来的乐观本性。为了从商品期货交易委员会的报告中得出有价值的结论,你必须将他们的现有仓位与其历史正常仓位进行对比。
\subsection*{合法的内部人交易}
单个内部人的买卖说明不了什么。例如,某位高管可能因为个人的重大支出而抛售所持公司股份,也可能因为行使股票期权而买入股票。研究合法内部人交易的分析师发现,只有当3个及以上的高管或大股东在一个月内持续买入或卖出时,内部人的买卖才有意义。他们的买卖行为表明,公司即将发生某些十分正面或负面的变化。如果有3个内部人在一个月内持续买进,这只股票就有望上涨;如果在一个月内有3个内部人持续卖出,这只股票则有望下跌。

\textbf{大量内部人的买入行为比起大量内部人的卖出行为更具有预测价值}。因为内部人卖出股票的理由可以有很多(多样经营、买新住房、送小孩读大学),但他们买入股票的主要理由只有一个,就是预期他们公司的股票价格未来将上升。
\subsection*{空头净额}
交易所报告的数据当中包括任意一只股票被卖空的数量。各股票被卖空数量的绝对值千差万别,我们可以通过比较被卖空的股票数量与该股票的流通数量(所有社会公众持有的可以交易的股票数量总和)来进行相互之间的比较分析。由此得到的数字——总流通股的卖空比例,倾向于在 1\% 和 2\% 之间变动。

另一种观测空头净额的有效方法就是将卖空量与日成交量做比较。通过这样,我们提出一个假设性的问题:当所有空头都决定回补,而其他买入者在场外观望,并且日成交量保持不变,那么此时需要多少天才能使空头完成回补并将空头头寸降到 0?这种“回补天数”通常在 1-2 日内震荡。

当你计划买入或卖空一只股票时,参考其总流通股的卖空比例和回补天数十分有意义。如果这两个值很高,说明空头一方已经十分拥挤了。市场上涨将吓坏那些空头,从而会恐慌性地进行回补,导致股票价格升得更高,这实际上有利于多头而不利于空头。

无论何时,当你寻找你将要买入的股票时,先看看该股票的总流通股的卖空比例和回补天数。这种指标在正常时不会给你提供十分有价值的信息,但这些指标与其常规值一旦发生偏离,将是十分有用的情报。
\figures{fig37-2}{这些数字反映出绿山咖啡公司的股票遭到了激进的卖空。请不要忘记,每一次卖空都是需要在某一时刻进行回补来平掉空头仓位的。或许机智的空头知道绿山咖啡公司遭遇到了很大的问题,但要是此时其股票反而上升了呢?很多空头此时会选择回补来平仓,当他们互相争夺着进行补仓时,股价又会高涨。不管其股票长期走势会怎样,短期内其价格将依然会上升。}

高卖空比例对任意一只股票的卖空都是危险的信号。将其拓展开来的话,如果你的技术指标建议你买入一只股票,该只股票的卖空比例很高,则会是一个积极的因素,它表明上升过程有更多所需的燃料。对于波段交易者来说,在选择自己需要买入和卖出的股票时将卖空数据包含进分析当中是有意义的。我本人在寻找潜在交易机会的时候总会关注这些数字。