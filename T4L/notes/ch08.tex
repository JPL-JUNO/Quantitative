\chapter{交易工具}
你所要做的最重要的市场决定之一就是选择什么品种去交易。不管是哪一类,你需要确保所选交易工具满足两个重要的条件:\textbf{流动性和波动性}。

流动性指与这一类别中的其他交易品种相比的日均成交量。日均成交量越大,你进入和退出交易就越容易。在流动性不好的股票中,你或许能建立浮盈的持仓,但当你退出时,却变成亏损,因为买卖价差特别大。

波动性是交易品种短期运动的范围。交易品种的波动性越高,交易机会就越多。受欢迎的股票往往波动性很大。许多公共事业部门公司的股票流动性很好,但因为波动性很低,而很难交易——它们常常在狭窄的价格区间里震荡。

衡量波动性的方法有许多,其中一个很实用的工具是“贝塔”($\beta$,Beta)。它是任意交易品种的波动性与其交易基准——比如说大盘指数——的波动性之间的比值。比如说,某只股票的贝塔值是 1,意味着它的波动性和标准普尔 500 指数的波动性一样。贝塔值为 2 的股票意味着当标准普尔指数上升 5\% 时,它可能会上升 10\%;但在标准普尔指数下跌 5\% 时,它也可能下跌 10\%。
\section{股票}
\section{交易所交易基金(ETF)}
ETF 行业一直对一个事实秘而不宣,就是实际存在两个 ETF 市场。一级市场仅仅对经授权的参与者开放——大型股票交易经纪商,他们和 ETF 发行者之间有进行大批量买卖的协议——常常以万份来计算。这些经纪商以批发的形式买入,然后以零售方式卖给你。你作为个人投资者只能坐在公共汽车的后排——也就是二级市场。
\section{期权}
需要记住很重要的一点是,期权的买方作为一个群体,长期来说是亏钱的,除了偶尔会出现几笔幸运的交易。在交易的另一边,期权的创造者整体能获得稳定的收益,尽管偶尔有损失。

期权是一种关于"希望"的生意。你可以买入希望也可以卖出希望。我是专业人士——所以我卖出希望。早上我来到交易大厅,找出市场所希望的是什么。然后对这些希望定价,再卖给他们。

期权的价格有两部分组成——内涵价值和时间价值。
\subsection*{立权者的选择}
时间是期权买入者的敌人。

既然每个期权都代表一个希望,那最好是销售那些不可能实现的希望。订立认购或认沽期权前,采取以下三个步骤。
\begin{enumerate}
    \item 分析订立期权所对应的标的证券。使用三重滤网交易系统来分析股票、期货或指数,判断其是呈现趋势还是非趋势变化。使用周线图和日线图的趋势跟随指标和震荡指标来识别趋势,侦测逆转,并设置目标价格。避免在股票要公布业绩时立权——在这些可能出现大幅波动的期间内,不要持有期权卖出头寸。
    \item 选择所立期权的种类。如果你分析认为当前是熊市,那么考虑卖出认购期权;如果你分析认为当前是牛市,那么考虑卖出认沽期权。当趋势是上涨的,那么卖出市场会反转下跌的希望;当趋势是下跌的,那么卖出市场会反转上涨的希望。当市场趋势很平,权利金也很低时,不要卖出期权——否则市场从区间震荡中突破出来时可能导致亏损。
    \item 估计在足够的安全系数下,一只股票需要涨多少才能改变趋势。然后在这个幅度以外设立行权价,并卖出相应期权。
\end{enumerate}

谨慎的立权者应当瞄准买进或卖出那些 Delta 值不高于 0.1 的期权。它表示在到期日期前期权只有 10\% 的可能性会触及行权价格。记住,作为立权者你不会希望标的证券触及行权价格:你要卖的是不会实现的希望。如果你觉得 10\% 的可能性太高了,别忘了 Delta 值是没有加入任何市场分析的。如果你还加入很好的技术分析,那么你的风险会比 Delta 值显示的要更低。\warning{Delta 还有这种意思?}
\subsection*{买入期权可否变得明智}
在预期会有剧烈下跌的情况下,专业人士偶尔也会买入认沽期权。当长期上升趋势开始转向时,和远洋客轮更改航线时一样,在趋势顶点附近可能会出现巨大的波动。当波动率飞涨时,即使是经验老到的交易者也难以控制好仓位,做好有效的止损。这时候买入认沽期权可以帮助你规避这个问题。

交易者预期价格要下跌,那他需要决定买入何种认沽期权。最佳选择是跟一般直觉反着来,要和大多数人不一样。
\begin{itemize}
    \item 先预估股票会跌到何种程度。只有会发生暴跌,认沽期权才值得购买。
    \item 避免购买离到期日还有超过两个月时间的认沽期权。只有当你预计会出现自由落体式的暴跌时,购买认沽期权才是有意义的。如果你预计是一个长期的下跌趋势,那么最好直接做空相应证券。
    \item 寻找那些毫无行权希望的低价认沽期权。看一下期权报价栏:认沽期权行权价越低,价格越便宜。起初,行权价每下降一格,认沽期权价格会比上一格便宜 25\% 甚至35\%。到最下面的价位时,每个价位之间只有很少一点价差,表明所有行权的希望都已从认沽期权中挤出,剩下的像一张便宜的彩票。这正是你所要的。
\end{itemize}
\section{价差合约(CFD)}
\section{期货}
\subsection*{套期保值}
套期保值是持有与其商品现货头寸相反方向的期货头寸。

套期保值者放弃了发横财的机会,但是也避免了价格波动的影响。套期保值者把价格风险转移给了进入市场的投机者,这些投机者被潜在的利润所吸引。有讽刺意味的是,有内部信息优势的套期保值者竟然对价格没有信心,而大众中的激进的门外汉会投大量的钱到期货中赌价格的变化。

\subsection*{供给、需求和季节性}
期货交易者必须了解他正在交易的市场的关键供需影响因素。例如,在农产品关键的生长和收获月份,他必须保持对天气的关注。在期货市场中,趋势交易者倾向于寻找供应驱动的市场,然而波段交易者则在需求驱动的市场中也可以做得很好。
\subsection*{地板和天花板}
与股票不同,商品交易很少低于一定的价格低点(地板)或高于一定的价格高点(天花板)。地板价取决于商品生产成本。不管是黄金还是糖,当一种商品价格低于该水平,矿工就会停止挖掘,农民就会停止播种。

天花板取决于替代品的成本。如果一种商品的价格上涨了,消费者便会转向其他替代商品。如果重要的动物饲料(如豆粕)变得太贵了,需求将会转向鱼粉;如果糖变得太贵了,需求就会转向玉米甜味剂。

为什么很少有人在地板价或天花板价的水平上进行交易呢?他们为什么不在靠近地板价的位置买入和在天花板价附近卖出,就像钓取水桶里的鱼一样轻松获利呢?首先,无论是地板价还是天花板价都不是石板一块那么稳定,市场可能短期内违背它。更重要的是,人性与这些交易是相冲突的。大部分投机者都没有勇气在市场上近乎沸腾屡创新高的地方卖出,或者在市场已经崩溃后的位置买入。
\subsection*{升水、反转和价差}
正常来说,近月合约比远月合约价格要便宜,这种关系叫作升水。

远月合约价格更高,体现了“持仓成本”——融资、存储和商品保险等成本。交割月份之间的这种差异称为溢价,套期保值者会密切关注它们。当供给收紧或需求增加时,人们开始买入近月合约,远月合约的溢价萎缩。有时近月合约变得比远月合约更加昂贵,市场反转了!出现了真正的短缺,\important{人们愿意付出额外的钱以更快地拿到想要的商品}。这种“反转”是商品市场出现牛市的最强烈标志之一。

当你寻找反转的时候,请记住,有一个市场其升水反转结构是常态的,那就是利率期货。因为那些持有资金头寸的人在持续收息,而不是支付财务和仓储费用。

期保值者是市场的主要空头力量,大多数投机者则是永远的多头,但场内交易者喜欢交易差价。差价交易表示在市场中买入某月合约的同时卖出其他月合约,也可以是做多一个品种的同时做空另一个相关品种。
\section{外汇}
大多数初学者在外汇超市开户,在那儿他们立刻陷入一个致命的弱点——你的经纪商是你的敌人。当你交易股票、期货、期权的时候,你的经纪商是你的代理人。他们执行你的交易指令,并收取一定的费用。而在外汇交易场所(价差合约也是同样情况),你的经纪商是你交易的对手。你赢他们就输,你输他们就赢。因为交易平台手中拥有更多的牌,他们有更多的方式来获得他们想要的结果。