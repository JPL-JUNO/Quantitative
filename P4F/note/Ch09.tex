\chapter{输入/输出操作\label{Ch09}}
\section{PyTables 的 I/O}
\subsection{使用压缩表}
使用 PyTables 的主要优点之一是压缩方法。使用压缩不仅能节省磁盘空间,还能改善 I/O 操作的性能。
\subsection{内存外计算}
PyTables 支持内存外计算,因此可以实现不适合于内存的基于数组的计算。为此,考虑以下基于 \verb|EArray| 类的代码。这种对象可在一维(行方向)上扩展,但列数(每行元素数)必须固定。

对于没有导致聚合的内存外计算,需要另一个同样组成(大小)的 EArray 对象。PyTables有一个特殊模块,可以高效处理数值表达式。这个模块叫作 Expr,它基于数值表达式库 numexpr。

\section{TsTables 的 I/O}
TsTables 软件包使用 PyTables 来为时间序列数据构建高性能存储。主要使用场景是“一次写入,多次检索”。这在金融分析中是典型的场景:数据在市场中创建,实时或者异步读取,并存储在磁盘上供以后使用。该数据可能用于较大的交易策略验证程序,这种程序需要一次又一次地使用历史金融时间序列数据的不同子集,此时快速的数据检索非常重要。

