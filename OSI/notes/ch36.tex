\chapter{波动率交易的基本原理}
几乎所有的股票、指数或期货合约的隐含波动率的图形都有相似的模式:一个交易范围。隐含波动率完全突破它的“正常”范围的唯一情况是,如果有显著的事件发生,它改变这个股票的运动方式的基本面要素的话(例如,有人要买这个企业,或者重要兼并,或者股票其他类型的价值稀释等)。
\section{波动率的定义}
波动率只是一个用来描写一只股票、期货或者指数的价格变化有多快的术语。当你就期权而谈到波动率的时候,有两类波动率是重要的。第一类是历史波动率(historical volatility),它是对标的工具在过去的价格变化快慢的衡量。另一类是隐含波动率,它是期权市场对这个标的物在这个期权的存续期内的波动率的预测。计算和比较这两种对波动率的衡量,可以在预测标的工具即将出现的波动率方面立刻对交易者有所帮助,这在决定今天的期权价格上是至关重要的。