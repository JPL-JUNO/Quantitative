\chapter{理论定价模型}
\section{布莱克-斯科尔斯模型}
\subsection{行权价格}
对于期权合约的执行价格不会存在任何疑问,因为执行价格是期权合约中清楚写明的,且在期权存续期内不会发生改变。\footnote{当然,有时交易所在股票拆分的情况下也会调整期权合约的执行价格。但实际上这并没有改变执行价格,因为执行价格与股价间的对应关系并没有改变。期权合约特征本质上并没有变化。}
\subsection{到期时间}
与执行价格类似,期权的到期日是固定的,不会发生改变。\footnote{我记得实践中是可以更改的}当然,每过一天都使期权合约离到期日越来越近,所以某种意义上到期时间是越来越短的。

在金融模型中,一年通常是标准时间单位。因此,到期时间作为年化数字输入 Black-Scholes 模型。如果我们用天数来表示时间,则必须通过将到期天数除以 365 来进行适当的调整。然而,大多数期权评估计算机程序已经将这种转换合并到软件中,因此我们只需要输入正确的到期剩余天数。

对于输入模型的准确天数似乎很难确定。之所以需要剩余天数,是因为两个原因:一是计算利息的需要,二是计算标的合约价格变动的需要。为计算评估市场价格变化速度的波动率变量,我们仅关心可能发生价格变动的交易日天数,只有在交易日中标的合约的价格才会变化,这需要将期权合约剩余时间中的周末和节假日去掉。另外,为计算利息我们要将期权合约到期时间剩余的每一天都计算在内,因为无论资金借入或借出,我们都要考虑交易日和非交易日的天数。

尽管交易者通常以天为单位表示到期时间,但交易者可能希望使用不同的衡量标准。特别是当到期日临近时,交易者可能更喜欢使用几个小时甚至几分钟。理论上,更精细的时间增量应该会产生更准确的值。但使用非常小的时间增量存在实际限制。随着时间的推移,我们输入理论定价模型的离散时间增量可能无法准确代表现实世界中时间的连续流逝。大多数交易者通过经验了解到,随着到期日的临近,理论定价模型的使用变得不太可靠,因为输入变得不太可靠。事实上,在临近到期时,许多交易者完全停止使用模型生成的值。
\subsection{标的合约价格}
与执行价格和到期时间不同,正确的标的合约价格通常难以确定。事实上,市场在任何时间点上都存在着买价和卖价,通常难以确定应该使用二者中的哪一个,或两个价格中间的某个价格。

尽管我们关注的是理论定价模型的使用,但我们应该强调,没有法律规定交易者必须基于或符合理论定价模型做出任何决策。交易者可以简单地买入或卖出期权,并希望交易结果有利。但使用定价模型的纪律交易者知道,他需要通过在基础合约中采取相反的市场头寸来对冲期权头寸。因此,他输入理论定价模型的基础价格应该是他认为可以进行相反交易的价格。如果交易者打算买入看涨期权或卖出看跌期权(两者都是多头市场头寸),他将通过卖出标的合约进行对冲。在这种情况下,他会希望使用接近买入价的价格,因为这是他可能出售标的资产的价格。另一方面,如果交易者打算卖出看涨期权或买入看跌期权(两者都是空头市场头寸),他将通过购买标的合约进行对冲。现在,他将希望使用接近要价的价格,因为这是他可能购买标的资产的价格。
\subsection{利率}
利率因素在期权理论定价过程中产生两个方面的影响:一是影响标的合约的远期价格,如果标的合约是股票型结算模式的,利率水平提高就相当于远期价格提高,从而使看涨期权价值增大、看跌期权价值降低;二是利率水平会影响期权的持有成本,如果期权是股票型结算模式的,利率水平提高就相当于期权合约价值降低。尽管利率在期权定价过程中产生两个方面的影响,多数情况下同一利率水平能反映两种影响,因此输入模型的是一个利率值。但在某些模型中需要根据两种影响输入不同利率水平,比如在外汇期权(外国货币利率产生一种作用,本国货币利率产生另外一种作用)理论定价模型中需要输入两种利率水平值,这就是布莱克–斯科尔斯模型的 Garman-Kohlhagen 版本所需要的。

利率产生两种作用的事实,意味着利率对于不同标的资产、不同结算方式的期权的重要性不同。比如说,利率对于股票期权价值就比期货期权价值的影响更大,利率水平提高使股票远期价格提高,但期货合约的远期价格并不因此发生变化。同时,股票型结算模式下利率水平提高期权价值降低。

期权定价中,交易者应该使用哪一种利率水平呢?教科书上常常建议使用无风险利率,也就是用于最有信誉的借款人的利率。在实践中,没有个人能够以与政府相同的利率借贷,因此使用无风险利率似乎不现实。为了确定更现实的利率,交易者可能会寻求利率合约的自由交易市场。
\subsection{股利}
为了对股票期权进行准确定价,交易者必须知道股利发放数量和除息日(ex-dividend date),交易者只有在除息日前持有股票才能获得股利,这里强调股权的所有权。深度实值期权虽然具有很多与股票类似的特征,但只有拥有股票才有获得股利的权利。在缺少其他信息的情况下,绝大多数交易者倾向于假设上市公司会延续以前的股利政策。
\subsection{波动率}