\chapter{个体心理}
\section{现实与幻觉}
\subsection*{交易时睁大双眼}
一厢情愿的力量甚于金钱。最近的研究证明,人们都有惊人的欺骗自己和回避现实的能力。
\begin{tcolorbox}
    你必须像分析你的交易那样分析你的心理情绪,以确保能做出正确的决定。你的交易必须建立在定义明晰的规则之上,你得学会规范资金管理,以确保在出现连续亏损后也不会被踢出局。
\end{tcolorbox}
\section{自我毁灭}
\subsection*{赌博}
相比于在赌马游戏中填数字,在金融市场中赌博有着更耀眼的光环,给人以高深莫测的感觉。
\subsection*{自毁}
当交易者陷入困境时,他们往往去责怪别人、坏运气或者其他随便什么。毕竟寻找导致自身失败的原因总是痛苦的。
\begin{tcolorbox}
    坚持写交易日志——记录每次建仓和清仓的理由,寻找你成功和失败的重复模式。
\end{tcolorbox}
\section{胜利者与失败者}
\subsection*{大海般的市场}
市场不知道你的存在,你的存在也无法影响市场。它不会故意关照你的财富,也不会特意伤害你的资金。你要做的,只是控制自己的行为。

经过一连串盈利之后,新手可能觉得他可以在交易市场上如鱼得水来去自如,他开始肆无忌惮地冒险,直到亏得一无所有。还有一种可能,业余交易者在经历连续损失之后,倍受打击,以至于当自己的交易系统给出明确的买卖信号时,他也无法下定决心做出指令。当交易让你得意忘形或垂头丧气时,你的神智已被遮蔽。当喜悦让你飘飘然时,你会非理性交易,然后亏损;当恐惧支配着你的心灵时,你会错过获利的机会。

作为交易者,当被市场整得死去活来的时候,该做的便是减少交易的规模。记住,尚在摸索学习中或感到紧张之时,千万要减小交易规模。

\subsection*{为你的生涯负责}
你可以根据自己的情况调整以下这个清单:
\begin{enumerate}
    \item 坚定自己在市场中长期作战的意念——从现在开始至少交易 20 年。
    \item 像海绵一般地学习,关注专家的观点,但对任何事情都要保持有益的怀疑态度。遇到有疑问的地方要刨根究底,而不是简单地接受专家的观点,或只理解他们字面上的意思。
    \item 不贪婪,不急于交易——要把你的时间用于学习,市场一直在这里,未来无尽的岁月中会有更多更好的机会。
    \item 培养分析市场的方法,换句话说,就是“如果 A 发生,那么 B 很可能会发生”。市场有很多维度,要使用多种解析方法来确认自己的交易决策。要学会用历史数据测试交易决策,随后在市场上真枪实弹地进行交易。市场瞬息万变,你需要的是根据牛市、熊市、震荡市等不同的特征采用不同的工具进行交易,同时还要有所区分。
    \item 建立一套资金管理计划。你的第一目标是必须长期生存下去,第二目标是资本的稳定增长,第三目标才是赚取高额利润。大多数交易者对第一目标和第三目标的重要性产生了混淆,将第三目标放在了第一位,更有甚者,都不知道第一目标和第二目标的存在。
    \item 要认识到交易者在任何交易系统中都是最为薄弱的一环。假如有条件的话,去匿名戒酒会学习一下如何避免损失,或者建立一套属于自己的方法来克制情绪化交易。
    \item 胜利者在思考、感受与行动上的方式与失败者是完全不同的。你必须深探自己的内心,驱赶那些幻觉,改变你原来的思考、感受与行动的方式。这样的改变通常都不容易,但如果你想成为一名专业交易者,你必须专注于自我改善和培养自己的个性。
\end{enumerate}