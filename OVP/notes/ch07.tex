\chapter{风险度量 1}
利率变化的影响可能会因标的合约和结算程序的不同而有所不同。

利率变化会以两种方式影响期权。首先,它可能会改变标的合约的远期价格。其次,它可能会改变期权的现值。在股票期权市场中,利率上升将提高远期价格,导致看涨期权价值上升,看跌期权价值下降。与此同时,更高的利率将降低两者的现值。看跌期权的价值显然会下降,因为这两种结果都会降低看跌期权的价值。但对于看涨期权而言,结果却具有相反的效果。更高的远期价格将导致看涨期权的价值增加,但更高的利率将降低看涨期权的现值。由于股票价格始终高于期权价格,因此远期价格的上涨总是比现值的降低产生更大的影响。因此,\important{股票看涨期权的价值将随着利率上升而上升,随着利率下降而下降。股票看跌期权的价值则恰恰相反,随着利率上升而下降,价值则上升}。

期权价值取决于交易者是用多头股票还是空头股票进行对冲,这一事实带来了大多数交易者希望避免的复杂情况。这为股票期权交易者提供了一条有用的规则:
\begin{tcolorbox}
    只要有可能,交易者就应该避免持有空头股票仓位。
\end{tcolorbox}
\section{Delta}
Delta($\Delta$)是衡量期权相对于标的合约变动方向的风险的指标。Delta 为正表示希望价格向上变动;Delta 为负表示希望价格向下变动。Delta 有几种不同的解释,任何一种解释都可能对交易者有用,具体取决于所执行的策略类型。
\subsection{变化率}
到期时,期权的价值恰好等于其内在价值。然而,在到期之前,期权的理论价值是一条曲线,随着期权进入价内非常深或远离价外,该曲线将接近内在价值,如 \autoref{fig7-4} 所示。当标的价格上涨时,曲线的斜率趋近于 +1;当标的价格下跌时,曲线的斜率趋近于零。任何给定标的价格的看涨期权的 Delta 就是曲线的斜率--期权价值相对于标的合约变动的变化率。

\figures{fig7-4}{看涨期权的理论值。}

看跌期权的特征与看涨期权相似,只是看跌期权的价值与标的市场走势相反。在 \autoref{fig7-5} 中,我们可以看到,当标的价格上涨时,看跌期权的价值会贬值;当标的价格下跌时,看跌期权的价值会上升。因此,看跌期权的 Delta 总是为负,范围从远价看跌期权的 0 到深价看跌期权的 -1。与看涨期权 Delta 一样,看跌期权 Delta 衡量的是看跌期权的价值相对于标的价格变化的变化率,但负号表示变化将与标的合约的方向相反。

\figures{fig7-5}{看跌期权的理论价值。}

尽管看涨期权的 Delta 值范围是 0 到 1.00,看跌期权的 Delta 值范围是 0 到 -1.00,但许多期权交易者通常将 Delta 值表示为整数,去掉小数点,我们也将遵循这一惯例\footnote{这一惯例起源于美国股票期权市场,股票期权交易者通常将一个 Delta 等同于一股股票。由于标的合约包含 100 股股票,因此交易者将 Delta 指定为 100。许多期货期权交易者也使用这种整数格式来表示 Delta。}。使用这种格式,看涨期权的 Delta 值将在 0 到 100 的范围内,看跌期权的 Delta 值将在 -100 到 0 的范围内。标的合约的 Delta 值始终为 100。

\section{Gamma}
Gamma ($\Gamma$) 有时也称为期权曲率,是标的价格变化时 Delta 的变化率。Gamma 通常以标的价格每变化一个点而增加或损失的 Delta 来表示,标的价格上涨时 Delta 增加的量等于 Gamma,标的价格下跌时 Delta 减少的量等于 Gamma。

\figures{fig7-6}{Delta 值}

虽然 Delta 是衡量期权价值在标的价格变动时如何变化的指标,但重要的是要记住,它
代表的是瞬时指标。它只适用于非常小的价格变动。如果标的价格发生相当大的变动,使用恒
定 Delta 估计期权的新价值将变得越来越不可靠。但是,如果我们也考虑 Gamma,我们可
以改进这个估计。

假设价格为 $S1$ 的看涨期权具有理论价值 $C$、 Delta 值为 $\Delta$ 和 Gamma 值为 $\Gamma$。如果标的价格从 $S1$ 变为 $S2$,期权的新价值是多少?一种方法可能是简单地将价格变化 $S2 – S1$ 乘以 $\Delta$,并将其添加到原始值 $C$
$$C+\Delta \times (S2-S1)$$

但这假设 Delta 是恒定的,但事实并非如此。随着标的价格从 $S1$ 移动到 $S2$,期权的 Delta 也在变化。当标的价格达到 $S2$ 时,期权的新 Delta 将是
$$\Delta +(S2-S1)\times \Gamma$$

从逻辑上来讲,应该使用标的合约价格从 $S1$ 到 $S2$ 过程中的平均 Delta 值:
$$[\Delta+\Delta+(S2-S1)\times \Gamma]/2=\Delta+(S2-S1)\times\Gamma/2$$
这不是一个精确的解决方案,因为 Gamma 也会随着标的价格的变化而变化,但它会比使用
恒定 Delta 产生更好的估计值。