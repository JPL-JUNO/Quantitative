\chapter{利用期权套保}
\section{复杂套保策略}
如果套保者认为期权定价过高(即隐含波动率过高),那买入期权显然是对套保者不利的。如果套保者的唯一目的是对冲价格下跌风险而不考虑上涨后的潜在收益,他应该避免使用期权并且在期货或远期市场上对冲头寸。然而,如果他还希望得到标的合约价格上涨时所产生的潜在收益,他必须先自问要保留多大的看多头寸。举例来说,他可能会决定保留现有 50\% 的看多头寸。这就意味着,他必须买入 Delta 值之和为 -50 的看跌期权。他可以买入 1 份 Delta 值为 -50 的看跌期权,或者买入 Delta 值之和为 -50 的多份虚值期权。但是在高隐含波动率的市场中,交易者通常尽可能少买入期权并尽可能多地卖出期权(这类似于构建比例垂直价差)。因此,与买入 Delta 值之和为 -50 的多个虚值期权相比,理论上买入 1 份 Delta 值为 -50 的看跌期权成本更低。如果套保者希望消除更多的方向性风险,比如 75\%,在这种情况下他就需要买入 1 份 Delta 值为 -75 的看跌期权。
\begin{tcolorbox}
    在其他因素相同的条件下,隐含波动率较高的市场上,套保者应该尽可能少地买入期权,尽可能多地卖出期权;相反地,在隐含波动率较低的市场上,套保者应该尽可能多地买入期权,尽可能少地卖出期权。
\end{tcolorbox}