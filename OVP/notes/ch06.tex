\chapter{波动率}
\section{解释波动率数据}
我们将波动率分为两类--与标的合约关联的\textbf{已实现波动率}(realized volatility),以及与期权关联的\textbf{隐含波动率}(implied volatility)。
\subsection{已实现波动率}
已实现波动率是某段时期内某基础合约价格变化百分比的年化标准差\footnote{为了将价格变化转化为连续复合收益,波动率通常使用对数价格变化来计算——当前价格除以之前价格的自然对数。在大多数情况下,价格变化百分比和对数价格变化之间几乎没有实际差异。}。当我们计算已实现波动率时,我们必须指定测量价格变化的间隔以及计算中要使用的间隔数。

随着期权交易者开始意识到波动率作为定价模型输入的重要性,波动率预测模型应运而生,旨在更准确地预测未来的实际波动率。

\figures{fig6-8}{标准普尔 500 指数 250 天历史波动率。}
当我们计算给定时间段内的波动率时,我们仍然可以选择衡量基础合约价格变化的时间间隔。交易员可能会考虑间隔的选择是否会影响结果,即使间隔涵盖相同的时间段。例如,我们可以查看合约的 250 天波动率、52 周波动率和 12 个月波动率。所有波动率都涵盖大约一年,但一个是根据每日价格变化计算的,一个是根据每周价格变化计算的,一个是根据每月价格变化计算的。

对于大多数基础合约而言,所选的间隔似乎不会对结果产生很大影响。
\subsection{隐含波动率}
与实际波动率(根据标的合约的价格变化计算)不同,隐含波动率来自期权在市场上的价格。从某种意义上说,隐含波动率代表了市场对期权有效期内标的合约未来已实现波动率的共识。

当我们求解期权的隐含波动率时,我们假设理论值(期权价格)以及除波动率之外的所有其他输入都是已知的。实际上,我们正在反向运行理论定价模型来求解未知波动率。实际上,这说起来容易做起来难,因为大多数理论定价模型不能反向工作。但是,当所有其他输入都已知时,有许多相对简单的算法可以快速求解隐含波动率。

隐含波动率不仅取决于理论定价模型的输入,还取决于所使用的理论定价模型。对于某些期权,不同的模型可能会产生截然不同的隐含波动率。当输入不是同时发生的,也会出现问题。如果期权一段时间内没有交易,而市场条件发生了变化,使用过时的期权价格将导致误导或不准确的隐含波动率。

虽然\textbf{权利金}(premium)一词实际上指的是期权的价格,但由于隐含波动率是从期权价格中得出的,因此交易员有时会交替使用权利金和隐含波动率。如果当前的隐含波动率按历史标准衡量较高或相对于基础合约的近期历史波动率较高,交易员可能会说权利金水平较高;如果隐含波动率异常低,他可能会说权利金水平较低。

尽管期权交易者有时会参考各种对波动率的解释,但其中两种解释尤为重要,即未来实际波动率和隐含波动率。标的合约的未来实际波动率决定了该合约期权的价值。隐含波动率反映了期权的价格。价值和价格这两个数字是所有交易者(而不仅仅是期权交易者)所关注的。如果合约价值高而价格低,交易者就会想成为买家。如果合约价值低而价格高,交易者就会想成为卖家。对于期权交易者来说,这通常意味着将预期的未来实际波动率与隐含波动率进行比较。如果隐含波动率相对于预期的未来波动率较低,交易者会倾向于购买期权;如果隐含波动率较高,交易者会倾向于出售期权。当然,未来波动率是未知的,因此交易者会查看历史波动率,如果有的话,还会查看预测波动率,以帮助对未来做出明智的猜测。但归根结底,未来实际波动率决定了期权的价值。