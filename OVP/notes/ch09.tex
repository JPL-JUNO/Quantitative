\chapter{风险度量 2}
\section{Theta}
在 \autoref{fig9-8} 中,我们可以看到,期权的 Theta 值在平值期权时最大。随着期权变为实值或虚值时,其 Theta 值会下降。由于期权的 Theta 值是其时间价值的函数,并且由于实值期权非常深和虚值期权的时间价值非常小,因此此类期权的 Theta 值非常低是合乎逻辑的。

还要注意,当所有其他条件相同时,行权价格较高的平值期权的 Theta 值要大于行权价格较低的平值期权。要理解其中的原因,请考虑两个看涨期权,一个的行权价为 10,另一个的行权价为 1,000,这两个期权都是平值期权,并且两个看涨期权的到期时间和隐含波动率都相同。哪个期权的价值更高?显然,1,000 的看涨期权的价值更高,因为它代表着购买更有价值资产的权利。因为这两个期权都是平值期权,因此仅由时间价值组成,所以 1,000 的看涨期权的 Theta 值必须大于 10 的看涨期权的 Theta 值。

\figures{fig9-8}{随着标的价格变化的期权的 Theta 值}

\autoref{fig9-9} 显示了实值期权、平值期权和虚值期权的理论价值随时间的变化。在期权生命周期的早期,每种期权的衰减率(理论价值图的斜率)相似。但在期权生命周期的后期,随着到期日的临近,实值期权和虚值期权的衰减率会减慢,而平值期权的衰减率会加快,在到期时接近无穷大。这些特征适用于看涨期权和看跌期权,如 \autoref{fig9-10} 所示

\figures{fig9-9}{期权的理论价值随时间的变化。}

\figures{fig9-10}{随时间变化期权的 Theta 值。}

\autoref{fig9-11} 显示了波动率变化对 Theta 的影响。如果我们忽略利息,波动率为 0,则任何期权的 Theta 都将为 0。随着波动率的增加,我们增加了时间溢价,同时增加了 Theta。

\figures{fig9-11}{波动率变化时的期权 Theta 值。}

请注意,平值期权的图形本质上是一条直线,其中的 Theta 与波动率成正比。对于平值期权,波动率为 20\% 时的 Theta 正好是波动率为 10\% 时的 Theta 的两倍。对于较高的执行价格(虚值看涨期权和实值看跌期权)或较低的执行价格(实值看涨期权和虚值看跌期权),情况不一定如此。随着波动率的下降,Theta 趋于下降,但可能在波动率为 0 之前就变为 0。

\autoref{fig9-11} 是由于当前标的合约价格距离相等的较高行权价和较低行权价的期权构成。请注意,较高执行价格的 Theta 值大于较低执行价格的 Theta 值,并且随着波动性的增加,两者之间的差异也会增大。如果看涨期权和看跌期权都是同等的虚值期权,则在对数正态分布的假设下,虚值看涨期权(较高执行价格)的时间溢价将高于虚值看跌期权(较低执行价格)。如果标的合约的价格没有变动,则时间溢价较高的期权(较高执行价格)的衰减速度必然比时间溢价较低的期权(较低执行价格)更快。

如果我们知道期权的现值,有什么方法可以估计期权的 Theta 值吗?目前没有方便的方法来估计实值期权和虚值期权的 Theta 值,但对于平值期权,我们知道 Theta 值与波动率成正比(\autoref{fig9-11})。