\chapter{波动率价差}
\section{跨式期权}
\textbf{跨式期权}(straddle)包含1份看涨期权多头和1份看跌期权多头,或者1份看涨期权空头和1份看跌期权空头,且所有期权合约的行权价格和到期时间均相同。如果是买入看涨期权和看跌期权,那么称为买入跨式期权;如果是卖出看涨期权和看跌期权,那么称为卖出跨式期权。
\figures{fig11-1}{随着时间流逝或波动率降低的跨式期权多头}

从 \autoref{fig11-1} 中,可以看出当市场标的远离行权价格时跨式期权多头价值上升,而如果市场没有发生变化则随着时间流逝跨式期权价值将变小。

跨式期权通常是利用 1:1 比例(1 份看涨期权对应 1 份看跌期权)。通过这种方式,价差策略就近似为 Delta 中性的。跨式期权也可以通过利用实值期权或虚值期权来构建。

例如,对于 1 份交易价格为 100 的标的合约,我们可以买入 9 月行权价格为 90 的跨式期权。如果 9 月行权价格为 95 的看涨期权是 Delta 为 75 的实值期权,同时 9 月行权价格为 95 的看跌期权是 Delta 为 -25 的虚值期权,那么头寸的总 Delta 为 $75-25=50$,这种结果可以视为\textbf{牛市跨式期权}(bull straddle)。如果想持有 Delta 中性的跨式期权,需要调整合约的数量,每买入 1 份看涨期权的同时买入 3 份看跌期权。更确切地说,这是一个\textbf{比例跨式期权},因为多头合约(看涨期权)与空头合约(看跌期权)的数量不相等。
\section{宽跨式期权}
与跨式期权一样,\textbf{宽跨式期权}(strangle)包含 1 份看涨期权多头和 1 份看跌期权多头,或者 1 份看涨期权空头和 1 份看跌期权空头,且所有期权的到期日均相同。但在宽跨式期权中,两份期权的行权价格却不相同。如果是两份期权都是买入的,称为宽跨式期权多头;如果两份期权都是卖出的,称为宽跨式期权空头。

如果宽跨式期权只用到期月份和行权价格来标示,那么对于所使用的具体期权合约就会产生一定的混淆。1 份 3 月行权价格为 90/110 的宽跨式期权可以是由 1 份 3 月行权价格为 90 的看跌期权和 1 份 3 月行权价格为 110 的看涨期权组成,也有可能是由 1 份 3 月行权价格为 90 的看涨期权和 1 份行权价格为 110 的看跌期权组成。为了避免这种混淆,\textbf{通常假设宽跨式期权由虚值期权组成}。

\figures{fig11-4}{随着时间流逝或波动率降低的宽跨式期权多头}
\section{蝶式期权}
\textbf{蝶式期权}(butterfly)就是一个由相同类型并具有相同到期时间,且合约间行权价格间距相等的 3 份期权组成的三腿价差。蝶式期权多头中,买入外部行权价格的期权合约,卖出内部行权价格的期权合约;蝶式期权空头中,操作正好完全相反。蝶式期权的构成比例是固定不变的,总是 $1\times 2\times 1$,也即由内部行权价格期权合约各 2 份、外部行权价格期权合约各 1 份所组成。如果构成比例不是 $1\times 2\times 1$,那么这个价差就不是蝶式期权。
\figures{fig11-6}{随着时间流逝或波动率降低的蝶式期权多头}

到期时,蝶式期权的价值总会在零和行权价格差之间。如果标的合约价格低于最低执行价或高于最高执行价,价差的价值为 0;如果标的合约价格正好等于中间的行权价格时,价差的价值达到最大。

当内部行权价格约等于平值时,蝶式期权接近于 Delta 中性。在这种情况下,蝶式期权多头的表现类似于跨式期权空头,而蝶式期权空头类似于跨式期权多头。不论是持有蝶式期权多头还是跨式期权空头,交易者都希望标的市场保持不变(-Gamma, +Theta),隐含波动率下降(-Vega)。但是,这里有一个重要区别:跨式期权的潜在收益或风险都是无限的,而蝶式期权则是有限的。蝶式期权的价值不会低于 0,也不会超过行权价格之差。

如果我们假设所有期权都是欧式期权,没有提前行权的可能性,那么具有相同行权价格和到期日的看涨蝶式期权和看跌蝶式期权将具有相同的到其表现,因此也具有相同的特征。
\section{鹰式期权}
正如蝶式期权可以被认为是具有有限风险或收益的跨式期权一样,\textbf{鹰式期权}(condor)也可以被认为是具有有限风险或收益的宽跨式期权。鹰式期权由 4 份期权组成,2 个内部行权价格的期权和 2 个外部行权价格的期权。鹰式期权的构成比例总是 $1\times 1\times 1\times 1$。尽管两个内部行权价格的差额可以变化,但是 2 个最低行权价格的差额一定要与 2 个最高行权价格的差额相等。与蝶式期权一样,鹰式期权中所有期权的到期时间和类型都相同。买入 2 个外部行权价格的期权、卖出两个内部行权价格的期权就构成了鹰式期权多头,鹰式期权空头则与之相反。

鹰式期权的到期价值不会低于 0,也不会高于 2 个高行权价格之差或 2 个低行权价格之差。买入鹰式期权的交易者需要付出零到差额之间的一定金额,并期待标的合约交价格在 2 个内部行权价格之间,这时鹰式期权将具有最大的价值。卖出鹰式期权的交易者将会收入一定金额,并期待标的合约价格在两端的行权价格之外,这时鹰式期权价值为零。

当标的合约价格位于 2 个内部行权价格中间时,鹰式期权近似于 Delta 中性。当所有期权为欧式期权时,看涨鹰式期权和看跌鹰式期权的价值和特征将是相同的。

\figures{fig11-7}{随着时间流逝或波动率降低的鹰式期权多头}

跨式价差、宽跨式价差、蝶式价差和鹰式价差都有对称的损益图(P\&L)。当构建 Delta 中性价差策略时——通常情况下,这些策略对于标的市场的变动方向没有偏好。宽式期权和跨式期权多头、蝶式期权和鹰式期权空头偏好标的市场发生变动以及隐含波动率增加(+Gamma,-Theta,+Vega)。跨式期权和宽跨式期权空头、蝶式期权和鹰式期权多头偏好标的市场不发生变动以及隐含波动率降低(-Gamma,+Theta,-Vega)。\autoref{fig11-9} 总结了这些特征。

\figures{fig11-9}{对称策略}
\section{比例价差}
在波动率价差中,交易者不能完全不关心标的市场的变动方向。交易者可能会认为向一个方向变动的可能性要大于向另一个方向变动的可能性。鉴于这个原因,交易者可能希望构建一个当标的向一个方向而不是另一个方向变动时能够最大化收益或最小化损失的价差策略。为了实现这个目标,交易者可以构建一个\textbf{比例价差}——买入并卖出不同数量的期权,所有期权都是同一类型的,且具有相同的到期时间。和其他的波动率价差一样,比例价差也是典型的 Delta 中性策略。

买入看涨期权多于卖出的价差称为\textbf{看涨比例价差}(call ratio spread)。\textbf{看跌比例价差}(put ratio spread)指的是买入看跌期权多于卖出看跌期权的价差,这种价差也希望标的合约价格发生变动。

买入期权多于卖出期权的比例价差有时被称为\textbf{反套利}(backspread)。不论这个价差是由看涨期权还是看跌期权构成,这种类型的价差策略总是希望标的合约价格发生变动(+Gamma,-Theta)和/或隐含波动率上升(+Vega)。

对于任一个最终获利的价差策略来说,构建策略时一定需要资金支出,这是这类价差的典型特征。的确,在传统理论定价模型的假设下,一个买入期权多于卖出期权的 Delta 中性比例价差总会导致现金收入。

\figures{fig11-10}{随着时间流逝或波动率降低的看涨比例价差(买入多于卖出)}

使用看涨期权构建价差的情况下,如果标的合约到期价格低于较低的行权价格,那么头寸将变得毫无价值。在使用看跌期权构建价差的情况下,如果标的合约到期高于较高行权价格,那么头寸将变得毫无价值。头寸价值不能低于 0 的事实限制了卖出看涨期权多于买入看涨期权时价格下跌的风险,以及当卖出看跌期权多于买入看跌时价格上升的风险。
\section{圣诞树形期权}
通过模仿宽跨式期权来构建在某个方向上限定风险和收益的策略,这样的价差被称为\textbf{圣诞树形期权}(Christmas tress)或\textbf{梯式期权}(ladders)

看涨圣诞树形期权包含了在较低行权价上买入(卖出)1 份看涨期权并在 2 个较高的行权价格上分别卖出(买入)1 份看涨期权。看跌圣诞树形期权包含了在较低行权价上买入(卖出)1 份看跌期权并在 2 个较高的行权价格上分别卖出(买入)1 份看跌期权。所有期权必须类型相同且到期时间相同,并通常选择时整个头寸为 Delta 中性的行权价格。当买入一份期权并卖出两份期权(圣诞树形期权多头)时,该头寸表现为空头宽跨式期权,但在一个方向上的风险有限。当卖出一份期权并买入两份期权(圣诞树形期权空头)时,该头寸表现为多头宽跨式期权,但在一个方向上的盈利潜力有限。

买入期权多于卖出的价差将倾向于标的市场波动和/或隐含波动率增加(+gamma、-theta、+vega)。卖出期权多于买入期权的价差将倾向于标的市场不波动和/或隐含波动率下降(-gamma、+theta、-vega)。

\figures{fig11-14}{随着时间流逝或波动率降低的看涨圣诞树形期权多头}

\autoref{fig11-18} 总结了非对称价差策略的特征。

\figures{fig11-18}{非对称价差策略}
\section{日历价差}
日历价差,有时也称为时间价差(time spreads)或水平价差(horizontal spreads),是由不同到期月份的期权头寸构成的。

最常见的日历价差类型包括两种行权价格和类型(看涨期权或看跌期权)都相同的期权的相反头寸当买入长期期权并卖出短期期权时,交易者是日历价差多头;当买入短期期权并卖出长期期权时,交易者是日历价差空头。因为长期期权通常比短期期权更贵,所以这与把需要指出资金的策略称为多头头寸、把导致资金收入的价差称为空头头寸的惯例是一致的。

\figures{fig11-22}{随着时间流逝,日历价差多头}

尽管日历价差通常是按照 1:1 的比例构建的(每卖出一份合约就买入一份合约),但交易者可能会将日历价差比率化,以反映对市场牛、熊或中性的市场的观点。为了便于讨论,我们将重点讨论 delta 大致中性的一对一日历价差(每卖出一份短期期权对应一份长期期权)。由于平值期权的 delta 值接近 50,因此最常见的日历价差由多头和空头平值期权组成。

日历价差的价值不仅取决于标的市场的变动,还取决于市场对未来市场变动的预期,这反映在隐含波动率中。

如果我们假设构成日历价差的期权近似于平值,则日历价差具有两个重要特征:
\begin{enumerate}
    \item 如果随着时间流逝,标的合约价格不发生变动,那么日历价差的价值将会增加。
    \item 如果隐含波动率上升,那么日历价差的价值将会增加;如果隐含波动率下降,日历价差的价值将会下降。
\end{enumerate}

日历价差的价值取决于,当短期期权价值下降时,长期期权尽可能保留尽可能多的时间价值。如果两种期权都保持平值,情况就会如此,因为平值期权总是具有最大的时间价值。当期权进入实值或虚值时,其时间价值就会消失。长期期权的时间价值总是大于短期期权。但是,如果标的合约的变动足够大,期权进入实值或虚值的幅度非常大,即使是长期期权最终也会损失几乎所有的时间价值。这将导致日历价差暴跌。

波动率的变化对长期期权的影响将大于对同等短期期权的影响。换句话说,长期期权的 Vega 值大于短期期权。对波动性变化的敏感程度的不同,导致当波动率上升时日历价差价值变宽,波动率下降时日历价差价值变窄。

持有日历价差多头头寸的交易者希望市场上出现两种明显矛盾的情况。首先,他希望标的合约保持不动,以便利用短期期权较长的时间衰减优势。其次,他希望每个人都认为市场将会变动,从而隐含波动率将会上升,导致长期期权的价格上涨速度快于短期期权。这种情况会发生吗?市场可能在所有人都认为它会发生变化的情况下保持不变吗?事实上,这种情况经常发生,因为对标的合约没有直接影响的事件,未来有可能会对标的合约产生影响。

隐含波动率的影响是时间价差与我们讨论过的其他价差类型的区别所在。多头跨式期权、多头宽跨式期权和空头蝶式期权都希望标的合约的波动率以及隐含波动率上升(+gamma,+vega)。空头跨式期权、空头宽跨式期权和多头蝶式期权都希望标的合约的波动率以及隐含波动率下降(-gamma,-vega)。但日历价差中,标的波动率和隐含波动率具有相反的影响。平静的市场或隐含波动率的上升将有利于多头日历价差(-gamma,+vega),而标的市场的大幅波动或隐含波动率的下降将有利于日历价差 (+gamma, –vega)。这种相反的效果正是日历价差具有独特特征的原因。

\figures{fig11-25}{随着波动率下降,日历价差空头}

考虑一个有四个期货月份的期货市场:三月、六月、九月和十二月。如果有连续月份,四月/六月日历价差将具有相同的标的合约,即六月期货。但三月/六月日历价差将有一个三月期权的标的合约,即三月期货,以及另一个六月期权的标的合约,即六月期货。尽管人们可能预计三月期货和六月期货会同时波动,但并不能保证它们会同时波动。特别是在商品市场,短期供需因素可能导致同一商品的期货合约朝不同的方向波动。除了波动性因素外,购买六月/三月看涨日历价差的交易者还必须考虑三月期货相对于六月期货上涨的可能性。

为了抵消期货合约与日历价差头寸相反的风险,商品期货市场中交易者通常会用期货市场的反向头寸来抵消日历价差。
\section{时间蝶式期权}
时间蝶式期权(有时简称为 time fly)由相同行权价格但三种不同到期日的期权组成。所有期权必须是同一类型(全部看涨期权或全部看跌期权),且期权不同到期日之间的间隔要近似相等。外部到期月份期权通常称为\textbf{翅膀}(wings),内部到期月份期权则称为\textbf{身体}(body)。

时间蝶式交易是由同时买入或卖出长期日历价差,并在短期日历价差中采取
相反仓位,其中每个价差都有一个共同的到期月份。

\figures{fig11-26}{随着时间流逝,时间蝶式期权多头}
\figures{fig11-27}{随着波动率的降低,时间蝶式期权多头}

\autoref{fig11-26} 和 \autoref{fig11-27} 显示了时间蝶式期权的价值随时间推移和波动率下降的变化。当标的合约远离执行价格时,价差的价值将暴跌,这意味着价差具有负的 gamma。因此,价差也必须具有正的 theta。最后,随着波动率的下降,价差的价值会下降,这意味着价差具有正的 vega。总之,时间蝶式期权多头具有与日历价差类似的特征。
\section{利率和股利变化的影响}