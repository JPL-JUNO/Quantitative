\chapter{确定远期和期货价格}
远期合约比期货合约更容易分析,这是因为对远期合约不需要每日结算(而只是在到期日一次性结算),因此我们首先从远期价格与即期价格之间的关系开始。幸运的是,当同一资产上的远期合约和期货合约有相同期限时,可以证明远期价格和期货价格通常非常接近。
\section{投资资产与消费资产}
在考虑远期合约与期货合约时,区分\textbf{投资资产}(investment asset)和\textbf{消费资产}(consumption asset)是很重要的。投资资产是指至少有一些交易员仅仅是为了投资目的而持有的资产。股票与证券显然是投资资产,黄金和白银也属于投资资产。注意投资资产并不是只能用来投资(例如,白银也有一些工业用途),但是,投资资产的一个条件是有些交易员持有它的唯一目的就是投资。而持有消费资产的目的主要是消费而不是投资。消费资产的例子包括铜、原油和猪肉。
\section{卖空交易}
在卖空交易中,持有卖空头寸的投资者必须向经纪人支付被卖空资产的所有收入(只是持有期间的收入,一开始卖空的收入不转入),像股票的股息和债券的券息等(这些收入是在一般情况下被卖空资产应得的收入),经纪人会将这些收入转入证券借出方的账户。

在卖空交易中,卖空方需要在经纪人那里开一个\textbf{保证金账户}(margin account),并在这个保证金账户中存入一定数量的现金或其他有价证券,以保证在股票价格上涨时投资者不会违约。这与第2章中讲过的期货保证金是类似的。投资者投入的保证金并不代表投资费用,这是因为经纪人会按投资者账户上的金额数量支付利息,如果支付的利率对投资者来讲不可接受,投资者可以在保证金账户中存入有价证券(像国债)来满足要求。卖出这些资产时的收入属于投资者,而且一般会作为初始保证金的一部分。
\section{假设与符号}
我们假定对于某些市场参与者而言,以下假设全部成立:
\begin{enumerate}
    \item 市场参与者进行交易时没有手续费。
    \item 市场参与者对所有交易净利润都使用同一税率。
    \item 市场参与者能够以同样的无风险利率借入和借出资金。
    \item 当套利机会出现时,市场参与者会马上利用套利机会。
\end{enumerate}

注意,我们并不要求这些条件对于所有市场参与者均成立。我们只要求这些条件对像大型投资银行这样的关键参与者成立或大致成立即可。正是因为这些关键参与者的行为以及他们寻找套利机会的积极心态决定了远期价格与即期价格之间的关系。

将采用以下符号:

$T$:远期或期货合约的期限(以年计);

$S_0$:远期或期货合约标的资产的当前价格;

$F_0$:远期或期货的当前价格;

$r$:按连续复利的无风险零息利率,这一利率的期限对应于合约的交割日(即 $T$ 年后)。无风险利率 $r$ 是指在无信用风险的前提下(即资金一定全被偿还的情况下),借入和借出资金的利率。
\section{投资资产的远期价格}
最容易定价的远期合约是既不提供任何中间收入,又不需要任何存储费用的投资资产上的合约。无股息股票和零息债券都属于这一类资产。