\chapter{确定远期和期货价格}
远期合约比期货合约更容易分析,这是因为对远期合约不需要每日结算(而只是在到期日一次性结算),因此我们首先从远期价格与即期价格之间的关系开始。幸运的是,当同一资产上的远期合约和期货合约有相同期限时,可以证明远期价格和期货价格通常非常接近。
\section{投资资产与消费资产}
在考虑远期合约与期货合约时,区分\textbf{投资资产}(investment asset)和\textbf{消费资产}(consumption asset)是很重要的。投资资产是指至少有一些交易员仅仅是为了投资目的而持有的资产。股票与证券显然是投资资产,黄金和白银也属于投资资产。注意投资资产并不是只能用来投资(例如,白银也有一些工业用途),但是,投资资产的一个条件是有些交易员持有它的唯一目的就是投资。而持有消费资产的目的主要是消费而不是投资。消费资产的例子包括铜、原油和猪肉。
\section{卖空交易}
在卖空交易中,持有卖空头寸的投资者必须向经纪人支付被卖空资产的所有收入(只是持有期间的收入,一开始卖空的收入不转入),像股票的股息和债券的券息等(这些收入是在一般情况下被卖空资产应得的收入),经纪人会将这些收入转入证券借出方的账户。

在卖空交易中,卖空方需要在经纪人那里开一个\textbf{保证金账户}(margin account),并在这个保证金账户中存入一定数量的现金或其他有价证券,以保证在股票价格上涨时投资者不会违约。这与第2章中讲过的期货保证金是类似的。投资者投入的保证金并不代表投资费用,这是因为经纪人会按投资者账户上的金额数量支付利息,如果支付的利率对投资者来讲不可接受,投资者可以在保证金账户中存入有价证券(像国债)来满足要求。卖出这些资产时的收入属于投资者,而且一般会作为初始保证金的一部分。
\section{假设与符号}
我们假定对于某些市场参与者而言,以下假设全部成立:
\begin{enumerate}
    \item 市场参与者进行交易时没有手续费。
    \item 市场参与者对所有交易净利润都使用同一税率。
    \item 市场参与者能够以同样的无风险利率借入和借出资金。
    \item 当套利机会出现时,市场参与者会马上利用套利机会。
\end{enumerate}

注意,我们并不要求这些条件对于所有市场参与者均成立。我们只要求这些条件对像大型投资银行这样的关键参与者成立或大致成立即可。正是因为这些关键参与者的行为以及他们寻找套利机会的积极心态决定了远期价格与即期价格之间的关系。

将采用以下符号:

$T$:远期或期货合约的期限(以年计);

$S_0$:远期或期货合约标的资产的当前价格;

$F_0$:远期或期货的当前价格;

$r$:按连续复利的无风险零息利率,这一利率的期限对应于合约的交割日(即 $T$ 年后)。无风险利率 $r$ 是指在无信用风险的前提下(即资金一定全被偿还的情况下),借入和借出资金的利率。
\section{投资资产的远期价格}
最容易定价的远期合约是既不提供任何中间收入,又不需要任何存储费用的投资资产上的合约。无股息股票和零息债券都属于这一类资产。

我们考虑一个投资资产上的远期价格,资产的当前价格为 $S_0$,并且不提供任何中间收入。采用我们前面的符号: $T$ 为期限,$r$ 为无风险利率,$F_0$ 为远期价格。$F_0$ 与 $S_0$ 的关系式为
\begin{equation}\label{eq5-1}
    F_0=S_0e^{rT}
\end{equation}
如果 $F_0 > S_0e^{rT}$,套利者可以通过买入资产并承约远期合约的空头来进行套利;如果 $F_0 < S_0e^{rT}$,套利者可以通过卖空资产并承约远期合约的多头来进行套利。

远期价格高于即期价格的原因是即期购买资产时,对远期合约期限内购买资产会产生融资费用。
\subsection{不允许卖空时会怎么样}
并不是所有的投资资产都可以用于卖空交易,而且有时在卖空时需要对所借的资产付出一定的费用。但这些情形对以上结果并没有影响。为了推导 \autoref{eq5-1},我们并不需要卖空资产,所需要的假设是只要有些投资者拥有这种资产的唯一目的是投资(由定义我们知道,对于投资资产这一假设永远是正确的)。如果远期价格太低,投资者会卖出资产并承约远期合约的多头。

假定某标的资产没有贮存费用与中间收入。如果 $F_0>S_0e^{rT}$,那么投资者可以采取以下交易策略:
\begin{itemize}
    \item 按利率 $r$ 借入 $S_0$ 美元,期限为 $T$;
    \item 买入一单位资产;
    \item 承约出售一单位资产的远期空头合约。
\end{itemize}
在时刻 $T$,卖出资产的价格为 $F_0$,这时偿还贷款所需资金为 $S_0e^{rT}$,投资者的盈利为 $F_0-S_0e^{rT}$。

接下来假定 $F_0<S_0e^{rT}$,这时拥有资产的投资者可以采取以下交易策略:
\begin{itemize}
    \item 以 $S_0$ 的价格卖出资产;
    \item 将所得资金以收益率 $r$ 进行投资,期限为 $T$;
    \item 承约购买一单位资产的远期多头合约。
\end{itemize}
在时刻 $T$,现金投资会涨至 $S_0e^{rT}$。投资者以 $F_0$ 价格买入资产,这个
投资者同一直保存资产的投资者相比,所得盈利为 $S_0e^{rT}-F_0$。

\section{已知收入}
我们考虑当资产给持有者提供完全可以预测的收入时的远期价格。这样的例子包括提供已知股息的股票以及带券息债券。

考虑一个买入当前价格为 900 美元的带息债券远期合约的多头。假定远期合约的期限为 9 个月,我们假定在 4 个月后将有 40 美元的券息付款,并且假定 4 个月期和 9 个月期的利率(连续复利)分别是 3\% 和 4\%。

首先假定远期价格比较高,为 910 美元。一个套利者可以借入 900 美元来买入债券,并且承约远期合约的空头。券息的现值为 $40e^{-0.03\times 4/12}=39.60$ 美元。在 $900$ 美元的价格中,其中有 $39.60$ 美元所对应的利率为每年 3\%,期限为 4 个月,而这笔资金在 4 个月时可用券息来偿还,其他部分资金(即 $860.40$ 美元)所对应的利率为 9 个月期的利率(即 4\%)。这笔资金在 9 个月后变为 $860.4e^{0.04\times 0.75}=886.60$。按照远期合约条款,套利者在远期合约中卖出证券可以收入 910 美元,因此盈利为
$$910-886.60=23.40 \$$$

    接下来假定远期价格相对较低,为 870 美元。一个套利者可以卖空债券并同时承约远期合约的多头,以 870 美元的价格购买债券。在卖空交易所得的 900 美元资金中,将其中 39.60 美元以 3\% 的利率投资 4 个月,在 4 个月时这笔资金足够可以偿还债券的券息。剩余的 860.40 美元以 4\% 投资 9 个月,在到期时这笔资金变为 886.60 美元。根据远期合约,投资者能够以 870 美元买入债券,然后将卖空交易进行平仓,投资者的收入为
    $$886.60-870=16.6 \$$$

我们可以将以上例子推广:当投资资产在远期合约期限内提供的收入贴现值为 $I$ 时,我们有以下关系式
\begin{equation}\label{eq5-2}
    F_0=(S_0-I)e^{rT}
\end{equation}
如果 $F_0>(S_0-I)e^{rT}$,套利者可以通过买入资产并且承约远期合约的空头来取得盈利;如果 $F_0<(S_0-I)e^{rT}$,套利者可以通过卖空资产并且承约远期合约的多头来获得盈利。如果不能卖空交易,拥有资产的投资者可以卖出资产,并同时承约远期合约的多头来获得盈利。
\section{收益率为已知的情形}
我们现在考虑远期合约标的资产支付已知的收益率(而非现金收入的情形),这意味着在中间收入的数量是当时资产价格的百分比。

定义 $q$ 为资产在远期期限内的平均年收益率,如果将所得收益再用于购买资产,所持资产的数量将会以 $q$ 的速度增长:在时刻 0 的 1 单位资产将在时间 $T$ 增长到 $e^{qT}$。考虑以下策略:
\begin{itemize}
    \item 在时间 0 借入 $S_0$ 并用来购买一单位资产
    \item 承约远期空头合约,在时间 $T$ 按 $F_0$ 的价格卖出 $e^qT$ 单位资产
    \item 在时间 $T$ 对远期合约平仓,卖出 $e^{qT}$ 单位资产。
\end{itemize}
该策略的盈利应当是 0,因此 $S_0e^{rT}=e^{qT}F_0$ 或者
\begin{equation}\label{eq5-3}
    F_0=S_0e^{(r-q)T}
\end{equation}
\section{远期合约定价}
在刚刚承约远期合约时,其价值为 0,但在承约合约之后,远期合约价值可能为正也可能为负。对银行或其他金融机构来讲,每天计算这些合约的价值是非常重要的(称为对合约按市场定价或逐日盯市,marking to the market)。采用前面引入的符号,假设 $K$ 是以前成交合约的交割价格,合约的交割日期是在从今日起 $T$ 年之后,$r$ 是期限为 $T$ 年的无风险利率,变量 $F_0$ 表示目前的远期价格,即假如在今天成交的话,合约的交割价格,我们还定义 $f$ 为远期合约在今天的价值。

清楚地理解变量 $F_0$、$K$ 和 $f$ 的含义非常重要。如果今天正好是合约的最初成交日,那么交割价格 $K$ 等于远期价格 $F_0$,而且合约的价值 $f$ 是 0。随着时间的推移,$K$ 保持不变(因为已经被合约确定),但远期价格 $F_0$ 却会变动,而且远期合约的价值 $f$ 可以变成或正或负。

对于远期合约的多头方(既可以是在投资资产上也可以是在消费资产上),合约的价值是
\begin{equation}\label{eq5-4}
    f=(F_0-K)e^{-rT}
\end{equation}

\autoref{eq5-4} 说明对于一个资产上的远期合约多头定价时,我们可以假定资产在远期合约到期时的价格等于远期价格 $F_0$。为了说明这一点,注意在做出这个假设之后,远期合约在 $T$ 时刻的收益为 $F_0-K$,贴现值 $(F_0-K)e^{-rT}$,这与 \autoref{eq5-4} 中的 $f$ 一致。

将 \autoref{eq5-4} 与 \autoref{eq5-1} 结合,我们可以得出在没有中间收入的资产上远期合约多头的价值为
\begin{equation}\label{eq5-5}
    f=S_0-Ke^{-rT}
\end{equation}
类似地,将 \autoref{eq5-4} 与 \autoref{eq5-2} 结合,我们可以得出在提供贴现值为 $I$ 的已知收入的资产上远期合约多头价值为
\begin{equation}\label{eq5-6}
    f=S_0-I-Ke^{-rT}
\end{equation}
最后,将 \autoref{eq5-4} 与 \autoref{eq5-3} 并用,我们可以得出提供收益率为 $q$ 的资产上的远期合约多头价值为
\begin{equation}\label{eq5-7}
    f=S_0e^{-qT}-Ke^{-rT}
\end{equation}
\section{股指期货价格}
通常可以将股指看成支付股息的投资资产,投资资产为构成股指的股票组合,股息等于构成资产所支付的股息。通常假定股息为已知收益率(而不是现金收入)。如果 $q$ 为股息收益率(按连续复利形式),\autoref{eq5-3} 给出的期货价格 $F_0$。
\section{货币上的远期和期货合约}
我们现在从美国投资者的角度来考虑外汇上的远期和期货合约,这里的标的资产为 1 单位的外币。定义变量 $S_0$ 为 1 单位外币的美元价格,$F_0$ 为 1 单位外币的美元远期或期货价格。外币具有以下性质:外币持有人可以收取货币发行国的无风险利率。例如,外币持有人可将货币投资于以外币计价的债券。我们定义 $r_f$ 为对应于期限 $T$ 的外币无风险利率,变量 $r$ 为对应于同样期限的美元无风险利率。

$F_0$ 与 $S_0$ 之间有以下关系式
\begin{equation}\label{eq5-9}
    F_0=S_0e^{(r-r_f)T}
\end{equation}
这就是在国际金融领域里著名的利率平价关系式,\autoref{fig5-1} 显示了该公式成立的原因。

\figures{fig5-1}{两种将外币在时刻 $T$ 转换成美元的方法,$S_0$ 为即期汇率,$F_0$ 为远期汇率,$r$ 和 $r_f$分别为美元和外币无风险利率}
\subsection*{将外汇作为提供已知收益率的资产}
注意,如果以 $r_f$ 代替 $q$,\autoref{eq5-9} 与 \autoref{eq5-3} 相同。这并非偶然:外币可以看成提供已知收益率的资产,这里的收益率为外汇的无风险利率。为了理解这一点,注意外汇提供的利息与外汇的价值有关。假定英镑利率为 5\%。对于美元投资者而言,这一外币提供以英镑计量的收入为 5\%。换句话讲,英镑是提供 5\% 收益率的资产。
\section{商品期货}
\subsection{收入和贮存费用}
由于黄金生产商的对冲策略,部分投资银行需要借入黄金。类似于像中央银行这样的黄金拥有者在借出黄金时会索取所谓的\textbf{黄金租借率}(gold lease rate)形式的利息,对于白银也是一样。因此,黄金和白银会给其拥有者带来收入。与其他商品一样,它们也需要支付贮存费用。

在没有贮存费用和收入时,\autoref{eq5-1} 给出了投资资产的远期价格。贮存费用可视为负收入。假定U为期货期限之间所有去掉收入后贮存费用的贴现值。由 \autoref{eq5-2} 得出
\begin{equation}\label{eq5-11}
    F_0=(S_0+U)e^{rT}
\end{equation}

如果在每个时刻的贮存费用(除去收入)都与商品价格成比例,这时的费用可看成负收益率。由 \autoref{eq5-3} 得出
\begin{equation}\label{eq5-12}
    F_0=S_0e^{(r+u)T}
\end{equation}
其中 $u$ 为除去资产所赚取的所有收益率后贮存费用占即期价格的比例。
\subsection{消费商品}
假设
\begin{equation}\label{eq5-14}
    F_0<(S_0+U)e^{rT}
\end{equation}
当商品为投资资产时,由于许多投资者拥有商品的目的是投资,所以当他们发现以上关系后,会采用以下交易策略来取得盈利:
\begin{enumerate}
    \item 卖出商品,节省贮存费用,并将所得资金按无风险利率投资;
    \item 承约远期合约的多头。
\end{enumerate}

当持有商品的主要目的不是投资时,以上的讨论不再适用。当个人和公司持有商品的目的是其消费价值而不是其投资价值时,他们不愿意出售商品并买入期货合约,因为期货合约并不能用于加工或其他形式的消费。因此,我们没有任何理由说 \autoref{eq5-14} 不能成立,我们所能肯定的只是以下关系式
\begin{equation}\label{eq5-15}
    F_0\leq (S_0+U)e^{rT}
\end{equation}
如果将贮存费用表示成即期价格的比例 $u$,等价表达式为
\begin{equation}\label{eq5-16}
    F_0\leq S_0e^{(r+u)T}
\end{equation}
\subsection{便利收益率}
因为商品持有者可能会认为持有商品比持有期货能提供更多的便利,因此 \autoref{eq5-15} 和 \autoref{eq5-16} 不一定成立。例如,某原油加工厂不太可能将持有原油期货合约与持有原油库存同等看待。库存原油可以用于原油加工,而持有的期货合约并不能用于这个目的。一般来讲,持有实物资产可以确保工厂的正常运作,并且从商品的暂时局部短缺中盈利,而持有一个期货合约却做不到这一点。由于持有商品而带来的好处有时称为商品所具有的\textbf{便利收益率}(convenience yield)。如果贮存成本为现金形式而且已知现值为 $U$,商品的便利收益率 $y$ 可由以下关系式来定义
$$F_0e^{yT}=(S_0+U)e^{rT}$$
如果单位商品的贮存成本为即期价格的比例 $u$,那么便利收益率 $y$ 可由以下关系式定义
$$F_0e^{yT}=S_0e^{(r+u)T}$$
即
\begin{equation}\label{eq5-17}
    F_0=S_0e^{(r+u-y)T}
\end{equation}

便利收益率反映了市场对将来能够购买商品的可能性的期望,商品短缺的可能性越大,便利收益率就越高。如果商品的用户拥有大量库存,在不久的将来出现商品短缺的可能性便会很小,这时便利收益率也会比较小。较低的库存一般会导致较高的便利收益率。
\section{持有成本}
期货价格与即期价格之间的关系式可由\textbf{持有成本}(cost of carry)来描述。持有成本包括贮存成本加上资产的融资利息,再减去资产所提供的收益。对于无股息的股票而言,持有成本为 $r$,这是因为股票既没有贮存费用也没有中间收入;对于股指而言,持有成本为 $r-q$,因为股指提供收益率为 $q$ 的中间收入;对于货币而言,持有成本为 $r-r_f$;对于提供中间收益率 $q$ 和贮存成本率为 $u$ 的资产而言,持有成本为 $r-q+u$,等等。
定义持有成本为$c$,对于投资资产,期货价格满足
\begin{equation}\label{eq5-18}
    F_0=S_0e^{cT}
\end{equation}
对于消费资产,期货价格满足
\begin{equation}\label{eq5-19}
    F_0=S_0e^{(c-y)T}
\end{equation}
其中 $y$ 为便利收益率。
\section{交割选择}
如果期货价格是期货期限的递增函数,由 \autoref{eq5-19} 得出 $c>y$,持有资产所带来的利益(包括去掉贮存费后的便利收益率)小于无风险利率。因此空头方越早交割资产越有利,此时收到资金后所得利息超出了持有资产所带来的好处。作为一般规则,在这种情况下计算期货价格时应当假设交割时间为交割期的开始。如果期货价格随期限的增加而减($c<y$),这时采取的标准会相反:空头方交割越晚越有利,因此在这种情况下对期货定价时应当使用这一假设。
\section{期货价格与预期未来即期价格}
我们将市场对于在将来某时刻资产即期价格的一般观点称为资产在这一时刻的\textbf{预期即期价格}(expected spot price)。
\subsection{凯恩斯和希克斯}
经济学家约翰·梅纳德·凯恩斯(John Maynard Keynes)和约翰·希克斯(John Hicks)提出,\textbf{如果对冲者倾向于持有空头而投机者倾向于持有多头,那么资产期货价格会低于预期未来即期价格}。这是因为投机者因承担风险而会索取补偿,他们只有在预期产生盈利时才会进行交易。平均来讲对冲者会有损失,因为期货可以减小风险,所以对冲者更容易接受亏损的事实。凯恩斯和希克斯指出,如果对冲者倾向于持有多头而投机者倾向于空头时,类似的原因可以说明期货价格会高于预期即期价格。
\subsection{现货溢价和期货溢价}
当期货价格低于预期未来即期价格时,这一情形叫\textbf{现货溢价}(normal backwardation);当期货价格高于预期未来即期价格时,这一情形叫\textbf{期货溢价}(contango)。但是,需要注意这些名词有时是指期货价格低于或高于当前即期价格(而不是预期未来即期价格)。