\chapter{金融数据及其特征}
\section{资产收益率}
大多数金融研究都是针对资产收益率,而不是资产价格. Campbell 等(1997)给出了使用资产收益率的两个主要原因. 首先,对于一个普通的投资者来说,资产收益率代表一个完全的、尺度自由的投资机会的总结和概括. 其次,资产收益率序列比价格序列更容易处理,前者有更好的统计特性. 然而,资产收益率有多种不同的定义.
\subsection*{单期简单收益率}
假设投资者在一个周期内拥有某种资产,从第 $t-1$ 天到第 $t$ 天,其简单毛收益率为:
$$1+R_t=\frac{P_t}{P_{t-1}} ~\text{or} ~P_t=P_{t-1}(1+R_t)$$

相对应的单期简单净收益率 (simple net return)或简单收益率(simple return)为:
\begin{equation}
    R_t=\frac{P_t}{P_{t-1}}-1=\frac{P_t-P_{t-1}}{P_{t-1}}
\end{equation}

\subsection*{多期简单收益率}
假设从第 $t-k$ 天到第 $t$ 天,这 $k$ 个周期内持有某种资产,则 $k$ 期简单毛收益率为:
\begin{equation}
    \begin{aligned}
        1+R_t[k] & =\frac{P_t}{P_{t-1}}=\frac{P_t}{P_{t-1}}\times\frac{P_{t-1}}{P_{t-2}}\times\cdots\times\frac{P_{t-k+1}}{P_{t-k}} \\
                 & =(1+R_t)(1+R_{t-1})\cdots(1+R_{t-k+1})                                                                           \\
                 & =\prod_{j=0}^{k-1}(1+R_{t-j})
    \end{aligned}
\end{equation}
这样,$k$ 期简单毛收益率是其包含的这 $k$ 个单期简单毛收益率的乘积,称为复合收益率 (compound return). $k$ 期简单净收益率为 $R_t[k]=(P_t -P_{t-k})/P_{t-k}$.

在实际中,确切的时间区间对讨论和比较收益率是非常重要的(例如月收益率还是年收益率). 若时间区间没有给出,这里隐含的假定时间区间为一年. 如果持有资产的期限为 $k$ 年,则(平均)年度化收益率定义为
$$\text{年化}R_t[k]=\left[\prod_{j=0}^{k-1}(1+R_{t-j})\right]^{1/k}-1$$

这是由它所包含的 $k$ 个单期简单毛收益率几何平均得到的,可用下式计算:
$$\text{年化}R_t[k]=\exp\left[\frac{1}{k}\sum_{j=0}^{k-1}(1+R_{t-j})\right]-1$$

因为算术平均值比几何平均值计算起来容易,并且单期收益率一般很小,所以我们可用一阶泰勒(Taylor)展开来近似表示年度化的收益率,则有
\begin{equation}
    \label{eq1-3}
    \text{年化}R_t[k]\approx\frac{1}{k}\sum_{j=0}^{k-1}R_{t-j}
\end{equation}
然而,在有些应用中,\autoref{eq1-3} 的近似精确度可能不够.