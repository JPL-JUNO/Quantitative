\chapter{Time Series Analysis and Forecasting\label{ch06}}
\section{Time series decomposition}
The components of time series can be divided into two types: systematic and non-systematic. The
systematic ones are characterized by consistency and the fact that they can be described and modeled.
By contrast, the non-systematic ones cannot be modeled directly.
The following are the systematic components:
\begin{itemize}
    \item \textbf{Level}—the mean value in the series.
    \item \textbf{Trend}—an estimate of the trend, that is, the change in value between successive time points at any given moment. It can be associated with the slope (increasing/decreasing) of the series. In other words, it is the general direction of the time series over a long period of time.
    \item \textbf{Seasonality}—deviations from the mean caused by repeating short-term cycles (with fixed and
          known periods).
\end{itemize}
The following is the non-systematic component:
\begin{itemize}
    \item \textbf{Noise}—the random variation in the series. It consists of all the fluctuations that are observed after removing other components from the time series.
\end{itemize}
The classical approach to time series decomposition is usually carried out using one of two types of models: additive and multiplicative.

An \textbf{additive model} can be described by the following characteristics:
\begin{itemize}
    \item Model's form—$y(t) = level + trend + seasonality + noise$
    \item Linear model—changes over time are consistent in size
    \item The trend is linear (straight line)
    \item Linear seasonality with the same frequency (width) and amplitude (height) of cycles over time
\end{itemize}

A \textbf{multiplicative model} can be described by the following characteristics:
\begin{itemize}
    \item Model's form—$y(t) = level * trend * seasonality * noise$
    \item Non-linear model—changes over time are not consistent in size, for example, exponential
    \item A curved, non-linear trend
    \item Non-linear seasonality with increasing/decreasing frequency and amplitude of cycles over time
\end{itemize}

Hodrick-Prescott 滤波器——虽然这种方法并不是真正的季节性分解方法,但它是一种数据平滑技术,用于消除与经济周期相关的短期波动。 通过消除这些,我们可以揭示长期趋势。 HP 滤波器常用于宏观经济学。 您可以在 \verb|statsmodels| 的 \verb|hpfilter| 函数中找到它的实现。

\section{Testing for stationarity in time series}
时间序列分析中最重要的概念之一是平稳性(stationarity)。 简单地说,平稳时间序列是一个属性不依赖于观察该序列的时间的序列。 换句话说,平稳性意味着某个时间序列的数据生成过程(data-generating process, DGP)的统计特性不随时间变化。

更正式地说,平稳性有多种定义,其中一些定义比其他定义更严格。 对于实际用例,我们可以使用一种称为弱平稳性(或协方差平稳性)的方法。 对于要归类为(协方差)平稳的时间序列,它必须满足以下三个条件:
\begin{itemize}
    \item 系列的平均值必须恒定
    \item 系列的方差必须是有限且恒定的
    \item 相同距离的周期之间的协方差必须恒定
\end{itemize}

ADF 和 KPSS 检验的一个潜在缺点是它们不允许结构中断的可能性,即数据生成过程的平均值或其他参数的突然变化。 Zivot-Andrews 测试允许该系列中出现单一结构断裂的可能性,但其发生时间未知。