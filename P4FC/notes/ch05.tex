\chapter{Technical Analysis and Building Interactive Dashboards\label{ch05}}
\section{Calculating the most popular technical indicators}
\textbf{Bollinger bands} are a statistical method, used for deriving information about the prices and volatility of a certain asset over time. To obtain the Bollinger bands, we need to calculate the moving average and standard deviation of the time series (prices), using a specified window (typically, 20 days). Then, we set the upper/lower bands at $K$ times (typically, 2) the moving standard deviation above/below the moving average. The interpretation of the bands is quite simple: the bands widen with an increase in volatility and contract with a decrease in volatility.

The \textbf{relative strength index (RSI)} is an indicator that uses the closing prices of an asset to identify oversold/overbought conditions. Most commonly, the RSI is calculated using a 14-day period and is measured on a scale from 0 to 100 (it is an oscillator). Traders usually buy an asset when it is oversold (if the RSI is below 30) and sell when it is overbought (if the RSI is above 70). More extreme high/low levels, such as 80–20, are used less frequently and, at the same time, imply stronger momentum.

The last considered indicator is the \textbf{moving average convergence divergence (MACD)}. It is a momentum indicator showing the relationship between two exponential moving averages (EMA) of a given asset's price, most commonly 26- and 12-day ones. The MACD line is the difference between the fast (short period) and slow (long period) EMAs. Lastly, we calculate the MACD signal line as a 9-day EMA of the MACD line. Traders can use the crossover of the lines as a trading signal. For example, it can be considered a buy signal when the MACD line crosses the signal line from below.

