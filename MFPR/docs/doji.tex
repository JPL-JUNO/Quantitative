\documentclass{article}
\usepackage{amsmath, amssymb}
\usepackage{ulem}
\usepackage{mathptmx}
\usepackage{ctex}
\usepackage{tcolorbox}
\usepackage{epigraph}
\usepackage{caption}
\usepackage{subcaption}
\usepackage{pdfpages}
\usepackage{graphicx}
\usepackage{multicol}
\setkeys{Gin}{width=0.75\textwidth}
\usepackage{listings}
\newtheorem{theorem}{Theorem}
\newtheorem{definition}{Definition}
\newtheorem{lemma}{Lemma}
\usepackage{tikz}
\usepackage{pifont}
\usepackage{threeparttable}
\usepackage{tabularx}
% \usepackage{algorithm}
\usepackage[lined,boxed,ruled]{algorithm2e}
\usepackage{bm}


% 在part页添加内容
\makeatletter
\let\old@endpart\@endpart
\renewcommand\@endpart[1][]{%
    \begin{quote}#1\end{quote}%
    \old@endpart}
\makeatother


\newcommand\tips[1]{\textcolor{green!50!black}{#1}}
\newcommand\notes[1]{\textcolor{blue!50!black}{#1}}
\newcommand\important[1]{\textcolor{red!90!black}{#1}}
\newcommand\warning[1]{\textcolor{orange!90!black}{#1}}

\newcommand\figures[2]{
    \begin{figure}
        \centering
        \includegraphics{../img/#1.png}
        \caption{#2}
        \label{#1}
    \end{figure}
}

\usepackage{enumitem}
% 去掉enumerate、itemize、description中的间隙
\setlist{noitemsep, topsep=0pt}

\tcbuselibrary{breakable}
\tcbset{breakable}
% \newtcblisting[auto counter, number within =chapter]{py}[1]{listing engine=minted,
% 	minted style=colorful,
% 	minted language=python,
% 	minted options={breaklines,autogobble,linenos,numbersep=3mm},
% 	colback=red!5!white,colframe=orange!50!blue,listing only, left=5mm,enhanced,
% 	title=Examples~\thetcbcounter~#1,
% 	breakable,
% 	enhanced
% 	%						 overlay={
% 	%						 	\begin{tcbclipinterior}
% 	%						 		\fill[red!30!white] (frame.south west)
% 	%						 		rectangle ([xshift=5mm]frame.north west);
% 	%							\end{tcbclipinterior}}
% }
\usepackage[
    top=1.23in,
    bottom=1in,
    right=1in,
    left=1in]
{geometry}

\usepackage{hyperref}
% 将引用的chapter改写为Chapter
\usepackage[english]{babel}
\addto\extrasenglish{
    \def\chapterautorefname{Chapter}
}
\addto\extrasenglish{
    \def\sectionautorefname{Section}
}
\usepackage{orcidlink}
\definecolor{SWJTU}{HTML}{025483}
% 全局取消段前缩进
% \setlength{\parindent}{0pt}
% \setlength{\parskip}{5pt}
\hypersetup{
    colorlinks=true,
    linkcolor=SWJTU,
    filecolor=SWJTU,
    urlcolor=SWJTU,
    citecolor=SWJTU,
}



\title{The Doji Pattern}
\begin{document}

\section*{The Doji Pattern}
\textit{doji} 这个词在日语中的意思是“错误”,这是合乎逻辑的,因为交易者将其标记为犹豫不决的模式。

\figures{fig6-2}{A bullish Doji}

\autoref{fig6-2} 显示了理论上的看涨十字星结构。在下跌的市场中,犹豫不决的蜡烛(例如十字星)的出现可能是方向改变的第一个线索。它必须由随后的看涨蜡烛来确认。

当需求开始失去动力并接近供给水平时,就会出现均衡,形成新高的动力丧失就证明了这一点。当十字星形态以收盘价等于开盘价的形式出现时,就会出现均衡,这表明双方都没有成功地将价格推向其期望的方向。
\subsection*{The Dragonfly Doji}
这种十字星的最高价格等于收盘价和开盘价。这意味着它是一个十字星,只有低廉的价格与其他十字星不同。这也是一种犹豫不决的模式,可以看涨也可以看跌,具体取决于之前的价格走势。

这种形态的问题在于,你可以用两种方式来思考它:要么市场没有创出更高的高点,因此未能继续走高并产生看跌倾向,要么市场未能收低并回到上方,给予看涨倾向。无论如何,它必须像对待所有十字星形态一样对待:作为反转形态。
\subsection*{The Gravestone Doji}
这与蜻蜓十字星相反。它遵循相同的直觉并具有相同的期望,这取决于之前的价格行为。在市场下跌的情况下,如果遇到墓碑十字线,那么你一定有看涨倾向。同样,如果你的市场正在上涨,并且你遇到了墓碑十字星,那么你一定有看跌倾向。

这好像是满足袋鼠尾的条件(短期信号)。
\subsection*{The Flat Doji}
通常发生在零售市场休市或市场交易临近假期时。低流动性和低交易量是扁平十字星的主要特征,它不太可能表明方向。然而,这仍然是一种市场犹豫不决的情况,其中四种价格彼此相等。这就是为什么它看起来像一条水平线。
\subsection*{The Double Doji}
可能会看到多个连续的十字星形态。连续的两个十字星形态意味着长期的犹豫不决,并且由于反向交易者仍在犹豫并努力迫使市场走自己的路,因此反转的信念可能会降低。\textbf{一般来说,两个十字星形态并不比一个十字星形态更有价值}。
\subsection*{The Tri Star Doji}
% 它的特点是具有三个十字星图案,其中中间的一个与另外两个之间存在间隙。

十字星形态并不完美,因为它可能会出现多次,但随后市场却没有反应。这是因为开盘价和收盘价之间的平衡并不一定意味着预期买卖活动的平衡。

十字星形态是一种常见结构,但本质上通常不具有预测性。\textbf{在依靠十字星模式进行未来反应之前,必须先定义当前的市场状况}。例如,在看涨趋势市场中,看跌十字星模式不太可能提供质量信号,但在横盘市场中,功效会增加(记住看不见的手)。
\section*{回测结果}
\end{document}