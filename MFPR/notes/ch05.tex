\chapter{现代趋势跟踪模式}
目标保持不变,即创造客观条件并对它们进行回测,以便对它们的频率和可预测性形成意见。需要记住的重要一点是,从一个市场到另一个市场,一种模式的可预测性是相当随机的,这意味着适用于 GBPUSD 的方法可能不适用于 EURGBP,因为每个市场都有不同的统计和技术特性。

因此,我所做的主要假设是,现代模式并不比经典模式好,也不比经典模式差。它们只是一种多元化工具,可以在分析中获得更多确认。这意味着,如果看到至少两种或三种模式(经典或现代)同时出现,应该更有信心进行交易。现在让我们开始使用这些新模式并探索它们的本质和代码。
\section{The Quintuplets Pattern}
五连线形态是一种确认潜在趋势的多 K 线结构。这种模式源于羊群心理以及反应失败\footnote{当市场参与者预计由于极端情况而出现逆转但市场仍保持其初始状态时,就会发生此事件。},这给予了它继续朝同一方向前进的额外推动力。它的特点是有五个连续的相同类型的小 K 线。该模式依赖于逐渐向上漂移的事件,其中市场似乎被高估或超买,但仍然毫发无伤并继续向上。

\figures{fig5-1}{A bullish Quintuplets pattern}

当波动性降低时,五连线形态就会出现,小型看涨 K 线就证明了这一点。

\autoref{ch04} 中讨论的三烛形态与五连烛形态之间的区别在于 K 线的数量及其大小。前者假设大动作之后是大动作,而后者假设缓慢而渐进的动作之后也是缓慢而渐进的动作。(这与金融市场中的波动性聚集事件同时发生。波动性聚集是波动性随时间的持续存在,其中低波动性通常随后是低波动性,而高波动性随后是高波动性。因此,两种模式都指向相同的方向,但处理不同的市场属性。

检测五连线形态很容易,但该模式可能不常见,因为需要满足许多条件。从算法上来说,需要满足以下条件:
\begin{itemize}
    \item 如果最近五个收盘价均大于其开盘价以及它们之前的收盘价,并且每个 K 线均遵循最大主体尺寸(箱体要小一点),则买入;
    \item 如果最近五个收盘价均小于其开盘价以及它们之前的收盘价,并且每个 K 线均遵循最大主体尺寸(箱体要小一点),则卖出;
\end{itemize}