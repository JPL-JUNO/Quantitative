\chapter{经典的趋势跟踪模式}
\begin{tcolorbox}
    对模式进行回测的方法是假设我在验证前一次收盘时的信号后,在下一次开盘时启动买入或卖出头寸。
\end{tcolorbox}
\section{Marubozu 趋势}
第一个经典的趋势跟踪模式是 Marubozu。这个词在日语中指的是秃头或短发。

区分 Marubozu K 线和普通 K 线相对容易,因为 Marubozu K 线没有任何影线。这意味着看涨的 Marubozu K 线具有相同的开盘价和最低价以及相同的收盘价和最高价。相比之下,看跌的 Marubozu K 线具有相同的开盘价和高价以及相同的收盘价和低价。\autoref{fig4-1} 显示了 Marubozu K 线的这两个情况。

\figures{fig4-1}{左边是看涨的 Marubozu K 线; 右侧是看跌的 Marubozu K 线}

Marubozu 模式通常发生在较短的时间范围内,因为波动和超出其范围的时间较短。分析 1 分钟和 5 分钟图表并将其与日线图进行比较时可以看出这一点。
\begin{tcolorbox}
    当开盘价和收盘价之间的时间较短时,K 线更有可能呈现 Marubozu 形态。
\end{tcolorbox}

当市场处于强劲的上升趋势时,由于对标的证券的巨大需求,它很少会创下更低的低点,并且通常会在高点附近收盘。一根 K 线中需求的最大力量是通过没有低点且以高点收盘来证明的,而这正是看涨 Marubozu 的全部内容。当一项资产以高价收盘时,您会得到买家渴望更多的信号,而当没有最低价低于开盘价时,您应该进一步确信没有人有兴趣卖出来价格推至低于开盘价。

同样,当市场处于强劲下跌趋势时,由于标的证券的供应量巨大,它很少会创下更高的高点,并且通常会在低点附近收盘。一根 K 线的最大供应力是通过没有高点且收于低点来证明的,正如看跌的 Marubozu 所证明的那样。
\section{The Three Candles Pattern}
三 K 线形态是一种趋势确认结构,在观察具有最小指定尺寸的三根相同颜色的 K 线后给出信号。看涨三 K 线形态具有三根大的看涨 K 线,每根 K 线的收盘价都高于前一根。相比之下,看跌三 K 线形态具有三根大看跌 K 线,每根 K 线的收盘价都低于前一根。
\figures{fig4-5}{看涨三 K 线}
该模式的直觉非常简单。它源于一种称为羊群效应的心理偏见,即市场参与者跟随趋势,因为其他人也在这样做。这并不意味着这种模式是基于人类的缺陷和缺乏努力;它只是指在人类中观察到的一种常见行为,他们倾向于遵循总体趋势。

\begin{tcolorbox}
    人们追随最新的趋势是因为它能带来心理上的回报。趋势跟踪交易者追随最新趋势,因为他们相信这会带来经济回报,或者因为他们有一种害怕错过的恐惧(FOMO,fear of missing out)。
\end{tcolorbox}

从算法上来说,需要满足以下条件:
\begin{itemize}
    \item 如果最近三个收盘价均大于其之前的收盘价,并且每个 K 线均满足最小箱体,则为买入信号。

    \item 如果最近三个收盘价均低于其之前的收盘价,并且每个 K 线均满足最小箱体,则为卖出信号。
\end{itemize}

K 线的箱体是收盘价和开盘价之间的绝对差值。
\begin{equation}
    \text{Candlestick body}_i = \text{Close price}_i - \text{Open price}_i
\end{equation}