\chapter{经典的趋势跟踪模式\label{ch04}}
\begin{tcolorbox}
    对模式进行回测的方法是假设我在验证前一次收盘时的信号后,在下一次开盘时启动买入或卖出头寸。
\end{tcolorbox}
\section{Marubozu 趋势}
第一个经典的趋势跟踪模式是 Marubozu。这个词在日语中指的是秃头或短发。

区分 Marubozu K 线和普通 K 线相对容易,因为 Marubozu K 线没有任何影线。这意味着看涨的 Marubozu K 线具有相同的开盘价和最低价以及相同的收盘价和最高价。相比之下,看跌的 Marubozu K 线具有相同的开盘价和高价以及相同的收盘价和低价。\autoref{fig4-1} 显示了 Marubozu K 线的这两个情况。

\figures{fig4-1}{左边是看涨的 Marubozu K 线; 右侧是看跌的 Marubozu K 线}

Marubozu 模式通常发生在较短的时间范围内,因为波动和超出其范围的时间较短。分析 1 分钟和 5 分钟图表并将其与日线图进行比较时可以看出这一点。
\begin{tcolorbox}
    当开盘价和收盘价之间的时间较短时,K 线更有可能呈现 Marubozu 形态。
\end{tcolorbox}

当市场处于强劲的上升趋势时,由于对标的证券的巨大需求,它很少会创下更低的低点,并且通常会在高点附近收盘。一根 K 线中需求的最大力量是通过没有低点且以高点收盘来证明的,而这正是看涨 Marubozu 的全部内容。当一项资产以高价收盘时,您会得到买家渴望更多的信号,而当没有最低价低于开盘价时,您应该进一步确信没有人有兴趣卖出来价格推至低于开盘价。

同样,当市场处于强劲下跌趋势时,由于标的证券的供应量巨大,它很少会创下更高的高点,并且通常会在低点附近收盘。一根 K 线的最大供应力是通过没有高点且收于低点来证明的,正如看跌的 Marubozu 所证明的那样。
\section{The Three Candles Pattern}
三 K 线形态是一种趋势确认结构,在观察具有最小指定尺寸的三根相同颜色的 K 线后给出信号。看涨三 K 线形态具有三根大的看涨 K 线,每根 K 线的收盘价都高于前一根。相比之下,看跌三 K 线形态具有三根大看跌 K 线,每根 K 线的收盘价都低于前一根。
\figures{fig4-5}{看涨三 K 线}
该模式的直觉非常简单。它源于一种称为羊群效应的心理偏见,即市场参与者跟随趋势,因为其他人也在这样做。这并不意味着这种模式是基于人类的缺陷和缺乏努力;它只是指在人类中观察到的一种常见行为,他们倾向于遵循总体趋势。

\begin{tcolorbox}
    人们追随最新的趋势是因为它能带来心理上的回报。趋势跟踪交易者追随最新趋势,因为他们相信这会带来经济回报,或者因为他们有一种害怕错过的恐惧(FOMO,fear of missing out)。
\end{tcolorbox}

从算法上来说,需要满足以下条件:
\begin{itemize}
    \item 如果最近三个收盘价均大于其之前的收盘价,并且每个 K 线均满足最小箱体,则为买入信号。

    \item 如果最近三个收盘价均低于其之前的收盘价,并且每个 K 线均满足最小箱体,则为卖出信号。
\end{itemize}

K 线的箱体是收盘价和开盘价之间的绝对差值。
\begin{equation}
    \text{Candlestick body}_i = \text{Close price}_i - \text{Open price}_i
\end{equation}
\section{The Tasuki Pattern}
Tasuki 形态是一种趋势确认形态,市场在未填补的缺口后发出持续信号。

缺口是价格行为的重要组成部分。它们的稀有程度因市场而异,例如,在货币市场中,它们通常发生在周末后市场开盘时或每当有重大公告时。在股票中,日复一日的跳空现象相当普遍。

缺口是两个连续收盘价之间的中断或缺口,主要是由于流动性较低而造成的。当市场交易价格为 100 美元,突然交易价格为 102 美元,而报价从未达到 101 美元,则形成了看涨缺口。这可以从图表中看出, K 线图之间似乎缺少一块。\autoref{fig4-8} 展现了看涨缺口。

\figures{fig4-8}{A bullish gap}

缺口有不同类型,当由于事后偏见(一种认知偏见,由于已经知道结果而导致分析师高估技术的预测能力)而出现时,对它们进行分类可能很困难:

\begin{description}
    \item[常见缺口]这些通常发生在横盘市场中。由于市场的均值回归动态,它们很可能会被填补。
    \item[突破性缺口]这些通常类似于常见的缺口,但缺口出现在图形阻力上方或图形支撑下方。它们标志着新趋势的加速。
    \item[失控缺口]这些通常发生在趋势内,但更多地确认趋势;因此,这是一个持续形态。
    \item[疲惫缺口]这些通常发生在趋势结束时且接近支撑位或阻力位。这是一个反转形态。
\end{description}

没有明确的方法可以在缺口出现时确定其类型,但这对于检测 Tasuki 模式并不是必需的。

\figures{fig4-10}{看涨 Tasuki 形态由三个 K 线组成,其中第一个是看涨 K 线,第二个是在第一个 K 线上方跳空的另一个看涨 K 线,第三个 K 线是看跌 K 线,但收盘价不低于第一个 K 线的收盘价。}

该形态的直觉与突破原理有关,突破某个阈值(无论是支撑位还是阻力位),在将其释放到初始方向之前,应该有一个朝向该阈值的引力。

每当看到市场跳空上涨,随后出现看跌 K 线图但未能完全缩小差距时,这可能是一个信号,表明卖方的力量不足以占据主导地位,从而产生看涨倾向。同样,每当看到市场向下跳空,随后出现看涨 K 线图但未能完全缩小差距时,这可能是一个信号,表明买家的实力不足以占据主导地位,从而产生看跌倾向。

算法上,条件可以这样判断,用于判断的三根 K 线分别用 $t-2,t-1,t$ 标识:
\begin{itemize}
    \item $t-2$ 与 $t-1$ 的收盘价大于开盘价,即两根阳线,并且 $t-1$ 的开盘价大于 $t-2$ 的收盘价,$t$ 的收盘价大于 $t-2$ 的收盘价小于 $t-1$的开盘价,那么确认为上涨趋势
    \item $t-2$ 与 $t-1$ 的收盘价小于开盘价,即两根阴线,并且 $t-1$ 的开盘价小于 $t-2$ 的收盘价,$t$ 的收盘价小于 $t-2$ 的收盘价大于 $t-1$的开盘价,那么确认为下跌趋势
\end{itemize}

请记住,这种模式相当罕见,应该以某种统计怀疑的态度来对待结果。
\section{The Three Methods Pattern}
Three Methods Pattern 是一种复杂的结构,主要由五个 K 线组成。上升的 Three Methods Pattern 应该出现在看涨趋势中,第一个 K 线是一个大型看涨 K 线,随后是三个小型看跌 K 线,通常包含在第一个 K 线的范围内。为了确认该形态,最后一根大看涨 K 线的收盘价必须高于第一根 K 线的高点。
\figures{fig4-12}{A rising Three Methods pattern}

从心理学上来说,该模式与超越或突破的概念有关,作为对初始走势的确认。交易者通常会推高价格,直到他们开始获利并平仓。这项活动对价格有平滑作用,应该通过一个称为修正或盘整的阶段来稳定价格。如果初始交易者恢复买入或卖出活动并设法超出盘整范围,你可以有一定的信心认为这一走势应该继续下去。
\section{The Hikkake Pattern}
Hikkake 是一个日语动词,意思是“欺骗”或“陷阱”,这个模式指的是一个实际的陷阱。该模式很复杂,根据文献由大约五个 K 线组成(一些研究表明它可能是多个 K 线的组合,但没有指定 K 线的数量)。

\figures{fig4-15}{看涨的 Hikkake}
看涨的 Hikkake(如 \autoref{fig4-15} 所示)以看涨 K 线开始,随后是完全嵌入第一个 K 线的看跌 K 线。然后,出现的两根 K 线的最高价不得超过第二根 K 线的最高价。最后,出现一根大看涨 K 线,其收盘价超过第二根 K 线的高点。这可以作为该形态的验证和上行确认。

尽管该模式本质上是主观的,但它的心理并不难理解。“陷阱”一词源于这样一个事实:对于认为市场已经遇到阻力并应该继续走低的交易者来说,看涨的 Hikkake 在历史上是一个看跌陷阱。因此,每当您看到最终 K 线突破高点并验证结构时,就可以对即将到来的更多强度有一定的信心。这是因为止损被取消,而且交易者正在关注突破阻力位的市场,这在理论上是一个强烈的看涨信号。

由于该模式非常罕见,因此没有足够的数据来评估。当然,条件可以放宽,以便出现更多信号,但是从理论上讲,如果你要这样做,你就必须将其命名为其他名称。
\section{小结}
\begin{table}
    \centering
    \caption{经典趋势跟踪策略}
    \label{tbl4-1}
    \begin{tabular}{lll}
        \hline
        Pattern               & 基本含义        & 出现频率 \\
        \hline
        Marubozu Pattern      & 缺上下影线       & 常见   \\
        Three Candles Pattern & 三连阳(阴)      & 常见   \\
        Tasuki Pattern        & 跳开          & 稀有   \\
        Three Methods Pattern & 一阳(阴)灭三阴(阳) & 稀有   \\
        Hikkake Pattern       & 陷阱          & 稀有   \\
        \hline
    \end{tabular}
\end{table}
\autoref{tbl4-1} 总结了经典趋势跟踪策略的含义和频率。