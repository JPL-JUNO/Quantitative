\chapter{经典的趋势跟踪模式}
\begin{tcolorbox}
    对模式进行回测的方法是假设我在验证前一次收盘时的信号后,在下一次开盘时启动买入或卖出头寸。
\end{tcolorbox}
\section{Marubozu 趋势}
第一个经典的趋势跟踪模式是 Marubozu。这个词在日语中指的是秃头或短发。

区分 Marubozu K 线和普通K 线相对容易,因为 Marubozu K 线没有任何影线。这意味着看涨的 Marubozu K 线具有相同的开盘价和最低价以及相同的收盘价和最高价。相比之下,看跌的 Marubozu K 线具有相同的开盘价和高价以及相同的收盘价和低价。\autoref{fig4-1} 显示了 Marubozu K 线的这两个情况。

\figures{fig4-1}{左边是看涨的 Marubozu K 线; 右侧是看跌的 Marubozu K 线}

Marubozu 模式通常发生在较短的时间范围内,因为波动和超出其范围的时间较短。分析 1 分钟和 5 分钟图表并将其与日线图进行比较时可以看出这一点。
\begin{tcolorbox}
    当开盘价和收盘价之间的时间较短时,K 线更有可能呈现 Marubozu 形态。
\end{tcolorbox}

当市场处于强劲的上升趋势时,由于对标的证券的巨大需求,它很少会创下更低的低点,并且通常会在高点附近收盘。一根烛台中需求的最大力量是通过没有低点且以高点收盘来证明的,而这正是看涨 Marubozu 的全部内容。当一项资产以高价收盘时,您会得到买家渴望更多的信号,而当没有最低价低于开盘价时,您应该进一步确信没有人有兴趣卖出来价格推至低于开盘价。

同样,当市场处于强劲下跌趋势时,由于标的证券的供应量巨大,它很少会创下更高的高点,并且通常会在低点附近收盘。一根烛台的最大供应力是通过没有高点且收于低点来证明的,正如看跌的 Marubozu 所证明的那样。
\subsection*{}
