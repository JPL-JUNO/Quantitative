\chapter{经典反转模式}
\section{The Doji Pattern}
\textit{doji} 这个词在日语中的意思是“错误”,这是合乎逻辑的,因为交易者将其标记为犹豫不决的模式。

\figures{fig6-2}{A bullish Doji}

\autoref{fig6-2} 显示了理论上的看涨十字星结构。在下跌的市场中,犹豫不决的蜡烛(例如十字星)的出现可能是方向改变的第一个线索。它必须由随后的看涨蜡烛来确认。

当需求开始失去动力并接近供给水平时,就会出现均衡,形成新高的动力丧失就证明了这一点。当十字星形态以收盘价等于开盘价的形式出现时,就会出现均衡,这表明双方都没有成功地将价格推向其期望的方向。
\subsection*{The Dragonfly Doji}
这种十字星的最高价格等于收盘价和开盘价。这意味着它是一个十字星,只有低廉的价格与其他十字星不同。这也是一种犹豫不决的模式,可以看涨也可以看跌,具体取决于之前的价格走势。

这种形态的问题在于,你可以用两种方式来思考它:要么市场没有创出更高的高点,因此未能继续走高并产生看跌倾向,要么市场未能收低并回到上方,给予看涨倾向。无论如何,它必须像对待所有十字星形态一样对待:作为反转形态。
\subsection*{The Gravestone Doji}
这与蜻蜓十字星相反。它遵循相同的直觉并具有相同的期望,这取决于之前的价格行为。在市场下跌的情况下,如果遇到墓碑十字线,那么你一定有看涨倾向。同样,如果你的市场正在上涨,并且你遇到了墓碑十字星,那么你一定有看跌倾向。

这好像是满足袋鼠尾的条件(短期信号)。
\subsection*{The Flat Doji}
通常发生在零售市场休市或市场交易临近假期时。低流动性和低交易量是扁平十字星的主要特征,它不太可能表明方向。然而,这仍然是一种市场犹豫不决的情况,其中四种价格彼此相等。这就是为什么它看起来像一条水平线。
\subsection*{The Double Doji}
可能会看到多个连续的十字星形态。连续的两个十字星形态意味着长期的犹豫不决,并且由于反向交易者仍在犹豫并努力迫使市场走自己的路,因此反转的信念可能会降低。\textbf{一般来说,两个十字星形态并不比一个十字星形态更有价值}。
\subsection*{The Tri Star Doji}
它的特点是具有三个十字星图案,其中中间的一个与另外两个之间存在间隙。

十字星形态并不完美,因为它可能会出现多次,但随后市场却没有反应。这是因为开盘价和收盘价之间的平衡并不一定意味着预期买卖活动的平衡。
\section{The Hammer Pattern}
从技术上讲,锤子形态只是四种不同(但相似)烛台图形的一个名称,即流星、上吊线、锤子和倒锤子。

看涨锤子线是具有长下影线但没有上影线的看涨 K 线。看跌锤子线是具有长上影线且没有下影线的看跌 K 线。
\figures{fig6-28}{A bullish Hammer}

对于连续的三根 K 线 $K1, K2, K3$,满足以下条件:
\begin{enumerate}
    \item $K1_c < K1_o$
    \item $(K2_c=K2_h) \land (K2_c > K2_o) \land ((K2_c-K2_o)<body) \land ((K2_o-K2_l)>2*wick)$
    \item $K3_c>K3_o$
\end{enumerate}
\begin{tcolorbox}[title=警告!,colframe=red!80!black]
    这个模式的出现次数太少,主要是 $(K2_c=K2_h)$ 太难满足。可以改进这个条件,允许存在一定的上影线。
\end{tcolorbox}