\chapter{技术分析引言}
\section{指标分析}
\subsection{RSI}
我们来讨论一下反向指标。相对强弱指数 (RSI) 最初由 J. Welles Wilder Jr. 提出,是最流行、最通用的有界指标之一。它主要用作反向指标,其中极值表明可以利用的反应。
使用以下步骤计算RSI(默认的计算周期是 14 条数据):
\begin{enumerate}
    \item 计算收盘价与前一日收盘价的变化。
    \item 将正净变化与负净变化分开。
    \item 计算正净变化和负净变化绝对值的平滑移动平均值。
    \item 将平滑后的正变化除以平滑后的绝对负变化,将此计算称为相对强度(RS)。
    \item 对每个时间应用归一化公式以获得 RSI:
          \begin{equation}
              RSI_i = 100 -\frac{100}{1 + RS_i}
          \end{equation}
\end{enumerate}

\begin{tcolorbox}
    平滑移动平均线是 RSI 创建者开发的一种特殊类型的移动平均线。它比简单移动平均线更平滑、更稳定。
\end{tcolorbox}

一般来说,RSI 默认使用 14 的周期,尽管每个交易者对此可能有自己的偏好。以下是如何使用该指标:
\begin{itemize}
    \item 当 RSI 显示为 30 或更低时,市场被视为超卖,并且可能会出现向上修正。
    \item 当 RSI 显示 70 或更高时,市场就被认为是超买,并且可能会出现下行修正。
    \item 当 RSI 超过或突破 50 时,新的趋势可能会出现,但这通常是一个薄弱的假设,并且本质上更理论化而不是实用。
\end{itemize}