\chapter{生产商套期保值策略}
对于生产商来讲,其面临的最大价格波动风险即商品价格下跌的风险。生产商在商品交易中拥有“天然的多头头寸”,他们拥有商品要出售,因此为规避价格下跌的风险,就要求生产商在期货市场上进行卖出套期保值操作,也可以将其称为生产者套期保值或空头套期保值。也可以称为空头套期保值是因为商品的生产者将在期货市场上通过空头头寸来对生产经营的现货头寸进行套期保值。
\section{运用期货合约进行卖出套期保值}
\subsection{正向市场中卖出套保}
在现货商品供应充足、库存量较大的正常情况下,期货价格通常要高于现货价格,或者远期合约价格通常要高于近月合约价格,因为有持仓费用的存在,我们通常称这类市场为正向市场或持仓费市场。

反向市场状况则刚好相反,即远期合约价格通常要低于近月合约价格,或者期货价格要低于现货价格。出现这列史称主要有两个原因:一是近期对某种商品的需求非常迫切,远大于近期产量及库存量,导致现货价格大幅度上扬,高于期货价格;二是预期将来该商品的供给会大幅增加,导致期货价格大幅下降,低于现货价格。