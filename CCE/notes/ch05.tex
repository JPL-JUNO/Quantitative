\chapter{酒田战法和其他蜡烛图组合}
\section{酒田战法}
\subsection{三山形态}
三山形态构成了市场的一个大型顶部,与西方技术分析中的“三重顶”形态比较类似,在三重顶形态中,价格上升和下跌各三次,形成了市场的顶部。三尊顶形态(san-son)与西方的头肩形顶部形态也类似。三尊顶形态有点儿像佛教中佛像的陈列,大殿里供奉的佛像通常有三尊,其中中间摆放的是一尊最大的,两边分别是小一点的佛像(见 \autoref{fig5-1a})。三山形态包含了西方技术分析中的“三重顶”形态,在这一形态中,价格三次向上测试,但随后都出现了一定幅度的调整,在三重顶形态中,三个顶部的高度是相同的,或者大致接近(见 \autoref{fig5-1b})。
\figures{fig5-1a}{}
\figures{fig5-1b}{}

\subsection{三川形态}
三川形态和三山形态正好相反。它同传统的三重底形态和头肩底形态类似。三川形态由三根位于市场底部的看涨蜡烛线组合而成,用于预测市场的转折点,这些 K 线组合包括启明星、白三兵等,在一些介绍酒田战法的日文著作中,也将启明星形态称作三川启明星形态 \autoref{fig5-2a}。

\figures{fig5-2a}{}
\subsection{三空形态}
在三空形态中(\autoref{fig5-3}),价格的跳空缺口意味着投资者进入和退出市场的时机。以向上跳空形态为例,市场出现底部后,当它再次上升时,投资者应该在出现第三个跳空缺口后做空。第一个向上跳空缺口意味着新入场的买方力量强大,第二个缺口代表继续有买方入场以及部分有经验的空头平仓,第三个跳空缺口是由犹豫的空头平仓和如梦初醒的多头买进造成的。酒田战法建议在第三个向上跳空的缺口后做空,因为买盘卖盘出现分歧以及随后市场出现超买的可能性越来越大。相反,在下降趋势中出现第三个向下跳空缺口后,投资者应该做多。在日语中,跳空缺口的弥补又被称为“anaume”,跳空缺口又被称为窗口(mado)。

\figures{fig5-3}{}
\subsection{三兵形态}
三兵形态指的是“向同一个方向站立的三名士兵”。白三兵是一个典型的看涨信号,表明市场正处于稳定的上升态势中,这种稳健的价格走势说明市场将进一步大幅走高。另外,酒田战法还给出了三兵形态的衰退形态,这些形态表明上升趋势的力度逐渐减弱,也就是通常所说的“上涨乏力”。三兵形态包含几种衰退形态,第一种衰退形态是“前进受阻形态”,该形态虽然跟白三兵形态比较类似,不同之处在于,该形态在第二天和第三天录得的 K 线都带有很长的上影线。第二个衰退形态则是“停滞形态”,在该形态中,同样也是第二天的 K 线带有很长的上影线,而第三天则是出现纺锤线,或者是十字星,这表明市场拐点的临近。

三兵形态还包括三只黑乌鸦形态和类三只乌鸦的形态。
\subsection{三法形态}
三法形态是指“市场休息或者休整的状态”。休整是我们平时说的除买入和卖出外的另外一种市场状态。

酒田战法为我们提供了一种清晰明确的图表分析法则,它成立的基础如下:
\begin{itemize}
    \item 在价格上上下下的波动中,市场将会继续沿着既定的大方向推进。这是我们分析蜡烛图形态的一个基础原则,在第6章中,我们将进一步提及。
    \item 推动市场上涨所消耗的能量要高于市场下跌的能量,这一点与物理学中物体可以借助自身重力下降的原理不谋而合。
    \item 没有只涨不跌的市场,也没有只跌不涨的市场。1991 年 9 月《福布斯观察》的一篇文章中曾写道:“在熊市中,聪明的投资者应该提醒自己,世界末日不会到来,市场不会无休止下跌;在牛市中,聪明的投资者也应该提醒自己,即使是参天大树,也不可能永无止境地生长。”套用一句谚语就是,天下没有不散的筵席。
    \item 市场价格有时会陷入停滞,也就是通常所说的横向盘整时期,此时,明智的交易员应该选择离场观望。
\end{itemize}
\section{其他蜡烛图组合}
在传统的技术分析中,有多种蜡烛图组合用来描述价格的走势。现在西方使用的蜡烛图形态大多由史蒂夫·尼森命名,虽然这些蜡烛图组合包含了很长一段交易日的价格数据,但我们通常只能把它们作为一般的预测市场的工具,投资者和交易员并不能利用这些组合形态获取准确的入场时机。当一个形态形成,尤其是反转形态,一定要找找能否找到其他证据佐证价格确实可能发生反转。另外,在形态形成到最终确认的一段很长的时间里,通常会出现一些干扰性的因素,请务必记住,在几乎所有的蜡烛图形态中,而且几乎可以肯定的是,在几乎所有的反转形态中,必须牢记它们与当下趋势或先前趋势的关系,在分析蜡烛图形态时,这些趋势会在很大程度上受到接下来的一些蜡烛图形态的影响。
\subsection{平头}
平头形态比较简单,由两根或两根以上的日蜡烛图组成,用于判断市场的顶部和底部。我们把两根具有相同最高点的日蜡烛图组成的形态称为平头顶(kenukitenjo);反之,把两根具有相同最低点的日蜡烛图组成的形态称为平头底(kenukizoko)。\tips{需要提醒投资者的是,平头顶部形态和平头底部形态不一定只包括两根蜡烛图,两根蜡烛图之间可以夹杂几根其他 K 线。}

利用平头顶部形态和平头底部形态,可以判定短期的支撑和阻力。支撑位是指在短期内可以阻止价格下跌的价位,阻力位是指在短期内会限制价格上涨的价位,两个价位都是市场前期走势中形成的。作为反转形态的一种,平头形态是一个较好的市场预测指标。在十字孕线形态中,如果两根蜡烛图的最高价(或者最低价)相等,那么十字孕线形态也可以成为平头形态。

另外,平头形态还可以演化为相同低价形态和竖状三明治形态。这两种看涨形态都是平头形态的演变,不同之处在于,它们的形态描述中引入的是收盘价的概念,而平头形态使用的是最高价或最低价。
\subsection{风高浪急形态}
风高浪急形态指的是具有长上影线的一系列 K 线组合。在一段上升趋势后,如果频繁出现射击之星、纺锤线或者墓碑十字星等,预示市场可能见顶。这类 K 线的出现说明市场正在失去方向感,不能再以更高价位收盘,反转趋势一触即发。通常而言,前进受阻形态也可能成为风高浪急形态的开始。
\subsection{塔形顶和塔形底}
塔形顶形态和塔形底形态都是由三根以上蜡烛线组成的形态,这个形态由标志性的大阳线或大阴线组成,随后蜡烛图的颜色逐渐发生变化,塔形的出现预示市场趋势即将发生反转。塔形底通常出现在下跌趋势中,先是出现了一系列的大阴线,但并不一定需要出现像三只黑乌鸦那样明显的下跌。在形态后期,大阴线逐渐转化为阳线,虽然此时市场反转的迹象还不是很明显,但是下降的趋势已经得到缓和,并且价格创出一段时期以来的新高。在塔形底形态形成的过程中,特别是在由阴线转化为阳线的调整期,通常会出现小实体 K 线,而这些小阳线或小阴线并未成为反转形态的一部分。同理,塔形顶正好完全相反。塔形的含义是指帮助我们判断形态的大阳线或大阴线。
\subsection{平底锅底部形态 \autoref{fig5-14}}
平底锅形态和塔形底比较类似,不同之处在于,该形态底部是由一系列小实体 K 线组成的,该形态呈圆弧形,K 线的颜色并不重要。在经历一段时间的底部徘徊后,出现一根向上跳空的阳线,由此确认反转和新一轮上涨趋势的开始。平底锅形态从形状上看,它确实有些像我们炒菜用的锅,底部的小实体 K 线类似于锅底,那根起决定作用的大阳线就是锅柄。

\figures{fig5-14}{平底锅底部形态}

圆形顶与平底锅底部形态正好相反。它与西方传统技术分析中的术语圆形顶部类似,通过一根具有向下跳空缺口的阴线,确认市场顶部。如果在跳空缺口后出现的阴线是一根执带线,那么未来看跌的倾向更为浓厚。
\subsection{高位跳空突破形态 \autoref{fig5-16} 和低位跳空突破形态}
高位跳空突破形态(见 \autoref{fig5-16})和低位跳空突破形态相当于日文中的突破,价格一开始在支撑或阻力附近盘整,随着时间的流逝,市场越发变得犹豫不决,盘整的时间越长,未来突破的力度也就越大,一旦盘整区间被突破,市场便会迅速确定方向。如果突破产生的缺口方向与盘整前的市场大方向一致,那么价格沿着原有趋势运行的可能性将更加确定。
\figures{fig5-16}{高位跳空突破形态}