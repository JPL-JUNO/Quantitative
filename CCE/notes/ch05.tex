\chapter{酒田战法和其他蜡烛图组合}
\section{酒田战法}
\subsection{三山形态}
三山形态构成了市场的一个大型顶部,与西方技术分析中的“三重顶”形态比较类似,在三重顶形态中,价格上升和下跌各三次,形成了市场的顶部。三尊顶形态(san-son)与西方的头肩形顶部形态也类似。三尊顶形态有点儿像佛教中佛像的陈列,大殿里供奉的佛像通常有三尊,其中中间摆放的是一尊最大的,两边分别是小一点的佛像(见 \autoref{fig5-1a})。三山形态包含了西方技术分析中的“三重顶”形态,在这一形态中,价格三次向上测试,但随后都出现了一定幅度的调整,在三重顶形态中,三个顶部的高度是相同的,或者大致接近(见 \autoref{fig5-1b})。
\figures{fig5-1a}{}
\figures{fig5-1b}{}

\subsection{三川形态}
三川形态和三山形态正好相反。它同传统的三重底形态和头肩底形态类似。三川形态由三根位于市场底部的看涨蜡烛线组合而成,用于预测市场的转折点,这些 K 线组合包括启明星、白三兵等,在一些介绍酒田战法的日文著作中,也将启明星形态称作三川启明星形态 \autoref{fig5-2a}。

\figures{fig5-2a}{}
\subsection{三空形态}
在三空形态中(\autoref{fig5-3}),价格的跳空缺口意味着投资者进入和退出市场的时机。以向上跳空形态为例,市场出现底部后,当它再次上升时,投资者应该在出现第三个跳空缺口后做空。第一个向上跳空缺口意味着新入场的买方力量强大,第二个缺口代表继续有买方入场以及部分有经验的空头平仓,第三个跳空缺口是由犹豫的空头平仓和如梦初醒的多头买进造成的。酒田战法建议在第三个向上跳空的缺口后做空,因为买盘卖盘出现分歧以及随后市场出现超买的可能性越来越大。相反,在下降趋势中出现第三个向下跳空缺口后,投资者应该做多。在日语中,跳空缺口的弥补又被称为“anaume”,跳空缺口又被称为窗口(mado)。

\figures{fig5-3}{}
\subsection{三兵形态}
三兵形态指的是“向同一个方向站立的三名士兵”。白三兵是一个典型的看涨信号,表明市场正处于稳定的上升态势中,这种稳健的价格走势说明市场将进一步大幅走高。另外,酒田战法还给出了三兵形态的衰退形态,这些形态表明上升趋势的力度逐渐减弱,也就是通常所说的“上涨乏力”。三兵形态包含几种衰退形态,第一种衰退形态是“前进受阻形态”,该形态虽然跟白三兵形态比较类似,不同之处在于,该形态在第二天和第三天录得的 K 线都带有很长的上影线。第二个衰退形态则是“停滞形态”,在该形态中,同样也是第二天的 K 线带有很长的上影线,而第三天则是出现纺锤线,或者是十字星,这表明市场拐点的临近。