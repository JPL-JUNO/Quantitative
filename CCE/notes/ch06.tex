\chapter{蜡烛图形态识别背后的哲学}
\section{蜡烛形态的识别}
\subsection{大阳或大阴蜡烛图的识别}
大阳(阴)蜡烛图的识别和判定有三种方法。在下列公式中,最小值指的是大阳(阴)交易日最低可接受的百分比。如果交易日实体的相对长度大于这个最小值,我们就可以认为该交易日是大阳(阴)日。
\begin{equation}
    \frac{\text{实体的长度}}{\text{价格}}-\text{最小值}
\end{equation}
该方法将考察日的价格和实际的股票或期货的价格进行比较。如果我们把最小值设为 5\%,价格为 100,那么大阳(阴)日就是指开盘价和收盘价之间的差额大于等于 5 的交易日。使用该方法时不需要用到以前的价格数据。
\begin{equation}
    \frac{\text{实体的长度}}{\text{最高价}-\text{最低价}}-\text{最小值}
\end{equation}
该方法将考察日的实体长度与当天最高价和最低价差额进行比较。如果考察日的蜡烛图有较长的上下影线,我们就不认为该交易日是大阳(阴)日。这种判别方法可以把带有较长上下影线的纺锤线从考察样本中剔除。如果和其他的判定方法搭配使用,该公式的有效性更高。

\begin{equation}
    \frac{\text{实体的长度}}{\text{最近 X 天的实体平均长度}}-\text{最小值}
\end{equation}
在该方法中,我们引入了一个参数(最近 X 天内的实体平均长度)来帮助我们确定考察日性质。X 的取值为 5-10。如果百分比设为 130\%,那么大阳(阴)日指的就是实体长度高于平均实体长度 30\% 的交易日。这种方法在实际的判定过程中经常使用,它的定义比较容易理解,而且考虑了短期的市场趋势。
\subsection{十字蜡烛图的识别}
从理论上讲,如果交易日内开盘价和收盘价相同就会形成十字蜡烛图。但是,由于最小价格波动单位等因素的限制,实际上的十字蜡烛图识别的条件可以适当放宽。下面的公式可以帮助读者开盘价和收盘价之差的百分比:
\begin{equation}
    \frac{\text{十字线的实体长度}}{\text{最高价}-\text{最低价}}-\text{最大值}
\end{equation}
在公式中,最大值是十字线长度相对于市场价格区间的最大可接受百分比。一般来说,差额在 1\% 至 3\% 之间就可以用来确定形态成立。