\chapter{反转形态}
我们在分析每一种蜡烛图形态的时候,都会用3根很小的垂直线段来表明市场此前的趋势方向,这些线段不应该作为形态间相互关系的直接参考。
\section{反转形态与持续形态}
我们在判断一个蜡烛形态组合是看涨倾向还是看跌倾向时,这里的倾向仅指随后的价格走势,而与此前的趋势方向无关,相同的蜡烛图形态出现在趋势的不同位置,所代表的含义也不尽相同。正因为如此,此前的趋势只是为我们进行了蜡烛图形态的确认,而不能作为预测此后价格运动的判断基础。无论出现了持续形态还是反转形态,交易者自行决定是否进行交易,即便你决定不进行交易,你也必须自主决策。
\section{标准格式}
\begin{table}[!ht]
    \centering
    \caption{蜡烛图形态标准格式}
    \begin{tabular}{|r|r|r|r|r|r|r|r|}
        \hline
        \textbf{形态名称}:                & \multicolumn{2}{r}{相同低价形态+} & ~                                 & ~     & ~     & \textbf{类型}: & R+            \\ \hline
        \textbf{趋势要求}:                & 是                           & \multicolumn{2}{r|}{\textbf{确认}:} & 不需要   & ~     & ~            & ~             \\ \hline
        \textbf{形态之间平均间隔天数(MDaysBP)}: & 590                         & ~                                 & ~     & ~     & ~            & ~     & ~     \\ \hline
        \multicolumn{8}{|c|}{形态统计来自 7275 致常见流通股票基于 1460 万个交易日中的数据}                                                                                     \\ \hline
        \textbf{间隔(日)}                & 1                           & 2                                 & 3     & 4     & 5            & 6     & 7     \\ \hline
        \textbf{获利比例(\%)}             & 69                          & 64                                & 62    & 61    & 60           & 59    & 58    \\ \hline
        \textbf{平均收益(\%)}             & 3.63                        & 4.71                              & 5.42  & 5.98  & 6.64         & 6.98  & 7.37  \\ \hline
        \textbf{亏损比例(\%)}             & 31                          & 36                                & 38    & 39    & 40           & 41    & 41    \\ \hline
        \textbf{平均亏损(\%)}             & -2.6                        & -3.42                             & -3.92 & -4.39 & -4.75        & -5.13 & -5.48 \\ \hline
        \textbf{净收益/净亏损 }             & 1.23                        & 1.43                              & 1.55  & 1.65  & 1.77         & 1.79  & 1.82  \\ \hline
    \end{tabular}
\end{table}

蜡烛图形态的名称:后面的“+”代表看涨形态,“-”代表看跌形态。类型:反转形态,符号为“R”,意指 reversal patterns。持续形态,符号为“C”,意指 continuation patterns。趋势要求分为两类:是、否。是否需要确认分为三种:必须、推荐、不需要。

我们将蜡烛图形态大致分为两类。一类形态单独出现时的成功率极高,这就是不需要确认的一类;另一类蜡烛图形态单独出现时的成功率较低,这就是需要确认的那一类。实际上在交易中还会出现一类蜡烛图形态,因为有些蜡烛图形态原先的成功率极高,但随着市场的演变,其成功率开始下降。这一类被我们列为推荐类,即经过确认后的成功率较为可靠。
\section{单日反转形态}
\subsection{锤子线和上吊线形态}
\subsubsection*{形态介绍}

\figures{fig3-1}{锤子线形态(又称锤头线)和上吊线形态都是由单一的蜡烛图组成的。它们都有较长的下影线,实体部分相对较小,并主要集中在当日交易价格区间的上半部分。}


\begin{table}[!ht]
    \centering
    \caption{蜡烛图形态标准格式}
    \begin{tabular}{|r|r|r|r|r|r|r|r|}
        \hline
        \textbf{形态名称}:                & \multicolumn{2}{r}{锤子线+} & ~                                 & ~     & ~     & \textbf{类型}: & R+            \\ \hline
        \textbf{趋势要求}:                & 是                        & \multicolumn{2}{r|}{\textbf{确认}:} & 必须    & ~     & ~            & ~             \\ \hline
        \textbf{形态之间平均间隔天数(MDaysBP)}: & 284                      & \multicolumn{2}{r|}{非常频繁}         & ~     & ~     & ~            & ~             \\ \hline
        \multicolumn{8}{|c|}{形态统计来自 7275 致常见流通股票基于 1460 万个交易日中的数据}                                                                                  \\ \hline
        \textbf{间隔(日)}                & 1                        & 2                                 & 3     & 4     & 5            & 6     & 7     \\ \hline
        \textbf{获利比例(\%)}             & 41                       & 42                                & 44    & 44    & 45           & 46    & 47    \\ \hline
        \textbf{平均收益(\%)}             & 3.06                     & 3.96                              & 4.47  & 5.33  & 5.93         & 6.51  & 6.96  \\ \hline
        \textbf{亏损比例(\%)}             & 59                       & 57                                & 56    & 56    & 55           & 54    & 53    \\ \hline
        \textbf{平均亏损(\%)}             & -3.25                    & -4.05                             & -4.66 & -5.17 & -5.17        & -6.15 & -6.52 \\ \hline
        \textbf{净收益/净亏损 }             & -0.57                    & -0.54                             & -0.47 & -0.46 & -0.46        & -0.33 & -0.23 \\ \hline
    \end{tabular}
\end{table}
锤子线形态通常出现在下跌趋势中,形如其名,意味着夯实市场的底部。上吊线形态通常出现在上升趋势中,形如其名,好像一个人被高高地挂在市场的上方。

另外,和锤子线形态相似的蜡烛图是探水竿线。这个单词在日语中还有“沿着绳子向上爬”和“向上牵引”的意思。就像用手用力拉起船锚一样,当你换手的时候,船锚的上升趋势会稍有停顿。\textbf{如果是一条探水竿线,其下影线的长度至少应当达到其实体长度的 3 倍;如果是一条锤子线,其下影线的长度则至少应当是其实体长度的 2 倍。}
\subsubsection*{形态识别的标准}
\begin{enumerate}
    \item 蜡烛图有很小的实体,并且位于交易价格的顶部。
    \item 实体部分的颜色并不重要。
    \item 下影线的长度应该远大于实体部分的长度,通常是实体部分的 2-3 倍。
    \item 蜡烛图不应该有上影线,即使有也很短。
\end{enumerate}
\subsubsection*{蜡烛图形态背后的交易情境及市场心理分析}
\paragraph{锤子线形态} 市场位于下降趋势中,即市场处于熊市状态,开盘后随着卖盘的不断涌出,价格一度快速下跌。但是随着卖盘的减少,价格开始逐渐企稳回升,并最终收于当日价格的高位附近,市场中空方的力量得到释放。在这种情况下,市场中的悲观情绪随之减少,交易者不愿再继续持有看跌的头寸。如果当日的收盘价高于开盘价,形成了带白色实体的锤子线,那么市场状况将对多方更为有利。我们可以用第二天是否有更高的开盘价和更高的收盘价来验证市场下跌趋势的改变。

\paragraph{上吊线形态}对于上吊线来说,市场位于上升趋势中,即市场处于牛市状态。开盘后,交易价格一直远在开盘价之下运动,最后在高点附近收盘,形成了很长的下影线,这样我们就会看到上吊线形态,下影线形成的原因就在于市场中交易者开始做空。如果第二天开盘价低于前一天的收盘价,多方就会找机会卖出头寸。根据史蒂夫·尼森的研究,我们知道确认上吊线为下跌信号的标志为,当天上吊线的实体是黑色的(即上吊线是阴线),其次第二天开盘价低于前一天的收盘价。
\subsubsection*{形态的灵活性}
极长的下影线、没有上影线、很短的实体部分(几乎形成了十字线)、此前急剧的单边趋势、蜡烛实体的颜色,这些都可以增加锤子线形态和上吊线形态所代表的当前市场趋势将开始反转的可能性。如果上述特征都出现在了锤子线形态中,我们也可以把锤子线称为探水竿线。通常,探水竿形态比锤子线形态更具看涨特征。

上吊线和锤子线实体部分的颜色也增加了趋势预测的力度。黑色上吊线比白色上吊线更具有看跌倾向,同样,白色锤子线也比黑色锤子线更具有看涨倾向。

对于像锤子线和上吊线,这种由单一蜡烛图组成的蜡烛图形态,等待市场确认信号的出现,是判断市场是否开始反转的关键。通常,我们利用次日的开盘价进行确认,为了保险起见也可以等到次日收盘价的出现。例如,假定某一交易日出现了锤子线,那么如果次日的收盘价高于前一日的收盘价,那么市场开始转入上升趋势的可能性将大大提高。

此外,下影线的长度至少应该是实体部分长度的2倍,上影线的长度应该在价格变化幅度的 5\%-10\% 之间,锤子线形态实体部分的底部应该低于此前趋势的低点,上吊线形态实体部分的高点应该高于此前趋势的高点。
\subsection*{执带线形态}
\figures{fig3-4}{执带线形态,也可以称为“光头开盘蜡烛图或光脚开盘蜡烛图”。看涨执带线形态是出现在下跌趋势中的白色光脚开盘蜡烛图(光脚阳线)。首先市场低开于前一天收盘价之下,多方的力量迅速得到凝聚,市场开始一路走高,最后市场小幅回调,以接近当日最高价的价格收盘。看跌执带线形态(是出现在上升趋势中的黑色光头开盘蜡烛图(光头阴线)。与前一种形态不同的是,首先市场以高于前一天收盘价的价格开市,然后一路下跌,最后市场小幅回调,以接近最低价的价格收盘。执带线形态的实体越长,说明市场反转的可能性越大。}

\subsubsection*{形态识别的标准}
\begin{enumerate}
    \item 执带线形态中的蜡烛图的一端没有影线。
    \item 对于白色看涨执带线形态来说,开盘价就是最低价,没有下影线。
    \item 对于黑色看跌执带线形态来说,开盘价就是最高价,没有上影线。
\end{enumerate}
\subsubsection*{蜡烛图形态背后的交易情境及市场心理分析}
市场在确定的趋势中运行,突然出现一个价格跳空缺口,该缺口的跳空方向同市场当时的发展趋势相同,这种情况在技术分析中也被称为衰竭跳空。从这个跳空缺口开始,市场就改变方向,原先的趋势方向不再延续,这一切促使场内的投资者开始重新考虑头寸的安排,他们或者获利了结,或者积极买入,这些行为更进一步加剧了市场的反转可能性。
\section{双日反转形态}
\subsection{吞没形态}
吞没形态是由两根蜡烛图组成的,两根线的实体颜色相反,第二根蜡烛图的实体将第一天的实体部分完全吞没,在这种形态中不必考虑上下影线的作用。由于从形态上说,第二天的实体部分完全包含了第一天的实体,所以我们也可以把这种形态称为抱线形态(daki)。如果这种形态出现在市场的顶部,或者是上升趋势中,我们认为它反映市场中交易者的心态正在出现变化,投资者更倾向于做空。

在吞没形态中,第一天的实体很短,而第二天的实体很长,这就说明第二天市场中价格波动更为剧烈,市场此前的趋势将很有可能结束。如果在一大段上升趋势后,出现了看跌吞没形态,这就说明大多数的多头已经入市交易,现在市场可能会转入下跌,因为没有足够的做多资金重新入市进一步推动市场继续上行。
\subsubsection*{形态识别的标准}
\begin{enumerate}
    \item 市场此前的趋势必须明确。
    \item  第二天的实体部分必须完全吞没前一天的实体。但是,这并不是说两个实体的顶部或底部不能相等,它只是意味着两个实体的顶部和底部不能都相等。
    \item  第一根实体的颜色应该能反映出前期市场的发展趋势:黑色(阴线)为下跌趋势,白色(阳线)为上升趋势。
    \item  在吞没形态中,第二根蜡烛图实体部分的颜色应该和第一根的相反。
\end{enumerate}
\subsubsection*{蜡烛图形态背后的交易情境及市场心理分析}
\paragraph{看跌吞没形态} 市场处在上升趋势中,出现了一根带很小实体的白色蜡烛图(阳线),同时成交量不大。次日开盘价创出新高,然后迅速回落。卖盘不断涌出,成交量也持续放大,最后收于前一日的开盘价以下。市场中看涨的情绪受到了打击,如果下一日(第三日)的市场价格仍然走低,那么上升趋势将可能发生反转。

看涨吞没形态与之类似,只是方向相反。
\subsubsection*{形态的灵活性}
在吞没形态中,如果第二天蜡烛图的实体部分不仅吞没了第一天蜡烛图的实体,还吞没了它的影线,那么我们判定反转趋势出现的成功概率就会大大增加。

第一天蜡烛图的颜色可以反映此前市场的趋势。在上升趋势中,第一天应该是一条阳线,在下跌趋势中则为阴线。第二天,即吞没日的蜡烛图颜色应该同第一天相反。

吞没意味着第一天K线的实体部分必须小于第二天出现的 K 线的实体,如果第一天的实体只有第二天实体的 70\%,或者更小,那么趋势反转的可能性将大大增加。
\subsubsection*{形态的简化}
看涨吞没形态可以简化为纸伞蜡烛图或锤子线形态,反映市场即将出现转折。如果第一天蜡烛图的实体很小,看跌吞没形态可以简化为类似启明星或墓碑十字线的形态。将看跌吞没形态和看涨吞没形态简化为单一的蜡烛图并不会改变它们看涨或看跌的倾向。
\subsection{孕线形态}
\begin{table}[!ht]
    \centering
    \caption{孕线看涨形态}
    \begin{tabular}{|r|r|r|r|r|r|r|r|}
        \hline
        \textbf{形态名称}:                & \multicolumn{2}{r}{孕线+} & ~                                 & ~     & ~     & \textbf{类型}: & R+            \\ \hline
        \textbf{趋势要求}:                & 是                       & \multicolumn{2}{r|}{\textbf{确认}:} & 不需要   & ~     & ~            & ~             \\ \hline
        \textbf{形态之间平均间隔天数(MDaysBP)}: & 69                      & \multicolumn{2}{r|}{非常频繁}         & ~     & ~     & ~            & ~             \\ \hline
        \multicolumn{8}{|c|}{形态统计来自 7275 致常见流通股票基于 1460 万个交易日中的数据}                                                                                 \\ \hline
        \textbf{间隔(日)}                & 1                       & 2                                 & 3     & 4     & 5            & 6     & 7     \\ \hline
        \textbf{获利比例(\%)}             & 41                      & 42                                & 44    & 44    & 45           & 46    & 47    \\ \hline
        \textbf{平均收益(\%)}             & 3.06                    & 3.96                              & 4.47  & 5.33  & 5.93         & 6.51  & 6.96  \\ \hline
        \textbf{亏损比例(\%)}             & 59                      & 57                                & 56    & 56    & 55           & 54    & 53    \\ \hline
        \textbf{平均亏损(\%)}             & -3.25                   & -4.05                             & -4.66 & -5.17 & -5.17        & -6.15 & -6.52 \\ \hline
        \textbf{净收益/净亏损 }             & -0.57                   & -0.54                             & -0.47 & -0.46 & -0.46        & -0.33 & -0.23 \\ \hline
    \end{tabular}
\end{table}
孕线形态是由与吞没形态完全相反的两根蜡烛图组成的。
\subsubsection*{形态识别的标准}
\begin{enumerate}
    \item 在市场趋势图中出现了大阴线或大阳线。
    \item 第一天长蜡烛的颜色并不重要,但是可以利用它判定出市场的趋势。
    \item 在大阴或大阳后的交易日,它的实体部分应该完全包含在前一天的实体范围内。同吞没形态一样,两根蜡烛图实体顶部或底部可以相同,但是顶部或底部不能同时相同。
    \item 在孕线形态中,第二根蜡烛图实体部分的颜色应该和前一根的相反。
\end{enumerate}
\subsubsection*{蜡烛图形态背后的交易情境及市场心理分析}
\paragraph{看涨孕线形态}
市场一直处于下降趋势中,这时一根长的黑色蜡烛图(大阴线)的出现坚定了空方的信心,当日的成交量不太大也不太小。第二天的开盘价高于前一天的收盘价,空头感到震惊,开始转变对市场的看法,一些倾向于短线操作的空头开始平仓,进一步推动市场价格上扬,由于后期进场的交易者认为该交易日是弥补前一日踏空的好机会,所以市场价格逐步温和升高。成交量也较前一交易日有所放大,这表明市场短期趋势将出现改变。如果第三天市场价格继续走高,说明趋势的反转得到了确认。
\paragraph{看跌孕线形态}
市场一直处于上升趋势中,这时一根长的白色蜡烛图(大阳线)的出现坚定了多方的信心,当日的成交量也随价格的上涨有所放大。第二天市场低开(以低于前一天收盘价的价格开盘),全天价格在一个很窄的范围内波动,收盘价更低,但是实体部分并未超过前一天的实体范围。鉴于这种突然恶化的趋势,一部分交易者开始怀疑市场能否维持上涨趋势,特别当第二天成交量开始下降时,更能说明市场趋势可能面临变化。如果第三天市场价格继续走低,说明此前上升趋势的反转得到了确认。
\subsection{十字孕线形态}
孕线形态是由一根长实体的蜡烛图和其后一根较短实体的蜡烛图组成的。该形态的重要性取决于两根蜡烛实体部分的相对大小。我们对十字蜡烛图的解释,它说明在该交易日内开盘价同收盘价接近相等,反映出市场处于一种犹豫不决的状态中。因此,我们可以得出这样一个结论:在大阴或大阳日后出现较小实体的蜡烛图,说明市场中多空双方的博弈并未能分出结果,没有一方占有明显的优势。市场中的犹豫不决和不确定性越高,趋势发生改变的可能性就越大。如果第二天的蜡烛图实体变成了十字线,孕线形态就会演变成十字孕线形态,也被称为孕线十字形态。十字孕线形态比单纯的孕线形态具有更强的反转倾向。
\figures{fig3-19}{看涨十字孕线形态}
\subsubsection*{形态识别的标准}
\begin{enumerate}
    \item 在市场趋势明确的情况下出现了大阴线或大阳线。
    \item 第二天出现了十字线(开盘价与收盘价相等)。
    \item 第二天出现的十字线没有超过第一天大阴线或大阳线的范围。
\end{enumerate}
\subsubsection*{蜡烛图形态背后的交易情境及市场心理分析}
市场原本在已经确定的趋势中运行,可是突然在某个交易日内,市场价格开始出现较大的波动,但是价格的波动并没有超过前一交易日的价格范围,更糟糕的是,收盘时价格被打压至开盘价,同时当日的成交量有所放大。这一切都说明市场内的交易者未能对市场下一步的走势达成共识,一个重要的反转信号就此出现。
\subsubsection*{形态的灵活性}
大阴或大阳可以反映市场的此前趋势。出现十字线的交易日内,当且仅当前一段交易期内没有大量出现十字蜡烛图,只有在这种前提下,开盘价和收盘价之间的价格差异才允许有 2\%-3\% 的差异,我们还将其认定为十字孕线形态。
\subsubsection*{相关的形态}
十字孕线形态很可能是上升三法形态或下降三法形态的开端,这要看后面几个交易日的价格走势情况。上升三法形态和下降三法形态是持续形态中的两种形态,这本身就是和十字孕线形态所表达的反转倾向相反的地方。也就是说,在出现十字孕线时,其本身具有反转倾向,但我们要根据随后交易日的具体走势判断,是否会形成具有持续倾向上升三法形态或下降三法形态,这一点请务必留意。
\subsection{倒锤子线形态和流星线形态}
倒锤子线形态也被称为倒锤头形态,是一种用来判断市场是否见底的形态。这种形态一般出现在下跌趋势中,它的出现代表着市场趋势可能会发生反转。与其他一些由单一蜡烛图或双蜡烛图组成的形态一样,如果要利用倒锤子线来判断市场的下跌是否终结,必须等待市场给出的其他确认信号,对于倒锤子线来说,我们必须等待市场给出其他的做多信号,如果次日市场以高于倒锤子线的实体最高价的价位开盘,这就是一个确定信号。由于倒锤头蜡烛图的收盘价接近当日的最低价,次日市场却一直走高,这就说明市场做多的力量较强,这样的确认信号就更为可靠。
\figures{fig3-22}{倒锤子线形态}
流星线形态也是由单一的蜡烛图组成的,利用它可以判断市场上升趋势是否结束,但它并不是最重要的反转信号。流星线从图形上看同倒锤子线一样,所不同的是流星线形态只能出现在市场的顶部。开盘后,多方的做多能量迅速被消耗,最后市场在全日最低点收盘。流星线的实体同前一日的蜡烛图实体之间形成了一个小的向上跳空缺口,因此我们在学习流星线形态的时候必须关注前一日的市场变化。如果要利用流星线作为反转指标,最好与前一日的蜡烛图形态一起共同考量。
\subsubsection*{形态识别的标准}
\paragraph{倒锤子线形态}
\begin{enumerate}
    \item 小的实体部分出现在当日价格范围的低位。
    \item 在下跌趋势中,图形中不必出现向下的跳空缺口,只要整体形态向下就可以判定为倒锤子线形态。
    \item 上影线的长度通常不超过实体部分长度的2倍。
    \item 蜡烛图上不存在下影线。
\end{enumerate}
\paragraph{流星线形态}
\begin{enumerate}
    \item 在向上的市场趋势中,开盘时出现一个向上的跳空缺口。
    \item 小的实体部分出现在全天价格范围的低位。
    \item 上影线的长度至少是实体部分长度的3倍。
    \item 蜡烛图上不存在下影线。
\end{enumerate}
\subsubsection*{蜡烛图形态背后的交易情境及市场心理分析}
\paragraph{倒锤子线形态} 市场原本在确定的下降趋势中运行,可是突然在某个交易日内,市场的开盘价同前一日的收盘价之间形成了一个向下的跳空缺口。市场上攻失败,最终报收于较低的价格。同锤子线形态和上吊线形态一样,次日的市场走势是判断市场是否能成功反转的关键。如果次日市场的开盘价高于倒锤子线实体最高价,说明潜在的反转可能性使得短线空头开始转而做多,这种行为会进一步刺激价格继续上涨。倒锤子线日很容易演变成看涨启明星形态中中间的那一天,我们会在本章稍后部分为读者详细讲解启明星形态。
\paragraph{流星线形态} 在上升趋势中,市场出现了一个向上的跳空缺口,价格创出新高,但是上升的趋势没有得到支持,价格开始回落,收盘时只是稍高于开盘价。这种向上的跳空开盘,和随后下跌的走势只能被认定为是看跌形态。毫无疑问,这种走势必然会刺激多方获利了结,落袋为安。
\subsection{刺透线形态}
刺透线形态是一种判断市场是否已经形成底部的重要标志,这种形态通常出现在下跌趋势中,由两根蜡烛图或两个交易日的价格变动组成。第一日是一根阴线,意味市场处于下跌趋势中。第二日是一根大阳线,开盘价即创出新低,然后市场价格一路走高,最终以阳线收盘,并收盘在前一日阴线的中间位置之上。在日语中,kirikomi 的意思是“消减”和“折返”。
\figures{fig3-27}{刺透线形态。在下降趋势中出现的大阴线说明市场仍处于下跌趋势之中。第二天市场向下跳空开盘更加证明了这一点,但随后市场开始走高,并且当日的收盘价高于前一天大阴线的中间位置。这种走势提示了市场可能将形成底部,空头们继续做空的信心受到打击,一些短线空头们开始获利平仓,这进一步促进了市场筑底。}
\subsubsection*{形态识别的标准}
\begin{enumerate}
    \item 刺透线形态出现的第一日是一根大阴线,说明市场仍处在下跌趋势之中。
    \item 第二日是一根大阳线,并且它的开盘价要低于前一日的最低价(而不是收盘价)。
    \item 第二日的收盘价应该高于前一日大阴线实体部分的中间位置。
\end{enumerate}
\subsubsection*{形态的灵活性}
第二日阳线实体应该达到前一天阴线长度的一半(即收盘在前日阴线实体部分的一半以上)。如果这一前提未能满足,交易者应该耐心等待更多的做多信号出现。刺透线形态不存在其他的变化形式,刺透线的实体必须达到前一天阴线实体长度的一半。同刺透线形态相似的还有切入线形态、待入线形态和插入线形态,正因为这些形态,刺透线的定义必须非常严格。这三种形态与刺透线形态极为类似,但这三种形态中,随后的交易日内市场上涨的动能不足,多头的力量不够强,无法改变原先的下跌趋势,反而由于上攻失败消耗了多头的力量和信心,因此它们属于看跌持续形态。

在刺透线形态中,阳线收盘价所覆盖阴线部分越多,市场出现转势的可能性就越大。如果阳线的实体整个超过了阴线,那么刺透线形态就演变成了看涨吞没形态。

\important{请牢记这条法则:刺透线形态是由一根大阴线和一根大阳线组成的,第二天阳线的收盘价必须要高于第一天阴线实体部分的中间位置。}
\subsubsection*{相关的形态}
同刺透线相似的形态还有切入线形态、待入线形态和插入线形态,但是这三种形态与刺透线形态不同,刺透线形态反映的是市场趋势即将发生变化,而另外三种形态更多地被认为是下跌趋势继续的信号。看涨吞没形态可以看成是刺透线形态的扩展形态,具有更强的反转倾向。
\subsection*{乌云盖顶形态}
乌云盖顶形态同刺透线形态正好相反。这种形态通常出现在市场的上升趋势中,它是一种看跌反转形态。第一日的阳线表明市场仍处于上升趋势之中,次日市场先是向上跳空开盘,随后逐步下探,最后在前一天阳线的实体中间位置以下收盘。请大家注意,这又是一种利用市场高低价进行定义的蜡烛图形态。第二日阴线的低点应低于第一日阳线的中心点。

同刺透线形态相似,开盘乌云盖顶形态也表明市场趋势即将发生反转,所不同的是对交易者心态的影响,因为市场首先高开,随后却一路走低,这说明市场上涨无力,很可能转入下跌。该形态不存在其他的变化。在日语中,kabuse 的意思是“掩盖起来”和“包裹住”。
\subsubsection*{形态识别的标准}
\begin{enumerate}
    \item 乌云盖顶形态的第一根蜡烛图是一根大阳线,这说明市场仍处在上升趋势中。
    \item 第二天是一根大阴线,它的开盘价要高于前一天的最高价(注意是最高价,不是收盘价)。
    \item 第二天的(阴线)收盘价应该低于前一天大阳线实体的中心点。
\end{enumerate}
\subsection{十字星形态}
\subsubsection*{形态识别的标准}
\begin{enumerate}
    \item 第一日是大阳线或大阴线。
    \item 第二日的价格跳空是顺着市场此前的趋势方向。
    \item 第二日的市场图形是十字线。
    \item 十字线的上下影线不能太长,特别是对看涨反转信号来说。
\end{enumerate}
\figures{fig3-33}{十字星形态是市场趋势即将发生变化的预警信号。第一根蜡烛图的实体部分很长,代表着此前市场趋势。在下降趋势中应该形成一个黑色实体,而在上升趋势中则形成白色实体。第二天市场沿着原趋势运动方向出现一个跳空缺口,然后价格回落到开盘价附近(或刚好等于开盘价)形成收盘价。市场原趋势的减弱立刻引起交易者的关注。}
\subsubsection*{蜡烛图形态背后的交易情境及市场心理分析}
以看跌十字星形态为例,说明十字星形态背后的市场心理。市场原本在已经确定的上升趋势中运行,一根大阳线的出现更确定了市场的既有趋势。第二天市场向上跳空开盘,但是全天价格波动不大,收盘时价格又回落到开盘价(或开盘价附近)。这种价格变动实际上消耗了市场多头(或者空头)的信心,投资者会重新考虑应该持有何种头寸,这种市场中犹豫的情绪,在图表上的表现就是十字线形态。如果再下一日市场以较低的价格开盘,说明市场趋势的反转过程已经开始了。
\subsubsection*{形态的灵活性}
如果第二天的跳空缺口在前一日的影线之下(下跌趋势中)或之上(上升趋势中),即次日开盘价低于前日最低价(下跌趋势),高于前日最高价(上升趋势),这样的信号就增大了趋势改变的可能性。形态中第一天蜡烛图实体的颜色也应该与以前的市场趋势相符。
\subsection{约会线形态}
\figures{fig3-38}{约会线形态,有时也被称为相逢线形态,此形态由两根颜色相反且具有相同收盘价的蜡烛图组成。}
\paragraph{看涨约会线形态} 该形态通常出现在下跌趋势中。该形态的第一根蜡烛图是一根大阴线,第二天市场向下跳空,然后低开高走,一路上扬,最后收盘于前日的收盘价处。从形态上来看,看涨约会线形态同刺透线形态有些类似,两者的不同之处在于第二天的反弹力度。约会线形态的反弹力度要小于刺透线形态,在刺透线形态中,第二天收盘价向上超越了前一天蜡烛图实体的中心点,而约会线形态中的第二天收盘价只涨到第一天的收盘价。另外,请注意,不要将看涨约会线形态同待入线形态混淆。

\paragraph{看跌约会线形态} 对于看跌约会线形态来说,与它类似的形态是乌云盖顶形态。在看跌约会线形态中,第二天市场向上跳空高开,创出新高,然后一路下行,最后收市于第一天的收盘价处,而乌云盖顶形态的第二根蜡烛图向下超越了前一根蜡烛图实体的中心点。
\subsubsection*{形态识别的标准}
\begin{enumerate}
    \item 两根蜡烛图都要具有实体部分,并代表各自的市场趋势。
    \item 第一根蜡烛图的颜色应与原趋势相符,下跌趋势中第一根蜡烛图应为黑色,上升趋势中第一根蜡烛图应为白色。
    \item 第二根蜡烛图的颜色与第一根蜡烛图颜色相反。
    \item 两日的收盘价相同。
    \item 这两根蜡烛图的实体应较长,即大阴线或大阳线。
\end{enumerate}
\subsubsection*{形态的灵活性}
在约会线形态中,两天的蜡烛图都应该有比较长的实体。然而在通常情况下,第二天的实体比第一天的要小,这种变化不会改变形态的意义,但无论哪种情况,我们都推荐对该形态进行确认。当然,如果两根蜡烛图都是我们在前一章讲过的光头或光脚收盘蜡烛图,那么形态的反转意味将更为强烈。
\subsection{信鸽形态 \autoref{fig3-43}}
\figures{fig3-43}{信鸽形态和孕线形态十分相似,它们都由两根蜡烛图组成,前一根蜡烛图吞没后一根蜡烛图。所不同的是,在孕线形态中,两根蜡烛图的颜色是相反的,而在信鸽形态中,两根蜡烛图的颜色保持一致,都是黑色的。}
\subsubsection*{形态识别的标准}
\begin{enumerate}
    \item 在下跌趋势中出现了一根大阴线。
    \item 小阴线(第二根阴线)完全被第一根阴线吞没。
\end{enumerate}

市场已经在下降的趋势中运行,第一天,大阴线的出现更证明了这种趋势。第二天,市场高开,盘中价格在前一天形成的价格区域中波动,最后以较低的价格收盘。在考虑第一天和它之前的市场走势后,我们认为第二天虽然价格仍在下降,但是反映出这种下跌的力度在减弱,它表明市场可能存在反转机会。
\subsection{俯冲之鹰 \autoref{fig3-46}}
\figures{fig3-46}{俯冲之鹰是一个两日看跌反转形态。建立这一形态是为了对看涨的信鸽形态提供一个补充。}
\subsubsection*{形态识别的标准}
\begin{enumerate}
    \item 在上涨趋势中形成一根大阳线。
    \item 两根蜡烛图的实体部分必须是白色的。
    \item 第二天蜡烛图的实体必须完全被第一天蜡烛图的实体所吞没。
    \item 这两根蜡烛图的实体应较长。
\end{enumerate}

俯冲之鹰形态的第一根蜡烛图是大阳线。第一根蜡烛图的中心点必须在 10 日移动平均线的上方,这意味着市场已经处于上涨趋势中。大阳线进一步强化了早已存在的看涨心态。第二日,开盘价较低,交易不太活跃,最终收盘于接近当天最高价附近。如果下一日(第三日)的开盘价低于前一日的收盘价,且第三日的收盘价低于第一日的收盘价,那么你就可以确认这一形态。

俯冲之鹰的两根蜡烛图的实体部分必须较“大”。蜡烛图的实体部分是指开盘价与收盘价之间的部分。在这一形态之中,“大”实体指实体部分在最高价与最低价的价格范围内所占比例超过 50\%。不要将长实体(即大阴或大阳)这一要求与一天中价格波动范围大的要求混为一谈。根据定义,俯冲之鹰形态的两根蜡烛图的影线应该相对较短。

在实际使用时,我们推荐对这一形态进行确认。
\subsection{相同低价形态}
\figures{fig3-49}{市场一直处在下跌的趋势中,第一天出现的大阴线更是证明了市场继续下跌的可能性,次日虽然市场跳空高开,但是这种上扬趋势未能延续,价格回落并最终收盘于前一日的低点。在相同低价形态中,两根蜡烛图的实体底部(收盘价)是相同的。这个形态预示市场底部已经确立,即使在未来几天再次试图创出新低,也仅仅是一种短期的试探性行为(例如市场中的毛刺,刺探出新低,但很少收出新低),而不能说明下跌趋势的延续。我们可以利用该形态找出市场的支撑位。}
\subsubsection*{形态识别的标准}
\begin{enumerate}
    \item 第一天出现一根大阴线。
    \item 第二天也是一根阴线,它的收盘价和前一天的收盘价相同。
\end{enumerate}

市场已经在下跌的趋势中运行了一段时间,第一天的大阴线再次证明了这种趋势。次日,市场高开,盘中的交易价格可能高于开盘价(这时就会出现上影线),但是收盘时市场仍然保持下跌,并且收盘价和前一天收盘价相同。相同低价形态是一种经典的短线支撑形态,通常会立刻引起短线空头的警觉。这些做空的投资者,一般会比较谨慎地持有空头头寸,一旦出现任何风吹草动,他们就会立即平仓,减少风险。所以,相同低价形态会使这些短线空头迅速平仓。

这种形态的有趣之处在于,虽然市场没有达成共识,但是心理作用促成市场最后在相同的价位收盘。
\subsection{相同高价形态}
\figures{fig3-52}{相同高价形态是一种两日看跌反转形态。建立这一形态是为了对相同低价形态进行补充。}
\subsubsection*{形态识别的标准}
\begin{enumerate}
    \item 在上涨趋势中,第一天出现一根大阳线。
    \item 第二天的收盘价与第一天的收盘价相同。
    \item 两天的蜡烛图都没有上影线或上影线很短。
\end{enumerate}

请注意:如果第二天的收盘价与第一天的收盘价之间的差别在 1/1000 之内,我们就可以视为两天的收盘价相同。因此,如果第一天的收盘价是 20 美元,那么第二天的收盘价可以在 19.98 美元和 20.02 美元之间。
\subsection{反冲形态 \autoref{fig3-55}}
\figures{fig3-55}{在反冲形态中,两根蜡烛图之间存在一个跳空缺口。看涨反冲形态由一根黑色光头光脚蜡烛图和一根白色光头光脚蜡烛图组成,看跌反冲形态由一根白色光头光脚蜡烛图和一根黑色光头光脚蜡烛图组成。}
\subsubsection*{形态识别的标准}
\begin{enumerate}
    \item 第一天出现一根光头光脚蜡烛图,第二天又出现一根光头光脚蜡烛图,但颜色与第一天相反。
    \item 两根蜡烛图之间必须存在一个跳空缺口。
\end{enumerate}

市场已经在一定的趋势中运行,突然某一天开盘时出现了一个跳空缺口,而且市场价格没有再回到以前的价格区域内,仍持续上升(或下降),最终形成一个很大的跳空缺口。

\begin{tcolorbox}
    我觉得这个有点问题!!!
\end{tcolorbox}
\subsection{白色一兵形态}
\figures{fig3-61}{白色一兵形态是一种两日看涨反转形态。白色一兵是一种并列排列且颜色相反的(tasuki)蜡烛图形态,这种形态中当日开盘价高于前一日的收盘价且当日的收盘价高于前一日的最高价。}
\subsubsection*{形态识别的标准}
\begin{enumerate}
    \item 白色一兵形态的第一根蜡烛图是大阴线。
    \item 第二天是一根大阳线,其开盘价等于或高于前一天的收盘价,而其收盘价接近当天的最高价,并高于前一天的最高价。
\end{enumerate}

白色一兵形态的第一根蜡烛图是大阴线,第一天价格波动范围的中心点在 10 日移动平均线之下,这意味着下降趋势已经确立。大阴线强化了这种既有的下跌趋势。

第二天出现了一根大阳线,其开盘价等于或高于前一天的收盘价,最后以接近当天最高价收盘,并且超过前一天的最高价。

从市场中的交易心理来说,下降趋势已经被破坏。如果下一日的价格继续走高,那么下跌趋势就很可能出现反转。

白色一兵的两根蜡烛图都是大阴线或大阳线。当价格波动范围出现下面的情形时,就可以被认定为出现长蜡烛图:
\begin{itemize}
    \item 高于中心点的 1.5\%;
    \item 高于前 5 日最高价和最低价差价平均值的 0.75 倍,这是确定某一日蜡烛图线长度的不同方法。
\end{itemize}
价格波动范围是指某一日最高价与当日最低价之差。中心点是某一日最高价与最低价的中间值。

两根蜡烛图的实体部分也要比较长。蜡烛图的实体是指开盘价与收盘价之间的部分。长实体是指实体占最高价与最低价之间价格波动范围的 50\% 以上。
\subsection{一只黑乌鸦形态}
\figures{fig3-63}{一只黑乌鸦形态是一种两日看跌反转形态。一只黑乌鸦是一种并列排列且颜色相反的(tasuki)蜡烛图形态,这种形态中当日开盘价低于前一日的收盘价且当日收盘价低于前一日的最低价。}
\subsubsection*{形态识别的标准}
\begin{enumerate}
    \item 一只黑乌鸦形态的第一根蜡烛图是大阳线。
    \item 第二天是一根大阴线,其开盘价等于或低于前一日的收盘价,收于当天的最低价附近,最终收盘价低于前一天的最低价。
\end{enumerate}
\section{三日反转形态}
\subsection{启明星形态和黄昏星形态}
\figures{fig3-66}{启明星形态是看涨反转信号,又被称为黎明之星,顾名思义,它预示着市场价格将一路走高。此形态第一天是一根大阴线,第二天是一根向下跳空的小阴线,第三天是一根阳线,它将价格推进到第一天黑色实体所表示的价格波动范围内。理想的启明星形态是第二天的图形(即“星线”)应该和第一天的图形之间形成向下的跳空缺口,而第三天的阳线则应该在第二天的小阴线之间出现一个向上的跳空缺口。}

\subsubsection*{形态识别的标准}
\begin{enumerate}
    \item 第一天蜡烛图的颜色应与原趋势方向相符,即如果原趋势是上升的,黄昏星形态第一天则为大阳线,若原趋势为下跌的,启明星第一天则为大阴线。
    \item 第二天蜡烛图应出现跳空,而其颜色并不重要。
    \item 第三天蜡烛图的颜色同第一天相反。
    \item 第一天的蜡烛图和很多时候第三天的蜡烛图都是大阴线或大阳线。
\end{enumerate}

\paragraph{启明星形态} 市场原本在已经确定的下降趋势中运行,一根大阴线的出现更加确定了这种趋势,市场看起来好像要继续下跌。但第二天市场向下跳空开盘,全天价格波动不大。蜡烛图的小实体反映了交易者对市场趋势的未来发展犹豫不决。第三天,市场以高于第二天收盘价的价格开盘,价格一路上行,并最终收盘于当日价格的高位,一个市场趋势反转的重要信号就此形成。

\figures{fig3-67}{黄昏星形态又被称为黄昏之星,同启明星形态正好相反,它是看跌反转的信号。它出现在市场的上升趋势之中,预示着上升趋势即将结束。第一天是一根大阳线,第二天是一根向上跳空的星线(或小阴线),要记住,星线的实体与前一天蜡烛图的实体之间要有跳空。星线较小的实体是市场犹豫不决的第一个信号。第三天向下跳空,其收盘价更低,这一形态就被确立了。同启明星形态一样,黄昏星第二天的图形应该和第一天的图形之间形成跳空缺口,而第三天与第二天之间也要出现跳空缺口。}

\paragraph{黄昏星形态} 黄昏星形态的市场心理分析同启明星形态正好完全相反。

理想的启明星形态和黄昏星形态应该是第二天的“星线”同第一天的蜡烛图产生价格跳空缺口,第三天的蜡烛图同第二天的“星线”产生跳空缺口。第二个跳空缺口这个条件有时不一定被满足,这时我们可以根据具体的市场走势灵活处理。

如果第三天蜡烛图的收盘价格深深地进入了由第一天蜡烛图实体部分所形成的价格区域内,并且当日的成交量能够有效放大,这些都说明市场趋势改变的可能性将增大。
\subsection{十字启明星形态和十字黄昏星形态}
\paragraph{十字启明星形态} 市场处在下降趋势中,第一天是一根大阴线,第二天是一颗十字星。同前面我们讲过的启明星形态一样,第三天的走势将确认市场是否发生反转。启明星形态和十字启明星形态都是典型的市场反转信号,而且十字启明星形态的市场反转倾向比启明星形态更强。

\paragraph{十字黄昏星形态} 市场处于上升趋势中,十字星后紧随着一根大阴线,并且收盘价进入了第一天蜡烛图的实体范围内,这是一种明显的顶部反转信号。一般来说,黄昏星形态中的星线还有一小段实体,而在十字黄昏星形态中,星线演化成了十字星。黄昏十字星形态更重要,它具有更强的市场反转意义。十字黄昏星形态也被称为南方十字线形态。
\subsubsection*{形态识别的标准}
\begin{enumerate}
    \item 同许多反转形态一样,第一天蜡烛图的颜色应与市场此前的趋势方向相符。
    \item 第二天的图形必须是十字星(同时存在跳空缺口)。
    \item 第三天的蜡烛图同第一天蜡烛图的颜色相反。
\end{enumerate}
\figures{fig3-71}{十字启明星形态}
\subsection{弃婴形态}
\figures{fig3-76}{这种形态和十字启明星形态、十字黄昏星形态非常相似,但是如果仔细对照两类形态的图形,就会发现,两者其实有一个重要的差别。第二天所形成的十字星的上下影线与第一天和第三天的蜡烛图并不重合,从图形上看,有两个明显的跳空缺口,这使得十字星就好像一个被遗弃的婴儿。同样,我们看到在顶部有一颗被遗弃的十字星。请大家注意,在这种形态中,十字星(包括上下影线)同周围的几根蜡烛图之间一定要有完整的跳空缺口。当然这只是理论上的情况,在现实中这种形态是极少出现的。}
\subsubsection*{形态识别的标准}
\begin{enumerate}
    \item 第一天的蜡烛图颜色应与市场原趋势相符。
    \item 第二天的图形是一个十字星,其上影线或下影线与前一日的上影线或下影线之间存在着跳空缺口。
    \item 第三天的蜡烛图同第一天蜡烛图的颜色相反。
    \item 第三天的跳空缺口同第二天的跳空缺口方向相反,并且影线不能重叠。
\end{enumerate}
\subsection{三星形态}
它是由三根十字星线组成的,第二根十字星线代表着市场的转势。这种形态在现实中极少出现,但是一旦出现就应该引起我们的警觉。
\subsubsection*{形态识别的标准}
\begin{enumerate}
    \item 三天的蜡烛图都是十字星线。
    \item 其中第二天的星线同第一天和第三天的蜡烛图形成两个跳空缺口,跳空缺口方向相反。
\end{enumerate}
\subsection{向上跳空两只乌鸦形态 \autoref{fig3-86}}
\figures{fig3-86}{该形态仅出现在上升趋势中。同大多数的看跌反转形态一样,在该形态中,第一根蜡烛图是白色的,它说明市场当时还是处在上升趋势中。但是随后的两根蜡烛图都同第一天的蜡烛图产生向上的跳空缺口,而且这两根蜡烛图都是黑色的,看上去就好像出现了两只黑色的乌鸦。第三天(第二根阴线)高开低走,最后在第二天的收盘价之下收盘。虽然第三天收于第二天的收盘价之下,但是它依然与第一天的阳线的最高点之间有一个向上的跳空缺口。简单地说,就是第二根阴线吞没了第一根阴线。}
\subsubsection*{形态识别的标准}
\begin{enumerate}
    \item 第一天的大阳线说明市场仍处在上升趋势中。
    \item 随后的阴线同第一天的阳线之间要形成向上的跳空缺口。
    \item 第二根阴线与第一根阴线之间存在一个向上的跳空缺口,并且最后收于第一根阴线的实体部分之下。第二根阴线的实体部分完全吞没了第一根阴线。
    \item 第二根阴线的收盘价仍然高于第一天阳线的收盘价。
\end{enumerate}

同大多数的看跌反转形态一样,第一天的阳线通常表示市场当时处于上升趋势中。第二天市场高开,但是多方的力量遭到打压,最后高开低走,形成阴线。市场内并未因此而出现恐慌,因为第二天的收盘价依然高于第一天的收盘价,这说明上升趋势仍在延续。第三天市场仍然高开,可是盘中迅速回落,最后在第二天的收盘价之下报收,但是这时的成交价仍然高于第一天的收盘价。在这种情况下,多头的力量被极大地消耗,投资者的信心开始动摇。市场中连续两天的收盘价出现了下降,交易者开始怀疑市场能否继续走高。这一切都预示着市场可能会出现反转。

如果第三天阴线的开盘价未能略低于第二天的开盘价,而且第三天的阴线实体不能同第一天的实体形成跳空缺口,它将变成持续形态中的铺垫形态。铺垫形态是一种看涨持续形态。

\subsection{向下跳空两只兔子形态 \autoref{fig3-89}}
\subsubsection*{形态识别的标准}
\begin{enumerate}
    \item 在下降趋势中,这一形态的第一根蜡烛图是一根大阴线。
    \item 第二天是一根向下跳空的阳线。
    \item 第三天也是一根阳线,其开盘价低于前一阳线的开盘价,收盘价高于前一阳线的收盘价。
\end{enumerate}
\figures{fig3-89}{向下跳空两只兔子形态是一种三日看涨反转形态。向下跳空是指第二天阳线的实体部分与第一天阴线实体部分之间出现的跳空缺口。最后两根阳线代表两只蠢蠢欲动、准备跳出窝的兔子。注意:向下跳空两只兔子形态很少出现。}

向下跳空两只兔子形态的第一根蜡烛图是大阴线,其价格变动的中心点低于 10 日移动平均线,表明下降趋势已经确立。大阴线强化了市场中早已存在的看跌心态。

第二天跳空低开,但是,盘中价格上涨,收盘时形成阳线。下跌趋势没有因为这一天收于阳线而改变,因为阳线的收盘价仍然低于第一天的收盘价。第三天的开盘价虽然更低,但在这一天市场看涨气氛很浓,收盘价高于前一天的收盘价。市场中连续两天的收盘价出现了上升,交易者开始怀疑市场能否继续下跌。这一切都预示着市场可能会出现反转。

第三天的蜡烛图实体必须完全吞没第二天蜡烛图的实体。另外,第三天的最高价与最低价也必须完全吞没第二天的最高价与最低价。所有三天的蜡烛图必须具有长实体。请注意:这一形态还要求第一天蜡烛图与第二天蜡烛图实体部分的跳空缺口大于第一天最高价与最低价差别的 10\%。
\subsection{奇特三川底部形态 \autoref{fig3-92}}
\subsubsection*{形态识别的标准}
\begin{enumerate}
    \item 第一天是一根大阴线。
    \item 第二天是孕线形态,但是注意实体的颜色也是黑色。
    \item 第二根蜡烛图有很长的下影线,意味着当天创出了新低。
    \item 第三天是一根小阳线,收盘价低于第二天收盘价。
\end{enumerate}
\figures{fig3-92}{奇特三川底部形态同启明星形态比较相似。市场处在下降趋势中,一根大阴线的出现更是确定了趋势方向。第二天市场高开,盘中创出最近一段时期以来的新低,但在接近当天最高价处收盘,形成了一根有较长下影线同时实体较小的蜡烛图。第三天市场低开,但是没有低于第二天创出的新低,当日收盘价比第二天收盘价略低,这样第三天就形成了一根实体较小的阳线。}
\subsection{奇特三山顶部形态}
\subsubsection*{形态识别的标准}
\begin{enumerate}
    \item 出现在上升趋势中,第一天是一根大阳线。
    \item 第二天低开,盘中创出新高,但收盘时价格接近当天的最低价,因此形成了一根拥有很长上影线的小阳线。
    \item 第三天高开,但没有超过第二天的最高价。
    \item 第三天形成一根实体相对较小的阴线,其收盘价比第二天的收盘价高。
\end{enumerate}
\figures{fig3-95}{奇特三山顶部形态是一种三日看跌反转形态。它是奇特三川底部形态的相反形态。}

该形态要求第二天蜡烛图的实体部分不小于其最高价与最低价范围的 27\%。流星线形态对实体大小的要求也是如此。
\subsection{白色三兵形态 \autoref{fig3-98}}
\figures{fig3-98}{白色三兵形态,一般被简称为白三兵。它是由一系列大阳线组成的,这些阳线的收盘价逐步攀升。如果次日的蜡烛图在前一日实体中心点之上形成开盘价,那么形态的反转倾向就更强。这种逐级上升的形态具有很强的反转意义,它预示上升趋势的开始,下跌趋势戛然而止。}
\subsubsection*{形态识别的标准}
\begin{enumerate}
    \item 连续出现三根大阳线,收盘价逐渐提高。
    \item 每一根阳线都在前一日的实体内开盘。
    \item 每天都在当日的最高点或接近最高点处收盘。
\end{enumerate}

第二天、第三天的开盘价都在前一天的实体范围之内,但最好是在实体的中心点以上开盘。请大家记住,每天开盘的时候,在前日收盘价以下总会存在一些卖盘,即总有交易者在前一天的收盘价之下做空。这种形态为交易者展示了一个“健康的”上涨趋势,这种上涨趋势总是伴随着一些卖盘的出现,因此次日的开盘价总是低于前日的收盘价,但这些卖盘在上升趋势中被逐渐消化,丝毫不能影响上升趋势的延续,因此每日的收盘价都是上升的,这种情况就是我们常说的稳步上涨。
\subsection{三只黑乌鸦形态}
\subsubsection*{形态识别的标准}
\begin{enumerate}
    \item 连续出现三根大阴线。
    \item 连续三天,收盘价不断创出新低。
    \item 每天都在前一天的实体价格范围内产生开盘价。
    \item 每天的收盘价都是最低价,或者是接近最低价。
\end{enumerate}
\figures{fig3-101}{三只黑乌鸦形态是白色三兵形态的相反形态,只需要将图形倒转过来即可。该形态通常出现在上升趋势中,三根大阴线顺次排列,收盘价逐步下跌。在该形态中,开盘价虽然略高于前一天的收盘价,但是盘中价格不断走低,收盘价屡创新低。当这种市场情况重复发生三次以后,市场趋势将发生反转的信号就十分明显了。值得注意的是,这种趋势不会加速下降,其间会伴随着多次反弹。}

在三只黑乌鸦形态中,如果第一根蜡烛图(阴线)的实体部分在前一天阳线的最高价之下,该形态的看跌反转倾向就更强烈。
\subsection{三只乌鸦接力形态}
三只乌鸦接力形态是三只黑乌鸦形态的特例。其不同之处在于,第二天的开盘价等于或接近第一天的收盘价,而第三天的开盘价等于或接近第二天的收盘价。
\subsubsection*{形态识别的标准}
\begin{enumerate}
    \item 三根大阴线逐级下降。
    \item 每一天蜡烛图的开盘价都是前一天蜡烛图的收盘价,第一根蜡烛图可以例外。
\end{enumerate}

三只乌鸦接力形态实际上反映了市场的恐慌性抛售。每天的收盘价都成为次日市场价格的阻力位,市场中多头溃不成军,再也无力组织反击,市场价格最终暴跌不止。
\figures{fig3-106}{市场实例:三只乌鸦接力-形态}
\subsection{前进受阻形态}
\subsubsection*{形态识别的标准}
\begin{enumerate}
    \item 连续出现三根大阳线,收盘价逐渐提高。
    \item 每一根阳线都在前一日的实体内开盘。
    \item 第二天和第三天带有很长的上影线,说明上升趋势的力度在逐渐减弱。
\end{enumerate}
\figures{fig3-107}{前进受阻形态源于白色三兵形态,是白色三兵形态的演化形式。与白色三兵形态不同,它通常出现在上升趋势中,白色三兵形态通常出现在下降趋势中。另一不同点在于前进受阻形态的第二根和第三根蜡烛图都有很长的上影线,反映出上升趋势很难延续,上方阻力重重,市场在逐渐走弱。这是市场在大幅上涨后经常出现的状况。两根带很长上影线的小阳线表明市场内的投资者信心不足,开始产生犹豫情绪,特别是在市场经历了长时间的上涨后,这种心理反应就更明显。}
\subsection{下降受阻形态}
下降受阻形态是一种三日看涨反转形态。建立这一形态是为了对前进受阻形态进行补充。
\subsubsection*{形态识别的标准}
\begin{enumerate}
    \item 这一形态出现在下降趋势之中,第一天是一根大阴线。
    \item 后面两天也是阴线,每天的收盘价都在前一天收盘价之下。
    \item 后两天的蜡烛图还具有较长的下影线。
\end{enumerate}

首先,蜡烛实体的长度一天比一天短。其次,每一天的开盘价都在前一天的实体之内(在这一形态中没有实体之间的跳空)。最后,第二天和第三天蜡烛图的下影线很长。特别是,下影线的长度一定要占当日最高价与最低价波动范围的 40\% 以上。另外,在第二天和第三天的收盘价不断下降的同时,收盘价之间的差距在不断缩小。这意味着下降趋势的强度在减弱,空头应该小心应对这种情况。

下降受阻形态的第一根蜡烛图必须具有长实体。
\subsection{深思形态}
\paragraph{看跌深思形态} 看跌深思形态同样也源于白色三兵形态,是白色三兵形态的演化形式。它通常由两根大阳线和一根小阳线或星线组成,而且前两根大阳线创出这段时期的新高。如果第三天的星线与第二天的大阳线之间能够形成一个向上的跳空缺口,那么深思形态的反转意义就更强烈。这根星线显示出市场对于是否能继续上升存在分歧,多空双方争夺不休。这种犹豫显示市场正在进行深思,从而判明下一步的发展方向。另外,深思形态经过确认后,还可以转化为黄昏星形态。

\paragraph{看涨深思形态} 看涨深思形态是一种三日看涨反转形态。在下降趋势中,这一形态的前两天都出现了大阴线。建立这一形态是为了对看跌深思形态进行补充。

\figures{fig3-114}{看涨深思形态}

\subsubsection*{形态识别的标准}
看跌深思形态
\begin{enumerate}
    \item 形态的第一天和第二天是大阳线。
    \item 第三天,市场在前一天的收盘价附近开盘。
    \item 第三天是一根纺锤线,或者是一根星线。
\end{enumerate}

同前进受阻形态一样,深思形态也为我们展示了市场处于弱势状态,它表明市场在短期内将走弱,所不同的是深思形态在第三天才显露出颓势,而前进受阻形态从第二天开始就显示出了弱势。深思形态通常发生在市场持续上涨了一段时间后,说明市场无力继续上涨。同前进受阻形态一样,市场趋势的减弱很难被确定,大家在判断和确认深思形态时要格外小心。

\paragraph{看涨深思形态}
\begin{enumerate}
    \item 在下降趋势中,形态的第一天是一根大阴线。
    \item 第二天仍然是一根大阴线。
    \item 第三天是一根星线或相对较短的阴线,可能与前一天的黑色实体部分存在一个跳空缺口。
\end{enumerate}

在两天连续收阴线后,当前的下降趋势似乎确定无疑,空方肯定会十分满意。下降趋势的强度吸引了新的空头,而第三天的开盘价等于或低于前一天的收盘价。第三天仍然是一根阴线。在下降趋势中连续收三根阴线,空头很可能会心满意足。

但是,通过仔细分析我们就会发现,看涨深思形态显示出当前下降趋势正在发出走弱的信号。首先,第三天的价格波动幅度小于第二天的波动幅度。特别是,它小于第二天价格波动幅度的 75\%。其次,第三天蜡烛图的实体小于第二天蜡烛图的实体。这一形态要求它最好小于第二天实体大小的 50\%。最后,尽管第三天的黑色实体可能与第二天的黑色实体之间存在跳空缺口,但缺口小于第二天价格波动范围的 20\%。
\subsection{两只乌鸦形态}
\figures{fig3-119}{两只乌鸦形态只能预测顶部反转或者看跌反转形态。第一天的阳线支持了市场原有的上升趋势。第二天,市场高开低走,但仍留下一个向上跳空缺口。第三天,市场在第二天蜡烛图的实体部分内开盘,然后一路下滑,最后弥补了第二天的向上跳空缺口,突破到第一天的实体部分内。这样的图形让我们想起了乌云盖顶形态。如果把第二天和第三天的阴线合并为一根大阴线,我们会发现两只乌鸦形态可以演化为乌云盖顶形态。事实上,在该形态中缺口被快速回补,说明市场原有的上升趋势将出现反转。}
\subsubsection*{形态识别的标准}
\begin{enumerate}
    \item 第一天的大阳线表明了原有的市场趋势。
    \item 第二天是阴线,而且出现了向上的跳空缺口。
    \item 第三天也是阴线。
    \item 第三天的开盘价在第二天的实体内产生,最后收盘于第一天的实体内。
\end{enumerate}

市场本来处在上升趋势中,可是虽然出现了向上跳空缺口,但是市场未能继续走强,多方能量被削弱,最后价格下跌形成了阴线。第三天虽然高开,但是依然未能高于前一天(第二天)的开盘价,市场中看涨情绪被侵蚀,短线交易者开始卖空,最后市场在第一天蜡烛图的实体内收盘。向上跳空缺口仅持续一天就被快速回补,说明空头的能量很强,短线市场将走弱。
\subsection{两只兔子形态}
\figures{fig3-122}{两只兔子形态是一种三日看涨反转形态。最后两根阳线代表两只准备跳出巢穴的兔子。两只兔子形态与两只乌鸦形态恰好相反。}
\subsubsection*{形态识别的标准}
\begin{enumerate}
    \item 市场处于下降趋势中,这一形态的第一天是一根阴线。
    \item 第二天是一根向下跳空的阳线。
    \item 第三天也是一根阳线,其开盘价位于第二天蜡烛图的实体内,而收盘价在第一天蜡烛图的实体内。
\end{enumerate}
两只兔子形态的第一天是一根大阴线。其波动范围的中心点位于10日移动平均线之下,这意味着市场处于下跌趋势中。大阴线强化了早已存在的看跌气氛。

第二天跳空低开,但新低点没有坚持住,这一天实际上形成了一根阳线。由于这一天阳线的收盘价仍然位于第一天的收盘价之下,所以下跌趋势并没有发生改变。

第三天的开盘价位于第二天的实体内,收盘价位于第一天的实体内。第一天与第二天出现了跳空缺口,按照传统的技术分析方法,向下跳空缺口的出现意味着下跌趋势将会持续,但在这一形态中,跳空缺口被快速地恢复,这一现象说明下跌趋势延续的可能性极小,趋势出现反转的可能性较大。

请注意,每一天与第二天蜡烛图实体之间的缺口必须大于第一天最高价与最低价之差的 10\%。
\subsection{三内升和三内降形态 \autoref{fig3-125}}
三内升和三内降形态实际上是孕线形态的进一步演化,是它的确认形态。
\figures{fig3-125}{三内升和三内降形态前两天的蜡烛图形态和孕线形态一致,第三天的蜡烛图反映市场未来的走势是看涨还是看跌。三内升形态是看涨孕线形态的演化,在该形态中,第三天的蜡烛图是一根阳线,并且收盘在第二天的最高价之上。三内降形态是看跌孕线形态的演化,在该形态中,第三天的蜡烛图是一根阴线,并且收盘在第二天的最低价之下。}
\subsubsection*{形态识别的标准}
\begin{enumerate}
    \item 首先判定孕线形态。
    \item 在三内升形态中,第三天是阳线,并且收盘价更高;在三内降形态中,第三天是阴线,并且收盘价更低。
\end{enumerate}

作为孕线形态的确认形态,三内升和三内降形态的灵活性形式与孕线形态是一致的。第一天蜡烛图实体对第二天蜡烛图实体的吞没量,可以帮助我们判定这个反转信号的强弱。
\subsection{三外升和三外降形态}
三外升和三外降形态实际上是吞没形态的演化,是它的确认形态。在三外升和三外降形态中,第三天的蜡烛图反映市场未来的走势是看涨还是看跌。
\figures{fig3-131}{作为吞没形态的确认形态,三外升和三外降形态说明吞没形态成功地预示了市场趋势的反转。}
\subsection{南方三星形态 \autoref{fig3-135}}
\subsubsection*{形态识别的标准}
\begin{enumerate}
    \item 第一天是一根带很长下影线的大阴线(类似锤子线,但是下影线没有达到实体的 2 倍)。
    \item 第二天的图形类似于第一天,只是有所缩小,而且没有低于第一天的最低价。
    \item 第三天是一根小的黑色光头光脚蜡烛图,开盘价和收盘价都在第二天的价格区间内。
\end{enumerate}
\figures{fig3-135}{这一形态显示出下跌趋势的减缓。价格每天都在下跌,连续创新低。第一天的长下影线表明市场下跌时,逢低介入的买盘很积极,第二天市场又高开,虽然市场的收盘价没能上升,但是盘中的最低价已经较上一个交易日有所提高,第三天是一根黑色光头光脚蜡烛图,并且被包含在第二天的价格波动范围之内。}

市场已经在下降趋势中徘徊了一段时期,创出新低后,多头开始顽强抵抗,最终收盘价高于最低价,形成很长的下影线。这种市场情境引起了一些短线多头的兴趣。第二天市场高开,短线客纷纷入场抢筹码。但是这种短期的上升趋势并没有得到持续,市场再一次开始下跌,最后收盘时,出现了收盘价的下跌,但第二天的收盘价却高于第一天的最低价,这引起了空头的警觉。因此第三天的市场价格波动不大,最终形成了一根较小的黑色光头光脚蜡烛图。
\subsection{北方三星形态}
北方三星形态是一个三日看跌反转形态。它是一个与南方三星形态相反的形态。
\subsubsection*{形态识别的标准}
\begin{enumerate}
    \item 这一形态由三根阳线组成,第二天与第三天的最高价逐日降低,而最低价逐日提高。
    \item 北方三星形态的第一天是一根大阳线。第一天价格的中心点位于 10 日移动平均线之上,这表明市场处于上升趋势中。
    \item 第一天上影线的长度应该超过这一天价格范围的 40\%,而且应该没有下影线或只有很短的下影线。如果有下影线,那么其长度不能超过当日价格范围的 7.5\%。
    \item 第二天的开盘价低于第一天的收盘价,然后价格不断上升,最终的收盘价高于第一天的收盘价。第二天的最高价低于第一天的最高价,其上下影线的要求与第一天相同。
    \item 第三天是一根光头阳线,开盘价和收盘价都位于第二天的价格范围内。光头阳线是一根没有或只有很短上下影线的阳线。
\end{enumerate}
由于这一形态在定义上有很多要求,因此它的出现十分罕见。实际上,我们必须放宽对蜡烛图微小变化的接受程度,否则这种形态可能永远也不会出现。我们知道,光头线出现的概率本来就不大,再加上对前两天的影线要求,则这种形态出现的概率就更低了。我们没有改变对前两天影线的要求,而是改变了对第三天蜡烛图实体部分的要求。我们不再要求第三天是一根光头阳线,而只要求其实体部分超过价格波动范围的 50\%。一般来说,光头阳线要求其实体部分大于该日价格波动范围的 90\%。
\subsection{竖状三明治形态}
在看涨竖状三明治形态中,两根阴线实体夹着一根阳线实体。两根阴线的收盘价必须相等。这样我们就找到了一个价格支撑位,趋势反转的可能性就会较大。
\figures{fig3-141}{看涨竖状三明治形态}

看跌竖状三明治形态是一种三日看跌反转形态。它是与看涨竖状三明治形态相反的形态。
\subsubsection*{形态识别的标准}
\paragraph{看涨竖状三明治形态}
\begin{enumerate}
    \item 市场处于下降趋势中,一根阳线在一根阴线后出现,并且阳线的开盘价在前一天阴线的收盘价之上。
    \item 第三天又出现了一根阴线,并且这根阴线的收盘价和第一根阴线的收盘价相等。
\end{enumerate}

市场已经在下降趋势中运行,第一天阴线的出现更是证明了这种趋势。第二天,市场高开高走,在最高价或接近最高价处收盘,形成一根阳线。这种走势说明此前的下降趋势可能会发生变化,短线空头的信心开始减弱。第三天市场高开,空头受到打压,市场上开始出现平仓盘,但价格随后一路下探,市场无力再创新低,最终收盘于第一天收盘价处。至此,场中的交易者已经确认了当前价格的支撑位,市场开始企稳,其后的走势会证明这种判断。

如果第一天阴线的实体部分比第三天阴线的实体小很多,看涨竖状三明治形态就可以简化为类似倒锤子线的形态。如果第一天阴线的实体部分较小,同时第三天阴线的实体长度是它的两三倍,那么竖状三明治形态就可以简化为看涨倒锤子线形态。如果不是这样,则该形态将简化为一根带有上影线的阴线,而它通常具有看跌的含义。所以,在实际使用时需要进一步对该形态进行确认。

\paragraph{看跌竖状三明治形态}
\begin{enumerate}
    \item 市场处于上涨趋势中,这一形态的第一天是一根阳线。
    \item 第二天是一根实体阴线,其开盘价低于前一天的收盘价,而其收盘价低于前一天的开盘价。
    \item 第三天是一根实体阳线,吞没了第二天的阴线。
\end{enumerate}

这一形态的关键点是第一天与第三天两根阳线的收盘价必须相等。看跌竖状三明治的第一天是一根阳线,其中心点位于10日移动平均线的上方,这意味着市场处于上升趋势中。第二天低开,然后一路下探,在最低价或接近最低价处收盘。这种走势表明上升趋势有可能发生逆转,投资者如果不清仓出场的话,也应该减少多头头寸。

强烈建议对其进行确认,因为这一形态可以简化为带有下影线而没有上影线的大阳线。
\subsection{挤压报警形态}
\section{}