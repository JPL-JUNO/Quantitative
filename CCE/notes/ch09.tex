\chapter{蜡烛图形态过滤}
基于单一的技术分析方法做出的交易决策往往太过轻率,很可能给交易者带来亏损,使用蜡烛图形态时当然也不例外。正如其他一些基于价格分析基础的技术分析指标一样,蜡烛图形态同样也不可能是永远正确的。但如果几个不同的技术分析指标给出相同的分析结果或交易信号,这时我们就有了更高的胜率。再强调一次,蜡烛图形态没有什么不同,将其与其他一些技术分析指标结合起来综合运用,这样能大幅提高我们交易时的胜算概率。
\section{过滤的定义}
我们可以对“过滤”进行这样的定义:利用其他技术指标或工具去除“虚假”的蜡烛图形态,或者去除一些早期的、“不成熟的”形态,这就是我们所说的过滤。由于蜡烛图形态和市场的趋势具有很强的关联性,所以在一个长期(持续时间长)的市场趋势中,蜡烛图形态总是不可避免地给出一些假信号,当然,其他的技术分析指标也不例外。大多数技术分析师都利用多种技术指标验证给出的买入或卖出信号。我们在使用蜡烛图形态时也应秉持同样的理念。

大多数指标都给出买入信号和卖出信号来帮助投资者制定交易决策。由于技术分析的数据来源于价格、成交量等不同的市场要素,所以技术指标的出现相对于市场的实际趋势会出现滞后。实际上,在买入信号或者卖出信号出现之前的区域才是进行交易的最佳区间,但在实盘交易中很难提前界定这个区域。另外,如果把技术指标的参数设置得过于严格,通常会出现许多错误的信号,或者频繁地出现各种信号。我们可以把市场交易信号出现之前的区域称为“预备信号区域”,在这个区域内我们可以做好交易的准备。

一旦指标达到预备信号区域,我们就可以将手指放在扳机上,等待随时交易。虽然我们不能确定指标会在预备信号区域内运行多长时间,但我们可以确定的是,一旦技术指标进入预备信号区域,或早或晚一定会出现市场交易信号(买入或卖出)。统计结果显示,指标在预备信号区域中停留的时间越长,给出的买入或者卖出的信号就越准确。

对于每一种技术指标来说,预备信号区域就是信号过滤区域,是它的指纹。就像每个人都有不同的指纹一样,每种技术指标的“指纹”也都不同。如果一个指标在买入的预备信号区域,这时我们只需观察看涨的蜡烛图形态。同样,如果一个指标在卖出预备信号区域,这时我们只需观察看跌的蜡烛图形态。
\subsection{预备信号区域}
如 \autoref{fig9-1} 所示,对于以临界值(极值)为判断标准的技术指标来说,预备信号区域指的是临界值到技术指标极值的一段区间。请注意,这一区域有上下两处,因为指标一般都有两个极值(超买和超卖)。

\figures{fig9-1}{以极值作为技术指标判断标准的预备信号区域}

如 \autoref{fig9-2} 所示,由于一些技术指标的数值一直围绕零线上下振荡,所以我们也把一些技术指标的预备信号区域定义为从越过零线开始到穿过移动平均线或者是交易信号的平滑趋势线为止的这一段区间。

\figures{fig9-2}{具有振荡特性的技术指标的预备信号区域}
\section{技术指标}
用于过滤蜡烛图形态的指标应该具有定义简单、操作简便的特性。它们必须以某种方式给出买入或卖出信号,例如,指标已经到达超买、超卖区域。威尔斯·韦尔德(Welles Wilder)的 RSI(相对强弱指标)和乔治·莱恩(George Lane)的 KD 指标(随机指标)都是相当好的蜡烛图形态过滤指标,因为这两个指标总是在上下限(0-100)之间运行。
\subsection{韦尔德的相对强弱指标}
相对强弱指标是反映目前价格运动相对强度的一个简单测量,它的值均在 0-100 之间。它基本上是上涨交易日和下跌交易日的平均值。韦尔德偏好使用14(交易日)作为相对强弱指标的默认参数(周期),这恰好是市场自然周期(每月)的一半。另外,他还认为相对强弱指标多在 30-70 之间波动。低于 30 时,说明市场处于超卖状态,即将出现向上的反转;高于 70 时,说明市场处于超买状态,即将出现向下的反转。

许多经典的图形形态,如头肩顶或头肩底形态,也会出现在相对强弱指标的图形中。如果价格持续上升,而 RSI 的值开始下降,就会出现市场价格和 RSI 值之间的背离。同时,在 RSI 的走势图中,就会出现 RSI 的头肩顶形态,特别是这种形态出现在指标上下限的时候,市场转势的概率较高。
\subsection{莱恩的随机指标:\%D}
从理论上讲,随机指标是一个摆动指标,用以测量在一个交易日中收盘价相对整个价格区间的位置。简单地说,就是当天的收盘价相对于前 $n$ 个交易日的平均价格区间来说处于什么位置。同相对强弱指标一样,随机指标的默认参数(即计算周期)也是 14 个交易日。

在市场中我们会观察到这样一种现象,即如果市场处在强劲的上升趋势中,收盘价往往在当日的最高价附近;如果处在猛烈的下跌趋势中,收盘价往往在当日的最低价附近,这种现象就是随机指标构建的基础。例如,当市场由上升趋势转入下跌趋势时,虽然市场仍能不断地创出新高,但收盘价却越来越接近当日的最低价。这正是随机摆动指标和其他摆动指标的不同,它反映的是一种相对强度,通过收盘价位置来测算趋势的相对强度。

随机指标是用 \%K、\%D 两条曲线构成的图形关系来分析市场价格走势的,对 \%K 值取三个周期简单移动平均值就能得出 \%D 值。理解随机指标在市场各个阶段的表现,才能正确地使用随机指标。通常,当 \%D 值突破上下临界值时,会出现一定的交易信号(当 \%D 值突破 75-85 的上限区间时,说明市场处于超买中,投资者应该开始考虑出场;当 \%D 值突破 15-25 的下限区间时,说明市场处于超卖中,投资者应该开始考虑逐渐建仓)。但是,实际上我们通常等 \%K 值突破了 \%D 值时,才真正进行交易。虽然在 \%D 值突破上下限时就给出了交易信号,但是我们等待两根线出现交叉是对市场走势的预测进行双重保险。
\subsection{过滤参数的设定}