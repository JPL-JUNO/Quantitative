\chapter{持续形态}
持续形态说明原先的趋势会持续(无论此前是上升还是下降),反转形态说明此前的趋势要被反转(无论此前是上升还是下降),这里的反转和持续都是相对于此前的趋势方向而言,而与此前的趋势是上升趋势还是下跌趋势无关。
\section{两日形态}
\subsection{分手线形态}
分手线形态是由两根具有相同开盘价但颜色相反的蜡烛图组成的,是约会线形态的相反形态。在该形态中,第二天的蜡烛图是一根执带线。看涨分手线形态中有一根白色的看涨执带线(光头收盘);看跌分手线形态中有一根黑色的看跌执带线(光脚开盘)。
\subsubsection*{形态识别的标准}
\begin{enumerate}
    \item 第一天蜡烛图的颜色和市场原有的趋势方向相反。
    \item 第二天蜡烛图的颜色和第一根蜡烛图的颜色相反。
    \item 两根蜡烛图的开盘价相同。
\end{enumerate}
\figures{fig4-1}{看涨分手线形态}

以看涨分手线为例,说明其背后的市场含义。当第一次出现大阴线的时候,市场处于上升趋势。对于处在强势上升中的市场来说,出现这样的市场走势难免会引起市场的许多猜测和争议。但第二天市场高开,并且开盘价与前一天的开盘价相同,随后市场一路走高,最后在接近最高点的位置高点收盘,这种走势的出现说明以前的市场走势并未因前一天的下跌而改变,原有的市场趋势还将继续。

组成分手线形态的蜡烛图最好是,但不是必须是两根长的蜡烛图。在强势分离线形态中,两根蜡烛图相重叠部分应该只是实体部分,而不应该出现上下影线重叠(即两根蜡烛是秃头或光脚蜡烛图)。\notes{没想出来如何出现影线的重叠,如果满足标准 3}

看涨分手线形态可以简化为一根带长下影线的阳线。这一形态本身就带有看涨的倾向,因此这一简化与看涨持续形态相符。看跌分手线形态可以简化为一根带有长上影线的阴线。这一形态带有看跌的倾向,因此这一简化与看跌持续形态相符。

\figures{fig4-3}{看涨分手线形态简化}

\subsection{待入线形态}
看跌待入线形态实际上是刺透线形态未充分演化的形态。这两个形态相似,区别在于看跌待入线形态中第二天的阳线只达到了前一天的最低价。不要将待入线形态与约会线形态混淆。看涨待入线形态是一个两日看涨持续形态。它与看跌待入线形态相反。

请注意:这一形态很少出现。

\paragraph{看跌待入线}
\begin{enumerate}
    \item 第一天的蜡烛图是一根大阴线,并且处于下跌趋势中。
    \item 第二天的蜡烛图是一根阳线,同时开盘价在前一天的收盘价之下。这根蜡烛图最好不是大阳线,否则形态可能演化成看涨约会线形态。
    \item 第二天收盘于第一天的最低价处。
\end{enumerate}

看跌待入线形态通常出现在下跌趋势中。第一天的大阴线的出现更是增强了这种看空的氛围。虽然第二天市场跳空低开,但下跌趋势未能延续,多方开始进行反攻,最后这种反弹在前一天的最低价处结束。这样的市场走势使那些入场抄底的投资者感到不安。市场很快会再次转入下跌趋势。

看跌待入线形态可以简化为一根带长下影线的大阴线。这种简化基本上符合该形态的看跌持续倾向。

\paragraph{看涨待入线}
\begin{enumerate}
    \item 市场处于上涨趋势中,第一天是一根大阳线。
    \item 第二天是一根阴线。它以高于第一天最高价的价格开盘,最后以第一天的最高价收盘。
\end{enumerate}

看涨待入线形态的第一天是一根大阳线。第一天价格波动范围的中心点位于 10 日移动平均线之上,这表明市场已经处于上涨趋势之中。第一天的大阳线进一步强化了上涨的趋势。

第二天的成交量越大,市场的下跌趋势延续的概率就越大。

第二天市场跳空高开,但随后转入下跌。尽管市场在某种程度上的确出现了短期的下跌走势,但价格未能有效跌破前一天的最高价。这种情况一定使那些当日入场抓顶的空头感到很不舒服。市场很快会再次转入上升趋势。

由于第二天是一根带有短下影线或没有下影线的阴线,因此它被称为光脚收盘蜡烛图。这要求第二天的价格波动范围(蜡烛图的长度)小于第一天的价格波动范围(蜡烛图的长度)。如果对第二天的价格波动范围没有限制,而且第二天的最低价进入第一天的实体范围之内,那么看涨待入线持续形态就会演变成看涨约会线反转形态。

看涨待入线形态可以简化为一根带长上影线的阳线。由于这一蜡烛图出现在上涨趋势中,因此可以被认为具有看涨倾向。这种简化基本上符合该形态的看涨持续倾向。
\subsection{切入线形态}
和看跌待入线形态相同,看跌切入线形态也是刺透线形态未完全演变的形态。第二天蜡烛图的白色实体在接近第一天阴线的收盘价处结束。若精确定义这种形态,应该是第二天蜡烛图的收盘价恰好和第一天的收盘价相等,或者是略高于第一天的收盘价。和看涨待入线形态相比,切入线第二天的收盘价要高一些,但高得不太多。如果第一天的收盘价同时也是它的最低价(形成光头或光脚收盘蜡烛图),那么看跌待入线形态和看跌切入线形态就相同了。

看涨切入线形态是一个两日看涨持续形态。它是看跌切入线形态的补
充。

\paragraph{看跌切入线形态}
\begin{enumerate}
    \item 第一天的蜡烛图是一根大阴线,此时市场处于下跌趋势中。
    \item 第二天的蜡烛图是一根阳线,而且在第一天的最低价之下开盘。
    \item 第二天的收盘价要略低于第一天的收盘价,事实上,两天的收盘价可以相等。
\end{enumerate}

看跌待入线形态可以简化为一根带较长下影线的大阴线,这种简化基本上符合该形态的看跌倾向。

\paragraph{看涨切入线形态}
\begin{enumerate}
    \item 第一天的蜡烛图是一根大阳线,此时市场处于上涨趋势中。
    \item 第二天是一根阴线。它的开盘价高于前一天的最高价,收盘价则刚好进入第一天的实体部分。
\end{enumerate}

看涨切入线形态可以简化为一根带较长上影线的阳线。由于这一形态出现在上涨趋势中,而这根蜡烛图可以被认为是看涨的,因此这种简化基本符合该形态的市场看涨倾向。


同看跌待入线形态一样,我们也可以把看跌切入线形态看成是刺透线形态的弱势开始形态。这种弱势形态虽然比待入线形态有所加强,但是仍不能证实市场将出现反转。另外,如果两根蜡烛图都是秃蜡烛图,那么该形态在图形上和看涨约会线形态有些相似。
\subsection{插入线形态}
看跌插入线形态是刺透线形态的第三种演化形式(前两种是待入线形态和切入线形态)。看跌插入线形态比看跌待入线形态和看跌切入线形态代表的市场反转的可能性要高,但是第二天蜡烛图的白色实体仍然没有突破第一天实体的中心点,因此我们不能认为插入线形态像刺透线形态一样也是市场反转形态。和看跌待入线、看跌切入线形态相比,在插入线形态中,第二天向下跳空要更低一些,这使得第二根蜡烛图成为一根大阳线,所以投资者在继续做空时应该首先对该形态进行确认。

\figures{fig4-18}{看跌插入线形态}
看涨插入线形态是一个两日看涨持续形态,它是对看跌插入线形态的补充。
\subsubsection*{形态识别的标准}
\paragraph{看跌插入线形态}
\begin{enumerate}
    \item 第一天的蜡烛图是一根大阴线,此时市场处在下跌趋势当中。
    \item 第二天的蜡烛图是一根阳线,而且在第一天的最低价之下开盘。
    \item 第二天的收盘价要高于第一天的收盘价,但是未超过第一天实体的中心点。
\end{enumerate}

看跌插入线形态可以简化为一根锤子线,这种蜡烛图的市场含义和简化前的形态完全相反。因为看跌插入线形态和刺透线形态十分接近,所以会发现该形态的简化形式不能成立。

\paragraph{看涨插入线形态}
\begin{enumerate}
    \item 第一天的蜡烛图是一根大阳线,此时市场处于上涨的趋势当中。
    \item 第二天的蜡烛图是一根阴线。其开盘价远高于第一天的最高价,然后一路下跌,收盘价在第一天的实体内部,但是不低于第一天蜡烛图实体的中心点。
\end{enumerate}

看涨插入线形态可以简化为一根流星线。这是简化成一根蜡烛图后不支持原来的看涨(在这个例子中是看涨)或看跌倾向的一个典型的例子。

尽管看涨插入线持续形态与看跌反转乌云盖顶形态相似,但两者之间存在着三个差别。差别主要体现在第二天的蜡烛图上:
\begin{itemize}
    \item 第二天的开盘价远高于第一天的最高价(我们要求第二天的开盘价比第一天的最高价要高,高出的幅度应大于第一天价格波动范围的 30\%);
    \item 第二天的收盘价在第一天蜡烛图实体的中心点之上;
    \item 第二天的收盘价接近当天的最低价。
\end{itemize}
\subsubsection*{关于待入线、切入线和插入线形态的附加说明}
为什么从刺透线形态演化而来的三种持续形态——看跌待入线形态、看跌切入线形态和看跌插入线形态——都不是市场反转的信号呢?同样,这三种形态的看涨形态为什么不能像乌云盖顶形态一样形成下跌反转的信号呢?

建立这三种形态是为了找出刺透线形态的相反形态。那么,为什么和乌云盖顶类似的形态不能成为市场反转的信号呢?在熟悉了市场顶部和底部的判断后,许多读者都能找到这类问题的答案。原因在于,\important{当市场处于底部时,反转的出现是迅速而狂热的,一旦反转开始,投资者会疯狂地入场抢筹;当市场处于顶部时,反转的出现是缓慢的,即使是出现了顶部反转信号,大多数的投资者还是心存侥幸,舍不得立即获利了结。}
\section{三日形态}
\subsection{向上跳空并列阴阳线形态和向下跳空并列阴阳线形态}
并列阴阳线通常在两种情况下出现:一种是在一根阳线之后,市场低开,然后逐渐走低,最后在前一天阳线之下收盘,形成阴线;一种是在一根阴线之后,市场高开,然后逐渐走高,最后在前一天阴线之上收盘,形成阳线。

向上跳空并列阴阳线如 \autoref{fig4-24} 所示,在第一根阳线之上,向上跳空出现一根阳线,然后是一根阴线,这根阴线的开盘价一定要在第二天阳线的实体范围内产生,最后的收盘价则要进入这个向上跳空的缺口内。在这个形态中,关键一点在于,第三天的阴线不能把向上跳空缺口完全填补,这样投资者就倾向于在收盘时持有股票。

\figures{fig4-24}{向上跳空并列阴阳线}
\subsubsection*{形态识别的标准}
\begin{enumerate}
    \item 市场一直在确定的趋势中运行,形成跳空缺口的两根蜡烛图应该具有相同颜色。
    \item 前两根蜡烛图的颜色代表市场原有的运行趋势。
    \item 第三天的蜡烛图应该在第二天蜡烛图的实体内产生开盘价,同时两者具有相反的颜色。
    \item 第三天的蜡烛图在第一天和第二天形成的跳空缺口内收盘,但是它并未将整个缺口填补。
\end{enumerate}

跳空并列阴阳线背后的交易情境及市场心理分析非常简单:由于在原有市场趋势中出现跳空缺口,虽然在调整日(第三天)市场试图回补缺口,但是没有成功,原来的市场趋势仍将继续。

虽然第一天蜡烛图的颜色没有后两天蜡烛图的颜色重要,但是如果第一天和第二天蜡烛图的颜色相同,跳空并列阴阳线形态的市场意义就更明确。

向上跳空并列阴阳线形态可以简化为一根具有白色实体的长蜡烛图,这种蜡烛图通常(在上升趋势中)具有看涨的市场含义,这支持了原形态的看涨意味。向下跳空并列阴阳线形态可以简化为一根大阴线,这根大阴线带有较长的下影线,只要下影线不是很长,这种大阴线就具有看跌的倾向。由于简化后的单一蜡烛图所具有的倾向不是很强烈,所以推荐对跳空并列阴阳线形态进行进一步确认。
\subsection{并列阳线形态}
并列的蜡烛图”既可以指阳线,又可指阴线,它们都表示市场处于一种盘整的趋势中。市场在调整的过程中,成交逐渐萎缩,等待新方向的出现。这一形态的重要之处在于,出现了两根并列的阳线,它们顺着当前的趋势同以前的蜡烛图之间形成了一个跳空缺口。

\paragraph{看涨并列阳线形态} \autoref{fig4-29} 是看涨并列阳线形态的示意图。如 \autoref{fig4-29} 所示,第二根和第三根阳线与第一根阳线之间有一个向上跳空的缺口,且这两根阳线的长短类似。不仅如此,这两天的开盘价也十分接近。所以看涨并列阳线形态也被称为向上跳空并列阳线形态。

市场处于上升趋势中,第一根大阳线的出现,更是坚定了市场投资者看多做多的决心;第二天市场向上跳空高开,并且一路攀升,最终在接近最高点收盘,但是第三天市场却低开高走,开盘价同前一天(第二天)的开盘价相近。这种低开表明,市场在上涨过程中,一些投资者开始恐高,他们希望获利了结,市场在清掉这部分获利盘后,抛压减轻,大多数的投资者开始坚决做多,价格一路回升;一些观望的投资者在看到这种情况后,也开始抢购筹码,最后推动市场回到前一天(第二天)的收盘价附近。

\figures{fig4-29}{看涨并列阳线形态}

\paragraph{看跌并列阳线形态} 它也被称为向下跳空并列阳线形态。虽然这里出现了阳线,似乎市场要上涨,但事实上这两根并列阳线只说明市场中的短线投资者在对卖空头寸进行平仓。同许多持续形态一样,看跌并列阳线形态表明市场在进行短期的调整,随后原趋势仍将延续。

市场处于下跌趋势中,第一根大阴线的出现更是增强了这种市场趋势;第二天,市场向下跳空低开,并且价格出现回升,但价格的上升并没有回补当天形成的跳空缺口;第三天,市场仍然低开,开盘价同前一天(第二天)的开盘价相近。这种情况表明市场加速下跌的势头已经减缓,于是短线空头开始平掉做空的头寸,这样多方的力量开始凝聚,市场价格再一次开始回升,只是仍然不能弥补前一天产生的向下跳空缺口。整个形态说明,此时做空头并未全部平仓,仍然有一部分投资者看空后市,同时反弹的力度和能量也不能坚定市场的做多决心,所以市场下跌在短期内仍然不会结束。



实际上,在市场中出现两根排列阴线的情况也比较常见。向下跳的并列阴线说明市场的发展趋势没有发生变化,市场仍将继续下跌。但是,投资者利用向下跳空并列阴线形态来判定市场进一步发展趋势的作用不大,因为这种形态的市场含义就是下跌。并列阳线形态的另一种变化形式是没有跳空缺口的并列阳线,但它通常出现在上升的市场趋势中。综上所述,我们可以把并列阳线形态看成是一种停顿形态(ikizumari narabiaka),它表明市场将在原有的趋势中构筑一个平台,进行一定的调整后,继续原有趋势。
\subsubsection*{形态识别的标准}
\begin{enumerate}
    \item 跳空缺口的方向同市场原有趋势方向一致。
    \item 第二天的蜡烛图是阳线。
    \item 第三天的蜡烛图同样也是阳线,而且和第二天的阳线长度近乎相等,同时开盘价相近。
\end{enumerate}

向上跳空并列阳线形态可以简化为一根大阳线,这根大阳线就明显具有看涨倾向。向下跳空并列阳线形态可以简化为一根大阴线,这根大阴线带有较长的下影线。由于下影线的存在,这根简化后的大阴线并不能完全证明市场的看跌倾向。所以,推荐读者对向下跳空并列阳线形态进行进一步确认。
\subsection{并列阴线形态}
\subsubsection*{形态识别的标准}
\paragraph{看涨并列阴线形态}
\begin{enumerate}
    \item 市场处于上涨趋势之中。这一形态的第一天是一根大阳线。
    \item 第二天是一根阴线,其开盘价高于第一天的收盘价。
    \item 第二天价格一路走低,但并没有完全填补缺口。
    \item 第三天以较高价格开盘,开盘价高于前一天价格波动范围的中心点。然而,价格一跌走低,以接近当天的最低价收盘。第三天也没有能够填补第一天与第二天形成的缺口。
\end{enumerate}

\figures{fig4-34}{看涨并列阴线形态。第二天跳空高开,然后一路走低,但收盘时并没有填补跳空缺口。第三天以较高价开盘,开盘价高于前一天价格波动范围的中心点。与第二天一样,第三天开盘后价格一路走低,但仍然没有填补第一天与第二天之间的缺口。第二天与第三天的收盘价基本相等。这一形态中的两根阴线可以看成是交易者在获利了结。一旦获利了结结束,上涨趋势将重启。}

在这一形态中,你应该确保第一天的价格波动范围大于前五天价格波动范围的平均值。第一天的蜡烛图必须拥有长实体。蜡烛图的实体是指介于开盘价与收盘价之间的部分,长实体是指实体部分占价格波动范围的比例超过 50\%。第二天和第三天的蜡烛图必须拥有实体,不能是十字星。

最后,这一形态第二天和第三天的价格波动范围与实体长度应该基本相同。同时我们还要求两根蜡烛图中较短的一根的价格波动范围要大于较长蜡烛图价格波动范围的 50\%,这意味着一天的价格波动范围永远也不可能超过另一天价格波动范围的 2 倍。我们还要求较短蜡烛图的实体长度超过较长蜡烛图实体长度的 50\%,这意味着一天的实体长度永远也不可能超过另一天实体长度的 2 倍。

\paragraph{看跌并列阴线形态}
\begin{enumerate}
    \item 市场处于下跌趋势之中。这一形态的第一天是一根大阴线。
    \item 第二天也是一根大阴线,其开盘价低于第一天的收盘价,因而在两根阴线实体之间形成一个缺口。
    \item 第三天的开盘价要高得多,但还是没有能够填补前两天形成的缺口。但是,第三天的价格一路走低,在接近当天最低价处收盘。
\end{enumerate}

\begin{tcolorbox}
    注意:两种形态都要求第一天和第二天实体之间的缺口大于第一天价格波动范围的 10\%。
\end{tcolorbox}
\subsection{向上跳空三法形态和向下跳空三法形态}
向上跳空三法形态和向下跳空三法形态的定义比较简单。它们看上去分别与向上跳空并列阳线和向下跳空并列阴线形态类似,这两种形态通常出现在强势市场中。形态前两天的蜡烛图具有相同的颜色,同时代表市场原有的趋势,两根蜡烛图之间存在一个跳空缺口。第三天,市场在第二天蜡烛图的实体范围内开盘,最后价格进入到第一天的价格区域中。第三根蜡烛图的颜色和前两天的相反,如果按技术分析的术语来说就是,在这一天跳空缺口得到了回补。

\figures{fig4-40}{向上跳空三法形态。市场原有趋势较为强劲,跳空缺口的出现更显示出市场强烈的上涨(下跌)意愿。第三天,市场在第二根蜡烛图的实体内开盘,并且完全回补了前一天形成的跳空缺口。虽然跳空缺口的出现可用来确定市场的支撑位或是阻力位,但在这里我们应该将这种回补看成是对原有市场趋势的一种认同。因为缺口是在一天中就被弥补的,虽然我们可以把第三天的市场运动看成是获利盘的回吐,或者是空头的平仓。}

\subsubsection*{形态识别的标准}
\begin{enumerate}
    \item 从第一天和第二天的蜡烛图来看,市场趋势在延续,同时两根蜡烛图之间形成了跳空缺口。
    \item 第三天的蜡烛图回补了先前出现的跳空缺口,而且颜色和前两天的相反。
\end{enumerate}

向上跳空三法形态可以简化为一根流星线;向下跳空三法形态可以简化为一根锤子线(。这两种简化后的蜡烛图都不符合原有形态的倾向,所以推荐读者在利用它们进行市场判断时,对形态进一步确认。
\subsection{战后休整形态}
看涨战后休整形态是一个三日看涨持续形态。建立这种形态,是为了对某一上涨趋势进行说明,这一类形态以一根大阳线开始,经过几天的振荡后,接着又是一根大阳线,然后又是几天的振荡。这种“上台阶式”的上涨趋势可能持续 3-8 周。同时,上涨的力量在逐步增强,你会看到连续出现几根向上跳空高开的阳线,以及少量的连续阴线,直到上涨的力量被耗尽(见 \autoref{fig4-45})。

\subsubsection*{形态识别的标准}
\begin{enumerate}
    \item 战后休整形态的第一天是一根大阳线。其价格波动范围的中心点位于 10 日移动平均线的上方,这意味着市场处于上涨趋势之中。
    \item 第一天的价格波动范围应该大于这一形态出现之前五天价格波动范围的平均值。
    \item 第一天的蜡烛图必须具有很长的实体。
\end{enumerate}

\figures{fig4-45}{战后休整形态。这一形态的第一天显示出买盘十分踊跃。你不希望所有的做多能量在这一天之内被耗尽,因此这一形态出现的前一天的阳线不可能比这一形态的第一根蜡烛图更长。另外,如果这一形态出现在持续的上涨趋势之中,要小心观察是否还有足够的上涨动能(空间)。}

这一形态的第二根和第三根蜡烛图代表着在第一天强势上涨后的休整,第二天和第三天的蜡烛图相对较短,其实体也较短。特别是,第二天与第三天的价格波动范围必须小于第一天价格波动范围的 75\%。这两根蜡烛图的实体部分必须小于当日价格波动范围的 50\%。

第二天和第三天的蜡烛图意味着休整,而不是走弱或延续第一天的强势。因此,第二天和第三天的收盘价必须都高于第一天价格波动范围的中心点。另外,第三天的最低价必须高于第一天价格波动范围的中心点。这要求第一天之后价格不会下跌过多。

为了确保第一天之后还具有一定的上涨动能但又不过于强势,你应该保证第二天的最高价高于第一天的收盘价。第二天可以是阳线,也可以是阴线,因此第二天的最高价可以是开盘价,也可以是收盘价。在任何一种情况下,第二天的跳空高开显示出还存在一些额外的购买意愿。为了限制第二天的强度,你应该保证第二天的最低价低于第一天的最高价。

再强调一次,第二天的走势不能过强也不能过弱,因此第三天的开盘价和收盘价必须低于第二天的最高价,且必须高于第二天的最低价。与第二天一样,第三天既可以是阳线,也可以是阴线。

\figures{fig4-47}{市场实例战后休整+形态}
\section{四日或更多日形态}
\subsection{上升三法形态和下降三法形态 \autoref{fig4-48}}
三法形态由看涨上升三法和看跌下降三法组成。这两种形态都属于持续形态,而不是反转形态,它们表明市场在原有的发展趋势中出现停顿,调整一段时间后,将沿原有趋势继续前进,而不会引发反转。我们可以把它们看成是市场在原有趋势中略做休整,然后重新加速上升或下降。

\paragraph{上升三法形态} 在第一天,市场形成一根大阳线之后,连续出现一组实体很短的蜡烛图,它们表明市场在原有的趋势中遇到了阻力。通常这一组蜡烛图都是小阴线,而且重要的图形特征是:这些蜡烛图的实体部分都未超过第一天的价格变动范围(最高价和最低价)。请大家记住,价格波动范围包括上影线和下影线。形态最后一天(第五天)的开盘价要高于前一个回调日(第四天)的收盘价,并且收盘价为这一段时期以来的市场新高。

\figures{fig4-48}{上升三法形态最早源于日本期货交易中著名的酒田战法。三法形态通常被认为是市场的一种休整状态,它在为随后而来的市场发展积蓄能量。套用一个时髦的词就是“市场在调整”。这种市场变动说明一些不坚定的投资者在进行获利了结(在下跌的市场中,坚持做空的投资者也可以获利),市场主力在将它们振荡出局后,将沿原有的市场发展方向加速前进。从调整日中较小的市场价格波动就可以看出这些投资者的犹豫不定。但一旦在调整日中市场不能形成新低,一些摇摆不定、拿不准主意的投资者就会开始买入,于是市场价格开始回升,并创出一段时期的新高。}

\subsubsection*{形态识别的标准}
\begin{enumerate}
    \item 形态第一天的蜡烛图(大阳线或大阴线)代表市场原有的趋势。
    \item 第一天的蜡烛图之后是一系列实体很短的蜡烛图,它们的颜色最好和第一根蜡烛图的颜色相反。
    \item 这些小实体的蜡烛图和原有的市场趋势相反(上升或是下降),但是都没有超越第一天蜡烛图的价格波动范围。
    \item 最后一天,市场回复原有的趋势,表现出强烈的上升或者下降欲望,创出市场新高或者新低。
\end{enumerate}

虽然从理论上讲,上升三法形态和下降三法形态是由五根蜡烛图组成的,但在现实中,严格意义上的三法形态很难见到。因此,大多数学者认为可以放宽调整日中的限制条件,即那三根小蜡烛图可以超过第一天的价格波动范围,但不能超出过多。如有可能还是应该严格按“形态识别的标准”中的条件来判定三法形态。另外,如果这些小蜡烛图没有同原有的趋势相反,而是相同,三法形态就演变成了铺垫形态,这种形态通常在上升趋势中出现。

和看涨上升三法形态相关的形态是铺垫形态。铺垫形态也是上升趋势的持续形态的一种,它允许调整日的图形有更大的变化余地,即那一系列小阴线不必非要局限在第一根大阳线的价格波动范围内,完全可以超越它,相对于第一天而言,调整日的市场仍保持一定的上涨趋势。所以和看涨三法相比,铺垫形态的市场看涨意味更强烈。
\subsection{铺垫形态 \autoref{fig4-53}}
\paragraph{看涨铺垫形态} 看涨铺垫形态是上升三法的演化形式。如 \autoref{fig4-53} 所示,该形态的前三天看起来有点儿像向上跳空两只乌鸦形态,不同的是,第二根阴线(第三天的蜡烛图)弥补了第二天的向上跳空缺口,进入第一天大阳线的价格范围内。第四天市场继续走低,蜡烛图的收盘价仍然停留在第一天大阳线的价格范围内。第五天市场向上跳空开盘,产生一个较大的跳空缺口,然后一路上涨,超过前三天一系列小阴线的价格区间,创出市场新高。这个形态表明市场虽然出现反复,但依然维持上升趋势,并且使市场重心向上抬高。作为一种持续信号,与上升三法相比,看涨铺垫形态显示出的信号强度更高。换句话说,相对上升三法形态,看涨铺垫形态代表的市场调整的力度较小,对原有趋势的破坏程度较低。

\figures{fig4-53}{当第一天大阳线出现的时候,市场一直在上升趋势之中。第二天,市场跳空高开,盘中价格窄幅振荡,最后收盘时略有下跌,这个下跌只是相对开盘价而言,但就一段时期的市场收盘价来说,仍然是市场新高。这种情况说明,市场虽然下跌,但只是上升过程中的休整,不会对原有趋势造成太大影响。对于一些谨慎的市场投资者来说,他们害怕市场上升将会结束,开始获利平仓,所以第三天和第四天市场都是低开低走,即使是这样,市场价格依然维持在第一天的开盘价之上,说明市场并未出现反转,下跌只是暂时现象。最后一天市场高开高走,多方一路凯歌,一举收复前三天的失地。该形态的市场含义说明,牛市行情并未结束,下跌只是上涨趋势中的正常调整。}

\subsubsection*{形态识别的标准}
\paragraph{看涨铺垫形态}
\begin{enumerate}
    \item 市场处于上涨趋势之中,这一形态的第一天是一根大阳线。
    \item 第二天的向上跳空缺口和较低的收盘价使蜡烛图看起来有些像星线。
    \item 随后两天进行休整,与上升三法相似。
    \item 第五天是一根阳线,并以市场新高收盘。
\end{enumerate}
\paragraph{看跌铺垫形态}
\begin{enumerate}
    \item 市场处于下跌趋势之中,这一形态的第一天是一根大阴线。
    \item 第二天是一根阳线,其实体与前一天的阴线实体之间存在一个缺口。
    \item 随后两天是两根相对较短的蜡烛图,每天的最高价与最低价都会比前一天更高。
    \item 第五天是一根大阴线,其开盘价低于第四天的收盘价,收盘价低于第二天的开盘价。
\end{enumerate}

虽然从图形上看,形态前三天和向上跳空两只乌鸦形态有些类似,但是第三天蜡烛图的收盘价仍然停留在第一天的实体范围内,消除了市场可能转势的可能性。另外,读者要注意该形态也有点儿像三只黑乌鸦形态,但最后一天的大阳线可以排除这种可能性。

\subsection{三线直击形态}
三线直击形态是由四根蜡烛图组成的形态,通常出现在确定的市场趋势中。读者可以把它看成是三只黑乌鸦形态(看跌反转形态)或者是白色三兵形态(看涨反转形态)的扩展形式。

\paragraph{看涨三线直击形态} 市场处在上 趋势中,而且连续三天出现阳线,市场不断创出一段时期以来的新高 但是第四天市场跳空高开,然后一路下跌,将前三天的上涨尽数吃掉, 且在第一天的开盘价之下收盘。如果市场在这四天之前 直处于强势上涨阶段,那我们就可以把这种回调看成是获利盘回吐,这一天也可以称为是“清算日”。但请读者记住,这一切必须建立在市场强势上升的基础之上。
\begin{enumerate}
    \item 前三天的蜡烛图组合看起来有些像白色三兵形态,市场持续保持上升趋势。
    \item 第四天市场高开低走,最终在第一天的开盘价之下收盘。
\end{enumerate}

\figures{fig4-59}{市场始终在原有的上升趋势(或下跌趋势)中运行,出现的白色三兵形态或三只黑乌鸦形态就是这一点的最好例证。第四天,市场沿原有的趋势开盘(如果市场原有的趋势是上升的,则指市场向上跳空高开,如果市场原有的趋势是下降的,则指市场向下跳空低开),但是由于获利盘的涌出(短线多头或空头的平仓),导致市场突然转向,向相反的方向运行。虽然这种转向是市场心理的一种表现,但是请读者注意,它会使市场前三天的上涨或下降化为乌有。这种强劲的转向使空方或多方的能量在短时间内得到释放,因此它不会改变市场原有的运行趋势。}

\paragraph{看跌三线直击形态}
\begin{enumerate}
    \item 前三天的蜡烛图组合看似三只黑乌鸦形态,市场持续保持下跌趋势。
    \item 第四天市场低开高走,最终在第一天的收盘价之上收盘。
\end{enumerate}

看涨三线直击形态可以简化为一根流星线,但它违背了原形态所具有的看涨倾向。看跌三线直击形态可以简化为一根锤子线,它和原形态的市场含义同样存在着矛盾之处。