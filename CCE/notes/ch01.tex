\chapter{简介}
\section{技术分析}
就技术分析而言,我们应该牢记这样一个原则:事物的后续发展常常和它们之前的表象并不一致。我们自以为了解的很多事实并不是真正的事实,一些看起来显而易见的事情,有时也并非如此。

当事情或形势与本人相关时,人们的关注度也会增大。我们会被我们的情绪或情感蒙骗,这些情绪会影响我们的感知。在我们投资组合的净值开始下降时,我们就会不断胡乱空想出一大堆的利空消息,诸如经济衰退、债务危机、战争、政府预算赤字、银行倒闭等。因此为了避免成为错觉和情绪的牺牲品,我们应该学习技术分析。

几乎所有的技术分析方法都可以提供有价值的市场信息,这些信息进一步揭示了市场行为,从而加深了投资者对市场的理解程度。同时,那些失败的惨痛经历也会使投资者认识到,仅仅通过技术分析并不能保证实际的交易成功。对于那些在交易中蒙受损失的投机者来说,失败的原因不仅仅是错误的市场分析,而且他们缺乏将市场分析转化为实际操作的能力。投资者必须学会克服恐惧、贪婪和欲望等因素,才能真正地将技术分析运用在实际的交易之中。具体来说,在我们遇到暂时不利的因素时,首先要学会控制住自己的焦躁情绪,不要轻易地放弃自己正确的分析结果。我们要秉持自己的原则,坚信自己的判断是通过正确的技术分析得到的,不要被别人的言语和市场的假象左右。
\section{日本蜡烛图分析}
在市场交易中,价格受交易者心理诸如恐慌、贪婪和欲望等各种情绪影响,但我们无法使用传统的统计学方法测量或量化市场整体的心理状况,因此我们就需要使用一种可以量化市场心理因素变化的技术分析方法。日本蜡烛图技术可以通过当下的价格走势直观地揭示投资者现在的心理变化。日本蜡烛图不仅仅为我们提供了易于识别辨认的各种形态组合,同时也形象地反映了市场上多空双方间的相互博弈。
\section{蜡烛图与柱状图之比较}
\subsection*{标准柱状图}
绘制标准的柱状图,我们需要下面四个要素:开盘价、最高价、最低价和收盘价,当然这些要素都是对应于特定的时间周期。在柱状图中,竖直线段的长度代表交易日中最高价和最低价之间的跨度(\autoref{fig1-1})。最高价指的是交易日内最高的成交价格;反之,最低价指的是交易日内最低的成交价格。