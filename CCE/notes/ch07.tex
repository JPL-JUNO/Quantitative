\chapter{蜡烛图形态识别的可靠性}
\section{成功的衡量标准}
在判断蜡烛图形态能否成功预测市场走势时,要遵循以下三个假定:
\begin{itemize}
    \item 蜡烛形态的基础是开盘价、收盘价、最高价、最低价这四个要素和其相互间的关系。
    \item 在进行蜡烛图形态的识别和确认前,首先应确认市场现有趋势。
    \item 正确地判定蜡烛图形态后才能正确地衡量该形态的市场预测能力。
\end{itemize}
为了使预测结果具有可信性,你首先应该明确自己是否了解当前的市场走势,其次能正确地判定蜡烛图形态。同时你还应该认识到,利用蜡烛图形态对未来市场走势进行预测,是一种存在风险和不确定的行为。
\subsection{市场趋势已知}
通常,我们把蜡烛形态划分为反转形态和持续形态。

我们对市场进行预测,实际上就是要判断,在一定的预测时间段内市场价格是按原有趋势方向发展,还是出现反转。预测时段的变化会影响形态对市场预测的准确性。预测时段是指我们希望对未来多少天以后的市场价格进行预测,从实际蜡烛图形态出现的时点到未来价格变化时点的时间跨度就是预测时段。
\subsection{市场趋势未知}
有时候,我们在利用蜡烛图形态进行预测时,并不知道市场的原有趋势。在这种情况下,进行市场预测看起来有些像在猜硬币的正反面。如果在市场趋势未知的前提下,无论是反转形态还是持续形态,它们对市场预测的准确率都不会高于 50\%。

\begin{tcolorbox}
    请谨记,要想利用蜡烛图形态进行价格预测,首先一定要了解当时的市场趋势。
\end{tcolorbox}
\section{蜡烛形态统计评级}
蜡烛图形态实际上是一张具有预测功能的、反映市场交易心理的图表(或窗口),如果交易者能正确运用蜡烛图,就可以从中得出很多有价值的预测结果。

由于市场中存在趋势,持续形态的成功率必然比反转形态的成功率高。这就是为什么在随后的表格中出现,持续形态的预测成功率高,可是评级却不高的原因。

为什么这一分析只利用最多未来 7 天(包括 7 天)的预测间隔?超过这一时间框架,蜡烛图的预测效果就会减弱。蜡烛图的本质就是对短期走势进行预测,超过这一时间的任何预测结果都纯粹是巧合。请记住,蜡烛图可以预测反转或持续形态,但并不能预测形态持续时间。

\begin{tcolorbox}[title=重点注意]
    相同低价+,分手线+,倒锤子线+,向下跳空两只兔子+(极少见),藏婴+,三次向下跳空+,深思+,十字星+,插入线+
\end{tcolorbox}
